\date{Padova, 23 Gennaio 2016}
\begin{letter}{
	Responsabile Bugbusters \\
	Università degli Studi di Padova \\
	Dipartimento di Matematica Pura ed Applicata \\
	Via Trieste, 63 \\
	35121 Padova
}
\signature{
\begin{center}
\textit{
Gruppo \textit{Amministratori} \GroupName{}
}
\end{center}
}
\opening{Egregio Responsabile,}
\begin{quotation}
Durante l'intensa fase di produzione dei documenti, l'organizzazione dei vari membri del gruppo \`e diventata difficile ed errori dovuti alla stesura manuale di alcuni documenti portava ad uno aumento considerevole dei tempi nella fase di correzione.

L'origine del problema era la mancanza di un'infrastruttura software in grado di fornire: un quadro d'insieme al \textit{Responsabile}, strumenti di individuazione e correzione degli errori per i \textit{Verificatori}, unificazione con servizi software esterni ed integrazione con il progetto \textit{MaaS}.


La soluzione che vi propongo \`e la progettazione e lo sviluppo di un programma ( chiamato da noi \textbf{Chronos} ), su interfaccia grafica, per debellare il rischio di riproposizione dei problemi sopra citati ed abbracciare i tre principi fondamentali della \textit{Software Engineering} durante tutto l'arco di sviluppo del progetto.


Per garantire la massima efficacia ogni membro del gruppo dovr\`a lavorare esclusivamente attraverso questo software.

Di seguito le elenco i requisiti con a fianco il livello strategico:
\begin{enumerate}
\item Interfaccia grafica sviluppata con le tecnologie \textit{NW.js} - \textbf{Obbligatorio};
\item Integrazione con \textit{Git}
      \begin{itemize}
        \item Inserire nei commit il task a cui si riferisce ed l'assegnatario - \textbf{Obbligatorio};
        \item Permettere il commit direttamente dall'interfaccia - \textbf{Desiderabile};
        \item Chiusura dell'issue su \textit{GitHub} dove \`e stata fatta richiesta del task - \textbf{Opzionale};   
      \end{itemize}
\item Integrazione con \textit{Aspell}
      \begin{itemize}
        \item Individuazione file con errori sintattici - \textbf{Obbligatorio};
        \item Correzione automatica - \textbf{Obbligatoria};
        \item Procedura guidata per la sostituzione manuale - \textbf{Opzionale};
      \end{itemize}
\item Indice di Gulpease sui vari documenti - \textbf{Obbligatorio};
\item Lista task aperti, con assegnatario e data di scadenza - \textbf{Obbligatorio};
\item Integrazione con \textit{Mephisto}
      \begin{itemize}
        \item Gestione parole glossario - \textbf{Obbligatoria};
        \item Gestione casi d'uso - \textbf{Obbligatoria};
        \item Possibile interrogare il database per la creazione dei documenti - \textbf{Obbligatoria};
      \end{itemize}
\item Struttura a plug-in - \textbf{Desiderabile};

\end{enumerate}

Rimaniamo a Sua completa disposizione per ogni ulteriore chiarimento.
\end{quotation}
\closing{Cordiali saluti,}
\end{letter}
