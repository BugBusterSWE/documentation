\begin{frame}

\begin{center}
  \usebeamerfont*{title} \usebeamercolor[fg]{title} \huge Analisi dei requisiti
\end{center}

\end{frame}

\section{Analisi dei requisiti}
\begin{frame}
  \frametitle{Scopo del progetto}

  \begin{center}
    Convertire MaaP in un servizio ed ampliare le funzionalità offerte agli utenti.
  \end{center}
\end{frame}

\begin{frame}
  \frametitle{Cos'è MaaP?}

  MaaP è un applicazione web destinata alle aziende, che permette di creare delle interfaccie HTML per la visualizzazione di dati aziendali.

\end{frame}

\begin{frame}
  \frametitle{Requisiti Principali}

  \begin{itemize}
  \item un utente deve poter iscrivere la propria azienda
  \item il proprietario dell'azienda può invitare dei collaboratori
  \item un utente può collegare la sua base di dati al servizio
  \item tramite un editor deve essere possibile creare delle viste per i propri dati
  \item deve essere accessibile la visualizzazione dei propri dati tramite le viste create
  \end{itemize}
\end{frame}


\begin{frame}
  \frametitle{Utenti del sistema}

  \begin{itemize}
  \item Super-Admin: proprietario del servizio
  \item Owner: proprietario di una Company
  \item Admin: amministratore di una Company
  \item Utente Base: utente registrato, dipendente di una Company
  \item Guest: utente temporaneo di una Company
  \end{itemize}

\end{frame}

\begin{frame}
  \frametitle{Utenti del sistema}
  Il Super-Admin deve poter:
  \begin{itemize}
  \item creare delle nuove Companies
  \item gestire le Companies esistenti
  \item gestire gli utenti esistenti
  \item aggiungere nuovi Super-Admin
  \end{itemize}

  immagine casi d'uso super admin
\end{frame}

\begin{frame}
  \frametitle{Utenti del sistema}
  Owner, Admin, Utente base possono:
  \begin{itemize}
  \item creare nuove viste utilizzando l'editor
  \item visualizzare e disporre le viste nella propria dashboard utente
  \end{itemize}

  immagine casi d'uso principali
\end{frame}

\begin{frame}
  \frametitle{Utenti del sistema}
  Inoltre:
  \begin{itemize}
  \item L'Owner può concedere e revocare privilegi di admin
  \item l'Admin e l'Owner possono gestire gli utenti della Company
  \end{itemize}

\end{frame}

\begin{frame}
  \frametitle{L'editor}

  \begin{itemize}
  \item Permette di creare intuitivamente le viste per gli utenti
  \item Deve permettere di esportare i dati in formati Json e CSV
  \item deve dare la possibilità di modifica agevole delle viste
  \end{itemize}

  immagine casi uso editor
\end{frame}

\section{Piano di Progetto}
\begin{frame}
  \usebeamerfont*{title} \usebeamercolor[fg]{title} \huge Piano di Progetto
\end{frame}

\begin{frame}
  \frametitle{Modello di ciclo di vita scelto}
  Il modello di ciclo di vita scelto è quello \textbf{incrementale} perchè:
  \begin{itemize}
  \item Permette di monitorare l'evoluzione del progetto
  \item si ottiene una base verificata alla fine di ogni processo
  \item Permette l'alternanza tra le attività
  \end{itemize}

  immagine ciclo di vita
\end{frame}


\begin{frame}
  \frametitle{Rischi individuati}
  Per una migliore organizzazione sono stati suddivisi per tipologie:
  \begin{itemize}
  \item Tecnologici
  \item Derivanti dalle persone
  \item Organizzativi
  \item Derivanti dal software
  \item Derivanti dai requisiti
  \item Derivanti da stime errate
  \end{itemize}
\end{frame}
  
\begin{frame}
    \frametitle{Rischi individuati - Derivanti dalle persone}

    Sono i rischi con probabilità maggiore. In particolare sono stati individuati rischi riguardanti:
    \begin{itemize}
    \item La disponibilità di tempo dei membri del gruppo
    \item La possibilità di conflitti tra elementi del gruppo
    \item Le lacune portate dalla poca conoscenza delle tecnologie applicate
    \end{itemize}

    immagine focus group
\end{frame}

\begin{frame}
  \frametitle{Rischi individuati - Rischi sul software}
  Rischi che derivano dal software utilizzato per sviluppare il progetto.
  Il rischio di indisposizione delle piattaforme Teamwork e GitHub su cui si appoggia il progetto, seppur la probabilità di verificarsi sia remota, porterebbe disagi notevoli per il progetto.

  immagine logo teamwork

  immagine logo github
\end{frame}

\begin{frame}
  \frametitle{rischi individuati - Derivanti dai requisiti}
    Punto di rischio con maggior impatto e probabilità per il progetto.
    I rischi sono dovuti a:
    \begin{itemize}
    \item Comprensione errata dei requisiti
    \item Cambiamento di alcuni dei requisiti individuati
    \end{itemize}    
\end{frame}

\begin{frame}
  \frametitle{Pianificazione del progetto}
  Seguirà la seguente ripartizione delle ore
  immagine ripartizione ore
\end{frame}

\begin{frame}
  \frametitle{Preventivo calcolato}
  Per la realizzazione dell'intero progetto si è preventivato quanto segue:

  immagine tabella preventivo

  immagine grafico costi progetto
\end{frame}


