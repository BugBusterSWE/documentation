\section{Introduzione}
\subsection{Scopo del documento}
Lo scopo di questo documento è evidenziare le funzionalit\`a che il prodotto dovr\`a presentare e descriverle formalmente. Tali requisiti sono emersi dal capitolato presentato e dagli incontri svolti con il Proponente.

\subsection{Scopo del progetto}
Lo scopo del \glossaryItem{progetto} è la realizzazione di un servizio per le aziende, raggiungibile in un server web, per la visualizzazione di dati aziendali. Tale \glossaryItem{progetto} si basa su \glossaryItem{MaaP} (\textbf{M}ongoDB \textbf{a}s an \textbf{a}dmin \textbf{P}roduct), un'applicazione già esistente che ha lo scopo di fornire una visualizzazione dei dati letti dal database \glossaryItem{mongoDB} in possesso dell'azienda. Il \glossaryItem{progetto} verte sulla conversione di \glossaryItem{MaaP} da applicazione web a servizio, estendendone le potenzialità con un editor per facilitare ai nuovi utenti la creazione di viste per i propri dati.

\subsection{Glossario} 
Ogni occorrenza di acronimi, dei termini tecnici o di dominio è evidenziata con il corsivo e marcata con la lettera G in pedice. Nel documento \Glossario sono riportati i significati corrispondenti.

\subsection{Riferimenti}
\subsubsection{Normativi}
\begin{itemize}
\item Normativa ISO 12207 per \glossaryItem{ciclo di vita del software} (\url{https://en.wikipedia.org/wiki/ISO/IEC_12207})
\item Swebok (\url{http://www.computer.org/web/swebok/v3})
\item Capitolato d'appalto (\url{http://www.math.unipd.it/~tullio/IS-1/2015/Progetto/C4.pdf})
\end{itemize}

