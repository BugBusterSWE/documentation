\section{\glossaryItem{Super-Admin}}

Di seguito vengono elencati i casi d'uso per il \glossaryItem{Super-Admin}.

%''Fig. 29 e seguenti: esistono casi d’uso senza codice identificativo.
%Questi non hanno alcuna descrizione associata. Tutti i diagrammi dei casi
%d’uso devono avere associata una relativa descrizione (come fatto fino a
%questo punto).''

\subsection{Casi d'Uso}

\subsubsection{Interazione ad alto livello tra sistema e \glossaryItem{Super-Admin}}

    \begin{figure}[h]
      \begin{center}
        \includegraphics[width=12cm]{res/img/UCSuperadmin/UC-S0.png}
      \caption{UC-S0 - Gestione \glossaryItem{Company}}
      \end{center} 
    \end{figure}    
    
    %Tabella 
    \begin{center}
      \bgroup
      \def\arraystretch{1.8}     
      \begin{longtable}{  p{3.5cm} | p{8cm} } 
        
        \hline
        \multicolumn{2}{ | c | }{ \cellcolor[gray]{0.9} \textbf{UC-S0 - Gestione \glossaryItem{Company}}} \\ 
        \hline
        
        \textbf{Attori Primari} & \glossaryItem{Super-Admin}.\\  
        \textbf{Precondizioni}  & L'applicazione mostra al \glossaryItem{Super-Admin} la pagina di gestione delle \glossaryItem{Company}.  \\ 
        
        \textbf{Postcondizioni} & L'applicazione ha dato la possibilità al \glossaryItem{Super-Admin} di interagire con la pagina di gestione delle \glossaryItem{Company}. \\ 
        \textbf{Scenario principale} & 1. Il \glossaryItem{Super-Admin} pu\`o creare una nuova \glossaryItem{Company}. (UC-S1) 
        
        2. Il \glossaryItem{Super-Admin} può visualizzare il dettaglio di una \glossaryItem{Company}. (UC-S2)
        
        3. Il \glossaryItem{Super-Admin} pu\`o modificare i dati di una \glossaryItem{Company}. (UC-S3)  \\ 
        
        \textbf{Estensioni} & 1. Fallisce l'inserimento di una nuova \glossaryItem{Company}.
        
        2. Fallisce la modifica dei dati di una \glossaryItem{Company}. \\
      \end{longtable}
      \egroup
    \end{center}

\subsubsection{Creazione di una nuova \glossaryItem{Company}}
    \begin{figure}[H]
      \begin{center}
        \includegraphics[width=12cm]{res/img/UCSuperadmin/UC-S1.png}
      \caption{UC-S1 - Creazione di una nuova \glossaryItem{Company}}
      \end{center} 
    \end{figure}    
    
    %Tabella 
    \begin{center}
      \bgroup
      \def\arraystretch{1.8}     
      \begin{longtable}{  p{3.5cm} | p{8cm} } 
        
        \hline
        \multicolumn{2}{ | c | }{ \cellcolor[gray]{0.9} \textbf{UC-S1 - Creazione di una nuova \glossaryItem{Company}}} \\ 
        \hline
        
        \textbf{Attori Primari} & \glossaryItem{Super-Admin}.\\  
        \textbf{Precondizioni}  & Il sistema fornisce al \glossaryItem{Super-Admin} un form di registrazione.  \\ 
        
        \textbf{Postcondizioni} & Il sistema ha aggiunto una nuova \glossaryItem{Company} e il suo \glossaryItem{Owner}. \\ 
        \textbf{Scenario principale} & 1. L'\glossaryItem{Attore} inserisce il nome della \glossaryItem{Company}.
        
        2. Il \glossaryItem{Super-Admin} inserisce la propria email.
        
        3. Il \glossaryItem{Super-Admin} inserisce una password. \\ 
        \textbf{Estensioni} & 1. la \glossaryItem{Company} esiste gi\`a. 
        
        2. L'email inserita \`e gi\`a stata usata. \\
      \end{longtable}
      \egroup
    \end{center}

%%DA FARE
    \subsubsection{Creazione di una nuova \glossaryItem{Company} - inserimento del nome della nuova \glossaryItem{Company}} 
    
    %Tabella 
    \begin{center}
      \bgroup
      \def\arraystretch{1.8}     
      \begin{longtable}{  p{3.5cm} | p{8cm} } 
        
        \hline
        \multicolumn{2}{ | c | }{ \cellcolor[gray]{0.9} \textbf{UC-S1.0 - Inserimento del nome della \glossaryItem{Company} }} \\ 
        \hline
        
        \textbf{Attori Primari} & \glossaryItem{Super-Admin}\\  
        \textbf{Scopo e descrizione} & L'applicazione offre un campo testo su cui il \glossaryItem{Super-Admin} può scrivere il nome della \glossaryItem{Company} \\
      
        \textbf{Precondizioni}  & L'applicazione offre un campo testo. \\ 
        
        \textbf{Postcondizioni} & Il \glossaryItem{Super-Admin} ha compilato il campo testo relativo al nome della \glossaryItem{Company} \\ 
     \end{longtable}
      \egroup
    \end{center}


\subsubsection{Creazione di una nuova \glossaryItem{Company} - inserimento dell'email dell'\glossaryItem{Owner} della nuova \glossaryItem{Company}} 
    
    %Tabella 
    \begin{center}
      \bgroup
      \def\arraystretch{1.8}     
      \begin{longtable}{  p{3.5cm} | p{8cm} } 
        
        \hline
        \multicolumn{2}{ | c | }{ \cellcolor[gray]{0.9} \textbf{UC-S1.0 - Inserimento della email dell'\glossaryItem{Owner} }} \\ 
        \hline
        
        \textbf{Attori Primari} & \glossaryItem{Super-Admin}\\  
        \textbf{Scopo e descrizione} & Il \glossaryItem{Super-Admin} può inserire l'indirizzo email dell'\glossaryItem{Owner} nel campo del form offerto dall'applicazione \\
      
        \textbf{Precondizioni}  & L'applicazione offre un campo testo. \\ 
        
        \textbf{Postcondizioni} & Il \glossaryItem{Super-Admin} ha compilato il campo testo relativo all'email dell'\glossaryItem{Owner} \\ 
     \end{longtable}
      \egroup
    \end{center}

    \subsubsection{Creazione di una nuova \glossaryItem{Company} - inserimento della password dell'\glossaryItem{Owner}} 
    
    %Tabella 
    \begin{center}
      \bgroup
      \def\arraystretch{1.8}     
      \begin{longtable}{  p{3.5cm} | p{8cm} } 
        
        \hline
        \multicolumn{2}{ | c | }{ \cellcolor[gray]{0.9} \textbf{UC-S1.0 - Inserimento della password dell'\glossaryItem{Owner} }} \\ 
        \hline
        
        \textbf{Attori Primari} & \glossaryItem{Super-Admin}\\  
        \textbf{Scopo e descrizione} & L'applicazione offre un campo testo offuscato su cui il \glossaryItem{Super-Admin} può scrivere la password dell'\glossaryItem{Owner} \\
      
        \textbf{Precondizioni}  & L'applicazione mostra un campo testo offuscato. \\ 
        
        \textbf{Postcondizioni} & Il \glossaryItem{Super-Admin} ha compilato il campo testo relativo alla password dell'\glossaryItem{Owner} \\ 
     \end{longtable}
      \egroup
    \end{center}






    \subsubsection{Visualizzazione in dettaglio di una \glossaryItem{Company}}
    \begin{figure}[H]
      \begin{center}
        \includegraphics[width=12cm]{res/img/UCSuperadmin/UC-S2.png}
      \caption{UC-S2 - Visualizzazione dettaglio di una \glossaryItem{Company}}
      \end{center} 
    \end{figure}    
    
    %Tabella 
    \begin{center}
      \bgroup
      \def\arraystretch{1.8}     
      \begin{longtable}{  p{3.5cm} | p{8cm} } 
        
        \hline
        \multicolumn{2}{ | c | }{ \cellcolor[gray]{0.9} \textbf{UC-S2 - Visualizzazione dettaglio di \glossaryItem{Company}}} \\ 
        \hline
        
        \textbf{Attori Primari} & \glossaryItem{Super-Admin}\\  
        \textbf{Precondizioni}  & L'applicazione mette a disposizione la pagina di visualizzazione dei dettagli di una \glossaryItem{Company}.  \\ 
        
        \textbf{Postcondizioni} & L'applicazione ha reindirizzato il \glossaryItem{Super-Admin} alla pagina di visualizzazione della \glossaryItem{Company} selezionata. \\
        
        \textbf{Scenario principale} & 1. Il \glossaryItem{Super-Admin} può modificare i dati della \glossaryItem{Company} (UC-S2.0);  
        
        2. Il \glossaryItem{Super-Admin} può aggiungere un nuovo utente della \glossaryItem{Company} (UC-S2.1);
        
        3. Il \glossaryItem{Super-Admin} può visualizzare uno specifico utente della \glossaryItem{Company} (UC-S2.2); 
        
        4. Il \glossaryItem{Super-Admin} pu\`o modificare i dati di uno specifico utente. (UC-S2.3) \\ 
        
        \textbf{Estensioni} & 1. Fallimento della modifica dei dati della \glossaryItem{Company};
        
        2. Fallimento dell'aggiunta di un nuovo utente;
        
        3. Fallimento modifica utente. \\
      \end{longtable}
      \egroup
    \end{center}


\subsubsection{Modifica dei dati di una \glossaryItem{Company}}
    \begin{figure}[H]
      \begin{center}
        \includegraphics[width=12cm]{res/img/UCSuperadmin/UC-S2.0.png}
      \caption{UC-S2.0 - Modifica dei dati di una \glossaryItem{Company}}
      \end{center} 
    \end{figure}    
    
    %Tabella 
    \begin{center}
      \bgroup
      \def\arraystretch{1.8}     
      \begin{longtable}{  p{3.5cm} | p{8cm} } 
        
        \hline
        \multicolumn{2}{ | c | }{ \cellcolor[gray]{0.9} \textbf{UC-S2.0 - Modifica dei dati di una \glossaryItem{Company}}}. \\ 
        \hline
        
        \textbf{Attori Primari} & \glossaryItem{Super-Admin}.\\  
        \textbf{Precondizioni}  & L'applicazione mostra il \textit{form} per la modifica dei dati della \glossaryItem{Company} selezionata.  \\ 
        
        \textbf{Postcondizioni} & Il sistema ha modificato il profilo della \glossaryItem{Company} sulla base dei dati inseriti dal \glossaryItem{Super-Admin}.  \\ 
        \textbf{Estensioni} & 1. Fallimento della modifica.
      \end{longtable}
      \egroup
    \end{center}

%INTERNO: 2.0.0 


\subsubsection{Aggiunta di un nuovo utente nella \glossaryItem{Company} selezionata}
    \begin{figure}[H]
      \begin{center}
        \includegraphics[width=12cm]{res/img/UCSuperadmin/UC-S2.1.png}
      \caption{UC-S2.1 - Aggiunta di un nuovo utente nella \glossaryItem{Company} selezionata}
      \end{center} 
    \end{figure}    
    
    %Tabella 
    \begin{center}
      \bgroup
      \def\arraystretch{1.8}     
      \begin{longtable}{  p{3.5cm} | p{8cm} } 
        
        \hline
        \multicolumn{2}{ | c | }{ \cellcolor[gray]{0.9} \textbf{UC-S2.1 - Aggiunta di un nuovo utente nella \glossaryItem{Company} selezionata}} \\ 
        \hline
        
        \textbf{Attori Primari} & \glossaryItem{Super-Admin}.\\  
        \textbf{Precondizioni}  & L'applicazione mostra il \textit{form} per l'aggiunta di un nuovo utente.  \\ 
        
        \textbf{Postcondizioni} & Il sistema ha aggiunto un nuovo utente associandolo alla \glossaryItem{Company} precedentemente selezionata.  \\ 
      \end{longtable}
      \egroup
    \end{center}

%

\subsubsection{Visualizzazione in dettaglio di un utente della \glossaryItem{Company}}
    \begin{figure}[H]
      \begin{center}
        \includegraphics[width=12cm]{res/img/UCSuperadmin/UC-S2.2.png}
      \caption{UC-S2.2 - Visualizzazione in dettaglio di un utente della \glossaryItem{Company}}
      \end{center} 
    \end{figure}    
    
    %Tabella 
    \begin{center}
      \bgroup
      \def\arraystretch{1.8}     
      \begin{longtable}{  p{3.5cm} | p{8cm} } 
        
        \hline
        \multicolumn{2}{ | c | }{ \cellcolor[gray]{0.9} \textbf{UC-S2.2 - Visualizzazione in dettaglio di un utente della \glossaryItem{Company}}} \\ 
        \hline
        
        \textbf{Attori Primari} & \glossaryItem{Super-Admin}\\  
        \textbf{Scopo e descrizione} & Il \glossaryItem{Super-Admin} entra nella pagina di visualizzazione di un utente, nella quale pu\`o ispezionarne il profilo; inoltre pu\`o essere reindirizzato alle pagine di modifica ed eliminazione dell' utente in questione. \\
        \textbf{Precondizioni}  & Il sistema presenta la pagina di visualizzazione in dettaglio di un utente.  \\ 
        
        \textbf{Postcondizioni} & Il sistema ha ricevuto l'input dal \glossaryItem{Super-Admin}.  \\ 
         \textbf{Scenario principale} & 1. Il \glossaryItem{Super-Admin} pu\`o modificare il profilo dell'utente (UC-S2.3); 
         
         2. Il \glossaryItem{Super-Admin} pu\`o eliminare l'utente. (UC-S2.4) \\
        
     
     \end{longtable}
      \egroup
    \end{center}

\subsubsection{Modifica del profilo di un utente}
    \begin{figure}[H]
      \begin{center}
        \includegraphics[width=12cm]{res/img/UCSuperadmin/UC-S2.3.png}
      \caption{UC-S2.3 - Modifica del profilo di un utente}
      \end{center} 
    \end{figure}    
    
    %Tabella 
    \begin{center}
      \bgroup
      \def\arraystretch{1.8}     
      \begin{longtable}{  p{3.5cm} | p{8cm} } 
        
        \hline
        \multicolumn{2}{ | c | }{ \cellcolor[gray]{0.9} \textbf{UC-S2.3 - Modifica del profilo di un utente }} \\ 
        \hline
        
        \textbf{Attori Primari} & \glossaryItem{Super-Admin}\\  
        \textbf{Scopo e descrizione} & Il \glossaryItem{Super-Admin} entra nella pagina di modifica dell'utente, nella quale ha la possibilit\`a
        di modificarne ruolo e password. \\
      
        \textbf{Precondizioni}  & L'applicazione offre un \textit{form} di modifica del profilo. \\ 
        
        \textbf{Postcondizioni} & Il sistema ha modificato il profilo dell'utente sulla base di quanto inserito dal \glossaryItem{Super-Admin}. \\ 
         \textbf{Scenario principale} & 1. Il \glossaryItem{Super-Admin} ha la possibilit\`a di modificare la tipologia dell'utente. \\
        
        
         \textbf{Estensioni} & 1. Uscita dalla pagina senza salvataggio delle modifiche.  \\
     
     \end{longtable}
      \egroup
    \end{center}

%INTERNO: 2.3.0 "Modifica del livello di permesso""

% \subsubsection{UC-S2.4}
%    %UC-S2.4: il dettaglio raggiunto è ridondante.
%    \begin{figure}[H]
%      \begin{center}
%        \includegraphics[width=12cm]{res/img/UCSuperadmin/UC-S2.4.png}
%      \caption{UC-S2.4 - Eliminazione di un utente}
%      \end{center} 
%    \end{figure}    
    
    %Tabella 
%    \begin{center}
%     \bgroup
%    \def\arraystretch{1.8}     
%      \begin{longtable}{  p{3.5cm} | p{8cm} } 
        
%        \hline
%        \multicolumn{2}{ | c | }{ \cellcolor[gray]{0.9} \textbf{UC-S2.4 - Eliminazione di un utente }} \\ 
%        \hline
        
%        \textbf{Attori Primari} & \glossaryItem{Super-Admin}\\  
%        \textbf{Scopo e descrizione} & L'\glossaryItem{Attore} entra nella pagina di eliminazione dell'utente, nella quale ha la possibilit\`a
%        di eliminarlo. \\
      
 %       \textbf{Precondizioni}  & L'applicazione richiede la conferma dell'eliminazione dell'utente. \\ 
        
%        \textbf{Postcondizioni} & Il sistema ha seguito le indicazioni dell'\glossaryItem{Attore}. \\ 
%         \textbf{Scenario principale} & 1. L'\glossaryItem{Attore} ha la possibilit\`a di confermare l'eliminazione; 
         
%         2. l'\glossaryItem{Attore} Ha la possibilit\`a ritirare la richiesta di eliminazione. \\
        
%         \textbf{Estensioni} & 1. Uscita dalla pagina senza salvataggio delle modifiche.  \\
     
%     \end{longtable}
%      \egroup
%    \end{center}

    

\subsubsection{UC-S4} 
    
    %Tabella 
    \begin{center}
      \bgroup
      \def\arraystretch{1.8}     
      \begin{longtable}{  p{3.5cm} | p{8cm} } 
        
        \hline
        \multicolumn{2}{ | c | }{ \cellcolor[gray]{0.9} \textbf{UC-S4 - Fallisce la creazione di una \glossaryItem{Company}}} \\ 
        \hline
        
        \textbf{Attori Primari} & \glossaryItem{Super-Admin}\\  
        \textbf{Scopo e descrizione} & Il'\glossaryItem{Super-Admin} ha inserito dei dati errati per la creazione della \glossaryItem{Company}, quindi il sistema comunica l'errore. \\
      
        \textbf{Precondizioni}  & L'applicazione ha ricevuto in input dei dati non validi per la creazione di una \glossaryItem{Company}. \\ 
        
        \textbf{Postcondizioni} & Il sistema mostra al \glossaryItem{Super-Admin} un messaggio di errore che comunica l'entità dell'input errato. \\ 
         \textbf{Scenario principale} & 1. Il \glossaryItem{Super-Admin} visualizza l'errore. \\
        
     \end{longtable}
      \egroup
    \end{center}


    \subsubsection{Fallisce la modifica di una \glossaryItem{Company}} 
    
    %Tabella 
    \begin{center}
      \bgroup
      \def\arraystretch{1.8}     
      \begin{longtable}{  p{3.5cm} | p{8cm} } 
        
        \hline
        \multicolumn{2}{ | c | }{ \cellcolor[gray]{0.9} \textbf{UC-S4 - Fallisce la modifica di una \glossaryItem{Company}}} \\ 
        \hline
        
        \textbf{Attori Primari} & \glossaryItem{Super-Admin}\\  
        \textbf{Scopo e descrizione} & Il \glossaryItem{Super-Admin} ha inserito dei dati errati per la modifica della \glossaryItem{Company} e il sistema comunica l'errore. \\
      
        \textbf{Precondizioni}  & L'applicazione ha ricevuto in input dei dati non validi per la modifica di una \glossaryItem{Company}. \\ 
        
        \textbf{Postcondizioni} & Il sistema mostra al \glossaryItem{Super-Admin} un messaggio di errore che comunica l'entità dell'input errato. \\ 
         \textbf{Scenario principale} & 1. Il \glossaryItem{Super-Admin} visualizza l'errore. \\
        
     \end{longtable}
      \egroup
    \end{center}


\subsubsection{Gestione degli altri \glossaryItem{Super-Admin}}
    \begin{figure}[H]
      \begin{center}
        \includegraphics[width=12cm]{res/img/UCSuperadmin/UC-S3.png}
      \caption{UC-S3 - Gestione di altri \glossaryItem{Super-Admin}}
      \end{center} 
    \end{figure}    
    
    %Tabella 
    \begin{center}
      \bgroup
      \def\arraystretch{1.8}     
      \begin{longtable}{  p{3.5cm} | p{8cm} } 
        
        \hline
        \multicolumn{2}{ | c | }{ \cellcolor[gray]{0.9} \textbf{UC-S3 - Gestione degli altri \glossaryItem{Super-Admin} }} \\ 
        \hline
        
        \textbf{Attori Primari} & \glossaryItem{Super-Admin}.\\  
        \textbf{Scopo e descrizione} & Il \glossaryItem{Super-Admin} è entrato nella pagina di gestione dei \glossaryItem{Super-Admin}. In questa pagina può aggiungere un nuovo \glossaryItem{Super-Admin},
vedere in un elenco quelli già presenti e vederne le informazioni in dettaglio cliccando nella voci corrispondenti dell'elenco. \\
        \textbf{Precondizioni}  & Il \glossaryItem{Super-Admin} entra nella pagina di gestione dei \glossaryItem{Super-Admin}.\\ 
        
        \textbf{Postcondizioni} & Il sistema ha preso in carico le indicazioni del \glossaryItem{Super-Admin}. \\ 
         \textbf{Scenario principale} & 1. Il \glossaryItem{Super-Admin} ha la possibilit\`a di aggiungere un nuovo \glossaryItem{Super-Admin}  
         
         2. Il \glossaryItem{Super-Admin} ha la possibilit\`a di visualizzare in dettaglio il profilo di un \glossaryItem{Super-Admin} esistente. \\
        
         \textbf{Estensioni} & 1. Fallimento dell'inserimento di un \glossaryItem{Super-Admin}.  \\
     
     \end{longtable}
      \egroup
    \end{center}


\subsubsection{Creazione di un \glossaryItem{Super-Admin}}
    \begin{figure}[H]
      \begin{center}
        \includegraphics[width=12cm]{res/img/UCSuperadmin/UC-S3.0.png}
      \caption{UC-S3.0 - Creazione di un \glossaryItem{Super-Admin}}
      \end{center} 
    \end{figure}    
    
    %Tabella 
    \begin{center}
      \bgroup
      \def\arraystretch{1.8}     
      \begin{longtable}{  p{3.5cm} | p{8cm} } 
        
        \hline
        \multicolumn{2}{ | c | }{ \cellcolor[gray]{0.9} \textbf{UC-S3.0 - Creazione di un \glossaryItem{Super-Admin} }} \\ 
        \hline
        
        \textbf{Attori Primari} & \glossaryItem{Super-Admin}.\\  
        \textbf{Scopo e descrizione} & Il \glossaryItem{Super-Admin} entra nella pagina di creazione di un altro \glossaryItem{Super-Admin}. 
        Qui deve aggiungere le informazioni (email, password) necessarie per la registrazione di un nuovo \glossaryItem{Super-Admin}. \\
      
        \textbf{Precondizioni}  &  Il \glossaryItem{Super-Admin} entra nella pagina di creazione di un nuovo \glossaryItem{Super-Admin} \\
        \textbf{Postcondizioni} & Il sistema ha creato un nuovo \glossaryItem{Super-Admin} nel database \\ 
         \textbf{Scenario principale} & 1. Il \glossaryItem{Super-Admin} inserisce l'email dell'utente da registrare; 
         
         2. Il \glossaryItem{Super-Admin} aggiunge una password nell'apposito campo.\\
        
         \textbf{Estensioni} & 1. Il \glossaryItem{Super-Admin} esce dalla pagina;  
         
         2. L'email inserita \`e gi\`a presente nel database e il nuovo \glossaryItem{Super-Admin} non pu\`o essere creato.\\ 
     
     \end{longtable}
      \egroup
    \end{center}

    %TODO: Aggiungere UC-S6-1 "Visualizazione profilo di un SuperAdmin"
    







\newpage
