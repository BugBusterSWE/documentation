% example & bla & blabla & blablabla \tabularnewline \hline
TS-R1O 1 & Verificare che l'utente non autenticato possa creare un account per la propria Company. & N.E & R1O \tabularnewline \hline    %1

TS-R1O 1.1 & Verificare che l'utente non autenticato possa inserire il nome della propria Company nel campo di testo dedicato, all'interno della procedura di creazione dell'account della Company. & N.E & R1O 1.1  \tabularnewline \hline   %2


TS-R1O 1.2 & Verificare che l'utente non autenticato possa inserire l'email della propria Company nel campo di testo dedicato, all'interno della procedura di creazione dell'account della Company. & N.E & R1O 1.2 \tabularnewline \hline   %3


TS-R1O 1.3 & Verificare che l'utente non autenticato possa inserire la password della propria Company nel campo di testo (offuscato) dedicato, all'interno della procedura di creazione dell'account della Company. & N.E & R1O 1.3 \tabularnewline \hline   %4

TS-R1O 2 & Verificare che l'utente non autenticato (dopo aver ricevuto un invito da parte dell'Owner) riesca a registrarsi a MaaS e ad accedere al proprio account & N.E & R10 2 \tabularnewline \hline %5

TS-R1O 2.1 & Verificare che l'utente non autenticato (dopo aver ricevuto un invito da parte dell'Owner) riesca a registrarsi a MaaS tramite una password. N.E & R10 2.1 \tabularnewline \hline %6

TS-R1O 2.2 & Verificare che l'utente non autenticato in possesso di un account presso MaaS possa accedere al servizio tramite l'interfaccia \textit{web} offerta dal servizio. & N.E & R10 2.2 \tabularnewline \hline %7

TS-R1O 2.3 & Verificare che l'utente non autenticato riesca a recuperare la password tramite email e codice segreto (inviato tramite email). & N.E & R10 2.3 \tabularnewline \hline %8

TS-R1O 2.4 & Verificare che l'utente autenticato possa modificare campi diversi del proprio profilo. & N.E & R10 2.4 \tabularnewline \hline %9

TS-R1O 2.4.1 & Verificare che l'utente autenticato possa modificare il campo email del proprio profilo. & N.E & R10 2.4.1 \tabularnewline \hline %10

TS-R1O 2.4.2 & Verificare che l'utente autenticato possa modificare la propria password. & N.E & R10 2.4.2 \tabularnewline \hline %11

TS-R1O 2.4.3 & Verificare che l'utente autenticato riesca a rimuovere il proprio account da MaaS. & N.E & R10 2.4.3 \tabularnewline \hline %12

TS-R1O 3 & Verificare che l'utente autenticato possa accedere all'editor ed eseguire una qualsiasi operazione su una DSL. & N.E & R10 3 \tabularnewline \hline %13

TS-R1O 3.1 & Verificare che l'utente autenticato possa eseguire una DSL in particolare (dopo l'accesso all'editor). & N.E & R10 3.1 \tabularnewline \hline %14

TS-R1O 3.2 & Verificare che l'utente autenticato (con ruolo superiore a Member) possa modificare una DSL specifica (dopo l'accesso all'editor). & N.E & R10 3.2 \tabularnewline \hline %15

TS-R1O 3.3 & Verificare che l'utente autenticato (con ruolo superiore a Member) possa creare una DSL specifica (dopo l'accesso all'editor). & N.E & R10 3.3 \tabularnewline \hline %16

TS-R1O 3.4 & Verificare che l'utente autenticato (con ruolo superiore a Member) possa leggere una DSL specifica (dopo l'accesso all'editor). & N.E & R10 3.4 \tabularnewline \hline %17

TS-R1O 3.5 & Verificare che l'Admin possa aggiungere (oppure togliere) i permessi di lettura (oppure scrittura) ad un utente di una Company su una DSL specifica. & N.E & R10 3.5 \tabularnewline \hline %18

%TS-R1O 3.6 & Verificare che l'Owner di una Company possa invitare un utente a registrarsi presso MaaS. & N.E & R10 3.6 \tabularnewline \hline %19   %%

%TS-R1O 4 & Verificare che l'utente autenticato possa accedere alla pagina Dashboard, visualizzarne il contenuto ed effettuare una delle operazioni permesse sugli elementi presenti. & N.E & R10 4 \tabularnewline \hline %20

%TS-R1O 4.1 & Verificare che l'utente autenticato possa visualizzare uno qualsiasi degli elementi della Dashboard (Cell, Document o Collection). & N.E & R10 4.1 \tabularnewline \hline %21

%TS-R1O 4.2 & Verificare che l'utente autenticato possa effettuare una delle operazioni permesse su uno qualsiasi degli elementi della Dashboard (Cell, Document o Collection). & N.E & R10 4.2 \tabularnewline \hline %22  


%TS-R1O 4.2.1 & Verificare che l'utente autenticato possa aggiungere, modificare o ordinare un valore in un elemento Cell. & N.E & R10 4.2.1 \tabularnewline \hline %23
%%

%TS-R1O 4.2.2 & Verificare che l'utente autenticato possa modificare o rimuovere un elemento Document. & N.E & R10 4.2.2 \tabularnewline \hline %24
%%

%TS-R1O 4.2.3 & Verificare che l'utente autenticato possa eseguire un Send mail/Export dalla pagina Document. & N.E & R10 4.2.3 \tabularnewline \hline %25
%% 

 
