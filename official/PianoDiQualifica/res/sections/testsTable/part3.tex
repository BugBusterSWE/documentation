% example & bla & blabla & blablabla \tabularnewline \hline
TS-R1D 16.10 & Verificare che DashRow abbia una rappresentazione grafica visibile sul browser. & N.E & R1D 16.10 \tabularnewline \hline
TS-R1D 16.11 & Verificare che una \glossaryItem{Action} definita dall'amministratore della piattaforma abbia una rappresentazione grafica visibile sul browser. & N.E & R1D 16.11 \tabularnewline \hline
TS-R1D 17 & Verificare che l'utente, in caso di errori, venga avvisato con un messaggio d'errore. & N.E & R1D 17 \tabularnewline \hline
TS-R1D 17.1 & Verificare che il messaggio d'errore, in caso di fallimento della \glossaryItem{validazione} della specifica \glossaryItem{DSL}, indichi il \glossaryItem{DSL} Element che genera l'errore. & N.E & R3D 14.1 \tabularnewline \hline
TS-R1D 18 & Verificare che ad ogni azione dell'utente compiuta nell'editor corrisponda la manipolazione della struttura della specifica \glossaryItem{DSL}. & N.E & R1D 18 \tabularnewline \hline
TS-R1D 19 & Verificare che l'applicazione mantenga la correttezza di una struttura di una specifica \glossaryItem{DSL}.  & N.E & R1D 19 \tabularnewline \hline
TS-R1D 20 & Verificare che l'utente possa usare il tipo di collegamento Riferimento tra due elementi della specifica \glossaryItem{DSL}. & N.E & R1D 20 \tabularnewline \hline
TS-R1D 21 & Verificare che l'utente possa usare il tipo di collegamento Associazione tra due elementi della specifica \glossaryItem{DSL}. & N.E & R1D 21 \tabularnewline \hline
TS-R1D 22 & Verificare che l'utente possa definire i valori degli attributi della specifica \glossaryItem{DSL} tramite l'editor. & N.E & R1D 22 \tabularnewline \hline
TS-R1D 23 & Verificare che l'utente possa aggiungere e rimuovere gli attributi dalla specifica \glossaryItem{DSL} tramite l'editor. & N.E & R1D 23 \tabularnewline \hline
TS-R1D 24 & Verificare che l'utente possa aggiungere o rimuovere una struttura del \glossaryItem{DSL} specifica tramite l'editor. & N.E & R1D 24 \tabularnewline \hline
TS-R1D 25 & Verificare che l'utente possa scrivere la funzione \glossaryItem{JavaScript} direttamente nell'editor. & N.E & R1D 25 \tabularnewline \hline
TS-R1D 26 & Verificare che l'utente, attraverso l'interfaccia grafica, possa essere in grado di salvare i metodi creati. & N.E & R1D 26 \tabularnewline \hline
TS-R1D 27 & Verificare che l'utente possa importare nel \glossaryItem{DSL} corrente una funzione \glossaryItem{JavaScript} precedentemente creata. & N.E & R1D 27 \tabularnewline \hline
TS-R1D 28 & Verificare che l'utente, attraverso l'interfaccia grafica, visualizzi l'insieme degli attributi di un \glossaryItem{DSL} Element, che possono essere obbligatori o opzionali e sono distinti attraverso una simbolo grafico. & N.E & R1D 28 \tabularnewline \hline
TS-R1D 29 & Verificare che l'utente possa scegliere qual è il tipo di dato rappresentato dal \glossaryItem{Cell} Element. & N.E & R1D 29 \tabularnewline \hline
TS-R1D 30 & Verificare che si possa importare una \glossaryItem{Action} definita dall'amministratore nella specifica \glossaryItem{DSL} corrente. & N.E & R1D 30 \tabularnewline \hline
TS-R1D 31 & Verificare che si possa selezionare la funzionalità dell' \glossaryItem{Action}. & N.E & R1D 31 \tabularnewline \hline
TS-R1D 32 & Verificare la possibilit\`a di impostare associazioni tra \glossaryItem{DSL} Element obbligatorie e non modificabili. & N.E & R1D 32 \tabularnewline \hline
TS-R1D 33 & Verificare che la \glossaryItem{DSL} creata dall'editor sia inviata al server \glossaryItem{MaaS}. & N.E & R1D 33 \tabularnewline \hline
TS-R1O 34 & Verificare che l'Admin possa aggiungere un nuovo database alla propria \glossaryItem{Company}. & N.E & R1D 33 \tabularnewline \hline
TS-R1O 34.1 & Verificare che l'Admin possa inserire il nome del database in \glossaryItem{fase} di creazione di un database. & N.E & R1O 34 \tabularnewline \hline
TS-R1O 34.2 & Verificare che l'Admin possa inserire l'host del database in \glossaryItem{fase} di creazione di un database. & N.E & R1O 34.1 \tabularnewline \hline
TS-R1O 34.3 & Verificare che l'Admin possa inserire la porta del database in \glossaryItem{fase} di creazione di un database. & N.E & R1O 34.2 \tabularnewline \hline
TS-R1O 34.4 & Verificare che l'Admin possa inserire lo username per l'accesso al database in \glossaryItem{fase} di creazione di un database. & N.E & R1O 34.3 \tabularnewline \hline
TS-R1O 34.5 & Verificare che l'Admin possa inserire la password per l'accesso al database in \glossaryItem{fase} di creazione di un database. & N.E & R1O 34.4 \tabularnewline \hline
TS-R1O 35 & Verificare che l'Admin possa rimuovere un database della \glossaryItem{Company}. & N.E & R1O 35 \tabularnewline \hline
TS-R1O 36 & Verificare che l'Admin possa modificare un database della \glossaryItem{Company}. & N.E & R1O 36 \tabularnewline \hline
TS-R1O 36.1 & Verificare che l'Admin possa modificare il nome del database in \glossaryItem{fase} di modifica di un database. & N.E & R1O 36.1 \tabularnewline \hline
TS-R1O 36.2 & Verificare che l'Admin possa modificare l'host del database in \glossaryItem{fase} di modifica di un database. & N.E & R1O 36.2 \tabularnewline \hline
TS-R1O 36.3 & Verificare che l'Admin possa modificare la porta del database in \glossaryItem{fase} di modifica di un database. & N.E & R1O 36.3 \tabularnewline \hline
TS-R1O 36.4 & Verificare che l'Admin possa aggiungere delle \glossaryItem{Collection} in \glossaryItem{fase} di modifica di un database. & N.E & R1O 36.4 \tabularnewline \hline
TS-R1O 36.5 & Verificare che l'Admin possa il nome della \glossaryItem{Collection} in \glossaryItem{fase} di modifica di un database. & N.E & R1O 36.5 \tabularnewline \hline
TS-R1D 36.6 & Verificare che l'Admin possa decidere se la \glossaryItem{Collection} è visibile dai member in \glossaryItem{fase} di modifica di un database. & N.E & R1D 36.6 \tabularnewline \hline
TS-R1O 36.7 & Verificare che l'Admin possa rimuovere una \glossaryItem{Collection} in \glossaryItem{fase} di modifica di una database. & N.E & R1O 36.7 \tabularnewline \hline
TS-R1O 36.8 & Verificare che l'Admin possa modificare una \glossaryItem{Collection} in \glossaryItem{fase} di modifica di un database. & N.E & R1O 36.8 \tabularnewline \hline
TS-R1O 36.9 & Verificare che l'Admin possa modificare il nome della \glossaryItem{Collection} in \glossaryItem{fase} di modifica di un database. & N.E & R1O 36.9 \tabularnewline \hline
TS-R1D 36.10 & Verificare che l'Admin possa modificare il livello di accesso di una \glossaryItem{Collection} in \glossaryItem{fase} di modifica di un database. & N.E & R1O 36.10 \tabularnewline \hline
TS-R1O 36.11 & Verificare che l'Admin possa modificare lo username di accesso al database in \glossaryItem{fase} di modifica di un database. & N.E & R1O 36.11 \tabularnewline \hline
TS-R1O 36.12 & Verificare che l'Admin possa modificare la password di accesso al database in \glossaryItem{fase} di modifica di un database. & N.E & R1O 36.12 \tabularnewline \hline
TS-R1O 37 & Verificare che l'utente possa visualizzare i database ai quali ha accesso. & N.E & R1O 37 \tabularnewline \hline
TS-R1O 38 & Verificare che l'utente possa visualizzare le Collections alle quali ha accesso. & N.E & R1O 38 \tabularnewline \hline
TS-R3O 1 & Verificare che siano stati scritti e rilasciati i manuali d'uso ed ogni altra documentazione tecnica (in lingua inglese) necessaria per l'utilizzo del prodotto. & N.E & R3O 1 \tabularnewline \hline
TS-R3O 2 & Verificare che per lo sviluppo del prodotto richiesto siano rispettate tutte le norme descritte nel documento \NormeDiProgetto. & N.E & R3O 2 \tabularnewline \hline
TS-R4O 1 & Verificare che il \glossaryItem{codice sorgente} sia reso pubblico sotto il controllo di \glossaryItem{versione} usando GitHub o Bitbucket. & N.E & R4O 1 \tabularnewline \hline
TS-R4O 2 & Verificare che lo Stack tecnologico da usare includa:  \newline
- \glossaryItem{Node.js} per il \glossaryItem{Back End}. \newline 
- \glossaryItem{MongoDB} (versione >= 3.x) per il database dell'applicazione.
& N.E & R4O 2 \tabularnewline \hline
TS-R4O 3 & Verificare che il \glossaryItem{deployment} sia effettuato su \glossaryItem{Heroku}. & N.E & R4O 3 \tabularnewline \hline
TS-R4O 4 & Verificare che il browser in uso sia \textbf{\textit{Google Chrome}} versione 49.X o \textbf{\textit{Mozilla Firefox}} versione 45.Y. & N.E & R4O 4 \tabularnewline \hline %
  \end{longtable} %problem with latex compiler. See issue #129
