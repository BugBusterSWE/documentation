% example & bla & blabla & blablabla \tabularnewline \hline
TS-R1D 16.10 & Verifica che DashRow abbia una rappresentazione grafica visibile sul browser. & N.E & R1D 16.10 \tabularnewline \hline
TS-R1D 16.11 & Verifica che una Action definita dall'amministratore della piattaforma abbia una rappresentazione grafica visibile sul browser. & N.E & R1D 16.11 \tabularnewline \hline
TS-R1D 17 & Verifica che l'utente, in caso di errori, venga avvisato con un messaggio d'errore. & N.E & R1D 17 \tabularnewline \hline
TS-R3D 14.1 & Verifica che il messaggio d'errore, in caso di fallimento della validazione della specifica DSL, indichi il DSL Element che genera l'errore. & N.E & R3D 14.1 \tabularnewline \hline
TS-R1D 18 & Verifica che ad ogni azione dell'utente compiuta nell'editor corrisponda la manipolazione della struttura della specifica DSL. & N.E & R1D 18 \tabularnewline \hline
TS-R1D 19 & Verifica che l'applicazione mantenga la correttezza di una struttura di una specifica DSL.  & N.E & R1D 19 \tabularnewline \hline
TS-R1D 20 & Verifica che l'utente possa usare il tipo di collegamento Riferimento tra due elementi della specifica DSL. & N.E & R1D 20 \tabularnewline \hline
TS-R1D 21 & Verifica che l'utente possa usare il tipo di collegamento Associazione tra due elementi della specifica DSL. & N.E & R1D 21 \tabularnewline \hline
TS-R1D 22 & Verifica che l'utente possa definire i valori degli attributi della specifica DSL tramite l'editor. & N.E & R1D 22 \tabularnewline \hline
TS-R1D 23 & Verifica che l'utente possa aggiungere e rimuovere gli attributi dalla specifica DSL tramite l'editor. & N.E & R1D 23 \tabularnewline \hline
TS-R1D 24 & Verifica che l'utente possa aggiungere o rimuovere una struttura del DSL specifica tramite l'editor. & N.E & R1D 24 \tabularnewline \hline
TS-R1D 25 & Verifica che l'utente possa scrivere la funzione JavaScript direttamente nell'editor. & N.E & R1D 25 \tabularnewline \hline
TS-R1D 26 & Verifica che l'utente, attraverso l'interfaccia grafica, possa essere in grado di salvare i metodi creati. & N.E & R1D 26 \tabularnewline \hline
TS-R1D 27 & Verifica che l'utente possa importare nel DSL corrente una funzione JavaScript precedentemente creata. & N.E & R1D 27 \tabularnewline \hline
TS-R1D 28 & Verifica che l'utente, attraverso l'interfaccia grafica, visualizzi l'insieme dei metodi a cui può accedere. & N.E & R1D 28 \tabularnewline \hline
TS-R1D 29 & Verifica che l'utente possa scegliere qual è il tipo di dato rappresentato dal Cell Element. & N.E & R1D 29 \tabularnewline \hline
TS-R1D 30 & Verifica che si possa importare una Action definita dall'amministratore nella specifica DSL corrente. & N.E & R1D 30 \tabularnewline \hline
TS-R1D 31 & Verifica che si possa selezionare la funzionalità dell' Action. & N.E & R1D 31 \tabularnewline \hline
TS-R1D 32 & Verifica la possibilit\`a di impostare associazioni tra DSL Element obbligatorie e non modificabili. & N.E & R1D 32 \tabularnewline \hline
TS-R1D 33 & Verifica che la DSL creata dall'editor sia inviata al server MaaS. & N.E & R1D 33 \tabularnewline \hline
TS-R1O 34 & Verifica che l'Admin possa aggiungere un nuovo database alla propria Company. & N.E & R1D 33 \tabularnewline \hline
TS-R1O 34.1 & Verifica che l'Admin possa inserire il nome del database in fase di creazione di un database. & N.E & R1O 34 \tabularnewline \hline
TS-R1O 34.2 & Verifica che l'Admin possa inserire l'host del database in fase di creazione di un database. & N.E & R1O 34.1 \tabularnewline \hline
TS-R1O 34.3 & Verifica che l'Admin possa inserire la porta del database in fase di creazione di un database. & N.E & R1O 34.2 \tabularnewline \hline
TS-R1O 34.4 & Verifica che l'Admin possa inserire lo username per l'accesso al database in fase di creazione di un database. & N.E & R1O 34.3 \tabularnewline \hline
TS-R1O 34.5 & Verifica che l'Admin possa inserire la password per l'accesso al database in fase di creazione di un database. & N.E & R1O 34.4 \tabularnewline \hline
TS-R1O 35 & Verifica che l'Admin possa rimuovere un database della Company. & N.E & R1O 35 \tabularnewline \hline
TS-R1O 36 & Verifica che l'Admin possa modificare un database della Company. & N.E & R1O 36 \tabularnewline \hline
TS-R1O 36.1 & Verifica che l'Admin possa modificare il nome del database in fase di modifica di un database. & N.E & R1O 36.1 \tabularnewline \hline
TS-R1O 36.2 & Verifica che l'Admin possa modificare l'host del database in fase di modifica di un database. & N.E & R1O 36.2 \tabularnewline \hline
TS-R1O 36.3 & Verifica che l'Admin possa modificare la porta del database in fase di modifica di un database. & N.E & R1O 36.3 \tabularnewline \hline
TS-R1O 36.4 & Verifica che l'Admin possa aggiungere delle Collection in fase di moodifica di un database. & N.E & R1O 36.4 \tabularnewline \hline
TS-R1O 36.5 & Verifica che l'Admin possa il nome della Collection in fase di modifica di un database. & N.E & R1O 36.5 \tabularnewline \hline
TS-R1D 36.6 & Verifica che l'Admin deve poter decidere se la Collection è visibile dai member in fase di modifica di un database. & N.E & R1D 36.6 \tabularnewline \hline
TS-R1O 36.7 & Verifica che l'Admin possa rimuovere una Collection in fase di modifica di una database. & N.E & R1O 36.7 \tabularnewline \hline
TS-R1O 36.8 & Verifica che l'Admin deve poter modificare una Collection in fase di modifica di un database. & N.E & R1O 36.8 \tabularnewline \hline
TS-R1O 36.9 & Verifica che l'Admin possa modificare il nome della Collection in fase di modifica di un database. & N.E & R1O 36.9 \tabularnewline \hline
TS-R1D 36.10 & Verifica che l'Admin possa modificare il livello di accesso di una Collection in fase di modifica di un database. & N.E & R1O 36.10 \tabularnewline \hline
TS-R1O 36.11 & Verifica che l'Admin possa modificare lo username di accesso al database in fase di modifica di un database. & N.E & R1O 36.11 \tabularnewline \hline
TS-R1O 36.12 & Verifica che l'Admin possa modificare la password di accesso al database in fase di modifica di un database. & N.E & R1O 36.12 \tabularnewline \hline
TS-R1O 37 & Verifica che l'utente possa visualizzare i database ai quali ha accesso. & N.E & R1O 37 \tabularnewline \hline
TS-R1O 38 & Verifica che l'utente possa visualizzare le Collections alle quali ha accesso. & N.E & R1O 38 \tabularnewline \hline
TS-R3O 1 & Verifica che siano stati scritti e rilasciati i manuali d'uso ed ogni altra documentazione tecnica (in lingua inglese) necessaria per l'utilizzo del prodotto. & N.E & R3O 1 \tabularnewline \hline
TS-R3O 2 & Verifica che per lo sviluppo del prodotto richiesto siano rispettate tutte le norme descritte nel documento \NormeDiProgetto. & N.E & R3O 2 \tabularnewline \hline
TS-R4O 1 & Verifica che il codice sorgente sia reso pubblico sotto il controllo di versione usando GitHub o Bitbucket. & N.E & R4O 1 \tabularnewline \hline
TS-R4O 2 & Verifica che lo Stack tecnologico da usare includa:  \newline
- \glossaryItem{Node.js} per il \glossaryItem{Back End}. \newline 
- \glossaryItem{MongoDB} (versione >= 3.x) per il database dell'applicazione.
& N.E & R4O 2 \tabularnewline \hline
TS-R4O 3 & Verifica che il deployment sia effettuato su Heroku. & N.E & R4O 3 \tabularnewline \hline %
  \end{longtable} %problem with latex compiler. See issue #129
