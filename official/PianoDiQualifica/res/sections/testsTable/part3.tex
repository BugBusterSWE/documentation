% example & bla & blabla & blablabla \tabularnewline \hline
TS-R1D R1D 13.10 & DashRow G deve avere una & N.E & R1D R1D 13.10 \tabularnewline \hline
TS-R1D 13.10 & Verifica che DashRow deve avere una rappresentazione grafica visibile sul browser. & N.E & R1D 13.10 \tabularnewline \hline
TS-R1D 13.11 & Verifica che una Action definita dall'amministratore della piattaforma ha una rappresentazione grafica visibile sul browser. & N.E & R1D 13.11 \tabularnewline \hline
TS-R1D 14 & Verifica che l'utente, in caso di errori, venga avvisato con un messaggio d'errore. & N.E & R1D 14 \tabularnewline \hline
TS-R3D 14.1 & Verifica che il messaggio d'errore, in caso di fallimento della validazione del DSL, indichi il DSL Element che genera l'errore. & N.E & R3D 14.1 \tabularnewline \hline
TS-R1D 15 & Verifica che ad ogni azione dell'utente compiuta nell'editor, corrisponde la manipolazione della struttura del DSL. & N.E & R1D 15 \tabularnewline \hline
TS-R1D 16 & Verifica che l'applicazione mantenga la correttezza di una struttura DSL. & N.E & R1D 16 \tabularnewline \hline
TS-R1D 17 & Verifica che l'utente possa usare il tipo di collegamento Riferimento tra due elementi DSL. & N.E & R1D 17 \tabularnewline \hline
TS-R1D 18 & Verifica che l'utente possa usare il tipo di collegamento Associazione tra due elementi DSL. & N.E & R1D 18 \tabularnewline \hline
TS-R1D 19 & Verifica che l'utente possa definire i valori degli attributi del DSL tramite l'editor. & N.E & R1D 19 \tabularnewline \hline
TS-R1D 20 & Verifica che l'utente possa aggiungere e rimuovere gli attributi dal DSL tramite l'editor. & N.E & R1D 20 \tabularnewline \hline
