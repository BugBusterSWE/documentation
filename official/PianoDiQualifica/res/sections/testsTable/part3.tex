% example & bla & blabla & blablabla \tabularnewline \hline
TS-R1D R1D 13.10 & DashRow G deve avere una & N.E & R1D R1D 13.10 \tabularnewline \hline
TS-R1D 13.10 & Verifica che DashRow deve avere una rappresentazione grafica visibile sul browser. & N.E & R1D 13.10 \tabularnewline \hline
TS-R1D 13.11 & Verifica che una Action definita dall'amministratore della piattaforma ha una rappresentazione grafica visibile sul browser. & N.E & R1D 13.11 \tabularnewline \hline
TS-R1D 14 & Verifica che l'utente, in caso di errori, venga avvisato con un messaggio d'errore. & N.E & R1D 14 \tabularnewline \hline
TS-R3D 14.1 & Verifica che il messaggio d'errore, in caso di fallimento della validazione del DSL, indichi il DSL Element che genera l'errore. & N.E & R3D 14.1 \tabularnewline \hline
TS-R1D 15 & Verifica che ad ogni azione dell'utente compiuta nell'editor, corrisponde la manipolazione della struttura del DSL. & N.E & R1D 15 \tabularnewline \hline
TS-R1D 16 & Verifica che l'applicazione mantenga la correttezza di una struttura DSL. & N.E & R1D 16 \tabularnewline \hline
TS-R1D 17 & Verifica che l'utente possa usare il tipo di collegamento Riferimento tra due elementi DSL. & N.E & R1D 17 \tabularnewline \hline
TS-R1D 18 & Verifica che l'utente possa usare il tipo di collegamento Associazione tra due elementi DSL. & N.E & R1D 18 \tabularnewline \hline
TS-R1D 19 & Verifica che l'utente possa definire i valori degli attributi del DSL tramite l'editor. & N.E & R1D 19 \tabularnewline \hline
TS-R1D 20 & Verifica che l'utente possa aggiungere e rimuovere gli attributi dal DSL tramite l'editor. & N.E & R1D 20 \tabularnewline \hline
TS-R1D 21 & Verifica che l'utente possa aggiungere o rimuovere una struttura del DSL tramite l'editor. & N.E & R1D 21 \tabularnewline \hline
TS-R1D 22 & Verifica che l'utente possa scrivere la funzione JavaScript direttamente nell'editor. & N.E & R1D 22 \tabularnewline \hline
TS-R1D 23 & Verifica che l'utente, attraverso l'interfaccia grafica, possa essere in grado di salvare i metodi creati. & N.E & R1D 23 \tabularnewline \hline
TS-R1D 24 & Verifica che l'utente possa importare nel DSL corrente una funzione JavaScript precedentemente creata. & N.E & R1D 24 \tabularnewline \hline
TS-R1D 25 & Verifica che l'utente, attraverso l'interfaccia grafica, visualizzi l'insieme dei metodi a cui può accedere. & N.E & R1D 25 \tabularnewline \hline
TS-R1D 26 & Verifica che l'utente possa scegliere qual'è il tipo di dato rappresentato dal Cell Element. & N.E & R1D 26 \tabularnewline \hline
TS-R1D 27 & Verifica che si possa importare una Action definita dall'amministratore nel DSL corrente. & N.E & R1D 27 \tabularnewline \hline
TS-R1D 28 & Verifica che si possa selezionare la funzionalità dell Action. & N.E & R1D 28 \tabularnewline \hline
TS-R1D 29 & Verifica che si possa impostare associazioni tra DSL Element obbligatorie non modificabili. & N.E & R1D 29 \tabularnewline \hline
TS-R1D 30 & Verifica che la DSL creata dall'editor sia inviata al server MaaS. & N.E & R1D 30 \tabularnewline \hline
TS-R3O 1 & Verifica che siano stati scritti e rilasciati i manuali d'uso ed ogni altra documentazione tecnica (in lingua inglese) necessaria per l'utilizzo del prodotto. & N.E & R3O 1 \tabularnewline \hline
TS-R3O 2 & Verifica che per lo sviluppo del prodotto richiesto siano rispettate tutte le norme descritte nel documento \NormeDiProgetto. & N.E & R3O 2 \tabularnewline \hline
TS-R4O 1 & Verifica che il codice sorgente sia reso pubblico sotto il controllo di versione usando github o bitbucket. & N.E & R4O 1 \tabularnewline \hline
