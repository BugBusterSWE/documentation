\section{Esiti delle revisioni}
\subsection{Revisione di progettazione - RPmin}
Di seguito verranno descritti i miglioramenti apportati ai documenti in seguito alle osservazioni fatte in sede di valutazione.
\paragraph*{Generali}
Sono state apportate le seguenti azioni correttive:
\begin{itemize}
\item \textbf{Lettera di presentazione}: il prezzo è stato mantenuto uguale in quanto quello riportato nella lettera di presentazione alla RP è il prezzo corretto. Quello mostrato in ingresso alla RR era stato erroneamente trascritto;
\item \textbf{Verbali}: corretta la denominazione dei file. Ora è 'AAAAMMGG';
\item \textbf{Registro delle modifiche}: il \glossaryItem{team} cercherà di indicare il numero di sezione modificata nel registro delle modifiche, al fine di poter tracciare ogni cambiamento;
\item \textbf{Correttezza tipografica}: è stato sviluppato uno script interno per correggere e normalizzare le accentazioni. Dal formato LaTex si è passati al carattere UTF-8. Inoltre sono state rimosse le G del glossario rimaste nei registri delle modifiche e normalizzati i titoli delle sezioni.
\end{itemize}

\paragraph*{Norme di \glossaryItem{progetto}}
In sede di revisione è stata segnalata una mancanza di profondità nei contenuti. Il \glossaryItem{team} ha quindi deciso di aggiungere numerose sezioni al fine di normalizzare maggiormente il lavoro e di fornire delle norme facilmente consultabili e comprensibili. In particolare sono state fatte le seguenti modifiche:
\begin{itemize}
\item aggiunta della descrizione del Lint usato per normalizzare la scrittura del codice di \glossaryItem{MaaS};
\item correzione ed approfondimento delle norme che hanno regolato il periodo di analisi dei requisiti;
\item correzione ed approfondimento delle norme che hanno regolato il periodo di progettazione;
\item aggiunta la descrizione degli strumenti utilizzati per supportare i \glossaryItem{processi};
\item aggiunte norme su come strutturare i documenti da consegnare;
\item aggiunta una norma che prevede di indicare in corsivo eventuali termini inglesi non presenti nel glossario;
\item aggiunte norme per la gestione di immagini e tabelle;
\item aggiunte norme sulla scrittura dei test;
\item aggiunto il \glossaryItem{processo} di \glossaryItem{validazione};
\item aggiunta la descrizione del sistema di assegnazione automatica di un verificatore.
\end{itemize}

\paragraph*{Analisi dei requisiti}
In sede di revisione sono stati segnalati numerosi problemi riguardanti l'assegnazione di codici identificativi e sulla descrizione testuale dei \glossaryItem{casi} d'uso. Il \glossaryItem{team} ha notato che questi problemi si presentavano con maggiore evidenza nella sezione dedicata al \glossaryItem{Super-Admin}; tale sezione è stata approfonditamente riguardata e corretta. I requisiti minimi hardware e software sono stati spostati nella sezione dedicata ai requisiti di vincolo, dato che questa parte è più "contrattuale" e prevale sulla parte narrativa, nella quale erano stati precedentemente inseriti. È stata infine eliminata la generalizzazione di UC-U16.1.

\paragraph*{Specifica tecnica}
Il \glossaryItem{team} è stato abbastanza soddisfatto della revisione del documento, che ha evidenziato errori più di "forma" che di "contenuto". L'azione correttiva si è concentrata sull'approfondimento e sullo spostamento di alcune sezioni. In particolare:
\begin{itemize}
\item sono stati rivisti vantaggi e svantaggi di alcune tecnologie utilizzate;
\item sono stati corretti i \glossaryItem{diagrammi} delle classi della sezione di backend, che riportavano erroneamente librerie esterne all'interno del \textit{package} dell'applicazione;
\item sono state descritte le relazioni fra i componenti individuati, e alcune interfacce sono state modificate in classi astratte;
\item è stata spostata la sezione riguardante i \textit{framework} dell'applicazione;
\item è stata rivista l'applicazione del pattern MVC;
\item è stato eliminato il comando del glossario nella descrizione del formato \glossaryItem{JSON};
\item sono stati descritti i \glossaryItem{diagrammi} di sequenza rappresentanti le \glossaryItem{API} offerte dal server;
\item la dipendenza circolare di Flux è stata descritta con maggiore precisione;
\item è stato aggiunto il diagramma riepilogativo del frontend;
\item sono stati corretti alcuni errori nei \glossaryItem{diagrammi} delle classi e di attività;
\item sono state descritte con maggiore precisione le motivazioni alla base dell'uso del design pattern Facade;
\item sono stati corretti altri piccoli errori di forma.
\end{itemize}

\paragraph*{Piano di \glossaryItem{Progetto}}
In sede di revisione sono stati notati gli sforzi di miglioramento, che purtroppo non sono stati sufficienti a rendere il documento di alta qualità. Durante questo periodo sono però stati corretti alcuni errori sintattici e approfondite le sezioni riguardanti il \glossaryItem{consuntivo} di periodo e il preventivo a finire.

\paragraph*{Piano di Qualifica}
In questo periodo il \glossaryItem{team} valuterà con precisione le metriche enunciate, con particolare attenzione a quelle sul codice del prodotto, e documenterà l'esito delle misurazioni effettuate. L'appendice C è stata corretta utilizzando per tutti i test il verbo "verificare" al posto del sostantivo "verifica", e non un \textit{mix} dei due.
