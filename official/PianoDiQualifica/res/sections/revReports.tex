\section{Esiti delle revisioni}
\subsection{Revisione di progettazione - RPmin}
Di seguito verranno descritti i miglioramenti apportati ai documenti in seguito alle osservazioni fatte in sede di valutazione.
\paragraph*{Generali}
Sono state apportate le seguenti azioni correttive:
\begin{itemize}
\item \textbf{Lettera di presentazione}: il prezzo è stato mantenuto uguale in quanto quello riportato nella lettera di presentazione alla RP è il prezzo corretto. Quello mostrato in ingresso alla RR era stato erroneamente trascritto;
\item \textbf{Verbali}: corretta la denominazione dei file. Ora è 'AAAAMMGG';
\item \textbf{Registro delle modifiche}: il \glossaryItem{team} cercherà di indicare il numero di sezione modificata nel registro delle modifiche, al fine di poter tracciare ogni cambiamento;
\item \textbf{Correttezza tipografica}: è stato sviluppato uno script interno per correggere e normalizzare le accentazioni. Dal formato LaTex si è passati al carattere UTF-8. Inoltre sono state rimosse le G del glossario rimaste nei registri delle modifiche e normalizzati i titoli delle sezioni.
\end{itemize}

\paragraph*{Norme di \glossaryItem{progetto}}
In sede di revisione è stata segnalata una mancanza di profondità nei contenuti. Il \glossaryItem{team} ha quindi deciso di aggiungere numerose sezioni al fine di normalizzare maggiormente il lavoro e di fornire delle norme facilmente consultabili e comprensibili. In particolare sono state fatte le seguenti modifiche:
\begin{itemize}
\item aggiunta della descrizione del Lint usato per normalizzare la scrittura del codice di \glossaryItem{MaaS};
\item correzione ed approfondimento delle norme che hanno regolato il periodo di analisi dei requisiti;
\item correzione ed approfondimento delle norme che hanno regolato il periodo di progettazione;
\item aggiunta la descrizione degli strumenti utilizzati per supportare i \glossaryItem{processi};
\item aggiunte norme su come strutturare i documenti da consegnare;
\item aggiunta una norma che prevede di indicare in corsivo eventuali termini inglesi non presenti nel glossario;
\item aggiunte norme per la gestione di immagini e tabelle;
\item aggiunte norme sulla scrittura dei test;
\item aggiunto il \glossaryItem{processo} di \glossaryItem{validazione};
\item aggiunta la descrizione del sistema di assegnazione automatica di un verificatore.
\end{itemize}

\paragraph*{Analisi dei requisiti}
In sede di revisione sono stati segnalati numerosi problemi riguardanti l'assegnazione di codici identificativi e sulla descrizione testuale dei \glossaryItem{casi} d'uso. Il \glossaryItem{team} ha notato che questi problemi si presentavano con maggiore evidenza nella sezione dedicata al \glossaryItem{Super-Admin}; tale sezione è stata approfonditamente riguardata e corretta. I requisiti minimi hardware e software sono stati spostati nella sezione dedicata ai requisiti di vincolo, dato che questa parte è più "contrattuale" e prevale sulla parte narrativa, nella quale erano stati precedentemente inseriti. È stata infine eliminata la generalizzazione di UC-U16.1.

\paragraph*{Specifica tecnica}
Il \glossaryItem{team} è stato abbastanza soddisfatto della revisione del documento, che ha evidenziato errori più di "forma" che di "contenuto". L'azione correttiva si è concentrata sull'approfondimento e sullo spostamento di alcune sezioni. In particolare:
\begin{itemize}
\item sono stati rivisti vantaggi e svantaggi di alcune tecnologie utilizzate;
\item sono stati corretti i \glossaryItem{diagrammi} delle classi della sezione di backend, che riportavano erroneamente librerie esterne all'interno del \textit{package} dell'applicazione;
\item sono state descritte le relazioni fra i componenti individuati, e alcune interfacce sono state modificate in classi astratte;
\item è stata spostata la sezione riguardante i \textit{framework} dell'applicazione;
\item è stata rivista l'applicazione del pattern MVC;
\item è stato eliminato il comando del glossario nella descrizione del formato \glossaryItem{JSON};
\item sono stati descritti i \glossaryItem{diagrammi} di sequenza rappresentanti le \glossaryItem{API} offerte dal server;
\item la dipendenza circolare di Flux è stata descritta con maggiore precisione;
\item è stato aggiunto il diagramma riepilogativo del frontend;
\item sono stati corretti alcuni errori nei \glossaryItem{diagrammi} delle classi e di attività;
\item sono state descritte con maggiore precisione le motivazioni alla base dell'uso del design pattern Facade;
\item sono stati corretti altri piccoli errori di forma.
\end{itemize}

\paragraph*{Piano di \glossaryItem{Progetto}}
In sede di revisione sono stati notati gli sforzi di miglioramento, che purtroppo non sono stati sufficienti a rendere il documento di alta qualità. Durante questo periodo sono però stati corretti alcuni errori sintattici e approfondite le sezioni riguardanti il \glossaryItem{consuntivo} di periodo e il preventivo a finire. 

\paragraph*{Piano di Qualifica}
In questo periodo il \glossaryItem{team} valuterà con precisione le metriche enunciate, con particolare attenzione a quelle sul codice del prodotto, e documenterà l'esito delle misurazioni effettuate. L'appendice C è stata corretta utilizzando per tutti i test il verbo "verificare" al posto del sostantivo "verifica", e non un \textit{mix} dei due.

\subsection{Revisione di qualifica - RQ}
L'esito della revisione ha evidenziato errori principalmente nel corpo dei documenti consegnati e non nell'organizzazione e nella progettazione del prodotto. Il team, soddisfatto del miglioramento della qualità di alcuni documenti, ha dunque dedicato un considerevole sforzo per correggere anche gli ultimi errori evidenziati. Di seguito verranno riportate le modifiche fatte.
\paragraph*{Generali}
Sono state apportate le seguenti azioni correttive:
\begin{itemize}
\item \textbf{Verbali}: il team si impegna a non mischiare inglese e italiano nella stesura dei verbali e a rendere più espliciti i riferimenti. Inoltre si cercherà di utilizzare lo stesso formalismo degli altri documenti.
\end{itemize}

\paragraph*{Norme di progetto}
In sede di revisione è stata segnalata una mancanza di profondità nei contenuti, soprattutto nei confronti della mancanza delle procedure di automazione. Il \glossaryItem{team} ha quindi deciso di inserire maggiori dettagli riguardo a tali procedure, in particolare:
\begin{itemize}
\item aggiunta della descrizione dei comandi usati per la generazione automatica dell'elenco delle tabelle e delle figure;
\item aggiunta della descrizione del processo per la generazione automatica del glossario dei termini;
\item aggiunta della descrizione delle regole per la stesura della documentazione del codice e degli strumenti automatici per la generazione dei siti web ad essa associati;
\item aggiunta della descrizione del tool di misurazione del codice.
\end{itemize}
Inoltre sono stati rimossi i riferimenti agli elenchi di figure e tabelle nell'indice del documento.

\paragraph*{Analisi dei requisiti}
Il documento è stato notevolmente migliorato nel periodo che ha portato alla RQ. I miglioramenti sono stati notati, ed ora il documento è conforme alle attese. Il team ha tuttavia notato un piccolo errore nella descrizione testuale del caso d'uso UC-S1. Tale errore è stato corretto.

\paragraph*{Specifica tecnica}
Anche in questo caso i miglioramenti del documento sono stati notati. Qualche dettaglio è però sfuggito alla revisione, e il team si è impegnato, in questo periodo, a correggere anche questi ultimi errori; in particolare:
\begin{itemize}
\item l'errore nel numero di versione è stato corretto;
\item è stata migliorata la descrizione delle tecnologie utilizzate;
\item sono state aggiunte le direzioni nel diagramma in figura 6;
\item alcuni diagrammi sono stati ritoccati in modo da risultare più compatti e leggibili.
\end{itemize}

\paragraph*{Manuale amministratore}
Essendo il manuale consegnato una bozza di quello definitivo il team si è impegnato a rendere più approfonditi e completi i contenuti. Inoltre sono state aggiunte le caratteristiche richieste per un documento formale.

\paragraph*{Manuale utente}
Essendo il manuale consegnato una bozza di quello definitivo il team si è impegnato a rendere più approfonditi e completi i contenuti. Inoltre sono state aggiunte le caratteristiche richieste per un documento formale.

\paragraph*{Manuale sviluppatore}
La consegna del manuale è stata insoddisfacente, in quanto è stato consegnato un confusionario archivio compresso contenente la documentazione di codice ed API. Questo ha portato alla mancanza di un punto di accesso ben preciso e reso difficile la consultazione. Inoltre sono stati evidenziati molti errori nella documentazione realizzata. Sono state quindi fatte le seguenti modifiche:
\begin{itemize}
\item corretti i doppi ''\'' negli URL (\textbf{U}niform \textbf{R}esource \textbf{L}ocator) delle API (\textbf{A}pplication \textbf{P}rogramming \textbf{I}nterface);
\item aggiunti gli esempi di risposta per le richieste di tipo GET;
\item le \textit{password} non sono inviate in chiaro;
\item rimosso \textit{HelloWorld};
\item migliorate le descrizioni delle API;
\item aggiunti esempi di input e output.
\end{itemize}

\paragraph*{Piano di \glossaryItem{Progetto}}
In sede di revisione è stata evidenziata una mancanza di profondità nei contenuti. Il team ha dunque cercato di approfondire gli argomenti e le considerazioni contenute nel documento. È stata inoltre introdotta una sezione riguardante i meccanismi di rendicontazione utilizzati.

\paragraph*{Piano di Qualifica}
In sede di revisione è stata evidenziata un insufficiente stato di avanzamento delle verifiche rispetto alle attese della RQ. Questa mancanza è in parte dovuta alle difficoltà che il team ha riscontrato nell'unire le tecnologie utilizzate per sviluppare MaaS. Ad ogni modo tutti i test sul codice scritto sono stati eseguiti nel periodo corrente al fine di garantire la qualità attesa e promessa. \\
Inoltre si è cercato di rendere migliore e più approfondito il corpo del documento.