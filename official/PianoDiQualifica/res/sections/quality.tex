\newpage
\section{Standard di qualità}
\subsection{Standard ISO/IEC 15504}\label{AppQualitaProcessi}
ISO/IEC 15504 denominato anche \textbf{\glossaryItem{SPICE}}, consiste di un insieme di normative e linee guida per lo sviluppo di \glossaryItem{processi} software.
Definisce un modello di riferimento per dare una valutazione complessiva della maturità dei \glossaryItem{processi} in un'organizzazione del settore IT.\\
\textbf{Livelli di maturità dei \glossaryItem{processi}}\\
Per ogni \glossaryItem{processo}, il modello ISO/IEC 15504 definisce un livello di maturità che può variare nella seguente scala:
\begin{enumerate}
\setcounter{enumi}{-1}
\item Incomplete process
\item Performed process
\item Managed process
\item Established process
\item Predictable process  
\item Optimizing process
\end{enumerate}
Ciascun livello di maturità viene misurato sulla base dei seguenti attributi, che possono essere non posseduti (0\% - 15\%), parzialmente soddisfatti (16\% - 50\%), largamente soddisfatti (51\% - 85\%), pienamente soddisfatti (86\% - 100\%):
\begin{enumerate}
\item Process performance
\item Performance management
\item Work product management
\item Process definition
\item Process \glossaryItem{deployment}
\item Process measurement
\item Process control
\item Process innovation
\item Process optimization
\end{enumerate}
\subsubsection{Ciclo di Deming} \label{AppPDCA}
Il ciclo \glossaryItem{PDCA} è un ciclo iterativo che si compone di 4 passi. Ha come scopo il controllo e il miglioramento continuo dei \glossaryItem{processi} (e dei prodotti risultanti da tali \glossaryItem{processi}). 
\begin{enumerate}
\item \textbf{Plan}: \glossaryItem{fase} di pianificazione. Si definisce cosa andrà realizzato e come andrà controllato, con lo scopo di produrre risultati conformi a determinati requisiti o a obiettivi di miglioramento.
\item \textbf{Do}: \glossaryItem{fase} di esecuzione. Si agisce in base a quanto pianificato nella \glossaryItem{fase} Plan.
\item \textbf{Check}: \glossaryItem{fase} di \glossaryItem{verifica}. Vengono controllati i prodotti e i \glossaryItem{processi} attuati nella \glossaryItem{fase} Do e confrontati con quanto pianificato nella \glossaryItem{fase} Plan. Le informazioni raccolte saranno utili per intervenire in tempo in caso di errori, per avere un controllo sullo stato dei \glossaryItem{processi} e dei prodotti, per migliorare i risultati.
\item \textbf{Act}: \glossaryItem{Fase} di “miglioramento continuo”. Se i risultati prodotti in Do sono diversi rispetto a quelli attesi dalla \glossaryItem{fase} Plan, allora probabilmente è necessario ripetere di nuovo il ciclo \glossaryItem{PDCA} e adottare delle azioni per il miglioramento continuo.
\end{enumerate}
\subsection{Standard ISO/IEC 9126}\label{AppQualitaProdotto}
Lo standard ISO/IEC 9126 consiste di una serie di normative e linee guida che hanno lo scopo di descrivere un modello di \glossaryItem{qualità} del software. 
Tale modello di \glossaryItem{qualità} è classificato da un insieme di 6 caratteristiche generali, a loro volta suddivise in sotto caratteristiche.
Si fa notare che lo standard descritto in questa sezione è stato sostituito dallo standard \href{http://www.iso.org/iso/iso_catalogue/catalogue_tc/catalogue_detail.htm?csnumber=35733}{ISO/IEC 25010:2011}, nel quale sono state apportate delle modifiche al modello di \glossaryItem{qualità} (sono state aggiunte/rimosse/modificate alcune caratteristiche e sotto caratteristiche).\\
Il \glossaryItem{team} BugBusters farà riferimento allo standard ISO/IEC 9126.
La norma tecnica relativa alla \glossaryItem{qualità} del software si compone di quattro parti:
\begin{itemize}
\item Modello della \glossaryItem{qualità} del software
\item Metriche per la \glossaryItem{qualità} esterna
\item Metriche per la \glossaryItem{qualità} interna
\item Metriche per la \glossaryItem{qualità} in uso
\end{itemize}
\textbf{Modello di \glossaryItem{qualità} - caratteristiche}
\begin{itemize}
\item \textbf{Funzionalità}: capacità di un prodotto software di fornire funzioni che soddisfino esigenze stabilite.
Sotto caratteristiche:
\begin{itemize}
\item Appropriatezza
\item Accuratezza
\item Interoperabilità
\item Conformità
\item Sicurezza
\end{itemize}
\item \textbf{Affidabilità}: capacità del prodotto software di mantenere un certo livello di prestazioni quando usato in date condizioni per un dato periodo.
Sotto caratteristiche:
\begin{itemize}
\item Maturità
\item Tolleranza agli errori
\item Resistenza agli errori
\item Aderenza
\end{itemize}
\item \textbf{Efficienza}: capacità di fornire appropriate prestazioni relativamente alla quantità di risorse usate.
Sotto caratteristiche:
\begin{itemize}
\item Comportamento rispetto al tempo
\item Utilizzo delle risorse
\item Conformità
\end{itemize}
\item \textbf{Usabilità}: capacità del prodotto software di essere capito, appreso e usato ottimamente dall'utente, sotto condizioni specificate.
Sotto caratteristiche:
\begin{itemize}
\item Comprensibilità
\item Apprendibilità
\item Operabilità
\item Attrattivà
\item Conformità
\end{itemize}
\item \textbf{Manutenibilità}: capacità del software di essere modificato, includendo correzioni, miglioramenti o adattamenti.
Sotto caratteristiche:
\begin{itemize}
\item Analizzabilità
\item Modificabilità
\item Stabilità
\item Testabilità
\end{itemize}
\item \textbf{Portabilità}: capacità del software di essere trasportato da un ambiente di lavoro ad un altro. (Ambiente che può variare dall'hardware al sistema operativo).
Sotto caratteristiche:
\begin{itemize}
\item Adattabilità
\item Installabilità
\item Sostituibilità
\item Conformità
\end{itemize}
\end{itemize}
\textbf{Metriche per la \glossaryItem{qualità} esterna}\\
Le metriche esterne, specificate nella norma ISO/IEC 9126-2, misurano i comportamenti del software sulla base dei test, dell'operatività e dell'osservazione durante la sua esecuzione, in funzione degli obiettivi stabiliti in un contesto tecnico rilevante o di business.\\
\textbf{Metriche per la \glossaryItem{qualità} interna}\\
La \glossaryItem{qualità} interna è specificata nella norma ISO/IEC 9126-3 e si applica al software non eseguibile (ad esempio il \glossaryItem{codice sorgente}) durante le \glossaryItem{fasi} di progettazione e codifica. 
Le metriche interne permettono di individuare eventuali problemi che potrebbero influire sulla \glossaryItem{qualità} finale del prodotto prima che sia realizzato il software eseguibile.\\
\textbf{Metriche per la \glossaryItem{qualità} in uso}\\
La \glossaryItem{qualità} in uso rappresenta il punto di vista dell'utente sul software. La \glossaryItem{qualità} in uso permette di abilitare specificati utenti a raggiungere specificati obiettivi con \glossaryItem{efficacia}, produttività, sicurezza e soddisfazione.
