\subsubsection{Validazione}
Di seguito vengono riportati gli esiti delle misurazioni effettuate sulle classi prodotte durante il periodo di Validazione.
Per quanto riguarda la misurazione della complessità ciclomatica del frontend, questa si riferisce solo all'insieme di classi e non è stata misurata per ogni singola classe come nel caso del backend. Questo perché lo strumento automatico utilizzato, complexity-report, viene eseguito sui file con estensione .js (di JavaScript) e non in quelli con estensione .ts (di TypeScript). La compilazione del codice scritto con React, tuttavia, produce un unico file JavaScript, bundle.js, che viene poi importato nelle pagine HTML. Per questo motivo lo script deve essere eseguito solo su questo file.

\paragraph{Dimensioni del prodotto software} \mbox{} \\
\begin{table}[H]
\centering
\begin{tabular}{|l|l|}
\cline{1-2}
\textbf{Backend} & 1126 \\ \cline{1-2} 
\textbf{Frontend} & 20066 \\ \cline{1-2} 
\textbf{Totale} & 21192 \\ \cline{1-2}
\end{tabular}
\caption{SLOC}
\end{table}

\paragraph{Complessità ciclomatica} \mbox{} \\
\subparagraph{Backend} \mbox{} \\
\begin{center}
\begin{longtable}{| >{\centering}p{7cm} | >{\centering}p{1.8cm} |}
\textbf{Classe} & \textbf{Valore} \tabularnewline \hline 
config::ChooseConfiguration & 4 \tabularnewline \hline
config::Configuration & 1 \tabularnewline \hline
config::DevConfiguration & 4 \tabularnewline \hline
config::ProdConfiguration & 4 \tabularnewline \hline
config::TestConfiguration & 4 \tabularnewline \hline
config::MongoConnection & 1 \tabularnewline \hline
lib::AuthenticationChecker & 6 \tabularnewline \hline
lib::DSLChecker & 11 \tabularnewline \hline
lib::LevelChecker & 5 \tabularnewline \hline
lib::Map & 1 \tabularnewline \hline
models::CompanyModel & 4 \tabularnewline \hline
models::CustomModelInterface & 1 \tabularnewline \hline
models::DatabaseModel & 1 \tabularnewline \hline
models::DSLModel & 1 \tabularnewline \hline
models::Model & 1 \tabularnewline \hline
models::MongooseConnection & 3 \tabularnewline \hline 
models::UserModel & 1 \tabularnewline \hline
routes::CompanyRouter & 1 \tabularnewline \hline
routes::DatabaseRouter & 1 \tabularnewline \hline
routes::DSLRouter & 1 \tabularnewline \hline
routes::UserRouter & 1 \tabularnewline \hline %
\caption{Misurazioni della complessità ciclomatica per le classi del backend}
\end{longtable}
\end{center}
\textbf{Esito}: Superato.

\begin{table}[H]
\centering
\begin{tabular}{|l|l|}
\cline{1-2}
\textbf{Media} & 3 \\ \cline{1-2}
\end{tabular}
\caption{Complessità ciclomatica media del frontend}
\end{table}
\textbf{Esito}: Superato.

\paragraph{Number of methods} \mbox{} \\
Nel calcolo del numero di metodi di una classe non sono contati quelli di eventuali superclassi dei quali non è stato fatto \textit{override}.
\subparagraph{Backend} \mbox{} \\
\begin{center}
\begin{longtable}{| >{\centering}p{7cm} | >{\centering}p{1.8cm} |}
\textbf{Classe} & \textbf{Valore} \tabularnewline \hline 
config::ChooseConfiguration & 1 \tabularnewline \hline
config::Configuration & 4 \tabularnewline \hline
config::Configuration & 1 \tabularnewline \hline
config::DevConfiguration & 1 \tabularnewline \hline
config::ProdConfiguration & 1 \tabularnewline \hline
config::TestConfiguration & 1 \tabularnewline \hline
config::MongoConnection & 6 \tabularnewline \hline
lib::AuthenticationChecker & 6 \tabularnewline \hline
lib::DSLChecker & 10 \tabularnewline \hline
lib::LevelChecker & 4 \tabularnewline \hline
lib::Map & 0 \tabularnewline \hline
models::CompanyModel & 3 \tabularnewline \hline
models::CustomModelInterface & 0 \tabularnewline \hline
models::DatabaseModel & 8 \tabularnewline \hline
models::DSLModel & 4 \tabularnewline \hline
models::Model & 7 \tabularnewline \hline
models::MongooseConnection & 3 \tabularnewline \hline 
models::UserModel & 9 \tabularnewline \hline
routes::CompanyRouter & 7 \tabularnewline \hline
routes::DatabaseRouter & 7 \tabularnewline \hline
routes::DSLRouter & 7 \tabularnewline \hline
routes::UserRouter & 10 \tabularnewline \hline %
\caption{Numero di metodi delle classi del backend}
\end{longtable}
\end{center}
\textbf{Esito}: Superato.

\subparagraph{Frontend} \mbox{} \\
\begin{center}
\begin{longtable}{| >{\centering}p{7cm} | >{\centering}p{1.8cm} |}
\textbf{Classe} & \textbf{Valore} \tabularnewline \hline 
actions::CompanyActionCreator & 7 \tabularnewline \hline
actions::SessionActionCreator & 2 \tabularnewline \hline
actions::UserActionCreator & 6 \tabularnewline \hline
stores::CompanyStore & 10 \tabularnewline \hline
stores::SessionStore & 8 \tabularnewline \hline
stores::UserStore & 7 \tabularnewline \hline
utils::Checker & 2 \tabularnewline \hline
utils::CompanyAPIs & 6 \tabularnewline \hline
utils::SessionAPIs & 1 \tabularnewline \hline
utils::UserAPIs & 6 \tabularnewline \hline
\caption{Numero di metodi delle classi del frontend}
\end{longtable}
\end{center}
\textbf{Esito}: Superato.

\paragraph{Variabili non utilizzate e/o non definite} \mbox{} \\
\begin{table}[H]
\centering
\begin{tabular}{|l|l|}
\cline{1-2}
\textbf{Backend} & 0 \\ \cline{1-2}
\textbf{Frontend} & 0 \\ \cline{1-2}
\textbf{Totale} & 0 \\ \cline{1-2}
\end{tabular}
\caption{Variabili non utilizzate e/o non definite}
\end{table}
\textbf{Esito}: Superato.

\paragraph{Numero di parametri per metodo} \mbox{} \\
\subparagraph{Backend} \mbox{} \\
\begin{center}
\begin{longtable}{| >{\centering}p{7cm} | >{\centering}p{1.8cm} | >{\centering}p{1.8cm} | >{\centering}p{1.8cm} |}
\textbf{Classe} & \textbf{Minimo} & \textbf{Massimo} & \textbf{Medio} \tabularnewline \hline 
config::ChooseConfiguration & 0 & 0 & 0 \tabularnewline \hline
config::Configuration & 0 & 3 & 0.75 \tabularnewline \hline
config::DevConfiguration & 2 & 2 & 2 \tabularnewline \hline
config::ProdConfiguration & 2 & 2 & 2 \tabularnewline \hline
config::TestConfiguration & 2 & 2 & 2 \tabularnewline \hline
config::MongoConnection & 5 & 0 & 0.83 \tabularnewline \hline
lib::AuthenticationChecker & 1& 3 & 2 \tabularnewline \hline
lib::DSLChecker & 2 & 2 & 2 \tabularnewline \hline
lib::LevelChecker & 1 & 3 & 2 \tabularnewline \hline
lib::Map & 0 & 0 & 0 \tabularnewline \hline
models::CompanyModel & 0 & 0 & 0 \tabularnewline \hline
models::CustomModelInterface & 0 & 0 & 0 \tabularnewline \hline
models::DatabaseModel & 0 & 5 & 1.37 \tabularnewline \hline
models::DSLModel & 0 & 1 & 0.25 \tabularnewline \hline
models::Model & 0 & 2 & 0.71 \tabularnewline \hline
models::MongooseConnection & 0 & 1 & 0.25 \tabularnewline \hline 
models::UserModel & 0 & 4 & 1.22 \tabularnewline \hline
routes::CompanyRouter & 0 & 2 & 1.42 \tabularnewline \hline
routes::DatabaseRouter & 0 & 2 & 1.42 \tabularnewline \hline
routes::DSLRouter & 0 & 2 & 1.42 \tabularnewline \hline
routes::UserRouter & 0 & 2 & 1 \tabularnewline \hline %
\caption{Numero di parametri per metodo delle classi del backend}
\end{longtable}
\end{center}
\textbf{Esito}: Superato.

\paragraph{Numero di parametri per metodo} \mbox{} \\
\subparagraph{Frontend} \mbox{} \\
\begin{center}
\begin{longtable}{| >{\centering}p{7cm} | >{\centering}p{1.8cm} | >{\centering}p{1.8cm} | >{\centering}p{1.8cm} |}
\textbf{Classe} & \textbf{Minimo} & \textbf{Massimo} & \textbf{Medio} \tabularnewline \hline 
actions::CompanyActionCreator & 1 & 3 & 1.85 \tabularnewline \hline
actions::SessionActionCreator & 0 & 1 & 0.5 \tabularnewline \hline
actions::UserActionCreator & 1 & 2 & 1.16 \tabularnewline \hline
stores::CompanyStore & 0 & 1 & 0.37 \tabularnewline \hline
stores::SessionStore & 0 & 1 & 0.25 \tabularnewline \hline
stores::UserStore & 0 & 1 & 0.28 \tabularnewline \hline
utils::Checker & 0 & 0 & 0 \tabularnewline \hline
utils::CompanyAPIs & 1 & 3 & 1.83 \tabularnewline \hline
utils::SessionAPIs & 2 & 2 & 2 \tabularnewline \hline
utils::UserAPIs & 1 & 2 & 1.16 \tabularnewline \hline
\caption{Numero di parametri per metodo delle classi del frontend}
\end{longtable}
\end{center}
\textbf{Esito}: Superato.
