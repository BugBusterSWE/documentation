% example & bla & blabla \tabularnewline \hline
%TI7, TI8, TI9, TI9, TI10, TI11, TI12, TI13

TI-B   & Verificare che il \textit{package} \textbf{Backend} (\textbf{MaaSServer}) si integri correttamente 
	 con le librerie \glossaryItem{Node.js} utilizzate. & N.E \tabularnewline \hline     % < TI 8

TI-BC  & Verificare che la componente \textbf{Config} permetta al server di avviarsi correttamente, e che 
	 il sistema gestisca correttamente il \textit{backend} del prodotto per fornire al \textit{frontend}
	 tutte le informazioni necessarie. & N.E \tabularnewline \hline   % < TI 7


TI-BR  & Verificare la corretta integrazione tra i \textit{package} \textbf{Backend}
         e il reindirizzamento adeguato delle richieste \textit{REST} destinate al server. & N.E \tabularnewline \hline  % < TI 10


TI-BM  & Verificare la corretta integrazione tra i \textbf{Models} e i \textbf{Routers} per la gestione dell'inserimento, della
	 modifica e dell'eliminazione dei dati. & N.E \tabularnewline \hline    % < TI 13


TI-BN  & Verificare che la componente \textbf{NodeMailer} si integri correttamente con \textbf{ExpressJS}
	 e con la componente principale del \textbf{Backend} (\textbf{MaaSServer}) & N.E \tabularnewline \hline      

TI-BM-M & Verificare che la componente \textbf{Mongoose} si integri correttamente con la componente \textbf{Models}.
	& N.E \tabularnewline \hline

TI-BL   & Verificare che la componente \textbf{Lib} si integri correttamente con \textbf{Routers} 
	  e venga inizializzata correttamente dalla componente principale del \textbf{Backend} (\textbf{MaaSServer}).
	& N.E \tabularnewline \hline


