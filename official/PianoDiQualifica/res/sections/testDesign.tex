\newpage
\section{Pianificazione dei test}

	Si vuole adottare una strategia di \glossaryItem{verifica} del software tramite test opportunamente predeterminati e che garantiscano almeno un test per ogni requisito. I test sono l’applicazione delle tecniche di analisi dinamica introdotte nelle \NormeDiProgetto; tali attività, oltre a richiedere l’esecuzione del programma, devono poter essere ripetibili, ossia tramite delle specifiche su come riprodurre i test vogliamo che il loro output sia deterministico. È importante che i test di unità vengano svolti in parallelo, dando precedenza alle unità che producono risultati utili alla comprensione del loro funzionamento integrato. L’ambiente di testing deve soddisfare tale obiettivo. L’attività di test deve produrre un log che specifica quando e chi ha eseguito il test e con quali input; l’insorgenza di failure deve essere tracciata e catalogata.

\subsection{Livelli di testing}
	Il testing del software viene suddiviso in livelli differenti, che si concretizzano in un'esecuzione bottom-up che avanza sequenzialmente alle attività di codifica e di \glossaryItem{validazione}. I test da effettuare sul prodotto sono di cinque tipi, ma gli ultimi due saranno specificati nella prossima revisione:

\begin{enumerate}
	\item Test di \glossaryItem{validazione} (TV): viene verificato che il prodotto soddisfi quanto richiesto dal proponente individuando delle macro azioni che un utente svolge comunemente sul sistema;
	\item Test di sistema (TS): sono test relativi al comportamento dell’intero sistema. Viene verificato che l'architettura generale funzioni complessivamente bene;
	\item Test di integrazione (TI): vengono verificate le componenti del sistema contenute nella \SpecificaTecnica, ossia viene verificato che i \glossaryItem{package} siano funzionanti e in grado di funzionare nel loro insieme;
	\item Test di unità (TU): viene testata ogni unità, ossia la più piccola parte di lavoro assegnabile ad un \glossaryItem{programmatore}. In questo \glossaryItem{progetto} una unità dovrebbe corrispondere ad una function o a un method;
	\item Test di regressione (TR): possono essere test di tutte le tipologie succitate, e devono mostrare il funzionamento del prodotto a seguito di una modifica.
\end{enumerate}

	La seguente figura illustra come i test elencati vengono distribuiti durante in ciclo di vita del prodotto.

	\begin{figure}[H]
		\centering
		\includegraphics[scale=0.5]{res/sections/V-model}
		\caption{V-model per il testing software}
	\end{figure}
	

\subsection{Test di sistema}
	Vengono qui descritti i test di sistema che andranno a verificare il funzionamento complessivo delle componenti. Nella seguente tabella, lo stato di ogni test è definito da N.E per non eseguito.

	%table of testsTable
	\begin{center}
  
  \begin{table}[H]
    \centering
    \begin{tabular}{ | >{\centering}p{3cm} | >{\centering}p{6cm} | >{\centering}p{1.5cm} | >{\centering}p{2cm} | }
      \textbf{Test Sistema} & \textbf{Descrizione} & \textbf{Stato} & \textbf{Requisito} \tabularnewline \hline
      

	% example & bla & blabla & blablabla \tabularnewline \hline
TS-R1O 1 & Verificare che l'utente non autenticato possa creare un account per la propria \glossaryItem{Company}. & S & R1O 1 \tabularnewline \hline    %1

TS-R1O 1.1 & Verificare che l'utente non autenticato possa inserire il nome della propria \glossaryItem{Company} nel campo di testo dedicato, all'interno della \glossaryItem{procedura} di creazione dell'account della \glossaryItem{Company}. & S & R1O 1.1  \tabularnewline \hline   %2


TS-R1O 1.2 & Verificare che l'utente non autenticato possa inserire l'email della propria \glossaryItem{Company} nel campo di testo dedicato, all'interno della \glossaryItem{procedura} di creazione dell'account della \glossaryItem{Company}. & S & R1O 1.2 \tabularnewline \hline   %3


TS-R1O 1.3 & Verificare che l'utente non autenticato possa inserire la password della propria \glossaryItem{Company} nel campo di testo (offuscato) dedicato, all'interno della \glossaryItem{procedura} di creazione dell'account della \glossaryItem{Company}. & S & R1O 1.3 \tabularnewline \hline   %4

TS-R1O 2 & Verificare che l'utente non autenticato (dopo aver ricevuto un invito da parte dell'Owner, o di un suo delegato) riesca a registrarsi a \glossaryItem{MaaS} e ad accedere al proprio account & S & R1O 2 \tabularnewline \hline %5

TS-R1O 2.1 & Verificare che l'utente non autenticato (dopo aver ricevuto un invito da parte dell'Owner, o di un suo delegato) riesca a registrarsi a \glossaryItem{MaaS} tramite una password. & S & R1O 2.1 \tabularnewline \hline %6

TS-R1O 3 & Verificare che l'utente autenticato possa accedere all'editor ed eseguire una qualsiasi operazione su una specifica \glossaryItem{DSL}. & S & R1O 3 \tabularnewline \hline %13

TS-R1O 3.1 & Verificare che l'utente autenticato possa eseguire una \glossaryItem{DSL} in particolare (dopo l'accesso all'editor). & S & R1O 3.1 \tabularnewline \hline %14

TS-R1O 3.2 & Verificare che l'utente autenticato (con ruolo superiore a Member) possa modificare una specifica \glossaryItem{DSL} (dopo l'accesso all'editor). & S & R1O 3.2 \tabularnewline \hline %15

TS-R1O 3.3 & Verificare che l'utente autenticato (con ruolo superiore a Member) possa creare una specifica \glossaryItem{DSL} (dopo l'accesso all'editor). & S & R1O 3.3 \tabularnewline \hline %16

TS-R1O 3.4 & Verificare che l'utente autenticato (con ruolo superiore a Member) possa leggere una specifica \glossaryItem{DSL} (dopo l'accesso all'editor). & S & R1O 3.4 \tabularnewline \hline %17

TS-R1O 3.5 & Verificare che l'Admin possa aggiungere (oppure togliere) i permessi di scrittura (oppure esecuzione) ad un utente di una \glossaryItem{Company} su una specifica \glossaryItem{DSL}. & S & R1O 3.5 \tabularnewline \hline %18

TS-R1O 3.6 & Verificare che l'Owner di una \glossaryItem{Company} possa invitare un utente a registrarsi presso \glossaryItem{MaaS}. & S & R1O 3.6 \tabularnewline \hline %19   

TS-R1O 4 & Verificare che l'utente autenticato possa accedere alla pagina \glossaryItem{Dashboard}, visualizzarne il contenuto ed effettuare una delle operazioni permesse sugli elementi presenti. & S & R1O 4
\tabularnewline \hline %20

TS-R1O 4.1 & Verificare che l'utente autenticato possa visualizzare uno qualsiasi degli elementi della \glossaryItem{Dashboard} (Cell, \glossaryItem{Document} o \glossaryItem{Collection}). & S & R1O 4.1 \tabularnewline \hline %21

TS-R1O 4.2 & Verificare che l'utente autenticato possa effettuare una delle operazioni permesse su uno qualsiasi degli elementi della \glossaryItem{Dashboard} (Cell, \glossaryItem{Document} o \glossaryItem{Collection}). & S & R1O 4.2 \tabularnewline \hline %22  


TS-R1O 4.2.1 & Verificare che l'utente autenticato possa aggiungere, modificare o ordinare un valore in un elemento \glossaryItem{Cell}. & S & R1O 4.2.1 \tabularnewline \hline %23


TS-R1O 4.2.2 & Verificare che l'utente autenticato possa modificare o rimuovere un elemento \glossaryItem{Document}. & S & R1O 4.2.2 \tabularnewline \hline %24


TS-R1O 4.2.3 & Verificare che l'utente autenticato possa eseguire un Send mail (o Export) dalla pagina \glossaryItem{Document}. & S & R1O 4.2.3 \tabularnewline \hline %25

TS-R1O 4.2.4 & Verificare che i Documents all'interno di una \glossaryItem{Collection} siano ordinabili in base a uno dei loro campi. & S & R1O 4.2.4 \tabularnewline \hline
TS-R1O 4.2.5 & Verificare che l'utente autenticato debba poter rimuovere una \glossaryItem{Collection}. & S & R1O 4.2.5 \tabularnewline \hline
TS-R1O 4.2.6 & Verificare che l'utente autenticato debba poter eseguire un'azione (send mail/export) dalla pagina \glossaryItem{Collection}. & S & R1O 4.2.6 \tabularnewline \hline
TS-R1O 5 & Verificare che l'applicazione mostri al Super-Admin la pagina di gestione di tutte le \glossaryItem{Company}. & S & R1O 5 \tabularnewline \hline
TS-R1O 5.1 & Verificare che il Super-Admin possa aggiungere una nuova \glossaryItem{Company}. & S & R1O 5.1 \tabularnewline \hline
TS-R1O 5.2 & Verificare che il Super-Admin possa modificare i dati di una \glossaryItem{Company}. & S & R1O 5.2 \tabularnewline \hline
 

	% example & bla & blabla & blablabla \tabularnewline \hline
% num test % desc % N.E & cod req
% TS-cod.req

TS-R1O 4.2.4 & Verificare che i Documents all'interno di una Collection siano ordinabili in base a uno dei loro campi. & N.E & R1O 4.2.4 \tabularnewline \hline

TS-R1O 4.2.5 & Verificare che l'utente autenticato debba poter rimuovere una Collection. & N.E & R1O 4.2.5 \tabularnewline \hline
TS-R1O 4.2.6 & Verificare che l'utente autenticato debba poter eseguire un'azione (send mail/export) dalla pagina Collection. & N.E & R1O 4.2.6 \tabularnewline \hline
TS-R1O 5 & Verificare che l'applicazione mostri al Super-Admin la pagina di gestione di tutte le Company. & N.E & R1O 5 \tabularnewline \hline
TS-R1O 5.1 & Verificare che il Super-Admin possa aggiungere una nuova Company. & N.E & R1O 5.1 \tabularnewline \hline
TS-R1O 5.2 & Verificare che il Super-Admin possa modificare i dati di una Company. & N.E & R1O 5.2 \tabularnewline \hline

	% example & bla & blabla & blablabla \tabularnewline \hline
TS-R1D R1D 13.10 & DashRow G deve avere una & N.E & R1D R1D 13.10 \tabularnewline \hline
TS-R1D 13.10 & Verifica che DashRow deve avere una rappresentazione grafica visibile sul browser. & N.E & R1D 13.10 \tabularnewline \hline
TS-R1D 13.11 & Verifica che una Action definita dall'amministratore della piattaforma ha una rappresentazione grafica visibile sul browser. & N.E & R1D 13.11 \tabularnewline \hline
TS-R1D 14 & Verifica che l'utente, in caso di errori, venga avvisato con un messaggio d'errore. & N.E & R1D 14 \tabularnewline \hline
TS-R3D 14.1 & Verifica che il messaggio d'errore, in caso di fallimento della validazione del DSL, indichi il DSL Element che genera l'errore. & N.E & R3D 14.1 \tabularnewline \hline
TS-R1D 15 & Verifica che ad ogni azione dell'utente compiuta nell'editor, corrisponde la manipolazione della struttura del DSL. & N.E & R1D 15 \tabularnewline \hline
TS-R1D 16 & Verifica che l'applicazione mantenga la correttezza di una struttura DSL. & N.E & R1D 16 \tabularnewline \hline
TS-R1D 17 & Verifica che l'utente possa usare il tipo di collegamento Riferimento tra due elementi DSL. & N.E & R1D 17 \tabularnewline \hline
TS-R1D 18 & Verifica che l'utente possa usare il tipo di collegamento Associazione tra due elementi DSL. & N.E & R1D 18 \tabularnewline \hline
TS-R1D 19 & Verifica che l'utente possa definire i valori degli attributi del DSL tramite l'editor. & N.E & R1D 19 \tabularnewline \hline
TS-R1D 20 & Verifica che l'utente possa aggiungere e rimuovere gli attributi dal DSL tramite l'editor. & N.E & R1D 20 \tabularnewline \hline
TS-R1D 21 & Verifica che l'utente possa aggiungere o rimuovere una struttura del DSL tramite l'editor. & N.E & R1D 21 \tabularnewline \hline
TS-R1D 22 & Verifica che l'utente possa scrivere la funzione JavaScript direttamente nell'editor. & N.E & R1D 22 \tabularnewline \hline
TS-R1D 23 & Verifica che l'utente, attraverso l'interfaccia grafica, possa essere in grado di salvare i metodi creati. & N.E & R1D 23 \tabularnewline \hline
TS-R1D 24 & Verifica che l'utente possa importare nel DSL corrente una funzione JavaScript precedentemente creata. & N.E & R1D 24 \tabularnewline \hline
TS-R1D 25 & Verifica che l'utente, attraverso l'interfaccia grafica, visualizzi l'insieme dei metodi a cui può accedere. & N.E & R1D 25 \tabularnewline \hline

	%  \end{longtable} %problem with latex compiler. See issue #129
\end{center}

