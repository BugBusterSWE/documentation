\newpage
\section{Pianificazione dei test}

	Si vuole adottare una strategia di \glossaryItem{verifica} del software tramite test opportunamente predeterminati e che garantiscano almeno un test per ogni requisito. I test sono l’applicazione delle tecniche di analisi dinamica introdotte nelle \NormeDiProgetto; tali attività, oltre a richiedere l’esecuzione del programma, devono poter essere ripetibili, ossia tramite delle specifiche su come riprodurre i test vogliamo che il loro output sia deterministico. È importante che i test di unità vengano svolti in parallelo, dando precedenza alle unità che producono risultati utili alla comprensione del loro funzionamento integrato. L’ambiente di testing deve soddisfare tale obiettivo. L’attività di test deve produrre un log che specifica quando e chi ha eseguito il test e con quali input; l’insorgenza di failure deve essere tracciata e catalogata.

\subsection{Livelli di testing}
	Il testing del software viene suddiviso in livelli differenti, che si concretizzano in un'esecuzione bottom-up che avanza sequenzialmente alle attività di codifica e di \glossaryItem{validazione}. I test da effettuare sul prodotto sono di cinque tipi, ma gli ultimi due saranno specificati nella prossima revisione:

\begin{enumerate}
	\item Test di \glossaryItem{Validazione} (TV): viene verificato che il prodotto soddisfi quanto richiesto dal proponente individuando delle macro azioni che un utente svolge comunemente sul sistema;
	\item Test di Sistema (TS): sono test relativi al comportamento dell’intero sistema. Viene verificato che l'architettura generale funzioni complessivamente bene;
	\item Test di Integrazione (TI): vengono verificate le componenti del sistema contenute nella \SpecificaTecnica, ossia viene verificato che i \glossaryItem{package} siano funzionanti e in grado di funzionare nel loro insieme;
	\item Test di Unità (TU): viene testata ogni unità, ossia la più piccola parte di lavoro assegnabile ad un \glossaryItem{programmatore}. In questo \glossaryItem{progetto} una unità dovrebbe corrispondere ad una function o a un method;
	\item Test di Regressione (TR): possono essere test di tutte le tipologie succitate, e devono mostrare il funzionamento del prodotto a seguito di una modifica.
\end{enumerate}

	La seguente figura illustra come i test elencati vengono distribuiti durante in ciclo di vita del prodotto.

	\begin{figure}[H]
		\centering
		\includegraphics[scale=0.5]{res/sections/V-model}
		\caption{V-model per il testing software}
	\end{figure}
	

\subsection{Test di Sistema}
	Vengono qui descritti i test di sistema che andranno a verificare il funzionamento complessivo delle componenti. Nella seguente tabella, lo stato di ogni test è definito da N.E per non eseguito.

	%table of testsTable
	\begin{center}
  
  \begin{table}[H]
    \centering
    \begin{tabular}{ | >{\centering}p{3cm} | >{\centering}p{6cm} | >{\centering}p{1.5cm} | >{\centering}p{2cm} | }
      \textbf{Test Sistema} & \textbf{Descrizione} & \textbf{Stato} & \textbf{Requisito} \tabularnewline \hline
      
      % example & bla & blabla & blablabla \tabularnewline \hline

	% example & bla & blabla & blablabla \tabularnewline \hline
TS-R1O 1 & & & R1O \tabularnewline \hline    %1
TS-R1O 1.1 & & & R1O 1.1 \tabularnewline \hline   %2
TS-R1O 1.2 & & & R1O 1.2 \tabularnewline \hline   %3
TS-R1O 1.3 & & & R1O 1.3 \tabularnewline \hline   %4

TS-R1O 2 & & & R10 2   

 

	% example & bla & blabla & blablabla \tabularnewline \hline
% num test % desc % N.E & cod req
% TS-cod.req

TS-R1O 4.2.4 & Verificare che i Documents all'interno di una Collection siano ordinabili in base a uno dei loro campi. & N.E & R1O 4.2.4 \tabularnewline \hline

TS-R1O 4.2.5 & Verificare che l'utente autenticato debba poter rimuovere una Collection. & N.E & R1O 4.2.5 \tabularnewline \hline
TS-R1O 4.2.6 & Verificare che l'utente autenticato debba poter eseguire un'azione (send mail/export) dalla pagina Collection. & N.E & R1O 4.2.6 \tabularnewline \hline
TS-R1O 5 & Verificare che l'applicazione mostri al Super-Admin la pagina di gestione di tutte le Company. & N.E & R1O 5 \tabularnewline \hline
TS-R1O 5.1 & Verificare che il Super-Admin possa aggiungere una nuova Company. & N.E & R1O 5.1 \tabularnewline \hline
TS-R1O 5.2 & Verificare che il Super-Admin possa modificare i dati di una Company. & N.E & R1O 5.2 \tabularnewline \hline
TS-R1O 6 & Verificare che il sistema mantenga un'associazione consistente tra un utente e una Company. & N.E & R1O 6 \tabularnewline \hline
TS-R1O 7 & Verificare che il Super-Admin possa visualizzare in dettaglio il profilo di un utente di una Company. & N.E & R1O 7 \tabularnewline \hline
TS-R1O 7.1 & Verificare che il Super-Admin possa modificare i dati di un utente. & N.E & R1O 7.1 \tabularnewline \hline
TS-R1O 7.2 & Verificare che il Super-Admin possa eliminare un utente. & N.E & R1O 7.2 \tabularnewline \hline
TS-R1O 7.3 & Verificare che il Super-Admin possa creare a sua volta un altro Super-Admin. & N.E & R1O 7.3 \tabularnewline \hline
TS-R1O 8 & Verificare che l'Owner di una Company possa rimuovere un utente. & N.E & R1O 8 \tabularnewline \hline
TS-R1O 9 & Verificare che l'Owner di una Company possa inserire manualmente un nuovo utente presso MaaS. & N.E & R1O 9 \tabularnewline \hline
TS-R1O 10 & Verificare che il sistema permetta all'utente di mantenere salvato un DSL precedentemente creato. & N.E & R1O 10 \tabularnewline \hline
TS-R1D 11 & Verificare che il sistema permetta all'utente di scaricare da Terminale una specifica DSL in un formato leggibile dall'editor. & N.E & R1D 11 \tabularnewline \hline
TS-R1D 12 & Verificare che sia possibile offrire un'interfaccia grafica per la manipolazione del DSL. & N.E & R1D 12 \tabularnewline \hline
TS-R1D 12.1 & Verificare che l'utente attriverso l'interfaccia grafica possa eseguire l'invio del DSL definito. & N.E & R1D 12.1 \tabularnewline \hline
TS-R1D 12.2 & Verificare che l'utente attraverso l'interfaccia grafica visualizzi l'insieme delle DSL a cui può accedere. & N.E & R1D 12.2 \tabularnewline \hline
TS-R1D 13 & Verificare che l'utente debba poter manipolare la struttura del DSL attraverso una rappresentazione grafica. & N.E & R1D 13 \tabularnewline \hline
TS-R1D 13.1 & Verificare che la Collection debba avere una rappresentazione grafica visibile sul browser. & N.E & R1D 13.1 \tabularnewline \hline
TS-R1D 13.2 & Verificare che una funzione in JavaScript debba avere una rappresentazione grafica visibile sul browser. & N.E & R1D 13.2 \tabularnewline \hline

	% example & bla & blabla & blablabla \tabularnewline \hline
TS-R1D R1D 13.10 & DashRow G deve avere una & N.E & R1D R1D 13.10 \tabularnewline \hline
TS-R1D 13.10 & Verifica che DashRow deve avere una rappresentazione grafica visibile sul browser. & N.E & R1D 13.10 \tabularnewline \hline
TS-R1D 13.11 & Verifica che una Action definita dall'amministratore della piattaforma ha una rappresentazione grafica visibile sul browser. & N.E & R1D 13.11 \tabularnewline \hline
TS-R1D 14 & Verifica che l'utente, in caso di errori, venga avvisato con un messaggio d'errore. & N.E & R1D 14 \tabularnewline \hline
TS-R3D 14.1 & Verifica che il messaggio d'errore, in caso di fallimento della validazione del DSL, indichi il DSL Element che genera l'errore. & N.E & R3D 14.1 \tabularnewline \hline
TS-R1D 15 & Verifica che ad ogni azione dell'utente compiuta nell'editor, corrisponde la manipolazione della struttura del DSL. & N.E & R1D 15 \tabularnewline \hline
TS-R1D 16 & Verifica che l'applicazione mantenga la correttezza di una struttura DSL. & N.E & R1D 16 \tabularnewline \hline
TS-R1D 17 & Verifica che l'utente possa usare il tipo di collegamento Riferimento tra due elementi DSL. & N.E & R1D 17 \tabularnewline \hline
TS-R1D 18 & Verifica che l'utente possa usare il tipo di collegamento Associazione tra due elementi DSL. & N.E & R1D 18 \tabularnewline \hline
TS-R1D 19 & Verifica che l'utente possa definire i valori degli attributi del DSL tramite l'editor. & N.E & R1D 19 \tabularnewline \hline
TS-R1D 20 & Verifica che l'utente possa aggiungere e rimuovere gli attributi dal DSL tramite l'editor. & N.E & R1D 20 \tabularnewline \hline
TS-R1D 21 & Verifica che l'utente possa aggiungere o rimuovere una struttura del DSL tramite l'editor. & N.E & R1D 21 \tabularnewline \hline
TS-R1D 22 & Verifica che l'utente possa scrivere la funzione JavaScript direttamente nell'editor. & N.E & R1D 22 \tabularnewline \hline
TS-R1D 23 & Verifica che l'utente, attraverso l'interfaccia grafica, possa essere in grado di salvare i metodi creati. & N.E & R1D 23 \tabularnewline \hline

	    \end{tabular}
  \end{table}
  
\end{center}

