\newpage
\section{Resoconto delle attività di verifica} \label{App:AppendixA}
	\subsection{Riassunto delle attività di verifica} \label{App:AppendixA}
		\subsubsection{Revisione dei requisiti} \label{App:AppendixA}
			
			L'attività di \glossaryItem{verifica} svolta dai verificatori è avvenuta come determinato dal \PianoDiProgetto al termine della stesura di ogni documento in ingresso alla revisione dei requisiti.
			
			La \glossaryItem{verifica} svolta sui documenti e sui \glossaryItem{processi} è avvenuta seguendo le indicazioni delle \NormeDiProgetto e misurando le metriche indicate in \hyperlink{metriche_documenti}{2.6.1}.
\subsection{Documenti} \label{App:AppendixB} 
\subsubsection{Analisi}
\paragraph*{Indice \glossaryItem{Gulpease}}
Di seguito vengono riportati gli esiti del calcolo dell'indice \glossaryItem{Gulpease} per il periodo di Analisi.
		\begin{table}[H]
			\centering
				\begin{tabular}{|l|l|l|ll}
					\cline{1-3}
					 \textbf{Macro-attività}  & \textbf{Indice \glossaryItem{Gulpease}}  & \textbf{Esito}  &  \\ \cline{1-3}
					 Norme di \glossaryItem{Progetto}  & 52 & Superato &  \\ \cline{1-3}
					 Studio Fattibilità & 57 & Superato &  \\ \cline{1-3}
					 Analisi dei Requisiti & 56 & Superato &  \\ \cline{1-3}
					 Piano di \glossaryItem{Progetto} & 63 & Superato &  \\ \cline{1-3}
					 Piano di Qualifica & 54 & Superato &  \\ \cline{1-3}
					 Glossario & 59 & Superato &  \\ \cline{1-3}
				\end{tabular}
				\caption{Esiti \glossaryItem{verifica} documenti - Analisi}
		\end{table}
\paragraph*{SV e BV}
Di seguito vengono riportati gli esiti del calcolo degli indici SV e BV per il periodo di Analisi.
		\begin{table}[H]
		\centering
		\begin{tabular}{|l|l|}
		\cline{1-2}
		\textbf{Scheduling Variance} & 0 ore \\ \cline{1-2}
		\textbf{Budget Variance} & -106 \euro{} \\ \cline{1-2}
		\end{tabular}
		\caption{SV e BV del periodo di Analisi}
		\end{table}

\subsubsection{Analisi Miglioramento}
\paragraph*{Indice \glossaryItem{Gulpease}}
Di seguito vengono riportati gli esiti del calcolo dell'indice \glossaryItem{Gulpease} per il periodo di Analisi Miglioramenti.
\begin{table}[H]
\centering
\begin{tabular}{|l|l|l|ll}
\cline{1-3}
\textbf{Macro-attività}  & \textbf{Indice \glossaryItem{Gulpease}}  & \textbf{Esito}  &  \\ \cline{1-3}
Norme di \glossaryItem{Progetto}  & 47 & Superato &  \\ \cline{1-3}
Studio Fattibilità & 57 & Superato &  \\ \cline{1-3}
Analisi dei Requisiti & 59 & Superato &  \\ \cline{1-3}
Piano di \glossaryItem{Progetto} & 57 & Superato &  \\ \cline{1-3}
Piano di Qualifica & 64 & Superato &  \\ \cline{1-3}
Glossario & 59 & Superato &  \\ \cline{1-3}
\end{tabular}
\caption{Esiti \glossaryItem{verifica} documenti - Analisi Miglioramenti}
\end{table}
\paragraph*{SV e BV}
Di seguito vengono riportati gli esiti del calcolo degli indici SV e BV per il periodo di Analisi Miglioramenti.
\begin{table}[H]
\centering
\begin{tabular}{|l|l|}
\cline{1-2}
\textbf{Schedule Variance} & -1 ore \\ \cline{1-2}
\textbf{Budget Variance} & -7 \euro{} \\ \cline{1-2}
\end{tabular}
\caption{SV e BV del periodo di Analisi Miglioramenti}
\end{table}


\subsubsection{Progettazione Architetturale}
\paragraph*{Indice \glossaryItem{Gulpease}}
Di seguito vengono riportati gli esiti del calcolo dell'indice \glossaryItem{Gulpease} per il periodo di Progettazione Architetturale.
\begin{table}[H]
			\centering
				\begin{tabular}{|l|l|l|ll}
					\cline{1-3}
					 \textbf{Macro-attività}  & \textbf{Indice \glossaryItem{Gulpease}}  & \textbf{Esito}  &  \\ \cline{1-3}
					 Norme di \glossaryItem{Progetto} & 48 & Superato &  \\ \cline{1-3}
					 Studio Fattibilità & 57 & Superato &  \\ \cline{1-3}
					 Analisi dei Requisiti & 59 & Superato &  \\ \cline{1-3}
					 Piano di \glossaryItem{Progetto} & 55 & Superato &  \\ \cline{1-3}
					 Piano di Qualifica & 66 & Superato &  \\ \cline{1-3}
					 Specifica Tecnica & 62 & Superato & \\ \cline{1-3}
					 Glossario & 59 & Superato &  \\ \cline{1-3}
				\end{tabular}
				\caption{Esiti \glossaryItem{verifica} documenti - Progettazione Architetturale}
		\end{table}
\paragraph*{SV e BV}
Di seguito vengono riportati gli esiti del calcolo degli indici SV e BV per il periodo di Progettazione Architetturale.
		\begin{table}[H]
		\centering
		\begin{tabular}{|l|l|}
		\cline{1-2}
		\textbf{Schedule Variance} & 0 ore \\ \cline{1-2}
		\textbf{Budget Variance} & -23 \euro{} \\ \cline{1-2}
		\end{tabular}
		\caption{SV e BV del periodo di Progettazione Architetturale}
		\end{table}

\subsubsection{Progettazione di Dettaglio e Codifica}
\paragraph*{Indice \glossaryItem{Gulpease}}
Di seguito vengono riportati gli esiti del calcolo dell'indice \glossaryItem{Gulpease} per il periodo di Progettazione di Dettaglio e Codifica.
\begin{table}[H]
			\centering
				\begin{tabular}{|l|l|l|ll}
					\cline{1-3}
					 \textbf{Macro-attività}  & \textbf{Indice \glossaryItem{Gulpease}}  & \textbf{Esito}  &  \\ \cline{1-3}
					 Norme di \glossaryItem{Progetto} & 46 & Superato &  \\ \cline{1-3}
					 Studio Fattibilità & 57 & Superato &  \\ \cline{1-3}
					 Analisi dei Requisiti & 59 & Superato &  \\ \cline{1-3}
					 Piano di \glossaryItem{Progetto} & 55 & Superato &  \\ \cline{1-3}
					 Piano di Qualifica & 64 & Superato &  \\ \cline{1-3}
					 Specifica Tecnica & 56 & Superato & \\ \cline{1-3}
					 Glossario & 60 & Superato &  \\ \cline{1-3}
				\end{tabular}
				\caption{Esiti \glossaryItem{verifica} documenti - Progettazione di Dettaglio e Codifica}
		\end{table}
\paragraph*{SV e BV}
Di seguito vengono riportati gli esiti del calcolo degli indici SV e BV per il periodo di Progettazione di Dettaglio e Codifica.
\begin{table}[H]
\centering
\begin{tabular}{|l|l|}
\cline{1-2}
\textbf{Schedule Variance} & -10 ore \\ \cline{1-2}
\textbf{Budget Variance} & -318 \euro{} \\ \cline{1-2}
\end{tabular}
\caption{SV e BV del periodo di Progettazione di Dettaglio e Codifica}
\end{table}
Analizzando i valori di SV e BV per il periodo di Progettazione di Dettaglio e Codifica, e confrontandoli con quelli rilevati nei precedenti periodi, si nota che in quest'occasione il \glossaryItem{team} ha lavorato in maniera più efficiente, portando ad un buon risparmio di tempo e denaro. Il risparmio orario, seppure esiguo (neanche 2 ore per ogni membro del \glossaryItem{team}), ha comunque portato ad un buon risparmio monetario. \\
\paragraph*{Numero di cambiamenti} 
In seguito alle indicazioni date in sede di revisione sono stati apportati alcuni cambiamenti importanti all'\AnalisiDeiRequisiti{} e alla \SpecificaTecnica. Si è deciso di escludere da quest'analisi i cambiamenti minori, come gli errori di battitura, le mancate normalizzazioni dei titoli, la mancanza di alcune descrizioni nella Specifica tecnica e così via. Sono stati infatti contati solo quei cambiamenti che hanno avuto impatto sull'architettura del prodotto e sull'implementazione dei \glossaryItem{casi} d'uso rilevati. \\
In particolare sono stati identificati come importanti i seguenti cambiamenti:
\begin{itemize}
\item le interfacce Models::Model e Config::Config sono diventate classi astratte;
\item rivisitazione dell'applicazione del pattern MVC;
\item eliminazione della generalizzazione tra \glossaryItem{casi} d'uso con base il caso d'uso UC-U16.1.
\end{itemize}
Le altre modifiche suggerite in \glossaryItem{fase} di revisione sono state, come già detto, ritenute di scarso impatto sul \glossaryItem{progetto}. Esse infatti hanno avuto un impatto maggiore sul miglioramento della qualità dei singoli documenti, non sulla progettazione o sui requisiti di \glossaryItem{MaaS}. \\
\begin{table}[H]
\centering
\begin{tabular}{|l|l|}
\cline{1-2}
\textbf{Esito}  & \textbf{Range di appartenenza}  \\ \cline{1-2}
3 & Ottimale \\ \cline{1-2}
\end{tabular}
\caption{Numero di cambiamenti del periodo di Progettazione di Dettaglio e Codifica}
\end{table}

\paragraph*{Produttività}
Di seguito vengono riportati gli esiti del calcolo degli indici di produttività per il periodo di Progettazione di Dettaglio e Codifica.
\begin{table}[H]
\centering
\begin{tabular}{|l|l|}
\cline{1-2}
\textbf{Di Documentazione (Manuali)} & 0.096 \\ \cline{1-2} % (ore di stesura manuali)/(numero righe di codice dei manuali)
\textbf{Di Documentazione (Codice)} & 0.07 \\ \cline{1-2} % (ore di stesura codice)/(numero righe di documentazione del codice => vedi commento SLOC misure codice) 
\textbf{Di Codifica} & 0.17 \\ \cline{1-2} % (ore di stesura del codice)/(numero di righe di codice => vedi misurazione sloc nella sezione di misurazione del codice)
\end{tabular}
\caption{Valori di produttività di documentazione e codifica del periodo di Progettazione di Dettaglio e Codifica}
\end{table}
Il valore molto basso della produttività di documentazione del codice è dovuto all'alto numero di righe di documentazione scritte, in particolare per le classi del \glossaryItem{package} routes, per le quali è stata inserita, oltre al TypeDoc di TypeScript anche la spiegazione di ciascuna \glossaryItem{API} implementata, completa di parametri, valori ritornati ed esempi d'uso (sia in caso di successo che di fallimento). Il totale di righe di documentazione scritte, infatti, ammonta a 2070, più del doppio delle righe di codice (pari a 855).

\subsubsection{Validazione}
\paragraph*{SV e BV}
Di seguito vengono riportati gli esiti del calcolo degli indici SV e BV per il periodo di Validazione.
\begin{table}[H]
\centering
\begin{tabular}{|l|l|}
\cline{1-2}
\textbf{Schedule Variance} & +11 ore \\ \cline{1-2}
\textbf{Budget Variance} & +76 \euro{} \\ \cline{1-2}
\end{tabular}
\caption{SV e BV del periodo di Validazione}
\end{table}
In quest'occasione il team ha sforato quanto pianificato, soprattutto per le difficoltà incontrate nel codificare MaaS. Tuttavia, sebbene sia stato speso più di quanto preventivato, i risparmi degli altri periodo hanno permesso al team di non superare il costo inizialmente preventivato di 13.647 \euro{}. In totale, infatti, il risparmio complessivo è stato di 355 \euro.
\paragraph*{Numero di cambiamenti} 
In seguito alle indicazioni date in sede di revisione sono stati apportati alcuni cambiamenti ai documenti, nessuno dei quali ha avuto grande impatto sul prodotto, in quanto gli errori segnalati riguardavano la profondità dei contenuti. L'unico cambiamento degno di nota è stato quello apportato al \ManualeSviluppatore{} per renderlo più facilmente consultabile e per migliorare la qualità delle descrizioni delle funzionalità esposte dal server.  \\
\begin{table}[H]
\centering
\begin{tabular}{|l|l|}
\cline{1-2}
\textbf{Esito}  & \textbf{Range di appartenenza}  \\ \cline{1-2}
1 & Ottimale \\ \cline{1-2}
\end{tabular}
\caption{Numero di cambiamenti del periodo di Validazione}
\end{table}
