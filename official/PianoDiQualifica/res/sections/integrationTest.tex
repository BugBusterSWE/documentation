\newpage
%introduction
%write here
\subsection{Test di Integrazione}
Questa tipologia di test è utile alla verifica del corretto funzionamento delle componenti definite in sede di \SpecificaTecnica.
Verrà utilizzata la seguente notazione:
	\begin{center}
		TI-\textit{IDComponente}
	\end{center}
dove \textit{IdComponente} corrisponde al codice identificativo (crescente) del componente preso in esame
dal test.

I \textit{Verificatori} hanno stabilito di definire i test di integrazione attraverso un approccio  \textit{top-down}. 

%DIAGRAMMA?

L'approccio \textit{top-down} obbliga a stendere i test di integrazione partendo dai moduli
di livello più alto, i quali dunque vengono integrati per primi;
in questo modo le componenti di livello più basso dovranno quindi essere simulate tramite degli
stub. Grazie all'integrazione incrementale delle componenti del sistema, è immediato determinare
la componente che crea problemi, e le funzioni di livello superiore sono sottoposte a test precedentemente.

\begin{figure}[H]
      \begin{center}
        \includegraphics[width=12cm]{res/sections/DiagrammaBackend}
      \caption{Struttura ad alto livello del \textit{backend}}
      \end{center}
\end{figure}



%table of integrationTest
\begin{center}
  
  \begin{table}[H]
    \centering
    \begin{tabular}{ | >{\centering}p{3cm} | >{\centering}p{6cm} | >{\centering}p{1.5cm} | >{\centering}p{2cm} | }
      \textbf{Test Sistema} & \textbf{Descrizione} & \textbf{Stato} & \textbf{Requisito} \tabularnewline \hline
      

% example & bla & blabla & blablabla \tabularnewline \hline
TS-R1O 1 & Verificare che l'utente non autenticato possa creare un account per la propria \glossaryItem{Company}. & S & R1O 1 \tabularnewline \hline    %1

TS-R1O 1.1 & Verificare che l'utente non autenticato possa inserire il nome della propria \glossaryItem{Company} nel campo di testo dedicato, all'interno della \glossaryItem{procedura} di creazione dell'account della \glossaryItem{Company}. & S & R1O 1.1  \tabularnewline \hline   %2


TS-R1O 1.2 & Verificare che l'utente non autenticato possa inserire l'email della propria \glossaryItem{Company} nel campo di testo dedicato, all'interno della \glossaryItem{procedura} di creazione dell'account della \glossaryItem{Company}. & S & R1O 1.2 \tabularnewline \hline   %3


TS-R1O 1.3 & Verificare che l'utente non autenticato possa inserire la password della propria \glossaryItem{Company} nel campo di testo (offuscato) dedicato, all'interno della \glossaryItem{procedura} di creazione dell'account della \glossaryItem{Company}. & S & R1O 1.3 \tabularnewline \hline   %4

TS-R1O 2 & Verificare che l'utente non autenticato (dopo aver ricevuto un invito da parte dell'Owner, o di un suo delegato) riesca a registrarsi a \glossaryItem{MaaS} e ad accedere al proprio account & S & R1O 2 \tabularnewline \hline %5

TS-R1O 2.1 & Verificare che l'utente non autenticato (dopo aver ricevuto un invito da parte dell'Owner, o di un suo delegato) riesca a registrarsi a \glossaryItem{MaaS} tramite una password. & S & R1O 2.1 \tabularnewline \hline %6

TS-R1O 3 & Verificare che l'utente autenticato possa accedere all'editor ed eseguire una qualsiasi operazione su una specifica \glossaryItem{DSL}. & S & R1O 3 \tabularnewline \hline %13

TS-R1O 3.1 & Verificare che l'utente autenticato possa eseguire una \glossaryItem{DSL} in particolare (dopo l'accesso all'editor). & S & R1O 3.1 \tabularnewline \hline %14

TS-R1O 3.2 & Verificare che l'utente autenticato (con ruolo superiore a Member) possa modificare una specifica \glossaryItem{DSL} (dopo l'accesso all'editor). & S & R1O 3.2 \tabularnewline \hline %15

TS-R1O 3.3 & Verificare che l'utente autenticato (con ruolo superiore a Member) possa creare una specifica \glossaryItem{DSL} (dopo l'accesso all'editor). & S & R1O 3.3 \tabularnewline \hline %16

TS-R1O 3.4 & Verificare che l'utente autenticato (con ruolo superiore a Member) possa leggere una specifica \glossaryItem{DSL} (dopo l'accesso all'editor). & S & R1O 3.4 \tabularnewline \hline %17

TS-R1O 3.5 & Verificare che l'Admin possa aggiungere (oppure togliere) i permessi di scrittura (oppure esecuzione) ad un utente di una \glossaryItem{Company} su una specifica \glossaryItem{DSL}. & S & R1O 3.5 \tabularnewline \hline %18

TS-R1O 3.6 & Verificare che l'Owner di una \glossaryItem{Company} possa invitare un utente a registrarsi presso \glossaryItem{MaaS}. & S & R1O 3.6 \tabularnewline \hline %19   

TS-R1O 4 & Verificare che l'utente autenticato possa accedere alla pagina \glossaryItem{Dashboard}, visualizzarne il contenuto ed effettuare una delle operazioni permesse sugli elementi presenti. & S & R1O 4
\tabularnewline \hline %20

TS-R1O 4.1 & Verificare che l'utente autenticato possa visualizzare uno qualsiasi degli elementi della \glossaryItem{Dashboard} (Cell, \glossaryItem{Document} o \glossaryItem{Collection}). & S & R1O 4.1 \tabularnewline \hline %21

TS-R1O 4.2 & Verificare che l'utente autenticato possa effettuare una delle operazioni permesse su uno qualsiasi degli elementi della \glossaryItem{Dashboard} (Cell, \glossaryItem{Document} o \glossaryItem{Collection}). & S & R1O 4.2 \tabularnewline \hline %22  


TS-R1O 4.2.1 & Verificare che l'utente autenticato possa aggiungere, modificare o ordinare un valore in un elemento \glossaryItem{Cell}. & S & R1O 4.2.1 \tabularnewline \hline %23


TS-R1O 4.2.2 & Verificare che l'utente autenticato possa modificare o rimuovere un elemento \glossaryItem{Document}. & S & R1O 4.2.2 \tabularnewline \hline %24


TS-R1O 4.2.3 & Verificare che l'utente autenticato possa eseguire un Send mail (o Export) dalla pagina \glossaryItem{Document}. & S & R1O 4.2.3 \tabularnewline \hline %25

TS-R1O 4.2.4 & Verificare che i Documents all'interno di una \glossaryItem{Collection} siano ordinabili in base a uno dei loro campi. & S & R1O 4.2.4 \tabularnewline \hline
TS-R1O 4.2.5 & Verificare che l'utente autenticato debba poter rimuovere una \glossaryItem{Collection}. & S & R1O 4.2.5 \tabularnewline \hline
TS-R1O 4.2.6 & Verificare che l'utente autenticato debba poter eseguire un'azione (send mail/export) dalla pagina \glossaryItem{Collection}. & S & R1O 4.2.6 \tabularnewline \hline
TS-R1O 5 & Verificare che l'applicazione mostri al Super-Admin la pagina di gestione di tutte le \glossaryItem{Company}. & S & R1O 5 \tabularnewline \hline
TS-R1O 5.1 & Verificare che il Super-Admin possa aggiungere una nuova \glossaryItem{Company}. & S & R1O 5.1 \tabularnewline \hline
TS-R1O 5.2 & Verificare che il Super-Admin possa modificare i dati di una \glossaryItem{Company}. & S & R1O 5.2 \tabularnewline \hline
 

% example & bla & blabla & blablabla \tabularnewline \hline
% num test % desc % N.E & cod req
% TS-cod.req

TS-R1O 4.2.4 & Verificare che i Documents all'interno di una Collection siano ordinabili in base a uno dei loro campi. & N.E & R1O 4.2.4 \tabularnewline \hline

TS-R1O 4.2.5 & Verificare che l'utente autenticato debba poter rimuovere una Collection. & N.E & R1O 4.2.5 \tabularnewline \hline
TS-R1O 4.2.6 & Verificare che l'utente autenticato debba poter eseguire un'azione (send mail/export) dalla pagina Collection. & N.E & R1O 4.2.6 \tabularnewline \hline
TS-R1O 5 & Verificare che l'applicazione mostri al Super-Admin la pagina di gestione di tutte le Company. & N.E & R1O 5 \tabularnewline \hline
TS-R1O 5.1 & Verificare che il Super-Admin possa aggiungere una nuova Company. & N.E & R1O 5.1 \tabularnewline \hline
TS-R1O 5.2 & Verificare che il Super-Admin possa modificare i dati di una Company. & N.E & R1O 5.2 \tabularnewline \hline

%  \end{longtable} %problem with latex compiler. See issue #129
\end{center}

