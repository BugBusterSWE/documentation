\newpage
%introduction
%write here
\subsection{Test di Integrazione}
Questa tipologia di test è utile alla \glossaryItem{verifica} del corretto funzionamento delle componenti definite in sede di \SpecificaTecnica.
Verrà utilizzata la seguente notazione:
	\begin{center}
		TI-\textit{IDComponente}
	\end{center}
dove \textit{IdComponente} corrisponde al codice identificativo (crescente) del componente preso in esame
dal test.

I \textit{Verificatori} hanno stabilito di definire i test di integrazione attraverso un approccio  \textit{top-down}. 

%DIAGRAMMA?

L'approccio \textit{top-down} obbliga a stendere i test di integrazione partendo dai \glossaryItem{moduli}
di livello più alto, i quali dunque vengono integrati per primi;
in questo modo le componenti di livello più basso dovranno quindi essere simulate tramite degli
stub. Grazie all'integrazione incrementale delle componenti del sistema, è immediato determinare
la componente che crea problemi, e le funzioni di livello superiore sono sottoposte a test precedentemente.

\begin{figure}[H]
      \begin{center}
        \includegraphics[width=12cm]{res/sections/DiagrammaBackend}
      \caption{Struttura ad alto livello del \textit{backend}}
      \end{center}
\end{figure}

\begin{figure}[H]

        \centering
        \includegraphics[width=12cm]{res/sections/DiagrammaFrontend}
        \caption{Struttura ad alto livello del \textit{frontend}}
\end{figure}

%table of integrationTest
\begin{center}
  
  \begin{table}[H]
    \centering
    \begin{tabular}{ | >{\centering}p{3cm} | >{\centering}p{6cm} | >{\centering}p{1.5cm} | >{\centering}p{2cm} | }
      \textbf{Test Sistema} & \textbf{Descrizione} & \textbf{Stato} & \textbf{Requisito} \tabularnewline \hline
      
      % example & bla & blabla & blablabla \tabularnewline \hline

% example & bla & blabla & blablabla \tabularnewline \hline
TS-R1O 1 & & & R1O \tabularnewline \hline    %1
TS-R1O 1.1 & & & R1O 1.1 \tabularnewline \hline   %2
TS-R1O 1.2 & & & R1O 1.2 \tabularnewline \hline   %3
TS-R1O 1.3 & & & R1O 1.3 \tabularnewline \hline   %4

TS-R1O 2 & & & R10 2   

 

% example & bla & blabla & blablabla \tabularnewline \hline
% num test % desc % N.E & cod req
% TS-cod.req

TS-R1O 4.2.4 & Verificare che i Documents all'interno di una Collection siano ordinabili in base a uno dei loro campi. & N.E & R1O 4.2.4 \tabularnewline \hline

TS-R1O 4.2.5 & Verificare che l'utente autenticato debba poter rimuovere una Collection. & N.E & R1O 4.2.5 \tabularnewline \hline
TS-R1O 4.2.6 & Verificare che l'utente autenticato debba poter eseguire un'azione (send mail/export) dalla pagina Collection. & N.E & R1O 4.2.6 \tabularnewline \hline
TS-R1O 5 & Verificare che l'applicazione mostri al Super-Admin la pagina di gestione di tutte le Company. & N.E & R1O 5 \tabularnewline \hline
TS-R1O 5.1 & Verificare che il Super-Admin possa aggiungere una nuova Company. & N.E & R1O 5.1 \tabularnewline \hline
TS-R1O 5.2 & Verificare che il Super-Admin possa modificare i dati di una Company. & N.E & R1O 5.2 \tabularnewline \hline
TS-R1O 6 & Verificare che il sistema mantenga un'associazione consistente tra un utente e una Company. & N.E & R1O 6 \tabularnewline \hline
TS-R1O 7 & Verificare che il Super-Admin possa visualizzare in dettaglio il profilo di un utente di una Company. & N.E & R1O 7 \tabularnewline \hline
TS-R1O 7.1 & Verificare che il Super-Admin possa modificare i dati di un utente. & N.E & R1O 7.1 \tabularnewline \hline
TS-R1O 7.2 & Verificare che il Super-Admin possa eliminare un utente. & N.E & R1O 7.2 \tabularnewline \hline
TS-R1O 7.3 & Verificare che il Super-Admin possa creare a sua volta un altro Super-Admin. & N.E & R1O 7.3 \tabularnewline \hline
TS-R1O 8 & Verificare che l'Owner di una Company possa rimuovere un utente. & N.E & R1O 8 \tabularnewline \hline
TS-R1O 9 & Verificare che l'Owner di una Company possa inserire manualmente un nuovo utente presso MaaS. & N.E & R1O 9 \tabularnewline \hline
TS-R1O 10 & Verificare che il sistema permetta all'utente di mantenere salvato un DSL precedentemente creato. & N.E & R1O 10 \tabularnewline \hline
TS-R1D 11 & Verificare che il sistema permetta all'utente di scaricare da Terminale una specifica DSL in un formato leggibile dall'editor. & N.E & R1D 11 \tabularnewline \hline
TS-R1D 12 & Verificare che sia possibile offrire un'interfaccia grafica per la manipolazione del DSL. & N.E & R1D 12 \tabularnewline \hline
TS-R1D 12.1 & Verificare che l'utente attriverso l'interfaccia grafica possa eseguire l'invio del DSL definito. & N.E & R1D 12.1 \tabularnewline \hline
TS-R1D 12.2 & Verificare che l'utente attraverso l'interfaccia grafica visualizzi l'insieme delle DSL a cui può accedere. & N.E & R1D 12.2 \tabularnewline \hline
TS-R1D 13 & Verificare che l'utente debba poter manipolare la struttura del DSL attraverso una rappresentazione grafica. & N.E & R1D 13 \tabularnewline \hline
TS-R1D 13.1 & Verificare che la Collection debba avere una rappresentazione grafica visibile sul browser. & N.E & R1D 13.1 \tabularnewline \hline
TS-R1D 13.2 & Verificare che una funzione in JavaScript debba avere una rappresentazione grafica visibile sul browser. & N.E & R1D 13.2 \tabularnewline \hline

    \end{tabular}
  \end{table}
  
\end{center}

