\subsection{Codice}
\subsubsection{Progettazione di Dettaglio e Codifica}
Di seguito vengono riportati gli esiti delle misurazioni effettuate sulle classi prodotte durante il periodo di Progettazione di Dettaglio e Codifica.
\paragraph{Dimensioni del prodotto software} \mbox{} \\
\begin{table}[H]
\centering
\begin{tabular}{|l|l|}
\cline{1-2}
\textbf{Backend} & 35 \\ \cline{1-2}
\textbf{Frontend} & 0 \\ \cline{1-2}
\textbf{Totale} & 35 \\ \cline{1-2}
\end{tabular}
\caption{KLOCS}
\end{table}

\paragraph{Complessità ciclomatica} \mbox{} \\
\subparagraph{Backend} \mbox{} \\
\begin{center}
\begin{longtable}{| >{\centering}p{7cm} | >{\centering}p{1.8cm} |}
\textbf{Classe} & \textbf{Valore} \tabularnewline \hline 
Models::Model & 1 \tabularnewline \hline
Models::MongooseConnection & 3 \tabularnewline \hline %
\caption{Misurazioni della complessità ciclomatica per le classi del backend}
\end{longtable}
\end{center}
\textbf{Esito}: Superato.

\subparagraph{Frontend} \mbox{} \\
\begin{center}
\begin{longtable}{| >{\centering}p{7cm} | >{\centering}p{1.8cm} |}
\textbf{Classe} & \textbf{Valore} \tabularnewline \hline %

\caption{Misurazioni della complessità ciclomatica per le classi del frontend}
\end{longtable}
\end{center}
\textbf{Esito}: Superato.

\paragraph{Number of methods} \mbox{} \\
\subparagraph{Backend} \mbox{} \\
\begin{center}
\begin{longtable}{| >{\centering}p{7cm} | >{\centering}p{1.8cm} |}
\textbf{Classe} & \textbf{Valore} \tabularnewline \hline 
Models::Model & 2 \tabularnewline \hline
Models::MongooseConnection & 4 \tabularnewline \hline %
\caption{Numero di metodi delle classi del backend}
\end{longtable}
\end{center}
\textbf{Esito}: Superato.

\subparagraph{Frontend} \mbox{} \\
\begin{center}
\begin{longtable}{| >{\centering}p{7cm} | >{\centering}p{1.8cm} |}
\textbf{Classe} & \textbf{Valore} \tabularnewline \hline %

\caption{Numero di metodi delle classi del frontend}
\end{longtable}
\end{center}
\textbf{Esito}: Superato.

\paragraph{Variabili non utilizzate e/o non definite} \mbox{} \\
\begin{table}[H]
\centering
\begin{tabular}{|l|l|}
\cline{1-2}
\textbf{Backend} & 0 \\ \cline{1-2}
\textbf{Frontend} & 0 \\ \cline{1-2}
\textbf{Totale} & 0 \\ \cline{1-2}
\end{tabular}
\caption{Variabili non utilizzate e/o non definite}
\end{table}
\textbf{Esito}: Superato.

\paragraph{Numero di parametri per metodo} \mbox{} \\
\subparagraph{Backend} \mbox{} \\
\begin{center}
\begin{longtable}{| >{\centering}p{7cm} | >{\centering}p{1.8cm} | >{\centering}p{1.8cm} | >{\centering}p{1.8cm} |}
\textbf{Classe} & \textbf{Minimo} & \textbf{Massimo} & \textbf{Medio} \tabularnewline \hline 
Models::Model & 0 & 0 & 0 \tabularnewline \hline
Models::MongooseConnection & 0 & 1 & 0.25 \tabularnewline \hline %
\caption{Numero di parametri per metodo delle classi del backend}
\end{longtable}
\end{center}
\textbf{Esito}: Superato.

\subparagraph{Frontend} \mbox{} \\
\begin{center}
\begin{longtable}{| >{\centering}p{7cm} | >{\centering}p{1.8cm} | >{\centering}p{1.8cm} | >{\centering}p{1.8cm} |}
\textbf{Classe} & \textbf{Minimo} & \textbf{Massimo} & \textbf{Medio} \tabularnewline \hline %

\caption{Numero di parametri per metodo delle classi del frontend}
\end{longtable}
\end{center}
\textbf{Esito}: Superato.