\newpage
\section{Visione generale della strategia di verifica}
La volontà del \glossaryItem{team} è quella di automatizzare il più possibile il lavoro di \glossaryItem{verifica}. Saranno quindi utilizzati dei \glossaryItem{tool}s adeguatamente configurati, con lo scopo di avere un riscontro affidabile e quantitativo che permetta di assicurare il grado di \glossaryItem{qualità} voluto. 

\subsection{Qualit\'a di prodotto}
\subsubsection{Obiettivi}
Lo standard ISO/IEC 9126\footnote{Si veda appendice \ref{AppQualitaProdotto} per approfondimenti} classifica la \glossaryItem{qualità} del software e definisce delle metriche per la sua misurazione. Il \glossaryItem{team} BugBusters ha scelto di utilizzare queste metriche, al fine di assicurare la \glossaryItem{qualità} del prodotto finale.

\subsubsection{Procedure}
Il controllo per la \glossaryItem{qualità} del prodotto definisce i seguenti processi:
\begin{itemize}
	\item \textbf{SQA} (\textbf{S}oftware \textbf{Q}uality \textbf{A}ssurance): si occupa di assicurare che i processi siano implementati secondo quanto pianificato e che siano forniti sistemi di misurazione dei processi;
	\item \textbf{\glossaryItem{verifica}}: si occupa di accertare che l'esecuzione dei processi non abbia introdotto degli errori, e accerta il rispetto delle regole, delle convenzioni e delle procedure;
	\item \textbf{\glossaryItem{validazione}}: si occupa di accertare che i prodotti realizzati siano conformi alle attese.
\end{itemize}

\subsubsection{Metriche}\label{MetricheSoftware}
\paragraph{Software}\mbox{}\\
Di seguito vengono indicate le metriche riguardanti il software che il \glossaryItem{team} BugBusters ha deciso di adottare.

\textbf{Dimensioni del prodotto software}\\

Rappresenta le dimensioni del prodotto software; generalmente \`e misurata in termini di migliaia di linee di codice (\glossaryItem{KLOC}s, Thousands \textbf{L}ine \textbf{O}f \textbf{C}ode), ma da alcuni anni è stata introdotto una nuova misura, legata al numero di funzionalità offerte, e quindi dal valore che il prodotto ha per l'utente. Questa seconda misurazione è espressa come numero di punti funzione (\glossaryItem{FP}s, \textbf{F}unction \textbf{P}oints). \\
\begin{center}
	\textit{Dimensioni del prodotto software} = \textit{\glossaryItem{KLOC}s} \\
	\textit{Dimensioni del prodotto software} = \textit{\glossaryItem{FP}s} 
\end{center}

\textbf{Complessità ciclomatica}\\

Permette di misurare la complessità del flusso di controllo di una componente (procedura, metodo, classe o modulo) calcolando tutti i possibili cammini indipendenti del grafo del flusso di controllo. Tale grafo è composto dai seguenti elementi:
\begin{itemize}
	\item i nodi corrispondono a gruppi indivisibili di comandi;
	\item gli archi collegano due nodi se e solo se il secondo gruppo di istruzioni può essere eseguito immediatamente dopo il primo.
\end{itemize}
Si vuole utilizzare tale metrica per limitare la complessità durante la fase di sviluppo e per determinare il numero di casi di test necessari per la \glossaryItem{verifica}, in quanto permette di dare un limite superiore al numero di test necessari per raggiungere la massima copertura della componente analizzata \footnote{Stein, C., G. Cox and L. Etzkorn, 2005. Exploring the Relationship between Cohesion and Complexity. J. Comput. Sci., 1: 137-144.}.
\begin{center}
	\textit{Complessità ciclomatica v(G)} = \textit{e} - \textit{n} + 2 * \textit{p}
\end{center}
Dove:
\begin{itemize}
	\item \textbf{e} è il numero di archi del grafo \textit{G};
	\item \textbf{n} è il numero di nodi del grafo \textit{G};
	\item \textbf{p} è il numero di componenti connesse del grafo \textit{G}.
\end{itemize}
\textbf{Range utilizzati}:
\begin{itemize}
	\item Range di accettazione : [0 - 15]
	\item Range ottimale: [0 - 10]
\end{itemize}

\textbf{Number of methods}\\

Rappresenta la media di occorrenze di metodi per \glossaryItem{package}: un \glossaryItem{package}, infatti, non dovrebbe contenere un numero eccessivo di metodi. Valori troppo elevati indicano la necessità di una migliore decomposizione del \glossaryItem{package}. \\

\textbf{Range utilizzati}:

\begin{itemize}
	\item Range di accettazione : [3 - 10]
	\item Range ottimale: [0 - 7]
\end{itemize}


\textbf{Bugs per linee di codice}\\

Rappresenta il numero di \glossaryItem{bug} trovati in un insieme di linee di codice: con la crescita del prodotto è utile misurare il rapporto tra i difetti trovati e la dimensione del prodotto (in termini di \glossaryItem{KLOC}s), tenendo conto del fatto che un incremento del sorgente comporta un aumento della possibilità di inserimento di un errore. \\

\textbf{Range utilizzati}:

\begin{itemize}
	\item Range di accettazione : [0 - 60]
	\item Range ottimale: [0 - 20]
\end{itemize}
Misurati sull'intera dimensione in \glossaryItem{KLOC}s del prodotto.\\

\textbf{Variabili non utilizzate e/o non definite}\\

Rappresenta il numero di variabili che vengono definite, ma non utilizzate, o viceversa. Questo viene considerato \glossaryItem{pollution}, e pertanto considerato inaccettabile. La misurazione avviene mediante un'analisi dell'\textbf{\glossaryItem{AST}} (\textbf{A}bstract \textbf{S}yntax \textbf{T}ree). \\

\textbf{Range utilizzati}:
\begin{itemize}
	\item Range di accettazione : [0 - 0]
	\item Range ottimale: [0 - 0]
\end{itemize}


\textbf{Numero di parametri per metodo}\\

Rappresenta il numero di parametri formali di un metodo. Un metodo con troppi parametri formali risulta complesso e poco mantenibile; pertanto è necessario che tale numero sia contenuto. \\

\textbf{Range utilizzati}:
\begin{itemize}
	\item Range di accettazione : [0 - 6]
	\item Range ottimale: [0 - 4]
\end{itemize}

\textbf{Numero di funzioni d'interfaccia per \glossaryItem{package}}\\

Rappresenta il numero di funzioni che il \glossaryItem{package} mette a disposizione dei suoi utenti. \\

\textbf{Range utilizzati}:
\begin{itemize}
	\item Range di accettazione : [0 - 20]
	\item Range ottimale: [1 - 10]
\end{itemize}

\paragraph{Documentazione}\mbox{}\\
\hypertarget{metriche_documenti}{}
La \glossaryItem{qualità} di un documento dipende fortemente dai suoi contenuti e dalla sua leggibilità. Tuttavia, a causa della scarsa esperienza dei membri del \glossaryItem{team}, valutarla risulta molto difficile. Si è pertanto scelto di affidarsi a parametri oggettivi e facilmente misurabili attraverso dei \glossaryItem{tool} automatici.\\

\textbf{Indice di leggibilità}\\

Per valutare la leggibilità di un documento si è scelto di usare l'indice \glossaryItem{Gulpease}, studiato appositamente per la lingua italiana.\\
Rispetto ad altri indici, questo ha il vantaggio di basarsi sulla lunghezza delle parole e non sulla loro divisione in sillabe, semplificando la valutazione automatica.\\
L'indice di \glossaryItem{Gulpease} è calcolato secondo la seguente formula:
\begin{center}
\begin{math}
	89 + 
		\dfrac	{300 * (\textit{numero delle frasi}) - 10 * (\textit{numero delle lettere})}
				{\textit{numero delle parole}}
\end{math}
\end{center}
I risultati sono compresi tra 0 e 100, dove un valore alto indica alta leggibilità, e viceversa. \\
In generale risulta che testi con indice:
\begin{itemize}
	\item \textbf{inferiore a 80} risultano difficili da leggere per chi ha la licenza elementare;
	\item \textbf{inferiore a 60} risultano difficili da leggere per chi ha la licenza media;
	\item \textbf{inferiore a 40} risultano difficili da leggere per chi ha la licenza superiore;
\end{itemize}
\textbf{Range utilizzati}:
\begin{itemize}
	\item Range di accettazione : [40 - 100]
	\item Range ottimale: [50 - 100]
\end{itemize}
Questi valori sono stati fissati tenendo conto delle soglie di cui sopra, e soprattutto del fatto che la documentazione del presente \glossaryItem{progetto} è destinata a persone sufficientemente preparate, competenti ed istruite.



\subsection{Qualit\'a di processo}
\subsubsection{Obiettivi}
Molto spesso prodotti scadenti derivano da processi scadenti. Per questo motivo, e per le seguenti ragioni, assicurare la \glossaryItem{qualità} dei processi è un obiettivo primario per il \glossaryItem{team} BugBusters:
\begin{itemize}
	\item aiuta ad ottimizzare l'uso di risorse;
	\item permette di contenere i costi;
	\item migliora la stima dei rischi e degli impegni.
\end{itemize}
Un \glossaryItem{processo} dovrebbe essere in grado di migliorare costantemente le proprie performance, che devono quindi essere costantemente misurabili e misurate. Inoltre, le attività di ciascun \glossaryItem{processo} e i costi associati devono essere in linea con quanto indicato nel \PianoDiProgetto. \\
Si è dunque deciso di perseguire la \glossaryItem{qualità} servendosi dei seguenti modelli: 
\begin{itemize}
	\item \glossaryItem{SPICE}\footnote{Si veda appendice \ref{AppQualitaProcessi} per approfondimenti} (\textbf{S}oftware \textbf{P}rocess \textbf{I}mprovement and \textbf{C}apability d\textbf{E}termination): definito nello standard ISO/IEC 15504, per poter valutare in modo oggettivo i processi dal punto di vista della maturità;
	\item \glossaryItem{PDCA}\footnote{Si veda appendice \ref{AppPDCA} per approfondimenti} (\textbf{P}lan \textbf{D}o \textbf{C}hek \textbf{A}ct): per il controllo delle attività di \glossaryItem{processo} ripetibili e misurabili e per la manutenibilità dei processi stessi incrementandone la \glossaryItem{qualità}.
\end{itemize}

\subsubsection{Procedure}
La pianificazione delle attività volte al miglioramento continuo dei processi sono descritte nel \PianoDiProgetto. Le linee guida per la gestione della \glossaryItem{qualità} del \glossaryItem{processo}, invece, seguono il modello PDCA e descrivono come devono essere attuate le procedure di controllo:
\begin{itemize}
	\item la pianificazione deve essere dettagliata, e le attività pianificate devono essere monitorate;
	\item le risorse necessarie per conseguire gli obiettivi devono essere definite;
	\item il miglioramento della \glossaryItem{qualità} del \glossaryItem{processo} deve essere verificato attraverso l'utilizzo di apposite metriche, che verranno descritte in seguito.
\end{itemize}

\subsubsection{Metriche}\label{MetricheProc}
\hypertarget{metriche_processi}{}
Le seguenti metriche rappresentano un indicatore volto a monitorare i tempi e i costi associati al \glossaryItem{progetto}. Sono metriche di tipo \glossaryItem{consuntivo} che danno un riscontro immediato sullo stato attuale: gli indici verranno valutati dal \textit{Responsabile} e, se necessario, verranno presi provvedimenti per sistemare la situazione.\\

\textbf{Schedule Variance}\\

Indica se si è in linea, in anticipo o in ritardo rispetto alla schedulazione delle attività di \glossaryItem{progetto}
pianificate.
\begin{center}
	\textit{Schedule Variance} = \textit{EV} - \textit{PV} 
\end{center}
Dove:
\begin{itemize}
	\item \textbf{EV}: indica il valore delle attivit\`a realizzate alla data corrente;
	\item \textbf{PV}: indica il costo pianificato per realizzare le attività di \glossaryItem{progetto} alla data corrente.
\end{itemize}
\`E un indicatore di efficacia soprattutto nei confronti del cliente. Se \begin{math}{SV > 0}\end{math} significa che il \glossaryItem{progetto} sta procedendo con maggior velocit\`a rispetto a quanto pianificato, viceversa se negativo. Alla fine del \glossaryItem{progetto} questo indice assumer\`a il valore 0, perch\`e in quel momento tutte le attivit\`a saranno state realizzate.\\

\textbf{Budget Variance}\\

Indica se alla data corrente si \`e speso di pi\`u o di meno rispetto a quanto previsto.\\
\begin{center}
	\textit{Budget Variance} = \textit{EV} - \textit{AC}
\end{center}
Dove:
\begin{itemize}
	\item \textbf{EV}: indica il costo pianificato per realizzare le attività di \glossaryItem{progetto} alla data corrente;
	\item \textbf{AC}: indica il costo effettivo sostenuto alla data corrente. \`E un indicatore con
un valore unicamente contabile e finanziario. Se BV > 0 significa che il \glossaryItem{progetto} sta spendendo
il proprio budget con minor velocità di quanto pianificato, viceversa se negativo.
\end{itemize}

\textbf{Produttività}\\

Rappresenta la produttività media delle risorse impiegate nelle diverse fasi del \glossaryItem{progetto}. Questa metrica può dunque riferirsi al \glossaryItem{progetto} in senso generale, oppure ad una specifica fase.\\ \`E utilizzata per valutare lo sforzo richiesto per lo sviluppo del \glossaryItem{progetto}, a fronte delle sue dimensioni.
\begin{center}
\begin{math}
	\textit{Produttività} = \frac	{\textit{Quantità di output ottenuto}}
									{\textit{Quantità di input utilizzato}}
\end{math}
\end{center}
Per quanto detto, i parametri "input" e "output" assumono valori diversi in base ai processi, e alle relative attività, cui la formula viene applicata:
\begin{itemize}
	\item \textbf{documentazione}:
	\begin{center} 
		\begin{math}
			\dfrac	{\textit{ore necessarie alla scrittura della documentazione}}
					{\textit{numero di parole scritte}}
		\end{math}
	\end{center}	
	\item \textbf{codifica}:
	\begin{center} 
		\begin{math}
			\dfrac	{\textit{ore necessarie alla scrittura del codice}}
					{\textit{numero di linee di codice scritte}}
		\end{math}
	\end{center}	
	\item \textbf{\glossaryItem{verifica}}:
	\begin{center} 
		\begin{math}
			\dfrac	{\textit{ore necessarie alle attivit\`a di \glossaryItem{verifica}}}
					{\textit{numero di anomalie rilevate}}
		\end{math}
	\end{center}		
	\item \textbf{\glossaryItem{validazione}}: 
	\begin{center}
		\begin{math}
			\dfrac	{\textit{numero di test di accettazione}}
					{\textit{numero di test di accettazione superati}}
		\end{math}
	\end{center}		
\end{itemize}

\textbf{Range utilizzati}: \\

I valori che si ottengono con questa metrica sono difficilmente valutabili in modo assoluto, ma assumono importanza se confrontati in momenti diversi del \glossaryItem{progetto}.

\textbf{Numero di cambiamenti apportati}\\

Rappresenta il numero di modifiche apportate al \glossaryItem{progetto} in corso d'opera. Queste modifiche possono riguardare i requisiti rilevati, le funzionalità, la progettazione, il codice e i manuali o documenti scritti, e sono causate, molto spesso, da un cambiamento dei requisiti, sia esso voluto dal \glossaryItem{committente} o derivato da un'errata interpretazione del fornitore. Misurare il numero di modifiche apportate al \glossaryItem{progetto} è dunque fondamentale al fine di valutare gli impatti sui tempi di realizzazione e sui costi.
\begin{center}
	\textit{Numero di cambiamenti apportati} = \textit{Numero di modifiche}
\end{center}
\textbf{Range utilizzati}:
\begin{itemize}
	\item Range di accettazione : [0 - 20]
	\item Range ottimale: [0 - 10]
\end{itemize}

\textbf{Copertura dei test}\\

Rappresenta il livello di copertura che i test eseguiti forniscono rispetto alle funzionalità del prodotto. \\
\begin{center}
\begin{math}
	\textit{Copertura dei test} = \frac{\textit{Numero di funzionalità \footnote{Si veda appendice \ref{appendice-qualitaDelProdottotestate}}}}{\textit{Numero totoale delle funzionalità disponibili}} * 100.
\end{math}
\end{center}
\textbf{Range utilizzati}:
\begin{itemize}
	\item Range di accettazione : [70 - 100]
	\item Range ottimale: [80 - 100]
\end{itemize}









\subsection{Strategia}
Nel \PianoDiProgetto vengono fissate delle scadenze che devono necessariamente essere rispettate: è dunque necessario definire un'efficace strategia di qualifica. I controlli saranno effettuati in maniera automatica secondo quanto previsto nelle \NormeDiProgetto. 

\subsection{Responsabilità}
La responsabilità della \glossaryItem{verifica} viene affidata al \textit{Responsabile} e ai \textit{Verificatori} secondo quanto previsto nel \PianoDiProgetto.

\subsection{Risorse}
Vengono consumati due tipi di risorse:
\begin{itemize}
	\item \textbf{umane}: in particolare il \textit{Responsabile} e il \textit{Verificatore}; le ore impiegate vengono contabilizzate e messe a calendario secondo quanto previsto dal \PianoDiProgetto e dalle \href{http://www.math.unipd.it/~tullio/IS-1/2015/Dispense/PD01.pdf}{regole} del \glossaryItem{progetto} didattico. Ai fini della qualifica, tuttavia, si può tralasciare l'aspetto economico, in quanto esso non rientra nel dominio del presente documento;
		\item \textbf{tecnologiche}: i \glossaryItem{tool} utilizzati per il controllo della \glossaryItem{qualità}. Le operazioni effettuate  consumeranno unità di calcolo considerate a costo nullo, in quanto le elaborazioni verranno effettuate su macchine per le quali non è richiesto nessun contributo e per un tempo non degno di nota.
\end{itemize}
