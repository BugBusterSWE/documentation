\section{Analisi dei rischi} 
Sezione in cui vengono elencati i \glossaryItem{rischi} incontrati durante lo svolgimento del progetto; vengono indicate inoltre le tecniche e gli strumenti adottati per la mitigazione o la rimozione dei rischi. Inoltre \`e presente la descrizione di ogni elemento informativo che il gruppo ha trovato sufficiente per la categorizzazione di un \glossaryItem{rischio}.
Ad ogni \glossaryItem{rischio} viene assegnato un identificativo nel formato seguente: 

\begin{center}
\textit{[codice gruppo][codice probabilit\`a \glossaryItem{rischio}][indice numerico crescente]}  
\end{center}

fornendo al lettore un'immediata comprensione della natura del \glossaryItem{rischio}.

Il \textit{codice gruppo} indica l'appartentenza ad una delle seguenti categorie:
\begin{itemize}
\item \textbf{T} - Rischi tecnologici: guasti o malfunzionamenti \textit{hardware};
\item \textbf{P} - Rischi sulle persone: riguardanti i membri del gruppo;
\item \textbf{M} - Rischi organizzativi: derivanti dall'ambiente e dalle necessit\`a organizzative;
\item \textbf{S} - Rischi sugli strumenti \textit{software}: problemi sul software impiegato a supporto del progetto;
\item \textbf{R} - Rischi sui requisiti: mal comprensione, mancanza o necessit\`a di aggiunta di nuovi requisiti;
\item \textbf{V} - Rischi sulle stime: sottostima dei tempi e dei costi necessari.
\end{itemize}

Il \textit{codice probabilit\`a \glossaryItem{rischio}} \`e associato alle seguenti probabilit\`a di \glossaryItem{rischio}: 
\begin{itemize}
\item \textbf{L} - Molto Bassa;
\item \textbf{S} - Bassa;
\item \textbf{M} - Media;
\item \textbf{H} - Alta.
\end{itemize}

L' \textit{indice numerico crescente} indica la posizione all'interno della sua categoria. 

\`E anche stato stabilito un \textit{Effetto} che il verificarsi di quel \glossaryItem{rischio} potrebbe causare al \glossaryItem{progetto}. L'\textit{Effetto} non viene indicato nell'identificativo del \glossaryItem{rischio}, in quanto criterio di ricerca meno rilevante. La scala degli effetti individuata è la seguente:
\begin{itemize}
\item Tollerabili;
\item Seri;
\item Catastrofici.
\end{itemize}

L'analisi dei rischi si suddivide in quattro parti:
\begin{itemize}
\item \textit{Identificazione}: breve descrizione del \glossaryItem{rischio} analizzato;
\item \textit{Probabilit\`a}': probabilit\`a del verificarsi del \glossaryItem{rischio};
\item \textit{Effetti}: tipologia dell'effetto;
\item \textit{Pianificazione}: la \glossaryItem{procedura} che il gruppo adotta in caso si verifichi il \glossaryItem{rischio};
\item \textit{Controllo}: metodo per la \glossaryItem{verifica} del successo della pianificazione.
\end{itemize}

\subsection{Rischi tecnologici}
\subsubsection{Guasto \textit{hardware} - TS1}
Ogni membro del gruppo \`e provvisto di un computer portatile o fisso in cui svolgere il proprio lavoro. Il \glossaryItem{rischio} potrebbe derivare da un guasto al computer di un elemento del gruppo, evento che gli renderebbe impossibile il proseguimento dei compiti assegnatigli.
\begin{enumerate}
\item \textit{Probabilit\`a}: Bassa;
\item \textit{Effetti}: Tollerabile; 
\item \textit{Pianificazione}: l'Universit\`a di Padova mette a disposizione dei laboratori informatici utilizzabili
in caso di necessit\`a;
\item \textit{Controllo}: il numero dei computer rimane invariato durante tutto lo sviluppo del software, quindi il \glossaryItem{rischio} resta invariato.
\end{enumerate}

\subsection{Rischi sulle persone}
\subsubsection{Problemi dei componenti del gruppo - PM1}
Ogni membro del gruppo ha impegni personali di vario tipo, dunque non potr\`a essere impegnato nel \glossaryItem{progetto} secondo degli orari fissi. Sono inoltre da prendere in considerazione periodi di assenza prolungati per motivi lavorativi o di salute.
\begin{enumerate}
\item \textit{Probabilit\`a}: Media;
\item \textit{Effetti}: Tollerabile;
\item \textit{Pianificazione}: nel caso in cui un elemento del gruppo sia impossibilitato a svolgere per un periodo limitato i propri \glossaryItem{task}, il responsabile provveder\`a a riassegnare i task ad altri membri;
\item \textit{Controllo}: utilizzo del calendario di gruppo per individuare le fasi critiche.
\end{enumerate}

\subsubsection{Problemi tra i componenti del gruppo - PM2}
Per tutti i componenti questo \glossaryItem{progetto} costituisce la prima esperienza con un gruppo numeroso di persone ed un lungo periodo di lavoro. La probabilit\`a di insorgenza di conflitti aumenta con la crescita del lavoro e l'avvicinamento delle date di consegna.
\begin{enumerate}
\item \textit{Probabilit\`a}: Media;
\item \textit{Effetti}: Seri;
\item \textit{Pianificazione}: nel caso di forte contrasto, sar\`a compito del \textit{Responsabile di \glossaryItem{progetto}} mediare al fine di risolvere il conflitto;
\item \textit{Controllo}: l'\textit{Amministratore} ha il compito di mantenere un clima cooperativo all'interno del gruppo.
\end{enumerate} 

\subsection{Scarsa conoscenza delle tecnologie - PA3}
Il \glossaryItem{progetto} \`e ad ampio spettro, quindi sicuramente si dovranno usare tecnologie non trattate durante il percorso di studi o affrontate individualmente.
\begin{enumerate}
\item \textit{Probabilit\`a}: Alta;
\item \textit{Effetti}: Seri;
\item \textit{Pianificazione}: ogni membro del gruppo \`e tenuto a studiare le tecnologie coinvolte nello sviluppo. Il proponente ha offerto la propria disponibilit\`a per la discussione degli argomenti pi\`u complessi;
\item \textit{Controllo}: il \textit{Responsabile} ha il compito di verificare il grado di conoscenza sulle tecnologie acquisito dai singoli membri del gruppo.
\end{enumerate}

\subsection{Rischi organizzativi}
\subsubsection{Rotazione dei ruoli - MS1}
Il cambio di ruolo pu\`o creare difficolt\`a ai componenti del gruppo, a causa della profonda modifica nelle responsabilit\`a e nelle competenze associate al ruolo da ricoprire. Ne potrebbe conseguire un periodo di abbassamento dell'\glossaryItem{efficacia} complessiva del gruppo.
\begin{enumerate}
\item \textit{Probabilit\`a}: Bassa;
\item \textit{Effetti}: Tollerabile;
\item \textit{Pianificazione}: la rotazione dei ruoli viene stabilita prima di avviare i lavori, perci\`o ogni membro del gruppo ha la possibilit\`a di studiare preventivamente i compiti e le metodologie del nuovo ruolo;
\item \textit{Controllo}: il \textit{Responsabile} \glossaryItem{verifica} che ogni membro del gruppo ricopra tutti i ruoli previsti dalle \textit{Norme di Progetto}.
\end{enumerate}

\subsection{Rischi sugli strumenti software}
\subsubsection{Piattaforme fuori servizio - SL1}
Le piattaforme coinvolte sono \textit{Teamwork} e \textit{GitHub}.
\begin{enumerate}
\item \textit{Probabilit\`a}: Molto bassa;
\item \textit{Effetti}: Catastrofici;
\item \textit{Pianificazione}: suddivisione delle piattaforme coinvolte.
  \begin{itemize}
    \item \textit{Teamwork} dichiara di appoggiarsi ai servizi di \textit{backup} offerti da \textit{Amazon};
    \item \textit{GitHub} dichiara di effettuare backup su tre differenti server, di cui uno fuori sede. All'indirizzo \href{https://help.github.com/articles/github-security/}{https://help.github.com/articles/github-security/} \`e possibile trovare tutte le informazioni rilasciate da \textit{GitHub} sulle misure di sicurezza adottate. Inoltre, ogni elemento del gruppo ha un copia in locale del \glossaryItem{progetto}, consentendo un recupero parziale o totale.
  \end{itemize}
\item \textit{Controllo}: ci si affida alle misure di sicurezza adottate dai servizi utilizzati.
\end{enumerate}

\subsubsection{Cambio di un programma - SS2}
Dati l'inesperienza nella gestione di un gruppo cos\`i numeroso e nell'utilizzo dei software adottati nel suo supporto, non si pu\`o escludere il passaggio ad un software considerato migliore di quello adottato fino ad un certo momento.
\begin{enumerate}
\item \textit{Probabilit\`a}: Bassa;
\item \textit{Effetti}: Seri;
\item \textit{Pianificazione}: prima dell'effettiva sostituzione, l'\textit{Amministratore} procede alla \glossaryItem{verifica} del soddisfacimento dei seguenti requisiti da parte del nuovo software:
  \begin{itemize}
    \item Retrocompatibile - tutto il materiale precedentemente prodotto deve essere rielaborato con il nuovo programma;
    \item Migliorante - il nuovo programma adottato dovr\`a consentire di svolgere lo stesso lavoro in tempo minori e/o con una \glossaryItem{qualit\`a} superiore.
  \end{itemize}
\item \textit{Controllo}: il \textit{Verificatore} controller\`a che il materiale prodotto con il nuovo software sia conforme ai criteri attesi.
\end{enumerate}

\subsection{Rischi sui requisiti}
\subsubsection{Comprensione errata dei requisiti - RM1}
Data l'inesperienza dei componenti del gruppo nell'attivit\`a di analisi dei requisiti, \`e possibile che un'errata comprensione dei requisiti comporti un'offerta non conforme alle richieste.
\begin{enumerate}
\item \textit{Probabilit\`a}: Media;
\item \textit{Effetti}: Seri;
\item \textit{Pianificazione}: i membri del gruppo sono tenuti a colmare le lacune sui fondamenti dell'analisi dei requisiti;
\item \textit{Controllo}: in caso di dubbi rivolgersi agli altri membri del gruppo per chiarimenti o contattare il Prof. Riccardo Cardin.
\end{enumerate}

\subsubsection{Modifica dei requisiti - RS2}
Dagli incontri con il proponente \`e emersa la possibilit\`a di modificare i requisiti richiesti nel capitolato.
\begin{enumerate}
\item \textit{Probabilit\`a}: Bassa;
\item \textit{Effetti}: Seri;
\item \textit{Pianificazione}: la modifica dei requisiti deve svolgersi precedentemente alla prima data di revisione del \glossaryItem{progetto}, in modo da non rallentare o bloccare i \textit{Progettisti} durante la stesura della \textit{Specifica Tecnica}. Un ulteriore controllo verr\`a effettuato prima dell'entrata nell'attivit\`a codifica;
\item \textit{Controllo}: lavorando a stretto contatto con il proponente si avr\`a una visione d'insieme pi\`u chiara, mitigando questo problema.
\end{enumerate}

\subsection{Rischi sulle stime}
\subsubsection{Sottostima dei tempi necessari}
Data l'inesperienza dei membri del gruppo nella pianificazione di \glossaryItem{progetto} e l'attuazione della stessa su un arco di tempo medio-lungo, la sottostima dei tempi necessari alla realizzazione del \glossaryItem{progetto} si presenta come \glossaryItem{rischio} concreto.
\begin{enumerate}
\item \textit{Probabilit\`a}: Alta;
\item \textit{Effetti}: Tollerabili;
\item \textit{Pianificazione}: i gruppi di attivit\`a pianificate, relative alle scadenze fissate dal committente, non ricoprono tutto l'arco di tempo a disposizione, lasciando uno \glossaryItem{Slack} prima di ogni consegna;
\item \textit{Controllo}: grazie ad una \glossaryItem{Dashboard}, il \textit{Responsabile} pu\`o verificare lo stato di avanzamento delle attivit\`a.
\end{enumerate}

