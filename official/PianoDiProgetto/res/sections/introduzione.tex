\section{Introduzione}
\subsection{Scopo del documento}
Questo documento ha l’obiettivo di identificare e dettagliare la pianificazione del gruppo
relativa allo sviluppo del \glossaryItem{progetto}. La ripartizione del carico di lavoro e di responsabilit\`a tra i
componenti del gruppo ed il conto economico preventivo sono oggetto di primo piano in tale documento.

\subsection{Scopo del prodotto}
Lo scopo del \glossaryItem{progetto} \`e la realizzazione di un \glossaryItem{Software as a Service} basato sul prodotto \glossaryItem{MaaP}. Il nome del prodotto atteso \`e \glossaryItem{MaaS}, ossia \glossaryItem{MongoDB} as an admin Service.

\subsection{Glossario}
Al fine di evitare ogni ambiguit\`a relativa al linguaggio impiegato nei documenti viene fornito il \Glossario, contenente la definizione dei termini marcati con una \textit{G} pedice.

\subsection{Riferimenti}
\subsubsection{Normativi}
\begin{itemize}
\item \NormeDiProgetto;

\item Capitolato d'appalto C4: \glossaryItem{MaaS}: \glossaryItem{MongoDB} as an admin Service: \\ 
\url{http://www.math.unipd.it/~tullio/IS-1/2015/Progetto/C4p.pdf}

\item Vincoli sull’organigramma del gruppo e sull’offerta tecnico-economica: \\
\url{http://www.math.unipd.it/~tullio/IS-1/2015/Progetto/PD01b.html}

\end{itemize}
\subsubsection{Informativi}
\begin{itemize}
\item IAN SOMMERVILLE, Software Engineering, Part 4: Software Management, 9th edition, Boston, Pearson Education, 2011;
\item The Guide to the Software Engineering Body of Knowledge V3 (SWEBOK).
\end{itemize}

\subsection{Ciclo di vita}
L’interesse del \glossaryItem{committente} è limitato al segmento di ciclo di vita che va dall’analisi dei requisiti al
rilascio del prodotto, escludendo dunque la successiva manutenzione ed il ritiro. Il modello di ciclo di
vita scelto \`e quello incrementale, ritenuto preferibile per le seguenti ragioni:
\begin{itemize}
\item Permette di monitorare l'evoluzione del \glossaryItem{progetto} senza mai eseguire \glossaryItem{iterazioni}, ma solo \glossaryItem{incrementi};
\item Alla conclusione di ogni \glossaryItem{processo} si ha una base verificata che ne permette un veloce successivo \glossaryItem{incremento};
\item Permette l'alternanza tra le attivit\`a;
\item Adottando questo modello, il proponente può, al termine di ogni fase, valutare
      il sistema prodotto fino a quel momento fornendo così un \textit{feedback} utile allo svolgimento
      delle fasi successive. 
\end{itemize}

\subsection{Scadenze}
Di seguito sono presentate le scadenze che il gruppo ha deciso di rispettare. Su queste si baserà la pianificazione del \glossaryItem{progetto}:
\begin{itemize}
\item Revisione dei Requisiti (RR): 2016-01-22;
\item Revisione di Progettazione (RP): 2016-04-11 presentandosi con Revisione di Progettazione minima; 
\item Revisione di Qualifica (RQ): 2016-05-16;
\item Revisione di Accettazione (RA): 2016-07-11; \\il gruppo non è stato in grado di fornire un prodotto completo
alla prima \textit{Revisione di Accettazione} a causa del verificarsi di alcuni rischi il cui piano di mitigazione
non ha portato a risultati concreti.
In particolare, il rischio derivante dall'utilizzo di tecnologie poco conosciute, e il rischio causato dall'assenza
prolungata di alcuni membri del gruppo a causa del tirocinio non sono stati mitigati come previsto dall'analisi dei rischi
contenuta nel \textit{Piano di Progetto}. \\
Con lo scopo di arrivare ad un prodotto finito entro la seconda \textit{Revisione di Accettazione} tutti i componenti del gruppo,
compresi quelli impegnati nel tirocinio, dovranno tassativamente dedicare una parte cospicua del loro tempo allo sviluppo
continuo di MaaS e alla sua documentazione.
\end{itemize}

\subsection{Ruoli e costi}
Durante lo sviluppo del \glossaryItem{progetto} vi sono diversi ruoli, che ogni membro del gruppo è tenuto a ricoprire almeno una volta, evitando conflitti d’interesse al momento della \glossaryItem{verifica}. Nelle \NormeDiProgetto sono descritte le responsabilità che competono ad ogni ruolo. I ruoli che ogni componente del gruppo ricoprirà in tempi diversi sono: \textit{Amministratore}, \textit{Analista}, \textit{Progettista}, \textit{Programmatore}, \textit{Responsabile} e \textit{Verificatore}.
Ciascun ruolo ha il proprio costo orario, come segnalato nei \textit{Vincoli sull’organigramma del gruppo e sull’offerta tecnico-economica}:
\url{http://www.math.unipd.it/~tullio/IS-1/2015/Progetto/PD01b.html}



