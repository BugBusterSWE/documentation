\section{Consuntivo}
Verranno indicate di seguito le spese effettivamente sostenute, relative alle spese rendicontate, sia per ruolo che per persona.

Sar\`a infine presentato un bilancio:
\begin{itemize}
\item \textbf{positivo} se il preventivo supera il \glossaryItem{consuntivo};
\item \textbf{negativo} se il \glossaryItem{consuntivo} supera il preventivo;
\item \textbf{in pari} se \glossaryItem{consuntivo} e preventivo si equivalgono.
\end{itemize}


%-----------------------------------------------------------------------------------------------------
%-------------------------------------- ANALISI ------------------------------------------------------
%-----------------------------------------------------------------------------------------------------
\subsection{Analisi}
\subsubsection{Consuntivo}
Verranno indicate le ore di ruolo e le spese effettivamente sostenute durante il periodo di \textit{Analisi}. Questi dati sono quindi relativi alle ore non rendicontate.

\begin{table}[H]
	\centering
	\begin{tabular}{ l c c }
		\textbf{Ruolo} & \textbf{Ore} & \textbf{Costo} \\
		\hline
		Amministratore & 33(-3) & 660(-60) \euro{} \\
		Analista & 50 & 1250 \euro{} \\
		Progettista & 8(+2) & 176(+44) \euro{} \\
		Programmatore & 0 & 0 \euro{} \\
		Responsabile & 33(-7) & 990(-210) \euro{} \\
		Verificatore & 36(+8) & 540(+120) \euro{} \\
		\hline
		\textbf{Totale \glossaryItem{Consuntivo}} & 160 & 3510 \euro{} \\
		\hline
		\textbf{Totale Preventivo} & 160 & 3616 \euro{} \\
		\hline
		\textbf{Totale (Differenza)} & 0 & 106 \euro{} \\
		\hline
	\end{tabular}
	\caption{Ore non rendicontate, differenza preventivo/consuntivo nel periodo di \textit{Analisi}}
\end{table}



\subsubsection{Conclusione}
Il \glossaryItem{team} ha impegnato otto ore in pi\`u per l'attivit\`a di \glossaryItem{verifica}, due ore in pi\`u per l'attivit\`a di progettazione, tre ore in meno per l'attivit\`a di amministrazione e sette ore in meno per l'attivit\`a di gestione di \glossaryItem{progetto}. Questo ha permesso un risparmio di 106,00 \euro{} portando il costo \glossaryItem{consuntivo} al di sotto di quello preventivato.

\newpage


%-----------------------------------------------------------------------------------------------------
%-------------------------------------- ANALISI MIGLIORAMENTI ----------------------------------------
%-----------------------------------------------------------------------------------------------------
\subsection{Analisi Miglioramenti}
\subsubsection{Consuntivo}
Verranno indicate le ore di ruolo e le spese effettivamente sostenute durante il periodo di \textit{Analisi Miglioramenti}.

\begin{table}[H]
	\centering
	\begin{tabular}{ l c c }
		\textbf{Ruolo} & \textbf{Ore} & \textbf{Costo} \\
		\hline
		Amministratore & 11(+3) & 220(+60) \euro{} \\
		Analista & 4 & 100 \euro{} \\
		Progettista & 3(-1) & 66(-22) \euro{} \\
		Programmatore & 0 & 0 \euro{} \\
		Responsabile & 1 & 30 \euro{} \\
		Verificatore & 11(-3) & 165(-45) \euro{} \\
		\hline
		\textbf{Totale \glossaryItem{Consuntivo}} & 29 & 574 \euro{} \\
		\hline
		\textbf{Totale Preventivo} & 30 & 581 \euro{} \\
		\hline
		\textbf{Totale (Differenza)} & 1 & 7 \euro{} \\
		\hline
	\end{tabular}
	\caption{Ore rendicontate, differenza preventivo/consuntivo in periodo di \textit{Analisi Miglioramenti}}
\end{table}



\subsubsection{Conclusione}
Il \glossaryItem{team} ha impegnato 3 ore in meno per l'attivit\`a di \glossaryItem{verifica}, un'ora ora in meno per l'attivit\`a di progettazione e tre ore in più per l'attivit\`a di amministrazione di \glossaryItem{progetto}. Questo ha permesso di risparmiare un'ora e 7,00 \euro{}, portando il costo \glossaryItem{consuntivo} al di sotto di quello preventivato.

\newpage


%-----------------------------------------------------------------------------------------------------
%-------------------------------------- PROGETTAZIONE ARCHITETTURALE ---------------------------------
%-----------------------------------------------------------------------------------------------------
\subsection{Progettazione Architetturale}
\subsubsection{Consuntivo}
Verranno indicate le ore di ruolo e le spese effettivamente sostenute durante il periodo di \textit{Progettazione Architetturale}.

\begin{table}[H]
	\centering
	\begin{tabular}{ l c c }
		\textbf{Ruolo} & \textbf{Ore} & \textbf{Costo} \\
		\hline
		Amministratore & 8 & 160 \euro{} \\
		Analista & 3(+1) & 75(+25) \euro{} \\
		Progettista & 141(-9) & 3102(-198) \euro{} \\
		Programmatore & 0 & 0 \euro{} \\
		Responsabile & 2(+2) & 60(+60) \euro{} \\
		Verificatore & 74(+6) & 1110(+90) \euro{} \\
		\hline
		\textbf{Totale \glossaryItem{Consuntivo}} & 228 & 4484 \euro{} \\
		\hline
		\textbf{Totale Preventivo} & 228 & 4507 \euro{} \\
		\hline
		\textbf{Totale (Differenza)} & 0 & 23 \euro{} \\
		\hline
	\end{tabular}
	\caption{Ore rendicontate, differenza preventivo/consuntivo nel periodo di \textit{Progettazione Architetturale}}
\end{table}



\subsubsection{Conclusione}
Il \glossaryItem{team} ha impegnato nove ore in meno per l'attivit\`a di progettazione, un'ora ora in più per l'attivit\`a di analisi, due ore in pi\'u per l'attivit\'a di gestione di \glossaryItem{progetto} e sei ore in pi\'u per l'attivit\'a di \glossaryItem{verifica}. Questo ha permesso di risparmiare 23,00 \euro{}, portando il costo \glossaryItem{consuntivo} al di sotto di quello preventivato.

%-----------------------------------------------------------------------------------------------------
%-------------------------------------- BILANCIO DELLE SPESE EFFETTIVE -------------------------------
%-----------------------------------------------------------------------------------------------------
\subsection{Bilancio delle Spese Effettive}
Di seguito è riportato il \textit{Bilancio delle Spese Effettive} complessivo dei periodi di \textit{Analisi}, \textit{Analisi Miglioramenti} e \textit{Progettazione Architetturale}.

\begin{table}[H]
	\centering
	\begin{tabular}{ l c c }
		\textbf{Ruolo} & \textbf{Ore} & \textbf{Costo} \\
		\hline
		Amministratore & 52 & 1040 \euro{} \\
		Analista & 57(+1) & 1425(+25) \euro{} \\
		Progettista & 152(-8) & 3344(-176) \euro{} \\
		Programmatore & 0 & 0 \euro{} \\
		Responsabile & 36(-5) & 1080(-150) \euro{} \\
		Verificatore & 121(+11) & 1815(+165) \euro{} \\
		\hline
		\textbf{Totale \glossaryItem{Consuntivo}} & 417 & 8568 \euro{} \\
		\hline
		\textbf{Totale Preventivo} & 418 & 8704 \euro{} \\
		\hline
		\textbf{Totale (Differenza)} & 1 & 136 \euro{} \\
		\hline
	\end{tabular}
	\caption{Differenza preventivo/consuntivo delle spese effettive, con il periodo di \textit{Analisi}}
\end{table}

Poichè si ricorda che le ore del periodo di \textit{Analisi} sono a carico del fornitore, di seguito è riportato il \textit{Bilancio delle Spese Effettive} complessivo dei soli periodi di \textit{Analisi Miglioramenti} e \textit{Progettazione Architetturale}.

\begin{table}[H]
	\centering
	\begin{tabular}{ l c c }
		\textbf{Ruolo} & \textbf{Ore} & \textbf{Costo} \\
		\hline
		Amministratore & 19(+3) & 380(+60) \euro{} \\
		Analista & 7(+1) & 175(+25) \euro{} \\
		Progettista & 144(-10) & 3168(-220) \euro{} \\
		Programmatore & 0 & 0 \euro{} \\
		Responsabile & 3(+2) & 90(+60) \euro{} \\
		Verificatore & 85(+3) & 1275(+45) \euro{} \\
		\hline
		\textbf{Totale \glossaryItem{Consuntivo}} & 257 & 5058 \euro{} \\
		\hline
		\textbf{Totale Preventivo} & 258 & 5088 \euro{} \\
		\hline
		\textbf{Totale (Differenza)} & 1 & 30 \euro{} \\
		\hline
	\end{tabular}
	\caption{Differenza preventivo/consuntivo delle spese effettive, senza il periodo di \textit{Analisi}}
\end{table}

\newpage