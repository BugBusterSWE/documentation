\section{Consuntivo} 
Verranno indicate di seguito le spese effettivamente sostenute, relative alle spese rendicontate, sia per ruolo che per persona.

Sar\`a infine presentato un bilancio:
\begin{itemize}
\item \textbf{positivo} se il preventivo supera il \glossaryItem{consuntivo};
\item \textbf{negativo} se il \glossaryItem{consuntivo} supera il preventivo;
\item \textbf{in pari} se \glossaryItem{consuntivo} e preventivo si equivalgono.
\end{itemize}

%-----------------------------------------------------------------------------------------------------
%-------------------------------------- BILANCIO DELLE SPESE EFFETTIVE -------------------------------
%-----------------------------------------------------------------------------------------------------
\subsection{Bilancio delle spese effettive}
\subsubsection{Consuntivo}
Di seguito \`e riportato il \glossaryItem{consuntivo} complessivo dei periodi di \textit{Analisi}, \textit{Analisi Miglioramenti}, \textit{Progettazione Architetturale} e \textit{Progettazione di Dettaglio e Codifica}.


\begin{table}[H]
	\centering
	\begin{tabular}{ l c c }
		\textbf{Ruolo} & \textbf{Ore} & \textbf{Costo} \\
		\hline
		Amministratore & 115(-18) & 2300(-360) \euro{} \\
		Analista & 60(+3) & 1500(+75) \euro{} \\
		Progettista & 284(-39) & 6248(-858) \euro{} \\
		Programmatore & 146(+56) & 2190(+840) \euro{} \\
		Responsabile & 45(-3) & 1350(-90) \euro{} \\
		Verificatore & 245(+1) & 3675(+15) \euro{} \\
		\hline
		\textbf{Totale \glossaryItem{Consuntivo}} & 895 & 16885 \euro{} \\
		\hline
		\textbf{Totale Preventivo} & 895 & 17263 \euro{} \\
		\hline
		\textbf{Totale (Differenza)} & 0 & 378 \euro{} \\
		\hline
	\end{tabular}
	\caption{Differenza preventivo/consuntivo delle spese effettive, con il periodo di \textit{Analisi}}
\end{table}


Poich\`e si ricorda che le ore del periodo di \textit{Analisi} sono a carico del fornitore, di seguito \`e riportato il \glossaryItem{consuntivo} complessivo dei soli periodi di \textit{Analisi Miglioramenti}, \textit{Progettazione Architetturale} e \textit{Progettazione di Dettaglio e Codifica}.

\begin{table}[H]
	\centering
	\begin{tabular}{ l c c }
		\textbf{Ruolo} & \textbf{Ore} & \textbf{Costo} \\
		\hline
		Amministratore & 82(-15) & 1640(-300) \euro{} \\
		Analista & 10(+3) & 250(+75) \euro{} \\
		Progettista & 276(-41) & 6072(-902) \euro{} \\
		Programmatore & 146(+56) & 2190(+840) \euro{} \\
		Responsabile & 12(+4) & 360(+120) \euro{} \\
		Verificatore & 209(-7) & 3135(-105) \euro{} \\
		\hline
		\textbf{Totale \glossaryItem{Consuntivo}} & 735 & 13375 \euro{} \\
		\hline
		\textbf{Totale Preventivo} & 735 & 13647 \euro{} \\
		\hline
		\textbf{Totale (Differenza)} & 0 & 272 \euro{} \\
		\hline
	\end{tabular}
	\caption{Differenza preventivo/consuntivo delle spese effettive, senza il periodo di \textit{Analisi}}
\end{table}




\subsubsection{Conclusione}
In generale si \`e riscontrato che si sono presentate delle discrepanze tra i costi presenti nel \glossaryItem{consuntivo} e quelli effettivi, portando il \textit{Bilancio delle spese effettive} al di sotto dei costi preventivati. 
Per la maggior parte delle attivit\`a la differenza \`e piccola, mentre per quelle di \textit{progettazione} e \textit{programmazione} risulta essere pi\`u elevata delle altre.
In particolare \`e stata sovrastimata l'attivit\`a di \textit{progettazione} e sottostimata quella di \textit{programmazione}.






\newpage
%-----------------------------------------------------------------------------------------------------
%-------------------------------------- ANALISI ------------------------------------------------------
%-----------------------------------------------------------------------------------------------------
\subsection{Analisi}
\subsubsection{Consuntivo}
Verranno indicate le ore di ruolo e le spese effettivamente sostenute durante il periodo di \textit{Analisi}. Questi dati sono quindi relativi alle ore non rendicontate.

\begin{table}[H]
	\centering
	\begin{tabular}{ l c c }
		\textbf{Ruolo} & \textbf{Ore} & \textbf{Costo} \\
		\hline
		Amministratore & 33(-3) & 660(-60) \euro{} \\
		Analista & 50 & 1250 \euro{} \\
		Progettista & 8(+2) & 176(+44) \euro{} \\
		Programmatore & 0 & 0 \euro{} \\
		Responsabile & 33(-7) & 990(-210) \euro{} \\
		Verificatore & 36(+8) & 540(+120) \euro{} \\
		\hline
		\textbf{Totale \glossaryItem{Consuntivo}} & 160 & 3510 \euro{} \\
		\hline
		\textbf{Totale Preventivo} & 160 & 3616 \euro{} \\
		\hline
		\textbf{Totale (Differenza)} & 0 & 106 \euro{} \\
		\hline
	\end{tabular}
	\caption{Ore non rendicontate, differenza preventivo/consuntivo nel periodo di \textit{Analisi}}
\end{table}



\subsubsection{Conclusione}
Durante la pianificazione di questo periodo sono state sovrastimate le attivit\`a di \textit{amministrazione} e di \textit{gestione di \glossaryItem{progetto}}, mentre sono state sottostimate quelle di \textit{progettazione} e di \textit{verifica}.
Questo ha permesso di risparmiare 106,00 \euro{}, portando il costo \glossaryItem{consuntivo} al di sotto di quello preventivato.
\newpage


%-----------------------------------------------------------------------------------------------------
%-------------------------------------- ANALISI MIGLIORAMENTI ----------------------------------------
%-----------------------------------------------------------------------------------------------------
\subsection{Analisi Miglioramenti}
\subsubsection{Consuntivo}
Verranno indicate le ore di ruolo e le spese effettivamente sostenute durante il periodo di \textit{Analisi Miglioramenti}.

\begin{table}[H]
	\centering
	\begin{tabular}{ l c c }
		\textbf{Ruolo} & \textbf{Ore} & \textbf{Costo} \\
		\hline
		Amministratore & 11(+3) & 220(+60) \euro{} \\
		Analista & 4 & 100 \euro{} \\
		Progettista & 3(-1) & 66(-22) \euro{} \\
		Programmatore & 0 & 0 \euro{} \\
		Responsabile & 1 & 30 \euro{} \\
		Verificatore & 11(-3) & 165(-45) \euro{} \\
		\hline
		\textbf{Totale \glossaryItem{Consuntivo}} & 29 & 574 \euro{} \\
		\hline
		\textbf{Totale Preventivo} & 30 & 581 \euro{} \\
		\hline
		\textbf{Totale (Differenza)} & 1 & 7 \euro{} \\
		\hline
	\end{tabular}
	\caption{Ore rendicontate, differenza preventivo/consuntivo in periodo di \textit{Analisi Miglioramenti}}
\end{table}



\subsubsection{Conclusione}
Durante la pianificazione di questo periodo sono state sovrastimate le attivit\`a di \textit{progettazione} e di \textit{verifica}, mentre \`e stata sottostimata quella di \textit{amministrazione}.
Questo ha permesso di risparmiare 1 ora e 7,00 \euro{}, portando il costo \glossaryItem{consuntivo} al di sotto di quello preventivato.


\newpage


%-----------------------------------------------------------------------------------------------------
%-------------------------------------- PROGETTAZIONE ARCHITETTURALE ---------------------------------
%-----------------------------------------------------------------------------------------------------
\subsection{Progettazione Architetturale}
\subsubsection{Consuntivo}
Verranno indicate le ore di ruolo e le spese effettivamente sostenute durante il periodo di \textit{Progettazione Architetturale}.

\begin{table}[H]
	\centering
	\begin{tabular}{ l c c }
		\textbf{Ruolo} & \textbf{Ore} & \textbf{Costo} \\
		\hline
		Amministratore & 8 & 160 \euro{} \\
		Analista & 3(+1) & 75(+25) \euro{} \\
		Progettista & 141(-9) & 3102(-198) \euro{} \\
		Programmatore & 0 & 0 \euro{} \\
		Responsabile & 2(+2) & 60(+60) \euro{} \\
		Verificatore & 74(+6) & 1110(+90) \euro{} \\
		\hline
		\textbf{Totale \glossaryItem{Consuntivo}} & 228 & 4484 \euro{} \\
		\hline
		\textbf{Totale Preventivo} & 228 & 4507 \euro{} \\
		\hline
		\textbf{Totale (Differenza)} & 0 & 23 \euro{} \\
		\hline
	\end{tabular}
	\caption{Ore rendicontate, differenza preventivo/consuntivo nel periodo di \textit{Progettazione Architetturale}}
\end{table}



\subsubsection{Conclusione}
Durante la pianificazione di questo periodo \`e stata sovrastimata l'attivit\`a di \textit{progettazione}, mentre sono state sottostimate quelle di \textit{analisi},\textit{gestione di \glossaryItem{progetto}} e di \textit{verifica}.
Questo ha permesso di risparmiare 23,00 \euro{}, portando il costo \glossaryItem{consuntivo} al di sotto di quello preventivato.

%-----------------------------------------------------------------------------------------------------
%-------------------------------------- PROGETTAZIONE DI DETTAGLIO E CODIFICA ------------------------
%-----------------------------------------------------------------------------------------------------
\subsection{Progettazione di Dettaglio e Codifica}
\subsubsection{Consuntivo}
Verranno indicate le ore di ruolo e le spese effettivamente sostenute durante il periodo di \textit{Progettazione di Dettaglio e Codifica}.

\begin{table}[H]
	\centering
	\begin{tabular}{ l c c }
		\textbf{Ruolo} & \textbf{Ore} & \textbf{Costo} \\
		\hline
		Amministratore & 50(-10) & 1000(-200) \euro{} \\
		Analista & 3(+2) & 75(+50) \euro{} \\
		Progettista & 120(-24) & 2640(-528) \euro{} \\ %(3*8*4)
		Programmatore & 141(+23) & 2115(+345) \euro{} \\ %(5*8*4 + altre)
		Responsabile & 3(+2) & 90(+60) \euro{} \\
		Verificatore & 54(-3) & 810(-45) \euro{} \\
		\hline
		\textbf{Totale \glossaryItem{Consuntivo}} & 361 & 6412 \euro{} \\
		\hline
		\textbf{Totale Preventivo} & 371 & 6730 \euro{} \\
		\hline
		\textbf{Totale (Differenza)} & 10 & 318 \euro{} \\
		\hline
	\end{tabular}
	\caption{Ore rendicontate, differenza preventivo/consuntivo nel periodo di \textit{Progettazione di Dettaglio e Codifica}}
\end{table}



\subsubsection{Conclusione}
Durante la pianificazione di questo periodo sono state sovrastimate le attivit\`a di \textit{amministrazione}, \textit{progettazione} e di \textit{verifica}, mentre sono state sottostimate quelle di \textit{analisi},\textit{programmazione} e di \textit{gestione di \glossaryItem{progetto}}.
Questo ha permesso di risparmiare 10 ore e 318,00 \euro{}, portando il costo \glossaryItem{consuntivo} al di sotto di quello preventivato.


%-----------------------------------------------------------------------------------------------------
%----------------------------------------------- VALIDAZIONE -----------------------------------------
%-----------------------------------------------------------------------------------------------------
\subsection{Validazione}
\subsubsection{Consuntivo}
Verranno indicate le ore di ruolo e le spese effettivamente sostenute durante il periodo di \textit{Validazione}.

\begin{table}[H]
	\centering
	\begin{tabular}{ l c c }
		\textbf{Ruolo} & \textbf{Ore} & \textbf{Costo} \\
		\hline
		Amministratore & 13(-8) & 260(-160) \euro{} \\
		Analista & 0(0) & 0(0) \euro{} \\
		Progettista & 12(-7) & 264(-154) \euro{} \\ 
		Programmatore & 5(+33) & 75(+495) \euro{} \\ 
		Responsabile & 6 & 180 \euro{} \\
		Verificatore & 70(-7) & 1050(-105) \euro{} \\
		\hline
		\textbf{Totale \glossaryItem{Consuntivo}} & 117 & 1905 \euro{} \\
		\hline
		\textbf{Totale Preventivo} & 106 & 1829 \euro{} \\
		\hline
		\textbf{Totale (Differenza)} & -11 & -76 \euro{} \\
		\hline
	\end{tabular}
	\caption{Ore rendicontate, differenza preventivo/consuntivo nel periodo di \textit{Verifica}}
\end{table}



\subsubsection{Conclusione}
Durante la pianificazione di questo periodo sono state sovrastimate le attivit\`a di \textit{amministrazione}, \textit{progettazione} e di \textit{verifica}, mentre \`e stata sottostimata quella di \textit{programmazione}.
Questo ha fatto aumentare di 11 ore e 76,00 \euro{} i costi, sorpassando quelli preventivati.
