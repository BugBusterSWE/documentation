\section{Consuntivo}
Verranno indicate di seguito le spese effettivamente sostenute, relative alle spese rendicontate, sia per ruolo che per persona.

Sar \`a infine presentato un bilancio:
\begin{itemize}
\item \textbf{positivo} se il preventivo supera il \glossaryItem{consuntivo};
\item \textbf{negativo} se il \glossaryItem{consuntivo} supera il preventivo;
\item \textbf{in pari} se \glossaryItem{consuntivo} e preventivo si equivalgono.
\end{itemize}


%-----------------------------------------------------------------------------------------------------
%-------------------------------------- ANALISI ------------------------------------------------------
%-----------------------------------------------------------------------------------------------------
\subsection{Analisi}
\subsubsection{Consuntivo}
Verranno indicate le ore di ruolo e le spese effettivamente sostenute durante il periodo di \textit{Analisi}. Questi dati sono quindi relativi alle ore non rendicontate.

\begin{table}[H]
	\centering
	\begin{tabular}{ l c c }
		\textbf{Ruolo} & \textbf{Ore} & \textbf{Costo} \\
		\hline
		Amministratore & 33(-3) & 660(-60) \euro{} \\
		Analista & 50 & 1250 \euro{} \\
		Progettista & 8(+2) & 176(+44) \euro{} \\
		Programmatore & 0 & 0 \euro{} \\
		Responsabile & 33(-7) & 990(-210) \euro{} \\
		Verificatore & 36(+8) & 540(+120) \euro{} \\
		\hline
		\textbf{Totale \glossaryItem{Consuntivo}} & 160 & 3510 \euro{} \\
		\hline
		\textbf{Totale Preventivo} & 160 & 3616 \euro{} \\
		\hline
		\textbf{Totale (Differenza)} & 0 & 106 \euro{} \\
		\hline
	\end{tabular}
	\caption{Ore non rendicontate, differenza preventivo/consuntivo nel periodo di \textit{Analisi}}
\end{table}



\subsubsection{Conclusione}
Il \glossaryItem{team} ha impegnato otto ore in pi\`u per l'attivit\`a di \glossaryItem{verifica}, due ore in pi\`u per l'attivit\`a di progettazione, tre ore in meno per l'attivit\`a di amministrazione e sette ore in meno per l'attivit\`a di gestione di \glossaryItem{progetto}. Questo ha permesso un risparmio di 106,00 \euro{} portando il costo \glossaryItem{consuntivo} al di sotto di quello preventivato.

\newpage


%-----------------------------------------------------------------------------------------------------
%-------------------------------------- ANALISI MIGLIORAMENTI ----------------------------------------
%-----------------------------------------------------------------------------------------------------
\subsection{Analisi Miglioramenti}
\subsubsection{Consuntivo}
Verranno indicate le ore di ruolo e le spese effettivamente sostenute durante il periodo di \textit{Analisi Miglioramenti}.

\begin{table}[H]
	\centering
	\begin{tabular}{ l c c }
		\textbf{Ruolo} & \textbf{Ore} & \textbf{Costo} \\
		\hline
		Amministratore & 11(+3) & 220(+60) \euro{} \\
		Analista & 4 & 100 \euro{} \\
		Progettista & 3(-1) & 66(-22) \euro{} \\
		Programmatore & 0 & 0 \euro{} \\
		Responsabile & 1 & 30 \euro{} \\
		Verificatore & 11(-3) & 165(-45) \euro{} \\
		\hline
		\textbf{Totale \glossaryItem{Consuntivo}} & 29 & 574 \euro{} \\
		\hline
		\textbf{Totale Preventivo} & 30 & 581 \euro{} \\
		\hline
		\textbf{Totale (Differenza)} & 1 & 7 \euro{} \\
		\hline
	\end{tabular}
	\caption{Ore rendicontate, differenza preventivo/consuntivo in periodo di \textit{Analisi Miglioramenti}}
\end{table}



\subsubsection{Conclusione}
Il \glossaryItem{team} ha impegnato 3 ore in meno per l'attivit\`a di \glossaryItem{verifica}, un'ora ora in meno per l'attivit\`a di progettazione e tre ore in più per l'attivit\`a di amministrazione di \glossaryItem{progetto}. Questo ha permesso di risparmiare un'ora e 7,00 \euro{}, portando il costo \glossaryItem{consuntivo} al di sotto di quello preventivato.

\newpage


%-----------------------------------------------------------------------------------------------------
%-------------------------------------- PROGETTAZIONE ARCHITETTURALE ---------------------------------
%-----------------------------------------------------------------------------------------------------
\subsection{Progettazione Architetturale}
\subsubsection{Consuntivo}
Verranno indicate le ore di ruolo e le spese effettivamente sostenute durante il periodo di \textit{Progettazione Architetturale}.

\begin{table}[H]
	\centering
	\begin{tabular}{ l c c }
		\textbf{Ruolo} & \textbf{Ore} & \textbf{Costo} \\
		\hline
		Amministratore & 8 & 160 \euro{} \\
		Analista & 3(+1) & 75(+25) \euro{} \\
		Progettista & 141(-9) & 3102(-198) \euro{} \\
		Programmatore & 0 & 0 \euro{} \\
		Responsabile & 2(+2) & 60(+60) \euro{} \\
		Verificatore & 74(+6) & 1110(+90) \euro{} \\
		\hline
		\textbf{Totale \glossaryItem{Consuntivo}} & 228 & 4484 \euro{} \\
		\hline
		\textbf{Totale Preventivo} & 228 & 4507 \euro{} \\
		\hline
		\textbf{Totale (Differenza)} & 0 & 23 \euro{} \\
		\hline
	\end{tabular}
	\caption{Ore rendicontate, differenza preventivo/consuntivo nel periodo di \textit{Progettazione Architetturale}}
\end{table}



\subsubsection{Conclusione}
Il \glossaryItem{team} ha impegnato nove ore in meno per l'attivit\`a di progettazione, un'ora ora in più per l'attivit\`a di analisi, due ore in pi\`u per l'attivit\`a di gestione di \glossaryItem{progetto} e sei ore in pi\`u per l'attivit\`a di \glossaryItem{verifica}. Questo ha permesso di risparmiare 23,00 \euro{}, portando il costo \glossaryItem{consuntivo} al di sotto di quello preventivato.

%-----------------------------------------------------------------------------------------------------
%-------------------------------------- PROGETTAZIONE DI DETTAGLIO E CODIFICA ------------------------
%-----------------------------------------------------------------------------------------------------
\subsection{Progettazione di Dettaglio e Codifica}
\subsubsection{Consuntivo}
Verranno indicate le ore di ruolo e le spese effettivamente sostenute durante il periodo di \textit{Progettazione di Dettaglio e Codifica}.

\begin{table}[H]
	\centering
	\begin{tabular}{ l c c }
		\textbf{Ruolo} & \textbf{Ore} & \textbf{Costo} \\
		\hline
		Amministratore & 50(-10) & 1000(-200) \euro{} \\
		Analista & 3(+2) & 75(+50) \euro{} \\
		Progettista & 120(-24) & 2640(-528) \euro{} \\ %(3*8*4)
		Programmatore & 141(+23) & 2115(+345) \euro{} \\ %(5*8*4 + altre)
		Responsabile & 3(+2) & 90(+60) \euro{} \\
		Verificatore & 54(-3) & 810(-45) \euro{} \\
		\hline
		\textbf{Totale \glossaryItem{Consuntivo}} & 361 & 6412 \euro{} \\
		\hline
		\textbf{Totale Preventivo} & 371 & 6730 \euro{} \\
		\hline
		\textbf{Totale (Differenza)} & 10 & 318 \euro{} \\
		\hline
	\end{tabular}
	\caption{Ore rendicontate, differenza preventivo/consuntivo nel periodo di \textit{Progettazione di Dettaglio e Codifica}}
\end{table}



\subsubsection{Conclusione}
Il \glossaryItem{team} ha impegnato dieci ore in meno per l'attivit\`a di amministrazione, due ore in pi\`u per l'attivit\`a di analisi, ventiquattro ore in meno per l'attivit\`a di progettazione, ventitre ore in pi\`u per l'attivit\`a di programmazione, due ore in pi\`u per l'attivit\`a di gestione di \glossaryItem{progetto} e tre ore in meno per l'attivit\`a di \glossaryItem{verifica}. Questo ha permesso di risparmiare 23,00 \euro{}, portando il costo \glossaryItem{consuntivo} al di sotto di quello preventivato.


%-----------------------------------------------------------------------------------------------------
%-------------------------------------- BILANCIO DELLE SPESE EFFETTIVE -------------------------------
%-----------------------------------------------------------------------------------------------------
\subsection{Bilancio delle Spese Effettive}
Di seguito \`e riportato il \textit{Bilancio delle Spese Effettive} complessivo dei periodi di \textit{Analisi}, \textit{Analisi Miglioramenti}, \textit{Progettazione Architetturale} e \textit{Progettazione di Dettaglio e Codifica}.

\begin{table}[H]
	\centering
	\begin{tabular}{ l c c }
		\textbf{Ruolo} & \textbf{Ore} & \textbf{Costo} \\
		\hline
		Amministratore & 102(-10) & 2040(-200) \euro{} \\
		Analista & 60(+3) & 1500(+75) \euro{} \\
		Progettista & 272(-32) & 5984(-704) \euro{} \\
		Programmatore & 141(+23) & 2115(+345) \euro{} \\
		Responsabile & 39(-3) & 1170(-90) \euro{} \\
		Verificatore & 175(+8) & 2625(+120) \euro{} \\
		\hline
		\textbf{Totale \glossaryItem{Consuntivo}} & 778 & 14980 \euro{} \\
		\hline
		\textbf{Totale Preventivo} & 789 & 15434 \euro{} \\
		\hline
		\textbf{Totale (Differenza)} & 11 & 454 \euro{} \\
		\hline
	\end{tabular}
	\caption{Differenza preventivo/consuntivo delle spese effettive, con il periodo di \textit{Analisi}}
\end{table}

Poich\`e si ricorda che le ore del periodo di \textit{Analisi} sono a carico del fornitore, di seguito \`e riportato il \textit{Bilancio delle Spese Effettive} complessivo dei soli periodi di \textit{Analisi Miglioramenti}, \textit{Progettazione Architetturale} e \textit{Progettazione di Dettaglio e Codifica}.

\begin{table}[H]
	\centering
	\begin{tabular}{ l c c }
		\textbf{Ruolo} & \textbf{Ore} & \textbf{Costo} \\
		\hline
		Amministratore & 69(+13) & 1380(-140) \euro{} \\
		Analista & 10(+3) & 250(+75) \euro{} \\
		Progettista & 264(-34) & 5808(-748) \euro{} \\
		Programmatore & 141(+23) & 2115(+345) \euro{} \\
		Responsabile & 6(+4) & 180(+120) \euro{} \\
		Verificatore & 139 & 2085 \euro{} \\
		\hline
		\textbf{Totale \glossaryItem{Consuntivo}} & 618 & 11470 \euro{} \\
		\hline
		\textbf{Totale Preventivo} & 629 & 11818 \euro{} \\
		\hline
		\textbf{Totale (Differenza)} & 11 & 348 \euro{} \\
		\hline
	\end{tabular}
	\caption{Differenza preventivo/consuntivo delle spese effettive, senza il periodo di \textit{Analisi}}
\end{table}

\newpage