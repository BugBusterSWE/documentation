\section{Consuntivo e preventivo a finire}
Verranno indicate di seguito le spese effettivamente sostenute, relative alle spese rendicontate, sia per ruolo che per persona.

Sar\`a infine presentato un bilancio:
\begin{itemize}
\item \textbf{positivo} se il preventivo supera il consuntivo;
\item \textbf{negativo} se il consuntivo supera il preventivo;
\item \textbf{in pari} se consuntivo e preventivo si equivalgono.
\end{itemize}


%-----------------------------------------------------------------------------------------------------
%-------------------------------------- ANALISI ------------------------------------------------------
%-----------------------------------------------------------------------------------------------------
\subsection{Fase Analisi}
\subsubsection{Consuntivo}
Verranno indicate le ore di ruolo e le spese effettivamente sostenute durante la \glossaryItem{fase} di Analisi. Questi dati sono quindi relativi alle ore non rendicontate.

\begin{table}[H]
	\centering
	\begin{tabular}{ l c c }
		\textbf{Ruolo} & \textbf{Ore} & \textbf{Costo} \\
		\hline
		Amministratore & 33(-3) & 660(-60) \euro{} \\
		Analista & 50 & 1250 \euro{} \\
		Progettista & 8(+2) & 176(+44) \euro{} \\
		Programmatore & 0 & 0 \euro{} \\
		Responsabile & 33(-7) & 990(-210) \euro{} \\
		Verificatore & 36(+8) & 540(+120) \euro{} \\
		\hline
		\textbf{Totale Consuntivo} & 160 & 3510 \euro{} \\
		\hline
		\textbf{Totale Preventivo} & 160 & 3616 \euro{} \\
		\hline
		\textbf{Totale (Differenza)} & 0 & 106 \euro{} \\
		\hline
	\end{tabular}
	\caption{Ore non rendicontate, differenza preventivo/consuntivo in \glossaryItem{fase} di Analisi}
\end{table}



\subsubsection{Conclusione}
Il \glossaryItem{team} ha impegnato otto ore in pi\`u per l'attivit\`a di verifica, due ore in pi\`u per l'attivit\`a di Progettista, tre ore in meno per l'attivit\`a di Amministratore e sette ore in meno per l'attivit\`a di Responsabile. Questo ha permesso un risparmio di 106,00 \euro{} portando il costo consuntivo al di sotto di quello preventivato.

\newpage


%-----------------------------------------------------------------------------------------------------
%-------------------------------------- ANALISI MIGLIORAMENTI ----------------------------------------
%-----------------------------------------------------------------------------------------------------
\subsection{Fase Analisi Miglioramenti}
\subsubsection{Consuntivo}
Verranno indicate le ore di ruolo e le spese effettivamente sostenute durante la \glossaryItem{fase} di Analisi Miglioramenti.

\begin{table}[H]
	\centering
	\begin{tabular}{ l c c }
		\textbf{Ruolo} & \textbf{Ore} & \textbf{Costo} \\
		\hline
		Amministratore & 11(+3) & 220(+60) \euro{} \\
		Analista & 4 & 100 \euro{} \\
		Progettista & 3(-1) & 66(-22) \euro{} \\
		Programmatore & 0 & 0 \euro{} \\
		Responsabile & 1 & 30 \euro{} \\
		Verificatore & 11(-3) & 165(-45) \euro{} \\
		\hline
		\textbf{Totale Consuntivo} & 29 & 574 \euro{} \\
		\hline
		\textbf{Totale Preventivo} & 30 & 581 \euro{} \\
		\hline
		\textbf{Totale (Differenza)} & 1 & 7 \euro{} \\
		\hline
	\end{tabular}
	\caption{Ore non rendicontate, differenza preventivo/consuntivo in \glossaryItem{fase} di Analisi Miglioramenti}
\end{table}



\subsubsection{Conclusione}
Il \glossaryItem{team} ha impegnato 3 ore in meno per l'attivit\`a di verifica, un'ora ora in meno per l'attivit\`a di Progettista e tre ore in più per l'attivit\`a di Amministratore. Questo ha permesso di risparmiare un'ora e 7,00 \euro{}, portando il costo consuntivo al di sotto di quello preventivato.
