\section {Meccanismi di Controllo e Rendicontazione}
Al fine di controllare e valutare il progresso del lavoro e delle
attività del progetto e semplificare il compito del \textit{Responsabile 
di Progetto} verranno utilizzati i seguenti strumenti:
\begin{itemize}
  \item \textbf{Teamwork}: questo strumento mette a disposizione
	un calendario interno, e sarà compito del \textit{Responsabile di Progetto}
	mantenerlo aggiornato con tutte le milestone e le date considerate importanti;
  \item \textbf{Diagrammi, tabelle e grafici}: per rendere più chiara ed efficace la pianificazione
	 del progetto, nel presente documento sono stati inseriti diagrammi di Gantt, tabelle e 
  	 grafici;
 
  \item \textbf{Rendicontazione delle ore di lavoro}: Teamwork fornisce una funzionalità
	per la rendicontazione delle ore di lavoro, permettendo così al \textit{Responsabile di Progetto}
	di avere una visione complessiva dello stato di avanzamento delle attività, ed eventualmente
	di riconsiderare la distribuzione del carico lavorativo in caso di problemi;

  \item \textbf{Riunioni interne}: sono necessarie per mantenere un confronto diretto tra i membri del gruppo
	e per monitorare l'avanzamento delle attività; 

  \item \textbf{Sistema  di ticketing}: per ottenere un controllo sistematico e quantificabile
   	 dello stato di avanzamento di ogni attività vengono utilizzati i sistemi di \textit{ticketing} 
  	 offerti dai servizi Teamwork e GitHub, 
  	 i quali andranno utilizzati secondo quanto definito nelle \textit{Norme di Progetto}.
\end{itemize}
