%\href{nome}{url}
\section{Processi primari}

	\subsection{Processo di fornitura}	
	\subsubsection{Oggetto del processo}
	L'oggetto del presente \glossaryItem{Processo} è la stesura dello \StudioDiFattibilita, compito
	di cui si occupa un \textit{Analista}.
	\subsubsection{Struttura dello \textit{Studio di Fattibilit\'a v\VersioneSF{}}}
		\begin{itemize}
			\item Capitolato scelto;
				\begin{enumerate}
					\item Descrizione
					\item Valutazione sul dominio del \glossaryItem{progetto}
					\item Motivazioni della scelta
				\end{enumerate}	
			\item Valutazioni sugli altri capitolati, con rispettivi aspetti positivi e negativi.
		\end{itemize}


	\subsection{Processo di sviluppo}
        \subsubsection{Scopo del processo}
        Lo scopo del presente \glossaryItem{processo} \`e la realizzazione di un elemento di sistema implementato come prodotto
        software. Il \glossaryItem{processo} in esame produce un software che soddisfa i requisiti architetturali, sottoposto
        in seguito alle attivit\`a di \glossaryItem{verifica} e \glossaryItem{validazione}.
        
        \subsubsection{Risultati principali del processo}
        Sono stati previsti i seguenti risultati positivi derivanti da una corretta implementazione del \textit{\glossaryItem{processo} di sviluppo}:

        \begin{itemize}
          \item Requisiti software ben delineati;
          \item Gli elementi tecnologici e di progettazione sono identificati;
          \item L'attivit\`a di codifica consiste in una mera traduzione dell'output proveniente dall'attivit\`a di progettazione,
            riducendo cos\`i i rischi provenienti dalla libert\`a dei \glossaryItem{programmatori}.
        \end{itemize}
        
		\paragraph*{1. Attività di Analisi dei Requisiti}
			\paragraph*{Scopo}
				Gli analisti del gruppo dovranno ricavare i requisiti utili per il \glossaryItem{progetto}
				dal capitolato e dagli incontri con il proponente, avendo come obiettivo la
				stesura del documento \textit{Analisi dei Requisiti}.
			        %parlare di: IEEE 830-1998: Recommended Practice
			        %for Software Requirements Specifications (slide 32 in ING_REQUISITI)
			        Inotre la specifica dei requisiti dev'essere conforme ai principi dello standard \textit{IEEE 830-1998}, espressi in otto qualit\`a
			        essenziali:
				\begin{itemize}
				\item Priva di ambiguit\`a (UNAMBIGOUS);
				\item Corretta (CORRECT);
				\item Completa (COMPLETE);
				\item Verificabile (VERIFIABLE);
				\item Consistente (CONSISTENT);
				\item Modificabile (MODIFIABLE);
				\item Tracciabile (TRACEABLE);
				\item Ordinata per rilevanza (RANKED).
				\end{itemize}
			        %slide 40 di ING_REQUISITI: Cause di abbandono
			        La stesura di un'Analisi dei Requisiti di qualit\`a \`e cruciale, infatti
			        da un rapporto dello Standish Group del 1995 si nota come due delle cause primarie
			        di abbandono dei progetti siano le seguenti:
				\begin{itemize}
				\item Requisiti incompleti;
				\item Volatilità di specifiche e requisiti.
				\end{itemize}
			
			
			%casi d'uso: si parte da 0 nella numerazione
			\paragraph*{Struttura del documento}
			Il documento prodotto dalla presente attivit\`a consister\`a in un elenco dei \glossaryItem{casi d'uso} individuati, 
                        ognuno avente la seguente forma:
				\begin{itemize}
				\item Codice identificativo: \textbf{UC-{X} x.y.z}
						\begin{itemize}
						\item \textbf{X}: rappresenta uno degli ambiti di riferimento individuati
						nell'Analisi dei Requisiti, in particolare:
							\begin{itemize}
							\item [] \textbf{U} = Ambito Utente;
							\item [] \textbf{S} = Ambito \glossaryItem{Super-Admin};
							\item [] \textbf{E} = Ambito Editor. 
							\end{itemize}
						\end{itemize}
				\item Titolo;
				\item Diagramma \glossaryItem{UML};
				\item Attori primari;
				\item Scopo e descrizione;
				\item Precondizioni;
				\item Postcondizione;
				\item Scenario principale;
                                \item Estensioni.
				\end{itemize}
			Saranno altresì esposti i requisiti, associati alle rispettive fonti, nella forma seguente:
				\begin{itemize}
				\item Codice identificativo: \textbf{R{X}{Y} x.y.z}
				\item Titolo;
				\item Descrizione;
                \item Fonti.
				\end{itemize}
			E viceversa:
				\begin{itemize}
				\item Fonte;
				\item Codice identificativo del requisito.
				\end{itemize}
			%parlare degli STRUMENTI utilizzati
			\paragraph*{Strumenti}
				Per i diagrammi dei casi d'uso verrà utilizzato il linguaggio \textit{UML 2.0},
				noto a tutti i membri del gruppo in quanto trattato durante il corso
				di \textit{Ingegneria del Software}.
				Si \`e scelto inoltre di adottare l'editor \glossaryItem{UML} \textit{Astah}.   %riferimento per astah
			
			
		\paragraph*{2. Attività di Progettazione}
			\paragraph*{Scopo}
		     %Costruzione a priori perseguendo la correttezza per costruzione
		     %invece che inseguendo la correttezza per correzione     <<  slide PROGETTAZIONE
		     %Procede dall'Analisi dei Requisiti   <<  slide PROGETTAZIONE	
		        La presente attivit\`a, che procede dall'Analisi dei Requisiti, persegue l'obiettivo
		        della correttezza del prodotto per costruzione, permettendo di ridurre al minimo le attività di correzione
		        in \glossaryItem{fase} di codifica.
                        I documenti elaborati in questa \glossaryItem{fase} sono:
                        \begin{itemize}
                        \item \glossaryItem{Specifica Tecnica}, nel quale vengono individuate le componenti macroscopiche del sistema,
                          divisione utile per garantire che il lavoro successivo proceda in modo parallelo e soddisfi i principi di ortogonalit\'a;
                        \item \glossaryItem{Definizione di Prodotto}, documento che fornisce ai \glossaryItem{programmatori} tutte le direttive. necessarie per la codifica
                        \end{itemize}
		    %si useranno Design patterns
		        Per la progettazione a livello di sistema si utilizzeranno \glossaryItem{design pattern} architetturali globalmente affermati.
                 
                        %diagrammi UML prodotti
                        \paragraph*{Diagrammi prodotti}
                        Nel corso di questa \glossaryItem{fase} verranno prodotti quattro tipi di diagrammi \glossaryItem{UML}:
                        \begin{itemize}
                        \item Diagrammi delle classi;
                        \item Diagrammi dei \glossaryItem{package};
                        \item Diagrammi di sequenza;
                        \item Diagrammi di attivit\`a.
                        \end{itemize}
                        I tipi di diagrammi sopra elencati andranno a far parte dei due documenti menzionati sopra.
		
		 
		 %parlare degli STRUMENTI utilizzati
		        \paragraph*{Strumenti}
		        Come per l'attivit\`a precedente e per gli stessi motivi, i membri del gruppo hanno stabilito l'utilizzo del linguaggio \glossaryItem{UML} 2.0 per i seguenti
		        diagrammi:
		        \begin{itemize}
			\item Diagrammi dei \glossaryItem{package}; 
			\item Diagrammi delle \textit{classi};	
			\item Diagrammi di \textit{attivit\`a}.
		        \end{itemize}
			 
		
		\paragraph*{3. Attività di Codifica}
                %citare https://github.com/RisingStack/node-style-guide/blob/master/README.md
                \paragraph*{Scopo}
                La presente attivit\`a ha come scopo la traduzione in codice sorgente dei risultati ottenuti in sede
                di \glossaryItem{Specifica Tecnica} e di \glossaryItem{Definizione di Prodotto}.
                In sede di \textit{Revisione dei Requisiti} non \`e ancora possibile fornire una visione precisa sulla presente attivit\`a,
                tuttavia per il codice \glossaryItem{Javascript} 
                si \`e stabilito di adottare le direttive presenti in \href{https://github.com/RisingStack/node-style-guide/blob/master/README.md}{RisingStack Node.js Style Guide()}.
     		%<<< slide 22/30 di ANALISI_STATICA
                Questa scelta \`e strettamente correlata alla teoria sull'\textit{Analisi Statica}, in quanto da essa si evince che:
                \begin{itemize}
		  \item L’adozione di \textit{standard} di codifica;
                    deve essere coerente con la scelta dei metodi di \glossaryItem{verifica} adottati,         
                     %<<< slide 30/30 di ANALISI_STATICA
		  \item L’\glossaryItem{efficacia} dei metodi di analisi statica \`e strettamente correlata alla qualit\`a di strutturazione del codice.
                \end{itemize}

		\paragraph*{\textit{10 regole d'oro}}
		Perseguendo l'obiettivo della semplificazione dell'analisi (sia statica sia dinamica),
           	i programmatori del gruppo cercheranno di essere quanto più conformi possibile alle 10 regole
           	espresse nel documento \textit{The Power of 10: Rules for Developing Safety-Critical Code}, che riportiamo di seguito per comodità:
                	\begin{enumerate}
                        	\item \textit{Usare costrutti di controllo di flusso i più semplici possibile}
                        	\item \textit{Usare iterazioni con limite superiore statico}
                        	\item \textit{Non usare allocazione dinamica di memoria dopo l'inizializzazione}
                        	\item \textit{Limitare la lunghezza di ogni singolo sottoprogramma a 60 linee}
                        	\item \textit{Controllare che tutti i parametri in ingresso e in uscita siano validi}
                        	\item \textit{Limitare al massimo la compilazione condizionale}
                        	\item \textit{Puntare ad un uso intensivo di asserzioni(almeno 2 per sottoprogramma)}
                        	\item \textit{Puntare al massimo livello architetturale di \textit{data hiding}}
                        	\item \textit{Limitare al massimo l'uso dei puntatori e il livello di dereferenziazione}
                        	\item \textit{Il codice va compilato da subito usando il compilatore nel modo più restrittivo}
			\end{enumerate}

                \paragraph*{Strumenti}
                Il gruppo ha stabilito di utilizzare i seguenti editor:
                \begin{itemize}
                  \item \textbf{Emacs}(\url{https://www.gnu.org/software/emacs/}) per la stesura del codice \LaTeX; 
                  \item \textbf{WebStorm}(\url{%https://www.jetbrains.com/webstorm/}) per il codice \glossaryItem{Javascript}.      
                \end{itemize}
