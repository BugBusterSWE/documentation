\newpage
\section{Lista di controllo}
Durante la verifica tramite walkthrough sono stati rilevati numerosi errori. I principali e più comuni vengono riportati di seguito, in modo da essere trattati tramite inspection e non più tramite walkthrough.
\begin{itemize}
\item \textbf{Norme stilistiche}:
	\begin{itemize}
	\item Elenco di voci: non termina con il punto e virgola o con il punto nel caso di ultimo elemento;
	\item Nome proprio di persona: le iniziali di nome o cognome non sono maiuscole;
	\item Riferimento ad un altro documento: non viene utilizzata la macro predisposta;
	\item Termini inglesi: non in corsivo.
	\end{itemize}
\item \textbf{Italiano}:
	\begin{itemize}
	\item Accenti nelle parole;
	\item Periodi: frasi troppo lunghe rendono difficile la lettura;
	\item Doppie negazioni: evitarle perché rendono complessa la lettura;
	\item Proponente e committente: evitare di confondere il significato;
	\item Periodo e fase: evitare di confondere il significato;
	\item Errori di battitura.
	\end{itemize}
\item \textbf{Inglese}:
	\begin{itemize}
	\item Correttezza delle frasi;
	\item Correttezza linguistica;
	\item Periodi: frasi troppo lunghe rendono difficile la lettura;
	\item Doppie negazioni: evitarle perché rendono complessa la lettura.
	\end{itemize}
\item \textbf{\LaTeX}:
	\begin{itemize}
	\item Ambiente verbatim: controllare che il testo inserito nell'ambiente \LaTeX{} verbatim non fuoriesca dalla pagina;
	\item Tabelle e figure: evitare che appaiano gli elenchi di tabelle e figure vuoti qualora non ci fossero tabelle e/o figure nel documento;
	\item Termini da glossario: evitare che la marcatura dei termini da glossario venga riportata anche nei titoli delle sezioni e/o nel registro delle modifiche.
	\end{itemize}
\item \textbf{UML}:
	\begin{itemize}
	\item Controllo ortografico: controllare attentamente l'ortografia. Non è possibile automatizzare tale controllo.
	\item Direzione delle frecce: deve essere sempre presente, se richiesta, e la direzione deve essere corretta.
	\item Consistenza della nomenclatura tra diagrammi e descrizioni testuali.
	\end{itemize}
\end{itemize}
