\section{Introduzione}
\subsection{Scopo del documento}
Questo documento definisce le norme, gli strumenti e le procedure che tutti i membri del gruppo devono adottare per la realizzazione del progetto. Tutti i membri sono tenuti a leggere il documento e ad applicare le regole ivi definite, al fine di garantire la conformità del materiale prodotto e di ridurre il numero di errori. Qualora siano necessarie modifiche a questo documento è richiesto tassativamente il tempestivo avviso agli altri membri del gruppo.

\subsection{Scopo del progetto}
Lo scopo del progetto è la realizzazione di un servizio per le aziende, raggiungibile in un server web, per la visualizzazione di dati aziendali. Tale progetto si basa su \textit{Maap}, un'applicazione già esistente che ha lo scopo di fornire la visualizzazione dei dati letti dal database \textit{mongoDB} in possesso dell'azienda. Il progetto verte sulla conversione di Maap da applicazione web a servizio, estendendone le potenzialità tramite un editor per facilitare ai nuovi utenti la creazione di viste per i propri dati.

\subsection{Ambiguit\`a}
Al fine di evitare ambiguità dovute al linguaggio impiegato nei documenti, viene fornito il \Glossario contente la definizione dei termini marcati.

\subsection{Riferimenti a documenti esterni}
\begin{itemize}
\item Normativa ISO 12207 per ciclo di vita del software (\url{https://en.wikipedia.org/wiki/ISO/IEC_12207})
\item Swebook (\url{http://www.computer.org/web/swebok/v3})
\item Capitolato (\url{http://www.math.unipd.it/~tullio/IS-1/2015/Progetto/C4.pdf})
\end{itemize}
