\section{Processi di supporto}
\subsection{Processo di documentazione}


%###############################################non modificato
\subsubsection{Scopo del processo}
Questo \glossaryItem{processo} ha come scopi la produzione e la manutenzione della documentazione necessaria durante il \glossaryItem{ciclo di vita del software}.
\subsubsection{Risultati attesi dal processo}
Tale \glossaryItem{processo} deve garanire:
\begin{itemize}
\item l'individuazione dei documenti da produrre per le varie \glossaryItem{fasi} del ciclo di sviluppo;
\item la corretta stesura dei documenti segnalati;
\item l'individuazione degli standard a cui tali documenti devono aderire.
\end{itemize}
%###################################################


%##########################################################OK
\subsubsection{1. Attivit\`a di implementazione di processo}    %NUOVO
  \paragraph*{Procedura}
  Per produrre la documentazione si è stabilito di utilizzare il \glossaryItem{linguaggio di markup} \LaTeX. Tale scelta è stata effettuata per permettere un miglior versionamento dei file e una formattazione uniforme dei testi.
  Per la realizzazione di ciascun documento \`e stato allestito un modello .tex reperibile presso il \glossaryItem{repository} dei documenti.

  \paragraph*{Linguaggio da utilizzare}
  Nel capitolato viene espresso l'obbligo di documentazione in lingua inglese. Il Responsabile di \glossaryItem{Progetto}, dopo aver contattato il \glossaryItem{Committente}, ha specificato che la richiesta di documentare in tale lingua è obbligatoria solo per i documenti rivolti all'utente finale. Si è deciso quindi che la documentazione interna (Norme di \glossaryItem{Progetto}) e la documentazione non riservata all'utente finale (Analisi dei Requisiti, Piano di Qualifica e Piano di \glossaryItem{Progetto}) venga stesa utilizzando la lingua italiana.  
 %###################################################





\subsubsection{2. Attivit\`a di progetto e sviluppo}   %NUOVO  
  \textbf{Struttura comune dei documenti} \\
      \textbf{Prima pagina} \\   
      La prima pagina del documento deve sempre essere conforme a questa struttura:
      \begin{itemize}
      \item Il \textbf{logo} ed il \textbf{nome} del gruppo devono essere riportati nella metà alta del documento, centrati orizzontalmente;
      \item Il \textbf{titolo} del documento al di sotto del logo e del nome del gruppo;
      \item \textbf{Tabella riassuntiva} (descrittiva) visibile subito dopo il titolo, centrata orizzontalmente e contenente
		le seguenti informazioni relative al documento:
        \begin{itemize}
        \item \glossaryItem{Versione} del documento;
        \item Nome e cognome dei \textbf{redattori} del documento;
        \item Nome e cognome dei \textbf{verificatori} del documento;
        \item Nome e cognome del \textit{Responsabile di Progetto}, il quale approva il documento;
        \item Tipologia d'uso del documento;
        \item Lista di distribuzione.
        \end{itemize}
    \item Breve descrizione del contenuto del documento.
      \end{itemize}

  \textbf{Diario delle modifiche} \\
  All'interno del documento deve essere presente, subito dopo la prima pagina, una sezione riguardante le modifiche apportate. Tale sezione consisterà in una tabella le cui righe devono essere cos\`i formattate:
  \begin{itemize}
  \item Sommario delle modifiche apportate;
  \item Nome e cognome dell'autore della modifica;
  \item Ruolo dell'autore della modifica;
  \item Data della modifica;
  \item \glossaryItem{Versione} del documento aggiornata a dopo la modifica.
  \end{itemize}
  Questa tabella deve essere ordinata per data in modo decrescente, affinché appaiano per prime le ultime modifiche subite dal documento, e la \glossaryItem{versione} riportata dalla prima riga sia la stessa riportata nella prima pagina del documento. \\

  \textbf{Indici} \\
  In ogni documento va obbligatoriamente inserito l'indice per le sezioni, in modo da agevolare la consultazione e permettere una lettura \textit{ipertestuale} 
  e non necessariamente sequenziale del documento.
  Per quanto concerne le appendici, esse non devono essere numerate ma indicate da una lettera maiuscola che verrà
  incrementata a partire dalla lettera \textbf{A} seguendo l’ordine alfabetico internazionale. \\
  
  \textbf{Formattazione generale delle pagine}  \\
  Ciascuna pagina del documento deve contenere un'intestazione riportante:
  \begin{itemize}
  \item Logo del gruppo;
  \item Nome del gruppo;
  \item Sezione corrente del documento.
  \end{itemize}
  Ciascuna pagina inoltre deve riportare anche un pié di pagina con le seguenti informazioni:
  \begin{itemize}
  \item Nome e \glossaryItem{versione} del documento;
  \item Pagina corrente del documento nel formato X di N, dove X è il numero di pagina corrente ed N è il numero di pagine totali.
  \end{itemize}
  
  \textbf{Note a pié di pagina} \\
  Per ciascuna pagina interna, qualora dovessero comparire delle note da esplicare, esse andranno segnalate in
  basso a sinistra della pagina corrente, riportate con il loro numero e la loro descrizione.
	

\paragraph*{Elementi grafici}
\begin{itemize}
	\item \textbf{Tabelle} \\
		Ciascuna tabella deve essere centrata orizzontalmente e deve contenere sotto di essa la propria
		didascalia (per semplificarne il tracciamento), nella quale deve comparire il numero della tabella corrente (incrementale in tutto
		il documento) e una breve descrizione del suo contenuto;
	\item \textbf{Immagini} \\
		Ciascuna immagine deve essere centrata orizzontalmente ed avere una larghezza fissa. Inoltre deve
		essere nettamente separata dai paragrafi che la seguono e la precedono, in modo da definire una
		chiara separazione tra testo e grafica e migliorare conseguentemente l'usabilità del documento. Essa dev’essere
		accompagnata da una didascalia analoga a quella descritta per le tabelle. Tutti i diagrammi
		UML vengono inseriti nel documento sotto forma di immagine.
	        Le immagini presenti nei documenti devono essere nel formato SVG.
		In questo modo si ottiene una qualità superiore dell'immagine, anche nel caso di ridimensionamento. 
		Qualora le immagini non possano essere salvate in formato vettoriale, si preferisce il formato PNG.

\end{itemize}

\paragraph*{Norme tipografiche}
In questa sezione vengono trattate le convenzioni riguardanti tipografia, ortografia e stile per il testo dei documenti. 
\begin{itemize}
\item \textbf{Parentesi}: il testo racchiuso tra parentesi non deve presentare spazi adiacenti al carattere di parentesi e non deve terminare con un simbolo di punteggiatura;
\item \textbf{Punteggiatura}: un carattere di punteggiatura non deve essere mai preceduto da uno spazio;
\item Le lettere maiuscole vanno poste dopo i punti, i punti di domanda, i punti esclamativi e all'inizio di ogni voce per gli elenchi puntati, oltre a dove previsto dalla lingua italiana (o dalla lingua inglese per la documentazione da consegnare in tale lingua). Il nome del \glossaryItem{team}, del \glossaryItem{progetto}, dei documenti, dei ruoli, delle \glossaryItem{fasi} di lavoro e le parole Proponente e \glossaryItem{Committente} devono sempre cominciare con lettera maiuscola;
  \item \textbf{Elenchi puntati}: devono terminare con punto e virgola, ad eccezione dell'ultimo che deve terminare con un punto.
\end{itemize}

\textbf{Stile di testo} 
\begin{itemize}
\item \textbf{Corsivo}  nei seguenti \glossaryItem{casi}:
  \begin{itemize}
  \item Citazioni;
  \item Abbreviazioni;
  \item Riferimenti ad altri documenti;
  \item Nomi di documenti;
  \item Nomi di programmi o \glossaryItem{framework};
  \item Ruoli di \glossaryItem{progetto};
  \item Nomi di societ\`a o aziende;
  \item Nome del gruppo;
  \item Vocaboli in lingua inglese.
  \end{itemize}
\item \textbf{Grassetto} per i seguenti \glossaryItem{casi}:
  \begin{itemize}
  \item Negli elenchi puntati per evidenziare il concetto sviluppato dal punto;
  \item Per evidenziare parole chiave.
  \end{itemize}
\item \textbf{Monospace} per formattare parti di codice o riportare indirizzi web e percorsi;
\item \textbf{Maiuscolo} per intere parole solo nei \glossaryItem{casi} di acronimi.
\end{itemize}

\textbf{Citazione di termini da glossario}
I termini presenti nel glossario possono essere utilizzati durante la stesura dei documenti riportandoli con la sintassi \glossaryItem{nome elemento}.


\textbf{Formati} 
\begin{itemize}
\item \textbf{Indirizzi email}: deve essere utilizzato il comando \LaTeX \textit{url};
\item \textbf{Date}: le date presenti nei documenti devono seguire la notazione standard dettata da ISO 8601:

  \begin{center}
    AAAA-MM-GG
  \end{center}
  
  dove:
  \begin{itemize}
  \item AAAA rappresenta l'anno con quattro cifre;
  \item MM rappresenta il mese con due cifre;
  \item GG rappresenta il gorno con due cifre.
  \end{itemize}
\item Sigle previste nella documentazione:
  \begin{itemize}
  \item \textbf{AdR} = Analisi dei Requisiti;
  \item \textbf{PdP} = Piano di \glossaryItem{Progetto};
  \item \textbf{NdP} = Norme di \glossaryItem{Progetto};
  \item \textbf{SF} = Studio di Fattibilità;
  \item \textbf{PdQ} = Piano di Qualifica;
  \item \textbf{ST} = Specifica Tecnica;
  \item \textbf{MU} = Manuale Utente;
  \item \textbf{DdP} = Definizione di prodotto;
  \item \textbf{RR} = Revisione dei Requisiti;
  \item \textbf{RP} = Revisione di Progettazione;
  \item \textbf{RQ} = Revisione di Qualifica;
  \item \textbf{RA} = Revisione di Accettazione.
  \end{itemize}
\item \textbf{Riferimenti a documenti}: qualora ci si riferisca ad un documento è necessario specificarne il nome completo e la \glossaryItem{versione} di riferimento.
\end{itemize}



%Le immagini presenti nei documenti devono essere nel formato \textit{SVG}. %glossario
%In questo modo si ottiene una qualità superiore dell'immagine, anche nel caso di ridimensionamento. Qualora le immagini non possano essere salvate in formato vettoriale, si preferisce il formato \textit{JPEG}.

\subsubsection{3. Attivit\`a di manutenzione}   %NUOVO
\paragraph*{Classificazione dei documenti}
\textbf{Documenti informali} \\
Un documento è informale dal momento della sua creazione fino alla sua approvazione dal Responsabile di \glossaryItem{Progetto}. Tali documenti sono da considerarsi esclusivamente ad utilizzo interno. \\
\textbf{Documenti formali}  \\
Un documento diventa formale al momento della sua approvazione da parte Responsabile di \glossaryItem{Progetto}. Un documento formale pu\`o essere distribuito alla propria linea di distribuzione.

\paragraph*{Versionamento dei documenti}
Ciascun documento deve essere corredato di un numero di \glossaryItem{versione} avente la seguente struttura:

\begin{center}
  \textbf{v X.Y.Z}
\end{center}

In cui:
\begin{itemize}
\item L'indice x deve essere incrementato dal Responsabile di \glossaryItem{Progetto} e indica l'approvazione del documento. Ad un cambio di indice per x, gli indici y e z vanno azzerati;
\item L'indice y viene incrementato dal verificatore del documento a seguito di ogni \glossaryItem{verifica}. Al suo \glossaryItem{incremento} deve corrispondere l'azzeramento dell'indice z;
\item L'indice z indica la modifica del documento da parte di un redattore.
\end{itemize}






%##################################################
\subsection{Processo di verifica}
\subsubsection{Scopo del processo}
Tale \glossaryItem{processo} ha come obiettivo confermare che ogni componente prodotta durante il \glossaryItem{ciclo di vita del software} sia conforme agli standard decisi dal \glossaryItem{team}.
\subsubsection{Risultati attesi dal Processo}
Il \glossaryItem{processo} di \glossaryItem{verifica} deve raggiungere questi obiettivi:
\begin{itemize}
\item Individuazione e applicazione di una strategia per la \glossaryItem{verifica};
\item I criteri che la \glossaryItem{verifica} deve accertare.
\end{itemize}

\subsubsection{1. Attivit\`a di verifica}
A seguito viene descritto il \glossaryItem{processo} di \glossaryItem{verifica} per la documentazione e per la codifica affinchè i Verificatori possano garantire che al termine del loro lavoro, la documentazione e il codice prodotto sia conforme alla struttura decisa dal \glossaryItem{Team} e che sia corretta rispetto alle regole decise dal Gruppo e dal punto di vista sintattico per il linguaggio deciso per il documento.

\paragraph*{Procedura di \glossaryItem{verifica} dei documenti}
\begin{itemize}
\item \textbf{Controllo sintattico e del periodo}: Un passaggio preliminare per la correzione di errori grammaticali viene garantito dal software \textit{GNU Aspell} in \glossaryItem{fase} di \glossaryItem{pre}-commit. Questo passaggio non risulta sufficiente a garantire la correttezza del documento perciò, in \glossaryItem{fase} di \glossaryItem{verifica}, il Verificatore incaricato deve sottoporre il documento in esame ad un walkthrough per evidenziare e correggere errori sintattici;
  \item \textbf{Rispetto delle Norme di \glossaryItem{Progetto}}: Ciascun Verificatore incaricato di esaminare la correttezza di un documento deve sottoporre il testo ad una \glossaryItem{verifica} delle convenzioni stilistiche e di forma scelte dal gruppo e reperibili nel documento Norme di \glossaryItem{Progetto} pi\`u aggiornato presente nel \glossaryItem{repository} dei documenti;
  \item \textbf{\glossaryItem{Verifica} dei vocaboli del Glossario}: Il Verificatore deve controllare la presenza effettiva dei termini marcati all'interno del Glossario e che questi presentino anche la loro descrizione;
  \item \textbf{In caso di \glossaryItem{anomalie}}: Il Verificatore deve aprire una issue nel \glossaryItem{repository} dei documenti dando riferimenti al redatore su come procedere per apportare le modifiche necessarie al documento esaminato.

\paragraph*{Procedura di \glossaryItem{verifica} della Progettazione Architetturale}
Il \textit{Verificatore} deve accertarsi che tutti i requisiti individuati nell'\AnalisiDeiRequisiti vengano soddisfatti
nell'architettura proposta nel documento di \SpecificaTecnica.
Nel caso in cui un requisito non sia soddisfatto, il \textit{Verificatore} deve comunicarlo ai \textit{Progettisti}, i quali
si faranno carico della revisione mirata dell'architettura. 

\end{itemize}

\paragraph*{Procedura di verifica del codice}
\begin{itemize}
\item \textbf{Controlli statici}: Vengono eseguiti controlli di tipo statico sul codice tramite l'utilizzo del \textit{linter} \textit{TsLint}, che permette di trovare eventuali errori di forma o logici prima della compilazione.

\item \textbf{Controlli dinamici}: Dopo una compilazione avvenuta con successo e dopo aver passato con successo i test di tipo statico vengono eseguiti i test di tipo dinamico tramite il programma \textit{mocha}, che si occupa di eseguire i test di unità e segnalare eventuali test falliti. È possibile generare un report di tipo \textit{xUnit}, che consente di avere un storico dell'esecuzione dei test e di tracciare il loro andamento. I test saranno scritti in TypeScript utilizzando la seguente struttura:
\begin{verbatim}
describe("NomeDellaClasse", () => {
    let subject : TipoDelSoggetto;


describe("NomeDellaClasse", () => {
    let subject : TipoDelSoggetto;

    beforeEach(function () : void {
        // Questa funzione permette di eseguire un'azione
        // prima di ogni test
    });

    //Un test
    describe("nomeTest", () => {
        it("CosaVieneVerificato", () => {

            // Codice per la verifica

            // Asserzione
            Chai.expect(UnValore).to.equal(UnAltro);
        });
    });
  
\end{verbatim}

\end{itemize}

\subsection{Processo di validazione}
\subsubsection{Scopo del processo}
Il processo di validazione serve per determinare se il prodotto finale è conforme a quanto è stato pianificato. Si procederà dapprima ad una validazione interna da parte dei componenti del team e poi sarà il proponente a rivalidare il prodotto, per garantire l'indipendenza del processo. \\
L'attività di validazione consiste nel condurre i test per poi analizzare i risultati e assicurarsi che il prodotto rispecchi le indicazioni per cui è stato sviluppato. \\
Per le norme sulla sintassi dei test si rimanda alla sezione dedicata al processo di verifica.

\subsubsection{Risultati attesi dal processo}
Il processo di validazione deve produrre una conferma sul fatto che il prodotto realizzato è conforme ai requisiti e soddisfa i bisogni espressi dal committente.

\subsubsection{1. Attività di validazione}
\paragraph*{Procedura di validazione}
I test verranno dapprima eseguiti dai verificatori, che ne tracceranno il risultato. Al termine dell'esecuzione dei test, il responsabile analizzerà i risultati e deciderà se accetarli, chiedere una ripetizione di alcuni test oppure di tutti, eventualmente con verificatori diversi. \\
Una volta accettati, i risultati verranno consegnati al proponente dal responsabile, che lo informerà sulle modalità di esecuzione indipendente della validazione.
