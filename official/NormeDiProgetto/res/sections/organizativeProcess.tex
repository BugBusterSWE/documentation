\section{Processi Organizzativi}

%nei \glossaryItem{processi} primari: come si fa lo studio di fattibililtà?

\subsection{Processo di gestione}

\subsubsection{Scopo}

Lo scopo di questo \glossaryItem{processo} è la realizzazione e la comunicazione del \PianoDiProgetto ai membri del gruppo. Tutto ciò aiuta nella determinazione dell'ambito della gestione del \glossaryItem{progetto} e delle attività tecniche, identifica gli output del \glossaryItem{processo} e ne stabilisce un calendario.

\subsubsection{Risultati osservabili}

Questo \glossaryItem{processo} produce:
\begin{itemize}

\item L'analisi di fattibilità degli obiettivi prefissati;
\item Una stima esatta dei \glossaryItem{task} necessari per completare il lavoro;
\item Un piano per l'esecuzione del lavoro programmato.

\end{itemize}

\subsubsection{Attività di pianificazione}

\paragraph*{Soglia oraria giornaliera} È stato deciso di porre la soglia oraria ai seguenti orari: la mattina dalle ore 8:30 alle ore 11:30, il pomeriggio dalle ore 14 alle 18. Le ore di lavoro al di fuori di questa fascia oraria saranno considerate come lavoro straordinario.

\paragraph*{Assegnazione dei \glossaryItem{task}} Il responsabile si occuperà di assegnare i \glossaryItem{task} in base alla disponibilità dei membri del gruppo e alla rotazione dei ruoli.

\paragraph*{Assegnazione \glossaryItem{verifica} delle pull request} Le pull request create dall'assegnatario di un \glossaryItem{task} per la \glossaryItem{verifica} sono assegnate in automatico. L'assegnatario di un \glossaryItem{task}, nel momento di creazione di una Pull Request, dovrà lasciare vuoto il campo \textit{Assignee}. Il procedimento d'assegnazione dovrà valutare il verificatore in base ai seguenti criteri:
\begin{itemize}
\item Il verificatore della Pull Request deve essere diverso dall'assegnatario del \glossaryItem{task}.
\item Il verificatore è un membro del gruppo \textit{BugBusters} con il numero minore di Pull Request assegnate.
\item In caso di riscontri multipli sarà selezionato il primo membro identificato come idoneo.
\end{itemize}

Il fine \`e di diminuire il tempo di \glossaryItem{verifica} rimuovendo il tempo richiesto dal responsabile per l'assegnazione della Pull Request.

\paragraph*{Rotazione dei ruoli}

Durante lo sviluppo del \glossaryItem{progetto} vi sono diversi ruoli che devono essere ricoperti dai membri del gruppo. Il \textit{Piano di \glossaryItem{Progetto}} si occupa di assegnare le attività principali e specifiche per ogni ruolo. Il \textit{Responsabile} di \glossaryItem{progetto} ha l'onere di far rispettare i ruoli assegnati durante l'attività, mentre il \textit{Verificatore} deve essere in grado di individuare eventuali incongruenze e di segnalarle opportunamente.
\paragraph*{Incongruenze tra ruoli e risoluzioni}
Esistono due tipi principali di incongruenze riscontrabili:
\begin{itemize}

\item Il compito svolto non fa parte dei compiti propri di un dato ruolo;
\item La stessa persona \glossaryItem{verifica} ciò che ha precedentemente prodotto.
\end{itemize}

È importante tenere conto inoltre di altri tre fattori per un buono svolgimento del \glossaryItem{progetto}:
\begin{itemize}

\item Una persona non deve impiegare più del 50\% delle ore di lavoro in un unico ruolo: questo impedirebbe una corretta rotazione dei ruoli;
\item Una persona non dovrebbe ricevere più di un \glossaryItem{task} al giorno, onde evitare un eccessivo carico di lavoro, che rischierebbe cos\`i di non essere svolto correttamente;
\item I ruoli verranno ruotati ogni due settimane.

\end{itemize}

Il responsabile nell'azione di assegnazione dei \textit{ticket} dovr\`a indicare il ruolo che il destinatario ricopre nell'adempimento di tale \glossaryItem{task}. Questo \`e necessario per il rendiconto economico. I ruoli sono assegnati basandosi sul \PianoDiProgetto.



\paragraph*{Codice univoco per le decisioni prese in sede di riunione}
Le decisioni prese durante le riunioni interne verranno tracciate con una notazione dedicata.
La notazione sar\`a del tipo: \\
RI-ggmm-s \\
dove:
\begin{itemize}
\item RI: identificatore per \textit{Riunione Interna};
\item gg: identificatore per il \textit{giorno};
\item mm: identificatore per il \textit{mese};
\item s: identificatore per il \textit{punto} nel verbale.
\end{itemize}



\paragraph*{Identificazione dei ruoli}

\begin{description}

\item[Amministratore] \hfill \\ L'\textit{Amministratore} equipaggia, organizza e gestisce l'ambiente di lavoro e di produzione. Collabora con il \textit{Responsabile} alla stesura delle \textit{Norme di \glossaryItem{Progetto}} e del \textit{Piano di \glossaryItem{Progetto}}, e si assicura di:
  \begin{itemize}

  \item Attuare scelte tecnologiche concordate con il \textit{Responsabile};
  \item Gestire le liste di distribuzione e assicurarne il rispetto;
  \item Controllare \glossaryItem{versioni} e configurazioni del prodotto;
  \item Risolvere i problemi legati alla gestione dei \glossaryItem{processi}.
    
  \end{itemize}

\item[Analista] \hfill \\ L'\textit{Analista} \`e responsabile dell'attivit\`a di analisi, e ha il compito di comprendere appieno il dominio applicativo. Si occupa di redigere lo \textit{Studio di Fattibilit\`a} e l'\textit{Analisi dei Requisiti}.

\item[Progettista] \hfill \\ Il \textit{Progettista} \`e il responsabile dell'attivit\`a di progettazione, ha una profonda conoscenza dello \textit{stack tecnologico} utilizzato e presenta adeguate competenze tecniche. Ha una forte influenza sugli aspetti tecnici e tecnologici del \glossaryItem{progetto}.

\item[Programmatore] \hfill \\ Il \textit{Programmatore} ha responsabilit\`a circoscritte, e si occupa dell' attivit\`a di codifica, rispettando le \textit{Norme di \glossaryItem{Progetto}}. Si occupa di realizzare il prodotto e le componenti di ausiliarie funzionali all'esecuzione delle prove di \glossaryItem{verifica} e \glossaryItem{validazione}.

\item[Responsabile] \hfill \\ Il \textit{Responsabile} ha l'ultima voce in capitolo per quanto concerne le decisioni sul \glossaryItem{progetto}, \`e il responsabile ultimo dei risultati, infatti approva l'emissione dei documenti, oltre a redigere insieme all'\textit{Amministratore} il \textit{Piano di \glossaryItem{Progetto}}. Il \textit{Responsabile} ha diverse responsabilit\`a, di seguito elencate:
  \begin{itemize}

  \item Pianificare e organizzare lo sviluppo del \glossaryItem{progetto}, stimando tempi, costi e assegnazione delle attivit\`a ai componenti del gruppo;
  \item Riportare lo stato del \glossaryItem{progetto} ai \glossaryItem{committenti};
  \item Analizzare i \glossaryItem{rischi} in cui è possibile incorrere, monitorarli e prendere provvedimenti a riguardo;
  \item Stabilire una \textit{way of working} per ogni componente del gruppo, per esercitare un'influenza positiva sulle performance del gruppo.
    
  \end{itemize}

\item[Verificatore] \hfill \\ Il \textit{Verificatore} organizza ed attua le attivit\`a di \glossaryItem{verifica} e controlla che le attivit\`a siano conformi alle norme definite. Redige nel \textit{Piano di Qualifica} la parte che documenta le attivit\`a svolte e i risultati ottenuti, confrontandoli con le metriche del \PianoDiQualifica. Le ore assegnate al verificatore devono corrispondere almeno al 30\% delle ore totali suddivise tra i ruoli.
  
\end{description}


%come organizziamo i \glossaryItem{task}
%come organizziamo le attività

\subsection{Processo di infrastruttura}

\subsubsection{Attività di organizzazione dell'infrastruttura}

\paragraph*{Sistema di versionamento scelto} Il gruppo ha stabilito di utilizzare per tutti i file che richiedono versionamento 
il \glossaryItem{CVS} \glossaryItem{Git}, tramite il servizio \glossaryItem{GitHub}. \\
Nonostante esistano varie alternative, come \textit{Mercurial}, \textit{Subversion} e \textit{Bazaar}, è stato scelto Git in quanto soddisfa
appieno le necessità di versionamento dei file per questo progetto e, inoltre, permette di versionare i file localmente, senza bisogno di una connessione
internet attiva. Le scelte di Git e GitHub sono state effettuate anche perché più membri del gruppo avevano già avuto l'opportunità 
di lavorare con tali strumenti. \\

\textbf{Utilizzo di Git/GitHub}
È stato stabilito che ogni commit debba appartenere a un \glossaryItem{task}. I \glossaryItem{task}, assegnati dal \textit{Responsabile} ai vari membri del gruppo corrispondono a \glossaryItem{issue} su \glossaryItem{GitHub}. All'apertura della \glossaryItem{issue} corrisponde alla creazione di un branch, il cui nome sarà conforme alla seguente nomenclatura: \textbf{issue\#\textit{numero issue}}. In caso di una attivit\`a, viene creata una issue su \glossaryItem{GitHub} che corrisponde ad un branch (creato dall'assegnatario dell'attivit\`a) con la seguente nomenclatura: \textbf{activity\#\textit{numeroissue}}. Da li, tutti i \glossaryItem{task} di quella attivit\`a deriveranno da quel branch. L'assegnatario del \glossaryItem{task}, durante il suo svolgimento, viene coadiuvato da un sistema di \glossaryItem{Continuous Integration}. Al termine dello svolgimento di tale \glossaryItem{task} l'assegnatario esegue una \glossaryItem{Pull Request}, seguita da un controllo di contenuto da parte di un verificatore umano. La Pull Request deve avere label e \glossaryItem{Milestone} uguali a quelle del \glossaryItem{task} corrispondente. In seguito il verificatore
chiude il \glossaryItem{task}. Al completamento di un \glossaryItem{task} la issue ad esso corrispondente va chiusa. In caso di conflitto il verificatore operer\`a congiuntamente all'assegnatario della issue per risolvere problema.
In questa maniera vengono inseriti dei \glossaryItem{Configuration Item} che sono stati verificati sia da sistemi di integrazione continua e sia da umani, coprendo rispettivamente la parte prettamente sintattica (scrittura di codice corretto) e la parte di contenuto (che una macchina non pu\`o controllare), comportando nel \glossaryItem{repository} l'inserimento di codice sempre corretto e verificato e generando \glossaryItem{baseline} sempre consistenti.

\paragraph*{Sistema di ticketing} I \glossaryItem{task}, come già accennato, vengono assegnati tramite la piattaforma \glossaryItem{TeamWork}. All'assegnazione di un ticket su questa piattaforma corrispondono l'apertura di una \glossaryItem{issue} su \glossaryItem{GitHub} ed una notifica tramite il sistema di messaggistica \glossaryItem{Slack}.

\paragraph*{Sistema di CI}Il sistema di \glossaryItem{CI} controlla che il contenuto di ogni \glossaryItem{repository} sia conforme alle \NormeDiProgetto. L'integrazione con \glossaryItem{GitHub} permette per ogni \glossaryItem{task} la \glossaryItem{verifica} di ciascun \textit{commit}. Attualmente, i sistemi di \glossaryItem{CI} utilizzati sono \glossaryItem{Travis} e \glossaryItem{Jenkins}.

\paragraph*{Sistema di \glossaryItem{Dashboard}}Il responsabile utilizza una \glossaryItem{dashboard} creata grazie ai seguenti strumenti:
\begin{itemize}

\item \glossaryItem{TeamWork} per l'assegnazione e la gestione dei \glossaryItem{task}
\item \glossaryItem{Chronos}, software sviluppato internamente al gruppo; è in grado di fornire al \textit{Responsabile} una visione globale dell'andamento dei \glossaryItem{processi}
\item \glossaryItem{Mephisto} \glossaryItem{progetto} proveniente dal \textit{fork} di \glossaryItem{Tracy}, software di \glossaryItem{tracciamento} dei requisiti sviluppato da un gruppo di questo corso. Tale software è stato accuratamente modificato e riadattato alle nostre necessità.
\end{itemize}

\paragraph*{Comunicazioni} 
\begin{itemize}
	\item \textbf{Interne} 
	Per le comunicazioni interne deve essere utilizzata l'applicazione Slack, la quale permette sia comunicazioni di gruppo sia messaggi
	privati. Essa inoltre mette a disposizione la possibilità di creare più canali di comunicazione su argomenti specifici (ad esempio per il documento 
	\textit{Norme di Progetto} può essere creato il canale ad esso dedicato). I membri del gruppo saranno dunque tenuti ad inviare i propri messaggi
	nel canale più adatto ad essi. \\
	Per le videochiamate vengono utilizzati gli applicativi \textit{Google Hangouts} e \textit{TeamSpeak}. \\
	I componenti del gruppo sono tenuti a prestare attenzione alla quantità e alla qualità dei messaggi spediti al fine 
	di non creare confusione e difficoltà di comunicazione. \\
	\item \textbf{Esterne}
	Per le comunicazioni esterne è stato creato un indirizzo di posta elettronica dedicato: \href{mailto:me@somewhere.com}{bugbusterswe@gmail.com}. \\
	Il \textit{Responsabile di Progetto} ha l'incarico di mantenere i contatti tra il team e le componenti esterne utilizzando 
	tale indirizzo di posta elettronica. Inoltre egli ha il dovere di informare i membri del gruppo delle discussioni avvenute con
	componenti esterne tramite l'applicazione Slack.
\end{itemize}

\subsection{Processo di miglioramento}

Come specificato ed approfondito nel \PianoDiQualifica, il gruppo ha stabilito di adottare il ciclo di Deming (\glossaryItem{PDCA}).
%da togliere?

\subsection{Processo di formazione}

Viene considerato fondamentale lo studio della teoria del \textit{Software Engineering} dalle fonti proposte durante il corso.
Inoltre, il \glossaryItem{team} ha stabilito che ogni membro del gruppo debba impegnarsi autonomamente nell'apprendimento delle tecnologie richieste dal \glossaryItem{progetto}.
