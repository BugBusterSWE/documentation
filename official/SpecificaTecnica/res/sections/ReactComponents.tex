\section{Componenti grafiche dell'editor}
\subsection{React branch}
Uno dei requisiti opzionali espressi nel capitolato consiste nell'usare React/Flux anziché Angular per implementare un front-end per MaaS basato su componenti riusabili.\\
React è una libreria JavaScript usata da Facebook e Instangram per creare interfacce utente ed è considerata la View del pattern MVC.\\
Il problema che React vuole risolvere è quello di costruire applicazioni facilmente scalabili e con dati che continuano a variare velocemente nel tempo.\\ 
L'architettura logica che usa React viene riassunta dalla seguente figura:
\begin{figure}[H]
  \centering
  \includegraphics[width=.8\textwidth]{res/img/react.png}
  \caption{Architettura logica di React}
  \label{fig:salvataggio}
\end{figure}
\textbf{Spiegazione}: lo stato dell'applicazione viene aggiornato da eventi a partire dal Server o dal browser. Questi eventi vengono indirizzati ai componenti React, scritti in JavaScript, i quali hanno il compito di descrivere la business logic dell'applicazione. React genera una rappresentazione virtuale del DOM che utilizza per stabilire quali operazioni eseguire con precisioni sul DOM vero e proprio; infine renderizza sul browser. L'uso di una rappresentazione virtuale del DOM rende React molto veloce poiché ad ogni cambiamento aggiorna solo le componenti interessate.\\
Disegnare la View di un'applicazione con React consiste quindi nel costruire componenti e a pensare a come incapsularli tra loro. Questo modo di lavorare produrrà codice che soddisferà il principio ``Separation of Concerns'', ne faciliterà il riuso, il testing e, in generale, sarà più facile da manutenere.
\subsection{Componenti}
Una componente React dovrebbe idealmente occuparsi di un solo compito, seguendo il principio ``Single Responsability''.\\
Le componenti individuate per l'interfaccia utente di MaaS sono le stesse rappresentate dal modello dei dati (un sottoinsieme di quello individuato dall'editor):
\begin{enumerate}
\item{\textbf{Dashboard}}
  Questo componente è la radice che contiene tutti gli altri e rappresenta graficamente la Dashboard utente.
\item{\textbf{DashRow}}  rappresentazione grafica di una DashRow, una riga di una DashBoard.
\item{\textbf{Collection}}  rappresentazione grafica di una Collection.
\item{\textbf{Index}}  rappresentazione grafica di un Index (un insieme di colonne)
\item{\textbf{Column}}  rappresentazione grafica di una colonna
\item{\textbf{Document}} rappresentazione grafica di un Document
\item{\textbf{Row}}  rappresentazione grafica di una Row (una riga di un Document)
\item{\textbf{Cell}}  rappresentazione grafica di una Cell
\item{\textbf{Action}}  rappresentazione grafica di una Action
\end{enumerate}
\subsection{Gerarchia}
Una volta individuate le componenti, è possibile organizzarle in una gerarchia. Le componenti che appaiono all'interno di un'altra componente appariranno in una relazione di padre/figlio nella seguente gerarchia.
\begin{itemize}
\item DashBoard
  \begin{itemize}
  \item Collection
    \begin{itemize}
    \item Index
      \begin{itemize}
      \item Column
      \end{itemize}
    \item Action
    \end{itemize}
  \item Document
    \begin{itemize}
    \item Row
    \item Action
    \end{itemize}
  \item Cell
  \end{itemize}
\end{itemize} 
Il modello di dati che viene passato alla componente di più alto livello (la Dashboard) viene poi passato alle componenti figlie mediante props (abbreviazione di properties).
Se viene apportato qualche cambiamento al modello dei dati, occorre refreshare la pagina e la user interface verrà aggiornata. Il flusso di dati unidireazionale che usa React permette di scrivere codice modulare e dalla veloce esecuzione.
\subsection{Stato}
Una componente può avere uno stato o ricevere dati attraverso props dai suoi padri.
Non è stato identificato uno stato in quanto le pagine la cui interfaccia è descritta da queste componenti non fanno altro che tradurre una specifica DSL (espressa in formato JSON) in una rappresentazione grafica. La Dashboard riceve quindi tale modello dei dati (la specifica DSL) e trasmette le informazioni opportune alle componenti figlie. Questo accade poichè le pagine utente, tutte tranne l'editor, sono statiche.
Le componenti che costituiscono la view dell'editor, invece, hanno uno stato rappresentato, ad esempio, dal valore che viene attribuito a un campo, o dal valore di una checkbox. In generale, lo stato raggruppa tutte le informazioni ``esterne'' che sono date dall'utente e, soprattutto, che variano nel tempo e danno dinamicità a una pagina.\\
Dati passati attraverso props non rappresentano uno stato così come dati che possono essere ottenuti da una computazione di dati già disponibili.\\

