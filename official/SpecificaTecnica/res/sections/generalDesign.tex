\section{Descrizione architettura}
\subsection{Metodo e formalismo di specifica}
Le scelte progettuali di MaaS sono state frtemente influenzate dallo stack tecnologico usato. \\
MaaS è basato su Node.js, e conseguentemente è scritto in JavaScript; questo linguaggio lascia molta libertà al programmatore nella scelta della tecnica da utilizzare per implemetare pattern come l'incapsulamento e l'ereditarietà. Al contrario di altri linguaggi orientati agli oggetti, come C++ o Java, non è presente un costrutto esplicito per definire classi, né un controllo statico dei tipi utilizzati. Per questo il team ha scelto, soprattutto per la codifica dell'editor visuale, di utilizzare il linguaggio TypeScript, che aggiunge, come già detto, la possibilità di definire classi ed interfacce. \\
La progettazione è stata influenzata pesantemente anche dal framework back end scelto, e consigliato dal proponente: LoopBack. Questo framework fornisce, di base, molte delle caratteristiche richieste per MaaS, come il salvataggio dei dati su MongoDB, la gestione di autenticazione e autorizzazioni degli utenti e la registrazione al sistema. LoopBack è basato su Node.js, e la sua progettazione è difficilmente rappresentabile in un diagramma delle classi: molto spesso, ad esempio, un parametro di una funzione è una funzione stessa. Questo a portato alla definizione, nel diagramma che spiega l'architettura del back end, di interfacce fittizie per indicare i parametri funzione (applicazione del design pattern Strategy). \\
La progettazione della parte back end ha quindi seguito un approccio top-down: inizialmente sono stati definiti i componenti principali del sistema a partire dalle classi presenti in LoopBack, le relazioni tra essi e infine l'interfaccia REST da esporre al client. \\
I diagrammi delle classi permettono di mostrare l'architettura generale del sistema, ma in un mondo orientato alle funzioni non sono sufficienti per descrivere l'intero sistema. Per questo, vengono mostrati anche diagrammi di sequenza e attività, che permettono di definire le interazioni tra le componenti, senza preoccuparsi della loro classificazione. In questo modo è possibile esprimere alcuni meccanismi tipici di un'applicazione REST like, come il modo in cui agiscono i middleware di LoopBack. 
%todo frontend!!!!!!!!!
I diagrammi di deployment, dei package, delle classi, di sequenza e di attività presentati utilizzano la specifica UML (\textbf{U}nified \textbf{M}odeling \textbf{L}anguage) 2.0.