\newpage
\section{Descrizione architettura}
\subsection{Metodo e formalismo di specifica}
Le scelte progettuali di MaaS sono state fortemente influenzate dallo stack tecnologico usato. \\
L'intera applicazione è stata progettata per essere scritta con un unico linguaggio: Typescript. Questo linguaggio è un superset di Javascript che viene compilato in normale Javascript. La scelta è ricaduta su Typescript perchè rende più leggibile ed intuitivo il codice prodotto, oltre a permettere i controlli statici che in Javascript non sono presenti.
La progettazione del backend è stata influenzata pesantemente anche dal framework scelto: ExpressJS. Questo framework è basato su Node.js e permette di creare velocemente ed intuitivamente dei webserver per l'esposizione di API REST. \\
Per esporre l'architettura dell'applicazione si procederà con approccio top-down, partendo cioè da una visione generale delle componenti che distinguono il sistema, per poi analizzare in dettaglio la conformazione di tali componenti. Per descrivere in maniera formale l'architettura verranno impiegati lo standard UML 2.0 per i diagrammi dei package e delle classi e lo standard UML 2.4 per i diagrammi di attività e sequenza. \\

Viene fatto uso inoltre di un codice a colori per distinguere la provenienza dei moduli dell'applicazione. In particolare:
\begin{itemize}
\item in colore \textbf{giallo} i moduli da implementare;
\item in colore \textbf{verde} vengono proposti i moduli/librerie importati dal Core delle tecnologie utilizzate e da terze parti;
\end{itemize}

I diagrammi delle classi che permettono di mostrare l'architettura generale del sistema vengono affiancati anche dai diagrammi di sequenza e attività, che permettono di definire le interazioni tra le componenti, senza preoccuparsi della loro classificazione. In questo modo è possibile esprimere alcuni meccanismi tipici di un'applicazione REST-like, come il modo in cui agiscono i middleware. 

\subsection{Architettura generale}
L'architettura del progetto si divide in una componente Client, rappresentata da un'applicazione frontend accessibile da un browser e in una componente WebServer, nella quale risiede il backend. 

\subsection{Conformazione generale dell'architettura}
L'architettura generale di MaaS si può dividere in 3 macrocomponenti:
\begin{itemize}
\item \textbf{Server REST} 
\item \textbf{Client} 
\item \textbf{Editor}
\end{itemize}
\begin{figure}[h]
\centering
\includegraphics[width=0.8\textwidth]{res/sections/GeneralArchitecture.png}
\caption{Diagramma di deployment per l'architettura}
\end{figure}
L'architettura proposta segue il Design Pattern MVC. In particolare i ruoli di Model e Controller verranno implementati a livello di server, mentre il ruolo di View viene affidato al frontend. L'interfaccia tra le due componenti verrà gestita grazie ad un set di API disposto dal server REST: in questo modo si garantisce una totale indipendenza tra frontend e backend e sono possibili sviluppi futuri anche in altre piattaforme (ad esempio in app mobile) senza dover creare dei sistemi di integrazione ad-hoc. \\
Le tre macrocomponenti verranno descritte in dettaglio in seguito su questo documento.
\subsection{Interfaccia REST-like}
Il backend si basa su uno stile REST-like, ovvero con le seguenti caratteristiche:
\begin{itemize}
\item stato dell'applicazione e funzionalità divisi in risorse web;
\item ogni risorsa è unica e indirizzabile attraverso un URI (\textbf{U}niform \textbf{R}esource \textbf{I}dentifier);
\item tutte le risorse sono condivise come interfaccia uniforme per il trasferimento di stato tra client e risorse. Questo trasferimento consiste in:
\begin{itemize}
\item un insieme vincolato di operazioni ben definite;
\item un insieme vincolato di contenuti, opzionalmente supportato da codice a richiesta;
\item un protocollo:
\begin{itemize}
\item client-server;
\item privo di stato;
\item memorizzabile in cache;
\item a livelli.
\end{itemize}
\end{itemize}
\end{itemize}
REST utilizza il concetto di risorsa (aggregato di dati con un nome, l'URI, e una rappresentazione interna), sulla quale è possibile invocare operazioni CRUD (\textbf{C}reate, \textbf{R}ead, \textbf{U}pdate, \textbf{D}elete) con la seguente corrispondenza:
\begin{table}[H]
\centering
\label{CRUD}
\begin{tabular}{| >{\centering}p{3cm} | >{\centering}p{5cm} | >{\centering}p{6cm} |}
\hline
\textbf{Risorsa} & \textbf{URI della collection} \newline es. http://maas.com/users & \textbf{URI della risorsa} \newline es. http://maas.com/users/10 \tabularnewline \hline
\textbf{GET} & Fornisce informazioni sui membri della collection. & Fornisce una rappresentazione dell'elemento della collection indicato. \tabularnewline \hline
\textbf{PUT} & Non usata. & Sostituisce una rappresentazione dell'elemento della collection indicato. Se non esiste lo crea.  \tabularnewline \hline
\textbf{POST} & Crea un nuovo elemento della collection. La URI del nuovo elemento è generata automaticamente. & Non usato. \tabularnewline \hline
\textbf{DELETE} & Non usata. & Cancella l'elemento della collection indicato. \tabularnewline \hline
\end{tabular}
\caption{Tabella delle operazioni CRUD}
\end{table}
Per la rappresentazione dei dati si è scelto di utilizzare JSON perché si integra molto bene con i framework utilizzati e con il linguaggio JavaScript. Questo non è vero per XML (e\textbf{X}tensible \textbf{M}arkup \textbf{L}anguage) o CSV (\textbf{C}omma \textbf{S}eparated \textbf{V}alues), che richiederebbero librerie specifiche. Inoltre JSON è molto meno verboso e molto più flessibile di XML, e si adatta molto bene al dominio dell'applicazione.
