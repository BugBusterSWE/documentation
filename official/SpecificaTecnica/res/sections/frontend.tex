\section{Frontend}
\subsection{Formalismo utilizzato}
%% DA FARE

\subsection{Descrizione generale}

Il frontend della nostra applicazione andrà a costituire il ruolo di View nel pattern MVC. In particolare tale componente dell'applicazione è costituita da un sottosistema che implementa l'architettura Flux proposta da Facebook. Tale architettura si basa sul creare un sistema che abbia un data-flow unidirezionale al fine di semplificare la struttura dell'applicazione stessa,di considerare le viste come uno snapshot di un dato lasso di tempo dei dati.
Nella progettazione secondo l'architettura Flux si è seguito in particolare il principio che nessuna classe modifichi direttamente lo stato di un'altra ma che creino semplicemente delle Action per comunicare il cambiamento.

IMMAGINE ARCHITETTURA FLUX FACEBOOK

In una architettura Flux vengono distinti 4 componenti fondamentali:

\begin{itemize}
\item Action: Rappresenta un messaggio tra le componenti
\item Dispatcher: funge da hub centrale per le action e si occupa di distribuirle al giusto store
\item Store: contengono la logica applicativa del frontend e lo stato dei dati dall'ultimo update. Si occupano di fornire i dati alle viste, quando queste li richiedono
\item View: Sono la parte visiva dell'applicazione e, nel nostro caso, saranno costituite da classi di React.
\end{itemize}

Dall'architettura sopra descritta sono stati individuati i seguenti package per il lato frontend di MaaS:

IMMAGINE ARCHITETTURA FRONTEND MAAS

\section{Descrizione dei package del frontend}
\subsection{WebAPIs}

IMMAGINE PACKAGE WEBAPIS

\paragraph*{Descrizione del package}
Il seguente package contiene tutte le classi che contengono i metodi per interagire con le API esposte dal server. 
\paragraph*{Classi contenute}
\begin{itemize}
\item UserAPIs
\item CompanyAPIs
\item DSLAPIs
\item DatabaseAPIs
\item WebAPIFacade
\end{itemize}

\subsection{UserAPIs}
\paragraph*{Descrizione della classe}
Classe che espone tutti i metodi per interagire con le API del server che riguardano gli utenti.

\paragraph*{Utilizzo}
Viene utilizzata sia per il login che per gestire le operazioni CRUD per le informazioni riguardanti gli utenti.

\paragraph*{Classi }
