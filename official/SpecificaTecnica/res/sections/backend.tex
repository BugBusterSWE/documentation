\section{Backend}
%\subsection{Formalismo utilizzato}
%L'utilizzo del framework ExpressJS come base per il server ha evidenziato la necessità di regolamentare alcune convenzioni utilizzate. Queste convenzioni riguardano il formalismo utilizzato per la rappresentazione in quanto concetti come moduli, middlewares e first-class function non hanno una corrispondenza diretta nei diagrammi UML.
%I moduli verranno rappresentati come classi.

%% DA FINIRE E DEFINIRE CON MATTEO

\subsection{Descrizione generale}
L'implementazione scelta per il backend dell'applicazione è un server con architettura REST. Ciò implica che:
\begin{itemize}
\item l'applicazione renda disponibili le sue funzioni in veste di risorse web;
\item ogni risorsa resa disponibile è indirizzabile univocamente utilizzando un indirizzo URL;
\item l'interfaccia delle risorse deve essere uniforme e deve garantire un insieme ben definito di operazioni e una gestione priva di stato delle operazioni.
\end{itemize}

Tale architettura permette l'indipendenza completa tra backend e frontend, permettendo così espansioni su altre piattaforme senza dover modificare il backend dell'applicazione.

L'architettura del server segue il design pattern MVC (\textbf{M}odel \textbf{V}iew \textbf{C}ontroller) per quanto concerne i ruoli di Model e Controller. 
Il ruolo di Controller tuttavia, non essendo implementato nativamente dal framework Express, verrà implementato da uno stack di middleware come suggerito dalla documentazione del framework stesso.
La parte View, invece, è rapprestata dal frontend, costituito da componenti definiti con React e che interagiscono secondo l'architettura Flux.

\subsection{Descrizione dei package}
Il seguente diagramma mostra l'architettura generale del backend.
\begin{figure}[H]
\centering
\includegraphics[width=0.8\textwidth]{res/sections/backend/generale.png}
\caption{Diagramma dei package}
\end{figure}

\subsubsection{Package MaaSApplication}
\paragraph*{Descrizione}
Package che racchiude tutta l'applicazione MaaS.

\paragraph*{Package contenuti}
\begin{itemize}
\item Models
\item Routes
\item Config
\item Lib
\end{itemize}

\paragraph*{Classi contenute}
\begin{itemize}
\item MaaSServer
\end{itemize}

\begin{figure}[H]
\centering
\includegraphics[width=0.8\textwidth]{res/sections/backend/collegamenti.png}
\caption{Diagramma dei package completo dei componenti per ciascun package}
\end{figure}

\subsection{Package Models}
\paragraph*{Descrizione}
Package che racchiude le classi che rappresentano la \textit{business logic} dell'applicazione. Ciascuna di queste classi implementa l'interafaccia Model dichiarata all'interno del package.
Ciascuna classe deve implementare un modello definito da MongooseJS e deve definire i metodi con cui interagire con i dati che la classe stessa rappresenta. \\
Queste classi rappresentano la parte \textbf{M} (Model) del design pattern MVC.

\paragraph*{Classi contenute}
\begin{itemize}
\item UserModel
\item DSLModel
\item CompanyModel
\item DatabaseModel
\end{itemize}

\paragraph*{Interfacce contenute}
\begin{itemize}
\item Model
\end{itemize}

\begin{figure}[H]
\centering
\includegraphics[width=0.8\textwidth]{res/sections/backend/models.png}
\caption{Diagramma delle classi del package Models}
\end{figure}

\paragraph{Model}
\paragraph*{Descrizione}
Interfaccia comune a tutti i modelli usati da MaaS.

\paragraph*{Utilizzo}
Viene utilizzata come base per UserModel, CompanyModel, DSLModel e DatabaseModel.

\paragraph*{Relazione con altri moduli}
\begin{itemize}
\item Mongoose
\item UserModel
\item CompanyModel
\item DSLModel
\item DatabaseModel
\end{itemize}

\subsubsection{UserModel}
\paragraph*{Descrizione}
Classe che racchiude la \textit{business logic} legata agli utenti. Implementa modello e schema definiti da MongooseJS.

\paragraph*{Utilizzo}
Il modello viene utilizzato sia per la rappresentazione di un utente nell'applicazione, sia per l'autenticazione nel sistema.

\paragraph*{Relazioni con altre classi}
\begin{itemize}
\item Mongoose
\item Mongoose-validator
\item Lib::AuthenticationChecker
\item Router::UserRouter
\end{itemize}

\subsubsection{CompanyModel}
\paragraph*{Descrizione}
Racchiude la business logic legata alle Company. Implementa modello e schema definiti da MongooseJS.

\paragraph*{Utilizzo}
Il modello rappresenta una Company nel sistema.

\paragraph*{Relazioni con altri moduli}
\begin{itemize}
\item Mongoose
\item Mongoose-validator
\item Router::CompanyRouter
\end{itemize}

\paragraph{DatabaseModel}
\paragraph*{Descrizione}

Racchiude la business logic legata al collegamento con i database delle Company. Implementa modello e schema definiti da MongooseJS.

\paragraph*{Utilizzo}
Il modello rappresenta la connessione ad un database aziendale di una Company. Contiene i dati per effettuare l'accesso al database e il riferimento alle collections definite su tale database, permettendo così all'utente di poter definire per ciascuna Collection la possibilità di interagirvi da parte di tutti i membri della propria Company o solo degli Admin.

\paragraph*{Relazioni con altri moduli}
\begin{itemize}
\item Mongoose
\item Mongoose-validator
\item Router::DatabaseRouter
\end{itemize}

\paragraph{DSLModel}
\paragraph*{Descrizione}

Racchiude la \textit{business logic} legata alle specifiche DSL. Implementa modello e schema definiti da MongooseJS.

\paragraph*{Utilizzo}
Il modello viene utilizzato per la rappresentazione delle specifiche DSL. Contiene i dati di tali specifiche e le funzioni per poter estrarre i dati richiesti da una specifica DSL.

\paragraph*{Relazioni con altri moduli}
\begin{itemize}
\item Mongoose
\item Mongoose-validator
\item Lib::DSLChecker
\item Router::DSLRouter
\end{itemize}


\subsubsection{Package Routes}
\paragraph*{Descrizione}
Il package Routes contiene le classi che implementano il router definito da ExpressJS, definendo le varie API esposte dal server.
I moduli vengono suddivisi in base al modello che utilizzano. 
Ciascuna classe inoltre ha il compito di definire i middleware da anteporre alla definizione delle API.\\
Le classi ricoprono il ruolo di Controller del design pattern MVC.

\paragraph*{Moduli contenuti}
\begin{itemize}
\item UserRouter
\item CompanyRouter
\item DSLRouter
\item DatabaseRouter
\item RouterFacade
\end{itemize}

\begin{figure}[H]
\centering
\includegraphics[width=0.8\textwidth]{res/sections/backend/routes.png}
\caption{Diagramma delle classi del package Routes}
\end{figure}

\paragraph{UserRouter}
\paragraph*{Descrizione}
Questa classe contiene la definizione degli endpoint riguardanti gli utenti. 

\paragraph*{Relazione con altri moduli}
\begin{itemize}
\item Lib::AuthenticationChecker
\item Model::UserModel
\end{itemize}

\paragraph{CompanyRouter}
\paragraph*{Descrizione}
Questo modulo contiene la definizione degli endpoint riguardanti le Company.

\paragraph*{Relazione con altri moduli}
\begin{itemize}
\item Lib::AuthenticationChecker
\item Model::CompanyModel
\end{itemize}

\paragraph{DSLRouter}
\paragraph*{Descrizione}
Questo modulo contiene la definizione delle API riguardanti le specifiche DSL.

\paragraph*{Relazione con altri moduli}
\begin{itemize}
\item Lib::AuthenticationChecker
\item Lib::LevelChecker
\item Model::DSLModel
\end{itemize}

\paragraph{DatabaseRouter}
\paragraph*{Descrizione}
Questo modulo contiene la definizione delle API riguardanti le connessioni ai database aziendali.

\paragraph*{Relazione con altri moduli}
\begin{itemize}
\item Lib::AuthenticationChecker
\item Lib::LevelChecker
\item Model::DatabaseModel
\end{itemize}

\paragraph{RouterFacade}
\paragraph*{Descrizione}
Oggetto che implementa il Facade Design Pattern. Tale oggetto incorpora tutte le API definite nelle classi Router.

\paragraph*{Relazione con altri moduli}
\begin{itemize}
\item UserRouter
\item CompanyRouter
\item DSLRouter
\item DatabaseRouter
\end{itemize}

\subsubsection{Package Config}
\paragraph*{Descrizione}
Questo package contiene le classi Configuration che contengono tutte le informazioni per inizializzare correttamente l'applicazione. 
Sarà poi l'applicazione a dover recuperare la corretta configurazione in base alla variabilie d'ambiente NODE\_ENV.

\paragraph*{Moduli contenuti}
\begin{itemize}
\item Configuration
\item DevConfiguration
\item TestConfiguration
\item ProdConfiguration
\end{itemize}

\paragraph*{Interfacce contenute}
\begin{itemize}
\item Configuration
\end{itemize}

\begin{figure}[H]
\centering
\includegraphics[width=0.8\textwidth]{res/sections/backend/config.png}
\caption{Diagramma delle classi del package Config}
\end{figure}

\paragraph{DevConfiguration}
\paragraph*{Descrizione}
Configurazione usata durante lo sviluppo.

\paragraph*{Utilizzo}
Viene utilizzata per configurare l'ambiente di lavoro durante lo sviluppo di MaaS.

\paragraph{TestConfiguration}
\paragraph*{Descrizione}
Configurazione usata durante il test.

\paragraph*{Utilizzo}
Viene utilizzata per configurare l'ambiente di lavoro durante la fase di test di MaaS.

\paragraph{ProdConfiguration}
\paragraph*{Descrizione}
Configurazione usata per il rilascio.

\paragraph*{Utilizzo}
Viene utilizzata per configurare l'ambiente di lavoro per la consegna di MaaS.

\paragraph{Configuration}
\paragraph*{Descrizione}
Interfaccia comune alle configurazioni usate da MaaS.

\paragraph*{Utilizzo}
Viene utilizzata come base per DevConfiguration, TestConfiguration e ProdConfiguration.

\paragraph*{Relazione con altri moduli}
\begin{itemize}
\item MaaSServer
\end{itemize}

\subsubsection{Package Lib}
\paragraph*{Descrizione}
Questo package contiene tutti i moduli di supporto al sistema e i middlewares generici per ExpressJS.

\paragraph*{Moduli contenuti}
\begin{itemize}
\item LevelChecker
\item AuthenticationChecker
\item DSLChecker
\end{itemize}


\begin{figure}[H]
\centering
\includegraphics[width=0.8\textwidth]{res/sections/backend/lib.png}
\caption{Diagramma delle classi del package Lib}
\end{figure}

\paragraph{LevelChecker}
\paragraph*{Descrizione}
Middleware che si occupa di verificare se l'utente che effettua una richiesta al server ha i livelli minimi di accesso per poterla eseguire.

\paragraph*{Utilizzo}
Viene utilizzato nelle \textit{routes} in cui deve essere garantito un livello utente minimo di accesso per portare a termine la richiesta.

\paragraph*{Relazione con altri moduli}
\begin{itemize}
\item Router
\end{itemize}

\paragraph{AuthenticationChecker}
\paragraph*{Descrizione}
Modulo che definisce due middleware: uno per effettuare il login, l'altro per l'autenticazione di una richiesta.

\paragraph*{Utilizzo}
Questo modulo viene utilizzato per definire l'\textit{endpoint} per effettuare il login all'applicazione e offre il middleware che permette di autenticare le richieste. Tale middleware si occuperà di estrarre il \textit{token} dalle richieste, verificarne la correttezza e aggiungere l'utente verificato nella richiesta per i prossimi middleware.

\paragraph*{Relazione con altri moduli}
\begin{itemize}
\item Router
\end{itemize}

\paragraph{DSLChecker}
\paragraph*{Descrizione}
Modulo che verifica la correttezza sintattica e di contenuto di una specifica DSL.

\paragraph*{Utilizzo}
Questo modulo viene richiamato in DSLModel per verificare che la specifica DSL che viene salvata o modificata sia valida per l'esecuzione.

\paragraph*{Relazione con altri moduli}
\begin{itemize}
\item DSLModel
\end{itemize}

\subsubsection{MaaSServer}
\paragraph*{Descrizione}
Classe principale dell'applicazione.

\paragraph*{Utilizzo}
Viene utilizzata questa classe per inizializzare l'applicazione. Questa classe ha la responsabilità di caricare gli altri moduli del backend correttamente e di inizializzare il webserver che li utilizza.
\paragraph*{Relazione con altri moduli}
\begin{itemize}
\item Config
\item Router::RouterFacade
\item BodyParser
\item ExpressJS
\end{itemize}


\subsubsection{Moduli integrati}
\paragraph{BodyParser}
\paragraph*{Descrizione}
Modulo di terze parti per la corretta lettura delle informazioni contenute nel \textit{body} delle richieste HTTP.

\paragraph*{Utilizzo}
Il modulo in questione viene utilizzato come middleware per ExpressJS e si occupa della corretta lettura delle informazioni contenute nel \textit{body} di una richiesta HTTP. Nel nostro caso verrà impiegato per la lettura dei dati del body in formato JSON.

\paragraph{NodeMailer}
\paragraph*{Descrizione}
Permette di inviare delle email.

\paragraph*{Utilizzo}
È usato per inviare email agli utenti di MaaS.

\subsubsection{Framework integrati}
\paragraph{SweetJS}
\paragraph*{Descrizione}
Tool che permettere la definizione di una serie di macro allo scopo di modellare una grammatica regolare traducibile in procedure JavaScript.

\paragraph*{Utilizzo}
Come usato in MaaP, definisce la sintassi del DSL. In MaaS il set di macro verrà esteso per soddisfare i nuovi requisiti.

\paragraph{ExpressJS}
\paragraph*{Descrizione}
Framework per la creazione e la gestione di webserver per l'esposizione di API RESTful.

\paragraph*{Utilizzo}
Utilizzato come base per la struttura del server.

\paragraph{Mongoose}
\paragraph*{Descrizione}
Framework per interfacciarsi con MongoDB.

\paragraph*{Utilizzo}
Utilizzato per la gestione dei dati su database MongoDB.

\subsection{API REST}
\subsubsection{Comunicazione tra client e server}
Per la creazione del backend di MaaS si è deciso di utilizzare Node.js e, in particolare, il framework ExpressJS, che permette la creazione semplificata di server REST. Il lato backend sarà quindi costituito da un insieme di API protette da strati diversi di sicurezza. \\
Ciascuna API del webserver fornirà una risposta in formato JSON per permettere la fruizione delle informazioni. Fornirà nello stesso formato anche gli eventuali messaggi di errore generati nel corpo dei metodi del server. Tali messaggi di errore saranno così composti: 
\begin{verbatim}
{
    "code":     [Codice definito nel protocollo HTTP, che identifica univocamente
                 la tipologia del problema]
    "message":  [Messaggio che definisce in dettaglio la tipologia dell'errore]
    ["data":   [Opzionale, trasporta i dati in cui si è verificato l'errore]]
}
\end{verbatim}
I codici di errore saranno del tipo 4xx (client error, la richiesta è sintatticamente scorretta o non può essere soddisfatta) o 5xx (server error, il server ha fallito nel soddisfare una richiesta apparentemente valida).
\subsubsection{Sicurezza}
Gli accessi alle API avranno 2 livelli di sicurezza. \\
Il primo livello è rappresentato dall'autenticazione: un utente non autenticato riceverà un errore se richiede una API protetta. Verrà implementato con l'utilizzo di PassportJS, un middleware per ExpressJS che permette l'autenticazione di utenti nel sistema. In particolare verrà utilizzata la strategia passport-local per l'accesso al server, cioè le credenziali dell'utente risiederanno nel database locale. \\
Il secondo livello è definito in base al ruolo di appartenenza di un utente. Si occuperà di controllare i permessi assegnati ad un utente autenticato e di verificare la possibilità che possa o meno interagire con la risorsa richiesta. \\
I ruoli utente ammessi nell'applicazione sono: 
\begin{itemize}
\item \textbf{Guest};
\item \textbf{Member};
\item \textbf{Admin};
\item \textbf{Owner};
\item \textbf{Super Admin}.
\end{itemize}
Per il ruolo di Super Admin è abilititato un set di API per la gestione dell'intera applicazione. Tali API non sono accessibili agli utenti con altri ruoli.
\newpage
\subsubsection{Scenari di accesso negato}
Di seguito verranno rappresentati scenari corrispondenti ad una negazione di accesso per ruolo non conforme alle attese o per errore nel login.
\paragraph{Errore di autenticazione}  \mbox{} \\
\textbf{Descrizione:} L'utente non autenticato cerca di accedere a MaaS, ma la procedura di login rileva un errore nelle credenziali inserite.
\begin{figure}[H]
\centering
\includegraphics[width=0.8\textwidth]{res/sections/backend/sequence/autenticazioneFallita.png}
\caption{Autenticazione fallita}
\end{figure}
\paragraph{Livello minimo: MEMBER} \mbox{} \\
\textbf{Descrizione:} Tentativo di accesso ad una risorsa visibile solo a utenti con ruolo almeno MEMBER da parte di un utente GUEST.
\begin{figure}[H]
\centering
\includegraphics[width=0.8\textwidth]{res/sections/backend/sequence/requireMemberFallita.png}
\caption{Livello minimo MEMBER}
\end{figure}
\paragraph{Livello minimo: ADMIN} \mbox{} \\
\textbf{Descrizione:} Tentativo di accesso ad una risorsa visibile solo a utenti con ruolo almeno ADMIN da parte di un utente GUEST o MEMBER.
\begin{figure}[H]
\centering
\includegraphics[width=0.8\textwidth]{res/sections/backend/sequence/requireAdminFallita.png}
\caption{Livello minimo ADMIN}
\end{figure}
\paragraph{Livello minimo: OWNER}  \mbox{} \\
\textbf{Descrizione:} Tentativo di accesso ad una risorsa visibile solo a utenti con ruolo almeno OWNER da parte di un utente GUEST, MEMBER o ADMIN.
\begin{figure}[H]
\centering
\includegraphics[width=0.8\textwidth]{res/sections/backend/sequence/requireOwnerFallita.png}
\caption{Livello minimo OWNER}
\end{figure}
\paragraph{Livello minimo: SUPERADMIN}  \mbox{} \\
\textbf{Descrizione:} Tentativo di accesso ad una risorsa accedibile solo da un SUPERADMIN da parte di un generico utente di MaaS.
\begin{figure}[H]
\centering
\includegraphics[width=0.8\textwidth]{res/sections/backend/sequence/requireSuperAdminFallita.png}
\caption{Livello minimo SUPERADMIN}
\end{figure}
\newpage
\subsection{API REST}
Di seguito sono descritte le API REST esposte dal server di MaaS. Si suppone che l'utente che richiede l'accesso alla risorsa descritta abbia i permessi necessari (ovvero che sia autenticato e che il suo ruolo sia conforme a quanto indicato). Qualora questo non fosse vero si ricadrebbe in uno degli scenari esposti precedentemente.
\subsubsection{Senza autenticazione}
\paragraph{Login}\mbox{}\\
\textbf{Tipologia:} POST \\
\textbf{API:} /api/login \\
\textbf{Descrizione:} Necessita di una richiesta con \textit{body} contenente email e password dell'utente. \\
\textbf{Scenario:} 
\begin{figure}[H]
\centering
\includegraphics[width=0.8\textwidth]{res/sections/backend/sequence/(POST)login.png}
\caption{Scenario del login}
\end{figure}

\newpage
\paragraph{Registrazione}\mbox{}\\
\textbf{Tipologia:} POST \\
\textbf{API:} /api/register/:unique\_code \\
\textbf{Descrizione:} Metodo per la creazione di un utente invitato da una Company. Necessita di una richiesta con \textit{body} contenete le informazioni per la creazione completa di un utente \\
\textbf{Scenario:} 
\begin{figure}[H]
\centering
\includegraphics[width=0.8\textwidth]{res/sections/backend/sequence/(POST)register.png}
\caption{Scenario della registrazione}
\end{figure}

\newpage
\paragraph{Creazione Company}\mbox{}\\
\textbf{Tipologia:} POST \\
\textbf{API:} /api/companies \\
\textbf{Descrizione:} Necessita di una richiesta con \textit{body} contenente le informazioni relative alla company e alla creazione del profilo del suo Owner. \\
\textbf{Scenario:} 
\begin{figure}[H]
\centering
\includegraphics[width=0.8\textwidth]{res/sections/backend/sequence/(POST)company.png}
\caption{Scenario della creazione company}
\end{figure}

\newpage
\subsubsection{User}

\paragraph{Inserimento utente}\mbox{}\\
\textbf{Tipologia:} POST \\
\textbf{API:} /api/companies/:company\_id/users \\
\textbf{Livello di accesso minimo:} OWNER \\
\textbf{Descrizione:} Necessita di una richiesta con \textit{body} contenente l'email e il livello di accesso dell'utente.\\
\textbf{Scenario:} 
\begin{figure}[H]
\centering
\includegraphics[width=0.8\textwidth]{res/sections/backend/sequence/(POST)user.png}
\caption{Scenario dell'inserimento utente in una Company}
\end{figure}

\newpage
\paragraph{Aggiornamento delle credenziali utente}\mbox{}\\
\textbf{Tipologia:} PUT \\
\textbf{API:} /api/companies/:company\_id/users/:user\_id/credentials \\
\textbf{Livello di accesso minimo:} GUEST \\
\textbf{Descrizione:} Metodo per la modifica delle credenziali di accesso di un utente. Un utente ha il permesso di cambiare solo le proprie credenziali. \\
\textbf{Scenario:} 
\begin{figure}[H]
\centering
\includegraphics[width=0.8\textwidth]{res/sections/backend/sequence/(PUT)credenzialiUtente.png}
\caption{Scenario dell'aggiornamento delle credenziali di un utente}
\end{figure}

\newpage
\paragraph{Cancellazione utente}\mbox{}\\
\textbf{Tipologia:} DELETE \\
\textbf{API:} /api/companies/:company\_id/users/:user\_id \\
\textbf{Livello di accesso minimo:} OWNER \\
\textbf{Descrizione:} Ritorna un messaggio di conferma dell'avvenuta cancellazione. \\
\textbf{Scenario:} 
\begin{figure}[H]
\centering
\includegraphics[width=0.8\textwidth]{res/sections/backend/sequence/(DELETE)user.png}
\caption{Scenario della cancellazione utente da una company}
\end{figure}

\newpage
\subsubsection{Company}
\paragraph{Dati di una Company}\mbox{}\\
\textbf{Tipologia:} GET \\
\textbf{API:} /api/companies/:company\_id/ \\
\textbf{Livello di accesso minimo:} GUEST \\
\textbf{Descrizione:} Ritorna le informazioni generali di una Company \\
\textbf{Scenario:} 
\begin{figure}[H]
\centering
\includegraphics[width=0.8\textwidth]{res/sections/backend/sequence/(GET)company.png}
\caption{Scenario di ottenimento dei dati di una company}
\end{figure}

\newpage
\paragraph{Aggiornamento dei dati di una company}\mbox{}\\
\textbf{Tipologia:} PUT \\
\textbf{API:} /api/companies/:company\_id/ \\
\textbf{Livello di accesso minimo:} ADMIN \\
\textbf{Descrizione:} Necessita di una richiesta con \textit{body} contenente le modifiche da apportare ai dati della Company. \\
\textbf{Scenario:} 
\begin{figure}[H]
\centering
\includegraphics[width=0.8\textwidth]{res/sections/backend/sequence/(PUT)company.png}
\caption{Scenario dell'aggiornamento dei dati di una Company}
\end{figure}

\newpage
\paragraph{Cancellazione di una Company}\mbox{}\\
\textbf{Tipologia:} DELETE \\
\textbf{API:} /api/companies/:company\_id/ \\
\textbf{Livello di accesso minimo:} OWNER \\
\textbf{Descrizione:} Ritorna un messaggio di avvenuta cancellazione. \\
\textbf{Scenario:} 
\begin{figure}[H]
\centering
\includegraphics[width=0.8\textwidth]{res/sections/backend/sequence/(DELETE)company.png}
\caption{Scenario della cancellazione di una Company}
\end{figure}

\newpage
\subsubsection{DSL}
\paragraph{Elenco delle specifiche DSL}\mbox{}\\
\textbf{Tipologia:} GET \\
\textbf{API:} /api/companies/:company\_id/DSLs \\
\textbf{Livello di accesso minimo:} GUEST \\
\textbf{Descrizione:} Ritorna un array contenente le specifiche DSL alle quali l'utente ha accesso in formato JSON. \\
\textbf{Scenario:} 
\begin{figure}[H]
\centering
\includegraphics[width=0.8\textwidth]{res/sections/backend/sequence/(GET)dsl.png}
\caption{Scenario dell'elenco delle specifiche DSL}
\end{figure}

\newpage
\paragraph{Lettura del codice di una specifica DSL}\mbox{}\\
\textbf{Tipologia:} GET \\
\textbf{API:} /api/companies/:company\_id/DSLs/:dsl\_id \\
\textbf{Livello di accesso minimo:} MEMBER \\
\textbf{Descrizione:} Ritorna il codice della specifica DSL richiesta in formato JSON. \\
\textbf{Scenario:} 
\begin{figure}[H]
\centering
\includegraphics[width=0.8\textwidth]{res/sections/backend/sequence/(GET)dslByID.png}
\caption{Scenario della lettura del codice di una specifica DSL}
\end{figure}

\newpage
\paragraph{Aggiunta di una specifica DSL}\mbox{}\\
\textbf{Tipologia:} POST \\
\textbf{API:} /api/companies/:company\_id/DSLs \\
\textbf{Livello di accesso minimo:} MEMBER \\
\textbf{Descrizione:} Necessita di una richiesta con \textit{body} contenente i dati necessari alla creazione della specifica DSL. \\
\textbf{Scenario:} 
\begin{figure}[H]
\centering
\includegraphics[width=0.8\textwidth]{res/sections/backend/sequence/(POST)dsl.png}
\caption{Scenario della creazione di una specifica DSL}
\end{figure}

\newpage
\paragraph{Aggiornamento del codice di una specifica DSL}\mbox{}\\
\textbf{Tipologia:} PUT \\
\textbf{API:} /api/companies/:company\_id/DSLs/:dsl\_id \\
\textbf{Livello di accesso minimo:} MEMBER \\
\textbf{Descrizione:} Necessita di una richiesta con body contenente i dati necessari alla modifica della specifica DSL identificata da dsl\_id. \\
\textbf{Scenario:}
\begin{figure}[H]
\centering
\includegraphics[width=0.8\textwidth]{res/sections/backend/sequence/(PUT)dsl.png}
\caption{Scenario dell'aggiornamento del codice di una specifica DSL}
\end{figure}

\newpage
\paragraph{Cancellazione di una specifica DSL}\mbox{}\\
\textbf{Tipologia:} DELETE \\
\textbf{API:} /api/companies/:company\_id/DSLs/:dsl\_id \\
\textbf{Livello di accesso minimo:} MEMBER \\
\textbf{Descrizione:} Ritorna un messaggio in formato JSON di avvenuta cancellazione. \\
\textbf{Scenario:} 
\begin{figure}[H]
\centering
\includegraphics[width=0.8\textwidth]{res/sections/backend/sequence/(DELETE)dsl.png}
\caption{Scenario della cancellazione di una specifica DSL}
\end{figure}

\newpage
\paragraph{Ottenimento della Dashboard di un utente}\mbox{}\\
\textbf{Tipologia:} GET \\
\textbf{API:} /api/companies/:company\_id/users/:user\_id/dashboard \\
\textbf{Livello di accesso minimo:} GUEST \\
\textbf{Descrizione:} Ritorna la Dashboard di un utente definita da una DSL. \\
\textbf{Scenario:}  
\begin{figure}[H]
\centering
\includegraphics[width=0.8\textwidth]{res/sections/backend/sequence/(GET)dashboard.png}
\caption{Scenario dell'ottenimento della Dashboard di un utente}
\end{figure}

\newpage
\paragraph{Esecuzione di una specifica DSL}\mbox{}\\
\textbf{Tipologia:} GET \\
\textbf{API:} /api/companies/:company\_id/DSLs/:dsl\_id/execute \\
\textbf{Livello di accesso minimo:} GUEST \\
\textbf{Descrizione:} Ritorna un JSON contenente i dati richiesti dalla specifica DSL e la struttura su cui inserire i dati. \\
\textbf{Scenario:} 
\begin{figure}[H]
\centering
\includegraphics[width=0.8\textwidth]{res/sections/backend/sequence/(GET)dslByIDex.png}
\caption{Scenario dell'esecuzione di una specifica DSL}
\end{figure}

\newpage
\subsubsection{Database}
\paragraph{Elenco dei database della Company}\mbox{}\\
\textbf{Tipologia:} GET \\
\textbf{API:} /api/companies/:company\_id/databases \\
\textbf{Livello di accesso minimo:} MEMBER \\
\textbf{Descrizione:} Ritorna un array contenente i nomi e gli id di ciascun database. \\
\textbf{Scenario:}
\begin{figure}[H]
\centering
\includegraphics[width=0.8\textwidth]{res/sections/backend/sequence/(GET)database.png}
\caption{Scenario dell'elenco dei database propri della Company}
\end{figure}

\newpage
\paragraph{Visualizzazione dati di un database}\mbox{}\\
\textbf{Tipologia:} GET \\
\textbf{API:} /api/companies/:company\_id/databases/:database\_id \\
\textbf{Livello di accesso minimo:} ADMIN \\
\textbf{Descrizione:} Ritorna tutte le informazioni relative al database richiesto. \\
\textbf{Scenario:} 
\begin{figure}[H]
\centering
\includegraphics[width=0.8\textwidth]{res/sections/backend/sequence/(GET)databaseById.png}
\caption{Scenario della visualizzazione dei dati di un database}
\end{figure}

\newpage
\paragraph{Aggiunta di un database}\mbox{}\\
\textbf{Tipologia:} POST \\
\textbf{API:} /api/companies/:company\_id/databases \\
\textbf{Livello di accesso minimo:} ADMIN \\
\textbf{Descrizione:} Necessita di una richiesta con \textit{body} contenete i dati relativi alla connessione del nuovo database. \\
\textbf{Scenario:}
\begin{figure}[H]
\centering
\includegraphics[width=0.8\textwidth]{res/sections/backend/sequence/(POST)database.png}
\caption{Scenario della creazione di una specifica DSL}
\end{figure}

\newpage
\paragraph{Aggiornamento di un database}\mbox{}\\
\textbf{Tipologia:} PUT \\
\textbf{API:} /api/companies/:company\_id/databases/:database\_id \\
\textbf{Livello di accesso minimo:} ADMIN \\
\textbf{Descrizione:} Metodo per aggiornare le informazioni relative alla connessione al database o per aggiornare l'elenco delle collezioni. \\
\textbf{Scenario:}
\begin{figure}[H]
\centering
\includegraphics[width=0.8\textwidth]{res/sections/backend/sequence/(PUT)database.png}
\caption{Scenario dell'aggiornamento di un database}
\end{figure}

\newpage
\paragraph{Cancellazione di un database}\mbox{}\\
\textbf{Tipologia:} DELETE \\
\textbf{API:} /api/companies/:company\_id/databases/:database\_id \\
\textbf{Livello di accesso minimo:} ADMIN \\
\textbf{Descrizione:} Elimina il database selezionato e tutte le specifiche DSL che lo utilizzano. \\
\textbf{Scenario:} 
\begin{figure}[H]
\centering
\includegraphics[width=0.8\textwidth]{res/sections/backend/sequence/(DELETE)database.png}
\caption{Scenario della cancellazione di un database}
\end{figure}

\newpage
\paragraph{Visualizzazione Collections di un database} \mbox{}\\
\textbf{Tipologia:} GET \\
\textbf{API:} /api/companies/:company\_id/databases/:database\_id/collections \\
\textbf{Livello di accesso minimo:} MEMBER \\
\textbf{Descrizione:} Ritorna un array di Collection relative ad un database a cui l'utente ha accesso. \\
\textbf{Scenario:} 
\begin{figure}[H]
\centering
\includegraphics[width=0.8\textwidth]{res/sections/backend/sequence/(GET)collection.png}
\caption{Scenario della visualizzazione Collections di un database}
\end{figure}

\newpage
\subsubsection{Super admin}
\paragraph{Ottenimento informazioni delle Company}\mbox{}\\
\textbf{Tipologia:} GET \\
\textbf{API:} /api/admin/companies \\
\textbf{Descrizione:} Restituisce un array di JSON contenenti le informazioni relative alle Company presenti nell'applicazione. \\
\textbf{Scenario:} 
\begin{figure}[H]
\centering
\includegraphics[width=0.8\textwidth]{res/sections/backend/sequence/(GET)companySA.png}
\caption{Scenario dell'ottenimento informazioni delle Company}
\end{figure}

\newpage
\paragraph{Aggiunta di un SuperAdmin}\mbox{}\\
\textbf{Tipologia:} POST \\
\textbf{API:} /api/admin/superadmins \\
\textbf{Descrizione:} Necessita di una richiesta con body contenente le informazioni relative al superadmin da creare. \\
\textbf{Scenario:} 
\begin{figure}[H]
\centering
\includegraphics[width=0.8\textwidth]{res/sections/backend/sequence/(POST)superadmin.png}
\caption{Scenario dell'aggiunta di un super admin}
\end{figure}

\newpage
\paragraph{Aggiunta di un utente}\mbox{}\\
\textbf{Tipologia:} POST \\
\textbf{API:} /api/admin/companies/:company\_id/users \\
\textbf{Descrizione:} Necessita di una richiesta con \textit{body} contenente le informazioni relative all'utente da creare per la Company individuata da company\_id. \\
\textbf{Scenario:} 
\begin{figure}[H]
\centering
\includegraphics[width=0.8\textwidth]{res/sections/backend/sequence/(POST)userSA.png}
\caption{Scenario dell'aggiunta di un utente}
\end{figure}
