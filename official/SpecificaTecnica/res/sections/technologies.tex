\section{Tecnologie utilizzate}  
MaaS � stato progettato utilizzando diverse tecnologie, alcune delle quali espressamente richieste dal proponente. Di seguito vengono elencate e descritte. Per ognuna di esse, inoltre, verranno presentanti vantaggi e svantaggi, al fine di fornire una pi� completa motivazione della scelta.
\begin{itemize}
\item \textbf{Node.js}, piattaforma per il back-end;
\item \textbf{React}, framework JavaScript per la realizzazione dele front-end;
\item \textbf{MongoDB}, database di tipo NoSQL per la parte di recupero e salvataggio dei dati;
\item \textbf{Mongoose}, libreria per interfacciarsi con il driver di MongoDB;
\item \textbf{HTML5}, linguaggio di markup per la creazione di pagine web;
\item \textbf{CSS3}, linguaggio di formattazione dei documenti HTML;
\item \textbf{Loopback}, framework Node.js per la crezione dinamica di API REST.
\end{itemize}
\subsection{Node.js}
L'utilizzo di Node.js � stato richiesto dal proponente. Si tratta di un sistema run-time cross-platform che utilizza il motore JavaScript V8 di Google Chrome e permette di realizzare facilemte applicazioni di rete scalabili e veloci. Grazie al suo modello event-driven, con chiamate di I/O non bloccanti risulta essere leggero ed efficiente.
\subsubsection{Vantaggi}
\begin{itemize}
\item \textbf{Approccio asincrono}: permette di accedere alle risorse del sistema operativo in modalit� event-driven e non tramite il classico processo orientato a processi e thread cocorrenti utilizzato dai classici server web. Ci� garantisce una maggiore efficienza dal punto di vista delle prestazioni, dato che durante le attese possono essere eseguire altre operazioni in modo asincrono.
\item \textbf{Architettura modulare}: � molto facile organizzare il lavoro in librerie, importare e combinare i moduli.
\end{itemize}
\subsubsection{Svantaggi}
\begin{itemize}
\item \textbf{Orientato all'I/O}: non � pensato per applicazioni che sfruttano in modo intensivo la CPU. Questo, tuttavia, non � il caso di MaaS.
\end{itemize}
\subsection{React}
L'utilizzo di React � stato richiesto dal proponente. � una libreria JavaScript usata per creare interfacce utente,e rappresenta la V del design pattern architetturale MVC. React risolve il problema di creare applicazioni di grandi dimensioni con dati che cambiano nel tempo.
\subsubsection{Vantaggi}
\begin{itemize}
\item \textbf{Semplice}: basta esprimere come l'applicazione dovrebbe apparire in ciascun momento, e React automaticamente gestir� tutti gli aggiornamenti dell'interfaccia utente.
\item \textbf{Dichiarativo}: quando i dati cambiano, React effettua un refresh concettuale della pagina e aggiorna solo i dati che sono cambiati.
\item \textbf{Facilit� di debugging}: attraverso una specifica estensione di Google Chrome.
\end{itemize}
\subsubsection{Svantaggi}
\begin{itemize}
\item \textbf{Verboso}: richiede di scrivere pi� codice rispetto alla semplice coppia HTML e JavaScript.
\item{Non � un framework completo}: non esiste un modello di gestione delle librerie all'interno di React, cosa che invece non � vera per altri framework simili, come Ember o AngularJS.
\end{itemize}
\subsection{MongoDB}
L'utilizzo di MongoDB � stato richiesto dal proponente. Si tratta di un database NoSQL open source scalabile e altamente performante di tipo document-oriented e schemaless, nel quale i dati sono archiviati sotto forma di documenti in stile JSON, con schemi dinamici e una struttura semplice e potente.
\subsubsection{Vantaggi}
\begin{itemize}
\item \textbf{Alte performance}: non ci sono join che rallentano operazioni di lettura o scrittura. 
\item \textbf{Affidabilit�}: � presente un meccanismo di replicazione su server.
\item \textbf{Schemaless}: non esistono schemi per i dati. Pertanto � pi� flessibile.
\item \textbf{Potenza espressiva}: permette di esprimre query complesse in un linguaggio non SQL.
\item \textbf{Map-Reduce}: permette di processare parallelamente i dati.
\item \textbf{Flessibilit�}: per i tipi di dato.
\end{itemize}
\subsubsection{Svantaggi}
\begin{itemize}
\item \textbf{Map-Reduce}: ancora non performante.
\item \textbf{Nessun supporto per le transazioni}: sono supportate alcune operazioni atomiche, ma a livello di documento.
\item \textbf{Nessun join}: va simulato via codice attraverso query multiple.
\item \textbf{Utilizzo di memoria}: maggiore rispetto ai database SQL perch� memorizza i nomi delle chiavi in ogni documento. 
\item \textbf{Problemi di concorrenza}: per le operazioni di scrittura viene creato un lock sull'intero database. Questo lock blocca anche le operazioni di lettura.
\end{itemize}
\subsection{Mongoose}
Mongoose � una libreria per interfacciarsi a MongoDB che permette di definire degli schemi per modellare i dati del database. Inoltre fornisce strumenti utili per la validazioni dei dati, per la definizione di queries e per il cast dei tipi predefiniti.
\subsubsection{Vantaggi}
\begin{itemize}
\item \textbf{Diffusione}: � la libreria pi� diffusa per interfacciarsi con MongoDB.
\item \textbf{Funzionalit� aggiuntive}: permette di definire strumenti per la validazione dei dati e per il cast dei tipi.
\end{itemize}
\subsubsection{Svantaggi}
\begin{itemize}
\item \textbf{Schema-based}: � basato sulla creazione di schemi per i documenti, e questo va contro uno dei principali vantaggi di MongoDB.
\end{itemize}
\subsection{HTML5}
� un linguaggio di markup per la strutturazione delle pagine web, pubblicato come W3C Recommendation da ottoble 2014.
\subsubsection{Vantaggi}
\begin{itemize}
\item \textbf{Raccomandazione W3C}.
\item \textbf{Creazione di pagine interattive}: soprattutto se usato insieme a CSS.
\end{itemize}
\subsubsection{Svantaggi}
\begin{itemize}
\item \textbf{Supporto}: non tutti i browser lo supportano allo stesso modo, e non tutte le caratteristiche definite sono ancora completamente supportate.
\end{itemize}
\subsection{CSS3}
� un linguaggio utilizzato per definire la formattazione di documenti HTML e XHTML. Le regole per la composizione di un foglio di stile CSS sono definite dal W3C a partire dal 1996. Inoltre permette di separare i contenuti delle pagine HTML dalla loro formattazione, assicurando una maggiore manutenibilit� e riutilizzo.
\subsubsection{Vantaggi}
\begin{itemize}
\item \textbf{Separazione tra contenuto e presentazione}.
\item \textbf{Raccomandazione W3C}.
\end{itemize}
\subsection{Loopback}
L'uso di Loopback � stato consigliato dal proponente. � un framework Node.js altamente estensibile ed open-source che permette, tra le altre cose, di creare API REST dinamiche e accedere a dati MongoDB.
\subsubsection{Vantaggi}
\begin{itemize}
\item \textbf{Velocit� di sviluppo}: la creazione di API REST � molto veloce.
\item \textbf{Configurabilit�}: pienamente configurabile secondo i bisogni dell'applicazione.
\item \textbf{Documentazione}: esaustiva e completa.
\item \textbf{Pronto all'uso}: ci sono molti moduli disponibili all'uso.
\end{itemize}
\subsubsection{Svantaggi}
\begin{itemize}
\item \textbf{Apprendimento}: pu� risultare difficile perch� ci sono molte parti gi� definite da imparare.
\end{itemize}