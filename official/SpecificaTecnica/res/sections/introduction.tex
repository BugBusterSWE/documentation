\section{Introduzione}
\subsection{Scopo del documento}
Il presente documento ha lo scopo di definire la progettazione ad alto livello del prodotto MaaS. A tale scopo viene definita l'architettura generale del sistema. Viene inoltre esplicitato il tracciamento tra le componenti software individuate e i requisiti presenti nel documento \AnalisiDeiRequisiti. 

\subsection{Scopo del prodotto}
L'obiettivo che si pone \glossaryItem{MaaS} (\textbf{M}ongoDB \textbf{a}s an \textbf{a}dmin \textbf{S}ervice) è quello di rendere il già funzionante \glossaryItem{MaaP} (\textbf{M}ongoDB \textbf{a}s an \textbf{a}dmin \textbf{P}latform) un servizio, ovvero quello di renderlo disponibile a tutti, senza richiederne l'installazione. \glossaryItem{MaaS} si propone di essere un'estensione di \glossaryItem{MaaP} anche dal punto di vista delle funzionalità offerte all'utente finale. Sarà in grado di supportare i ruoli basilari di una company, permettendo ad utenti diversi di eseguire operazioni diverse. \\
\glossaryItem{MaaS} verrà realizzato utilizzando principalmente Node.js e MongoDB.

\subsection{Glossario}
Ogni occorrenza di acronimi, dei termini tecnici o di dominio è evidenziata con il corsivo e marcata con la lettera G in pedice. Nel documento \Glossario sono riportati i significati corrispondenti.

\subsection{Riferimenti}
Di seguito sono elencati i riferimenti sui quali si basa il presente documento e che sono stati utilizzati dal team BugBusters per lo studio delle tecnologie coinvolte.

\subsubsection{Normativi}
\begin{itemize}
\item \textbf{Norme di progetto}: \NormeDiProgetto;
\item \textbf{Capitolato d'appalto C4}: RedBabel, \glossaryItem{MaaS} 
\item \textbf{Analisi dei Requisiti}: \AnalisiDeiRequisiti;
	
\subsubsection{Informativi}
\begin{itemize}
\item \textbf{Slides del corso di Ingegneria del Software mod. A}: \glossaryItem{Qualità} del software \url{http://www.math.unipd.it/~tullio/IS-1/2015/Dispense/L08.pdf};
\item \textbf{Slides del corso di Ingegneria del Software mod. A}: \glossaryItem{Qualità} del \glossaryItem{processo} \url{http://www.math.unipd.it/~tullio/IS-1/2015/Dispense/L09.pdf};
\item \textbf{Software Engineering 9th - I. Sommerville (Pearson, 2011)};
%todo definire fonti di studio delle tecnologie
\end{itemize}
