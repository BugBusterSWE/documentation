\section{Introduzione}
\subsection{Scopo del documento}
Il presente documento ha lo scopo di definire la progettazione ad alto livello del prodotto \glossaryItem{MaaS} e i protocolli di comunicazione fra le componenti dell'architettura logica del prodotto. A tale scopo viene definita l'architettura generale del sistema, esplicitando il \glossaryItem{tracciamento} tra le componenti software individuate e i requisiti presenti nel documento \AnalisiDeiRequisiti. 

\subsection{Scopo del prodotto}
L'obiettivo che si pone \glossaryItem{MaaS} (\textbf{M}ongoDB \textbf{a}s an \textbf{a}dmin \textbf{S}ervice) è quello di rendere il già funzionante \glossaryItem{MaaP} (\textbf{M}ongoDB \textbf{a}s an \textbf{a}dmin \textbf{P}latform) un servizio, ovvero quello di renderlo disponibile a tutti, senza richiederne l'installazione. \glossaryItem{MaaS} si propone di essere un'estensione di \glossaryItem{MaaP} anche dal punto di vista delle funzionalità offerte all'utente finale. Sarà in grado di supportare i ruoli basilari di una \glossaryItem{company}, permettendo ad utenti diversi di eseguire operazioni diverse. \\
\glossaryItem{MaaS} verrà realizzato utilizzando principalmente \glossaryItem{Node.js} e \glossaryItem{MongoDB}.

\subsection{Glossario}
Ogni occorrenza di acronimi, dei termini tecnici o di dominio è evidenziata con il corsivo e marcata con la lettera G in pedice. Nel documento \Glossario sono riportati i significati corrispondenti.

\subsection{Riferimenti}
Di seguito sono elencati i riferimenti sui quali si basa il presente documento e che sono stati utilizzati dal \glossaryItem{team} BugBusters per lo studio delle tecnologie coinvolte.

\subsubsection{Normativi}
\begin{itemize}
\item \textbf{Norme di \glossaryItem{progetto}}: \NormeDiProgetto;
\item \textbf{Capitolato d'appalto C4}: RedBabel, \glossaryItem{MaaS} 
\item \textbf{Analisi dei Requisiti}: \AnalisiDeiRequisiti;
\end{itemize}
\subsubsection{Informativi}
\begin{itemize}
\item \textbf{Software Engineering 9th - I. Sommerville (Pearson, 2011)};
\item \textbf{Slides del corso di Ingegneria del Software mod. A}: \glossaryItem{Diagrammi} delle classi \url{http://www.math.unipd.it/~tullio/IS-1/2015/Dispense/E03.pdf};
\item \textbf{Slides del corso di Ingegneria del Software mod. A}: \glossaryItem{Diagrammi} dei \glossaryItem{package}\url{http://www.math.unipd.it/~tullio/IS-1/2015/Dispense/E04.pdf};
\item \textbf{Slides del corso di Ingegneria del Software mod. A}: \glossaryItem{Diagrammi} di sequenza \url{http://www.math.unipd.it/~tullio/IS-1/2015/Dispense/E05.pdf};
\item \textbf{Slides del corso di Ingegneria del Software mod. A}: \glossaryItem{Diagrammi} di attività \url{http://www.math.unipd.it/~tullio/IS-1/2015/Dispense/E06.pdf};
\item \textbf{Slides del corso di Ingegneria del Software mod. A}: Design pattern strutturali: Decorator, Proxy, Facade, Adapter \url{http://www.math.unipd.it/~tullio/IS-1/2015/Dispense/E07.pdf}
\item \textbf{Slides del corso di Ingegneria del Software mod. A}: Design pattern creazionali: Singleton, Builder, Abstract Factory \url{http://www.math.unipd.it/~tullio/IS-1/2015/Dispense/E08.pdf}
\item \textbf{Slides del corso di Ingegneria del Software mod. A}: Design pattern comportamentali: Observer, Template Method, Command, Strategy, Iterator \url{http://www.math.unipd.it/~tullio/IS-1/2015/Dispense/E09.pdf}
\item \textbf{UML Distilled - Martin Fowler (Pearson, Addison Wesley, 2004)};
\item \textbf{Learning \glossaryItem{UML} 2.0 - Kim Hamilton, Russell Miles (O’Reilly, 2006)};
\item \textbf{Design Patterns - E. Gamma, R. Helm, R. Johnson, J. Vlissides (Pearson Education, Addison-Wesley, 1995};
\item \textbf{\glossaryItem{Node.js}}, \url{https://nodejs.org/dist/latest-v5.x/docs/api/};
\item \textbf{Reactjs}, \url{https://facebook.github.io/react/docs/getting-started.html};
\item \textbf{MongoDB}, \url{https://docs.mongodb.org/manual/};
\item \textbf{HTML5}, \url{http://www.w3schools.com/html/html5\_intro.asp};
\item \textbf{CSS3}, \url{http://www.w3schools.com/css/css3\_intro.asp};
\item \textbf{Materialize}, \url{http://materializecss.com/};
\item \textbf{ExpressJS}, \url{http://expressjs.com/en/4x/api.html};
\item \textbf{PassportJS}, \url{http://passportjs.org/docs};
\item \textbf{TypeScript}, \url{http://www.typescriptlang.org/Handbook};
\item \textbf{Flux}, \url{https://github.com/facebook/flux};
\item \textbf{API \glossaryItem{REST}}, \url{https://github.com/tfredrich/RestApiTutorial.com/raw/master/media/RESTful\%20Best\%20Practices-v1\_2.pdf}.
\end{itemize}
