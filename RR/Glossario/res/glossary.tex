\newglossaryentry{sistematico}
{
name=sistematico,
description={Colui che sa far quella cosa - darsi delle regole - approcciarsi in al problema con metodo. Ci\`o contribuisce all'efficienza e all'efficacia}
}

\newglossaryentry{stakeholder}
{
name=stakeholder,
description={Dall'inglese, portatore d' interesse. È l'insieme di persone a vario titolo coinvolte nel ciclo di vita del Software con influenza sul prodotto}
}

\newglossaryentry{processo}
{
name=processo, 
description={Multiple definizioni:
\begin{enumerate}
\item Detto in inglese come \textit{Way of working}. Esso \`e rappresentabile come un automa a stati, dove ogni stato rappresenta uno stadio del ciclo di vita del processo
\item Un insieme di attività interconnesse, che trasforma uno o più input in output consumando risorse. [SWEBok 8-1]
\end{enumerate}
}
}

\newglossaryentry{procedura}
{
name=procedura,
description={Una procedura \'e un ordinato insieme di passi o, alternativamente, controlli del lavoro per eseguire il task}
}


\newglossaryentry{sink}
{
name=sink,
description={Dall'inglese ``Punto di fine", scarico da cui non esco pi\`u (ovvero uno stato finale)}
}

\newglossaryentry{source}
{
name=source,
description={Dall'inglese, stato che ammette solo archi in uscita}
}

\newglossaryentry{ciclovitasoftware}
{
name=ciclo di vita del software,
description={Sono gli stati che il prodotto assume dal concepimento al ritiro}
}

\newglossaryentry{efficacia}
{
name=efficacia,
description={Definizioni multiple:
\begin{enumerate}
\item Conformit\`a al contratto. Si garantisce cio\`e ci\`o che si deve fare, ed \`e determinata dal grado di conformit\`a del progetto rispetto alle norme vigenti e agli obiettivi prefissati. L'efficacia \`e direttamente proporzionale alla quantit\`a di risorse impiegate
\item L'efficacia \`e il rapporto tra l'output attuale e quello attesso, prodotto dal processo, attivit\`a o compito [SWEBok-v3 8.4.1].
\end{enumerate}
}
}

\newglossaryentry{efficienza}
{
name=efficienza,
description={Definizioni multiple: 
\begin{enumerate}
\item L'efficienza \`e la capacit\`a di azione o di produzione con il minimo di scarto, di spesa, di risorse e di tempo impiegati. \`E inversamente proporzionale alla quantit\`a di risorse impiegate nell'esecuzione delle attivit\`a richieste.
\item L'efficienza \`e il rapporto tra le risorse consumate e quelle attese o desiderate nel compiere un processo, attivit\`a o compito [SWEBok-v3 8.4.1].
\end{enumerate}
}
}

\newglossaryentry{bestpractice}
{
name=best practice,
description={Prassi (modo di fare) che per esperienza e per studio abbia mostrato di garantire i miglio risultati in circostanze note e specifiche}
}

\newglossaryentry{iterazione}
{
name=iterazione,
description={Procedere per iterazioni significa operare raffinamenti o rivisitazioni. Essa \`e associabile a un'operazione, ad un qualcosa fatto prima (gi\`a fatto). Questa operazione \`e potenzialmente distruttiva e ha caratteristiche molto pericolose, perch\`e non sa garantire come finir\`a ed \`e una ripetizione di una cosa che ho gi\`a fatto},
plural=iterazioni
}

\newglossaryentry{incremento}
{
name=incremento,
description={Avvicinamento alla meta che si compie in due modi: aggiungendo o togliendo. Procedere per incrementi significa aggiungere a un impianto base. Un incremento non pu\`o mai tornare sui suoi passi, ed \`e preferibile rispetto alla iterazione, perch\`e pianifica i passi e ci\`o significa che si arriver\`a a una fine},
plural=incrementi
}

\newglossaryentry{prototipo}
{
name=prototipo,
description={Originario, abbozza, serve per capire se si sta andando in una direzione giusta o no. Esistono due tipi di prototipi: usa e getta da usare solamente se il beneficio \`e molto maggiore del costo per produrla, altrimenti se si presta ad essere la soluzione, anche se pu\`o essere una base per una iterazione},
plural=prototipi
}
\newglossaryentry{riuso}
{
name=riuso,
description={Due tipi di riuso: opportunistico (in stile copy-paste, a basso coso ma scarso impatto), altrimenti l'altro uso \`e quando si sa cosa prendo, so perch\`e lo prendo e so cosa fa. Fare software \`e fondalmentalmente riuso. \`e quindi una delle attivit\`a pi\`u importanti di SWE, assume una connotazione positiva},
plural=riusi
}

\newglossaryentry{controllodiversione}
{
name=controllo di versione,
description={--Da completare successivamente--}
}

\newglossaryentry{disciplinato}
{
name=disciplinato,
description={Saper prevedere i costi. Avere una quantit\`a credibile, seguendo le regole. Essere disciplinati significa anche seguire un ordine preciso degli stati nel ciclo di vita del software},
plural=disciplinati
}

\newglossaryentry{controllodeiprocessi}
{
name=controllo dei processi,
description={Luogo in cui si pongono delle regole per essere sempre efficaci e disciplinati}
}

\newglossaryentry{trigger}
{
name=trigger,
description={Evento che causa il cambiamento di arco nel ciclo di sviluppo del software. Attivit\`a la quale fa cambiare lo stato dell'automa}
}

\newglossaryentry{fase}
{
name=fase,
description={Durata temporale entro uno stato di ciclo di vita o in una transizione tra essi}
}


\newglossaryentry{pre-condizione-cascata}
{
name=pre-condizione,
description={Nel modello a cascata, la pre-condizione \`e ci\`o che \`e verificato prima di entrare in un certo stato}
}

\newglossaryentry{post-condizione-cascata}
{
name=post-condizione,
description={Nel modello a cascata, \`e ci\`o che dev'essere vero dopo lo svolgimento delle attivit\`a}
}

\newglossaryentry{meta-modello}
{
name=meta-modello,
description={Insieme di regole, vincoli e teorie utilizzate per la modellazione di una classe di problemi con astrazione dal mondo reale}
}

\newglossaryentry{casoduso}
{
name=caso d'uso,
description={Tecniche per individuare i requisiti funzionali. Queste Tecniche devono essere comprensibili anche all'utente committente. Il caso d'uso descrive l'insieme di funzionalit\`a del sistema come sono percepite dagli utenti}
}

\newglossaryentry{scenario}
{
name=scenario,
description={Rappresenta una sequenza di passi che descrivono iterazioni tra gli utenti e il sistema}
}

\newglossaryentry{attore}
{
name=attore,
description={Elemento esterno al sistema che interagisce con sistema}
}

\newglossaryentry{progetto}
{
name=progetto,
description={Insieme di tre elementi importanti: \begin{enumerate}
\item Insieme ordinato di compiti da svolgere
\item I compiti da svolgere sono pianificati da inizio a fine
\item I vincoli che vengono tenuti conto quando si pianifica nascono da quanto tempo ho a disposizione per l'intero progetto e quali strumenti \`e possibile utilizzare per dare i risultati attesi
\end{enumerate}
}
}

\newglossaryentry{rischio}
{
name=rischio,
description={\`e il non aver tenuto conto che le cose possono non andare come avevamo considerato}
}

\newglossaryentry{slack}
{
name=slack,
description={Margine tra inizio e fine di un'attivit\`a. In italiano \`e sinonimo di lasco}
}

\newglossaryentry{primaryprocess}
{
name=processo primario,
description={I processi primari includono processi sofware per: \begin{itemize}
\item Sviluppo
\item Operazioni o funzioni
\item Mantenimento del software
\end{itemize}
[def. SWEBok-v3 8-2.1.3]}
}

\newglossaryentry{supportinprocess}
{
name=processo di supporto,
description={Processi di supporto sono applicati discontinuamente o continuamente durante il ciclo di vita del software a supporto dei processi primari; questi includono:\begin{itemize}
\item Gestione configurazione
\item Controllo della qualit\`a
\item Verifica e validazione
\end{itemize}
[def. SWEBok-v3 8-2.1.3]}
}

\newglossaryentry{organizationalprocess}
{
name=processo organizzativo,
description={I processi organizzativi provvedono al supporto all'ingegneria del software. Includono: \begin{itemize}
\item Formazione
\item Analisi di misura del processo
\item Gestione dell'infrastruttura
\item Portfolio e riuso
\item Organizzazione miglioramento dei processi
\item Gestione del modello del ciclo di vita del software
\end{itemize}
[def. SWEBok-v3 8-2.1.3]}
}

\newglossaryentry{SDLC}
{
name=ciclo di vita dello sviluppo software (SDLC),
description={Un ciclo di vita dello sviluppo software include i processi software usati per specificare e trasformare requisiti software in un prodotto software finito [def. SWEBok-v3 8-2]}
}

\newglossaryentry{SPLC}
{
name=ciclo di vita del prodotto software (SPLC),
description={Un ciclo di vita del prodotto software include un SDLC pi\`u addizionali processi software che provvedono al: \begin{itemize}
\item distribuzione
\item mantenimento
\item supporto
\item evoluzione
\item ritiro
\end{itemize}
e tutti gli altri processi di inizio al ritiro, includendo processi di gestione per il controllo della configurazione e della qualit\`a applicati durante il ciclo di vita del prodotto software [def. SWEBok-v3 8-2]}
}

\newglossaryentry{camminocritico}
{
name=cammino critico,
description={Sequenza di attivit\`a-progetto che ha lo slack pi\`u piccolo}
}

\newglossaryentry{configuration}
{
name=configuration,
description={Definizioni multiple:
\begin{enumerate}
\item La configuration si basa sul concetto di sistema e la si ha dall'inizio di uno sviluppo (conception) fino alla fine (uso operativo) Ogni pezzo del sistema ha il suo perch\`e e della conception fino all'uso operativo ha diverse configuration. Si hanno tante configuration  potenzialmente in base alle configurazioni che si avranno, e si deve decidere quando averranno i cambiamento di scarsa o molta importanza. Questa decisione avviene attraverso le milestone.
\item Una "Software Configuration" \`e l'insieme delle funzionalit\`a e delle caratteristiche di hardware o software cos\`i come indicate nella documentazione o raggiunte in un prodotto. [SWEBok 6-6]
\end{enumerate}
}
}

\newglossaryentry{configurationitem}{
name=configuration item,
description={Definizioni multiple:
\begin{enumerate}
\item Un configuration item \`e un elemento o un'aggragazione di hardware e/o software che pu\`o essere gestito come una singola entit\`a.[SWEBok 6-6]
\item Un configuration item \`e qualsiasi cosa associata ad un progetto software (progettazione, codice, dati di test, documentazione) che sia stato messo sotto un controllo di configurazione. Spesso un configuration item ha diverse versioni, e ha un nome univoco. [Sommerville, pag 684]
\end{enumerate}
}
}

\newglossaryentry{softwareconfigurationitem}{
name=software configuration item,
description={Un software configuration item \`e un'entit\`a software che \`e stata stabilita come configuration item.[SWEBok 6-6]}
}

\newglossaryentry{SCM}{
name=software Configuration Management (SCM),
description={L'SCM è un processo di supporto al ciclo di vita di un software che avvantaggia la gestione di progetto, le attività di sviluppo e manutenzione, l'attività di garanzia di qualità, così come gli utenti e i clienti del prodotto finale.[SWEBok 6-1]}
}

\newglossaryentry{milestone}
{
name=milestone,
description={Le milestone servono per fissare dei punti di avanzamento significativi rispetto agli obiettivi stabiliti e al tempo a disposizione.
Un progettatore assegna milestone che hanno una distanza tale per cui arrivarci significa raggiungere un punto importante: infatti, ogni milestone corrispone a una specifica configurazione del sistema.
Ogni milestone ha un proprio nome se associata e una configurazione detta \textit{baseline}}
}

\newglossaryentry{baseline}
{
name=baseline,
description={Definizioni multiple:
\begin{enumerate}
\item La baseline \`e un punto di avanzamento certo, dal quale non si torna mai indietro. Viene visto come un punto di situazione certa dalla quale si potr\`a soltanto avanzare senza mai retrocedere
\item La baseline \`e una versione approvata di un configuration item che \`e stata formalmente progettata e definita (/sistemata, "fixed") in un momento specifico del ciclo di vita del configuration item. [SWEBok 6-7]
\item Una baseline \`e una collezione delle versioni dei componenti che compongono un sistema. Le baseline sono controllate, il che significa che le versioni dei componenti che compongono il sistema non possono essere cambiate e che \`e sempre possibile ricreare una baseline a partire dai componenti che la costituiscono. [Sommerville, pag 684]
\end{enumerate}
}
}

\newglossaryentry{versione}{
name=versione,
description={Definizioni multiple:
\begin{enumerate}
\item La versione di un elemento software \`e un'istanza identificata dell'elemento stesso. Pu\`o essere pensata come uno stato di un elemento in evoluzione.[SWEBok 6-7]
\item Una versione \`e un'istanza di un configuration item che differisce, in qualche modo, dalle altre istanze di quell'item. Le versioni hanno sempre un'identificatore unico, che spesso \`e composto dal nome del configuration item pi\`u un numero di versione. [Sommerville, pag 684]
\end{enumerate}
}
}

\newglossaryentry{controllodiconfigurazione}{
name=controllo di configurazione,
description={\`E il processo che garantisce che le versioni di un sistema e i componenti siano registrati e mantenuti in modo da poter gestire i cambiamenti e poter identificare e memorizzare tutte le versioni dei componenti durante il tempo di vita del sistema. [Sommerville, pag 684]}
}

\newglossaryentry{codeline}{
name=codeline,
description={\`E un insieme di versioni di un componente software e dei configuration item dai quali dipende. In altre parole \`e una sequenza di versioni di codice sorgente nella quale le versioni successive derivano dalle precedenti. [Sommerville, pag 684/690]}
}

\newglossaryentry{mainline}{
name=mainline,
description={\`E una sequenza di baseline che rappresenta le differenti versioni di un sistema. [Sommerville, pag 684]}
}

\newglossaryentry{release}{
name=release,
description={\`E una versione di un sistema rilasciata ai consumatori. [Sommerville, pag 684]}
}

\newglossaryentry{workspace}{
name=workspace,
description={\`E un'area di lavoro privata dove il software pu\`o essere modificato senza influenze da parte degli altri sviluppatori che stanno modificando lo stesso software. [Sommerville, pag 684]}
}

\newglossaryentry{branching}{
name=branching,
description={\`E la creazione di una nuova codeline a partire da una esistente. Le due codeline possono essere sviluppate in modo indipendente. [Sommerville, pag 684]}
}

\newglossaryentry{merging}{
name=merging,
description={\`E la creazione di una nuova versione di un componente ottenuta unendo versioni separate in codeline differenti. Queste codeline possono essere state create da una precedente ramificazione (branch).  [Sommerville, pag 684]}
}

\newglossaryentry{systembuilding}{
name=system building,
description={\`E la creazione di una nuova versione eseguibile del sistema attraverso la compilazione e il "linkaggio" di versioni appropriate dei componente e delle librerie che compongono il sistema. [Sommerville, pag 684]}
}

\newglossaryentry{stimaqualitativa}{
name = stima qualitativa,
description = {La stima si basa sul giudizio di esperti [SWEBok - 8.3.2]}
}

\newglossaryentry{stimaquantitativa}{
name = stima quantitativa,
description = {La valutazione della stima la si assegna attraverso un punteggio sulla base delle analisi di risultati che indicano il raggiungimento dell'obiettivo e l'esito di un processo definito [SWEBok - 8.3.2]}
}

\newglossaryentry{classificazionefasi}{
name = classificazione a fasi, 
description = {La classifcazione di un processo software \`e stabilita assegnando la stessa valutazione di maturit\`a a tutti i processi all'interno di un specifico livello [SWEBok - 8.3.4]}
}

\newglossaryentry{classificazionecontinua}{
name = classificazione continua,
description = {La classificazione avviene assegnando una valutazione ad ogni processo d'interesse [SWEBok - 8.3.4]}
}

\newglossaryentry{produttivita}{
name = produttivit\`a,
description = {Il rapporto tra l'output prodotto e le risorse consumate [SWEBok-v3 8.4.1], ovvero \[\frac{efficacia}{efficienza}\]}
}

\newglossaryentry{tracciamento}{
name = tracciamento,
description = {Procedimento tramite il quale per ogni baseline si sa ci\`o che si \`e fatto e perch\`e e si conosce la qualit\`a del lavoro svolto [def. Prof. Vardanega]}
}

\newglossaryentry{framework}{
name = framework,
description = {(in italiano: quadro di lavoro) Insieme di regole che costruiscono una soluzione coerente [def. Prof. Vardanega]}
}

\newglossaryentry{javascript}{
name = javascript,
description = {linguaggio di scripting orientato agli oggetti, usato prevalentemente nella programmazione Web lato client.}
}

\newglossaryentry{mongodb}{
name = mongodb,
description = {\textit{DBMS} di tipo \textit{NoSQL} tra i pi\`u diffusi.}
}

\newglossaryentry{NoSQL}{
name = NoSQL,
description = {(acronimo per Not-Only-SQL) è il nome di un tipo di \textit{DBMS} che non prevede soltanto l'utilizzo del \textit{modello relazionale}
              utilizzato dai sistemi classici di tipo \textit{SQL}.}
}

\newglossaryentry{node.js}{
name = node.js,
description = {Framework %todo: alla parola del glossario
            per la creazione di applicazioni distribuite. Utilizza JavaScript %todo: parola di glossario
            come linguaggio di scripting e gestisce le attese I/O in modo asincrono.
}
}


\newglossaryentry{angular.js}{
name = angular.js,
description = { Framework \iffalse link to glossary\fi open-source \iffalse link to glossary\fi  
               per lo sviluppo di applicazioni web. Fornisce una piattaforma per l'implementazione
               del \textit{pattern} \iffalse link to glossary\fi MVC (Model-View-Controller)
               lato \textit{client}.
}
}

\newglossaryentry{react.js}{
name = react.js,
description = { Libreria Javascript %todo: link to glossary
            
}
}


\newglossaryentry{back end}{
name = back end,
description = { Termine inglese che indica la parte di un sistema \textit{software} che si occupa della memorizzazione
            e del recupero dei dati.
            
}
}


%todo: aggiungere 'front end'

\newglossaryentry{dsl}{
name = domain specific language,
description = { (acronimo per \textbf{D}omain \textbf{S}pecific \textbf{L}anguage) Linguaggio per computer
            specializzato per uno specifico dominio di applicazione.
}
}

\newglossaryentry{react.js}{
name = react.js,
description = { React è una libreria Javascript per la creazione di interfacce utente realizzata da Facebook. 
Tale libreria ha lo scopo di implementare la parte visiva dei dati nel pattern MVC.
I suoi punti di forza sono la semplicità del suo utilizzo e la modularità dei suoi componenti
}
}

\newglossaryentry{qualita}{
name = qualità,
description= {Insieme di caratteristiche di un'entità che ne determinano la capacità di soddisfare esigenze espresse o implicite.}
}

\newglossaryentry{team}{
name = team,
description= {Insieme di persone.}
}

\newglossaryentry{verifica}{
name = verifica,
description = {Attività volta alla ricerca di consistenza, correttezza e completezza.}
}

\newglossaryentry{validazione}{
name = validazione,
description = {Controllo effettuato sul software, per controllare se tutti i requisiti previsti sono stati coperti.}
}

\newglossaryentry{codicesorgente}{
name = codice sorgente,
description = {Testo di un programma scritto in un linguaggio di programmazione da parte di un programmatore in fase di programmazione.}
}

\newglossaryentry{programmatore}{
name = programmatore,
description = {Persona che codifica un algoritmo in uno specifico linguaggio di programmazione.}
}

\newglossaryentry{programmazione}{
name = programmazione,
description = {Insieme delle attività e tecniche che una o più persone specializzate, i programmatori, svolgono per creare un software scrivendo il relativo codice sorgente in un certo linguaggio di programmazione.}
}

\newglossaryentry{linguaggiodiprogrammazione}{
name = linguaggio di programmazione,
description = {Insieme di regole usato per scrivere il codice sorgente di un software.}
}

\newglossaryentry{bycorrection}{
name = by correction,
description = {Ottenere la correttezza di un software procedendo per correzioni, ovvero applicando un metodo iterativo. Si tratta di un metodo errato in quanto fa perdere molto tempo e non garantisce a priori la correttezza finale.}
}

\newglossaryentry{funzioneaziendale}{
name = funzione aziendale,
description = {Insieme di attività svolte all'interno dell'azienda.}
}

\newglossaryentry{committente}{
name = committente,
description = {Figura che commissiona un lavoro.}
}

\newglossaryentry{maas}{
name = mongodb as an admin service,
description = {Acronimo di \textbf{M}ongoDB} \textbf{a}s an \textbf{a}dmin \textbf{S}ervice.}
}

\newglossaryentry{maap}{
name = mongodb as an admin platform,
description = {Acronimo di \textbf{M}ongoDB} \textbf{a}s an \textbf{a}dmin \textbf{P}latform.}
}

\newglossaryentry{anomalia}{
name = anomalia,
description = {Presenza di elementi non riconducibili al modello prototipo di una classificazione o al normale svolgimento di determinate funzioni.}
}

\newglossaryentry{overhead}{
name = overhead,
description = {Risorse accessorie, richieste in sovrappiù rispetto a quelle strettamente necessarie per ottenere un determinato scopo in seguito all'introduzione di un metodo o di un processo più evoluto o più generale.}
}

\newglossaryentry{responsivita}{
name = responsività,
description = {Velocità di risposta.}
}

\newglossaryentry{proattivo}{
name = proattivo,
description = {Chi opera con il supporto di metodologie e strumenti utili a percepire anticipatamente i problemi, le tendenze o i cambiamenti futuri, al fine di pianificare le azioni opportune in tempo.}
}