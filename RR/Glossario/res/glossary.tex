\newglossaryentry{sistematico}
{
name=sistematico,
description={Colui che sa fare quella cosa - darsi delle regole - approcciarsi al problema con metodo. Ci\`o contribuisce all'efficienza e all'efficacia}
}

\newglossaryentry{stakeholder}
{
name=stakeholder,
description={Dall'inglese, portatore d'interesse. È l'insieme di persone coinvolte a vario titolo nel ciclo di vita del software, con influenza sul prodotto}
}

\newglossaryentry{processo}
{
name=processo, 
description={definizioni multiple:
\begin{enumerate}
\item Detto in inglese \textit{Way of working}. Esso \`e rappresentabile come un automa a stati, dove ogni stato rappresenta uno stadio del ciclo di vita del processo;
\item Un insieme di attività interconnesse, che trasforma uno o più input in output consumando risorse. [SWEBok 8-1]
\end{enumerate}
}
}

\newglossaryentry{procedura}
{
name=procedura,
description={Una procedura \'e un insieme ordinato di passi o, alternativamente, controlli del lavoro per eseguire il task}
}


\newglossaryentry{sink}
{
name=sink,
description={Dall'inglese ``Punto di fine", scarico da cui non è più possibile uscire (ovvero uno stato finale)}
}

\newglossaryentry{source}
{
name=source,
description={Dall'inglese, stato che ammette solo archi in uscita}
}

\newglossaryentry{ciclovitasoftware}
{
name=ciclo di vita del software,
description={Insieme degli stati che il prodotto assume dal concepimento al ritiro}
}

\newglossaryentry{efficacia}
{
name=efficacia,
description={Definizioni multiple:
\begin{enumerate}
\item Conformit\`a al contratto. Si garantisce quindi ci\`o che si deve fare, ed \`e determinata dal grado di conformit\`a del progetto alle norme vigenti e agli obiettivi prefissati. L'efficacia \`e direttamente proporzionale alla quantit\`a di risorse impiegate;
\item L'efficacia \`e il rapporto tra l'output attuale e quello attesso, prodotto dal processo, attivit\`a o compito [SWEBok-v3 8.4.1].
\end{enumerate}
}
}

\newglossaryentry{efficienza}
{
name=efficienza,
description={Definizioni multiple: 
\begin{enumerate}
\item L'efficienza \`e la capacit\`a di azione o di produzione con il minimo di scarto, di spesa, di risorse e di tempo impiegati. \`E inversamente proporzionale alla quantit\`a di risorse impiegate nell'esecuzione delle attivit\`a richieste;
\item L'efficienza \`e il rapporto tra le risorse consumate e quelle attese o desiderate nel compiere un processo, attivit\`a o compito [SWEBok-v3 8.4.1].
\end{enumerate}
}
}

\newglossaryentry{bestpractice}
{
name=best practice,
description={Prassi (modo di fare) che per esperienza e per studio abbia mostrato di garantire i migliori risultati in circostanze note e specifiche}
}

\newglossaryentry{iterazione}
{
name=iterazione,
description={Procedere per iterazioni significa operare raffinamenti o rivisitazioni. Essa \`e associabile a un'operazione, ad un qualcosa fatto prima. Questa operazione \`e potenzialmente distruttiva e ha caratteristiche potenzialmente dannose, perch\`e non sa garantire come finir\`a ed \`e una ripetizione di una cosa gi\`a fatta in precedenza},
plural=iterazioni
}

\newglossaryentry{incremento}
{
name=incremento,
description={Avvicinamento alla meta che si compie in due modi: aggiungendo o togliendo. Procedere per incrementi significa aggiungere ad un impianto base. Un incremento non pu\`o mai tornare sui suoi passi, ed \`e preferibile rispetto all'iterazione, perch\`e pianifica i passi; ci\`o significa che si arriver\`a a una fine},
plural=incrementi
}

\newglossaryentry{prototipo}
{
name=prototipo,
description={Bozza, serve per capire se si sta andando nella direzione giusta o no. Esistono due tipi di prototipi: usa e getta, da usare solamente se il beneficio \`e molto maggiore del costo per produrla, altrimenti, se si presta ad essere la soluzione, anche se pu\`o essere una base per una iterazione},
plural=prototipi
}
\newglossaryentry{riuso}
{
name=riuso,
description={Due tipi di riuso: opportunistico (in stile copy-paste, a basso costo ma scarso impatto), oppure consapevolee finalizzato ad uno scopo preciso. Fare software \`e fondalmentalmente riuso. \`E quindi una delle attivit\`a pi\`u importanti nell'Ingegneria del Software, e assume una connotazione positiva},
plural=riusi
}

\newglossaryentry{controllodiversione}
{
name=controllo di versione,
description={--Da completare successivamente--}
}

\newglossaryentry{disciplinato}
{
name=disciplinato,
description={Saper prevedere i costi. Avere una quantit\`a credibile, seguendo le regole. Essere disciplinati significa inoltre rispettare un ordine preciso negli stadi del ciclo di vita del software},
plural=disciplinati
}

\newglossaryentry{controllodeiprocessi}
{
name=controllo dei processi,
description={Luogo in cui si pongono delle regole per essere sempre efficaci e disciplinati}
}

\newglossaryentry{trigger}
{
name=trigger,
description={Evento che causa il cambiamento di arco nel ciclo di sviluppo del software. Attivit\`a in grado di modificare lo stato dell'automa}
}

\newglossaryentry{fase}
{
name=fase,
description={Durata temporale entro uno stato del ciclo di vita o in una transizione tra essi}
}


\newglossaryentry{pre-condizione-cascata}
{
name=pre-condizione,
description={Nel modello a cascata, la pre-condizione \`e ci\`o che viene verificato prima di entrare in un certo stato}
}

\newglossaryentry{post-condizione-cascata}
{
name=post-condizione,
description={Nel modello a cascata, \`e ci\`o che dev'essere vero dopo lo svolgimento delle attivit\`a}
}

\newglossaryentry{meta-modello}
{
name=meta-modello,
description={Insieme di regole, vincoli e teorie utilizzate per la modellazione di una classe di problemi con astrazione dal mondo reale}
}

\newglossaryentry{casoduso}
{
name=caso d'uso,
description={Tecnica per individuare i requisiti funzionali. Queste tecniche devono essere comprensibili anche al committente. Il caso d'uso descrive l'insieme di funzionalit\`a del sistema come percepite dagli utenti}
}

\newglossaryentry{scenario}
{
name=scenario,
description={Rappresenta una sequenza di passi che descrivono le interazioni tra gli utenti e il sistema}
}

\newglossaryentry{attore}
{
name=attore,
description={Elemento esterno al sistema che interagisce con esso}
}

\newglossaryentry{progetto}
{
name=progetto,
description={Insieme di tre elementi importanti: \begin{enumerate}
\item Sequenza ordinata di compiti da svolgere;
\item I compiti da svolgere sono pianificati da inizio a fine;
\item I vincoli di cui si tiene conto quando si pianifica nascono dalla disponibilità di tempo e di strumenti per il progetto.
\end{enumerate}
}
}

\newglossaryentry{rischio}
{
name=rischio,
description={evento potenzialemente dannoso che potrebbe verificarsi in corso d'opera}
}

\newglossaryentry{slack}
{
name=slack,
description={Margine tra inizio e fine di un'attivit\`a.}
}

\newglossaryentry{primaryprocess}
{
name=processo primario,
description={I processi primari includono processi sofware per: \begin{itemize}
\item Sviluppo;
\item Operazioni o funzioni;
\item Mantenimento del software;
\end{itemize}
[def. SWEBok-v3 8-2.1.3]}
}

\newglossaryentry{supportingprocess}
{
name=processo di supporto,
description={I processi di supporto sono applicati discontinuamente o continuamente durante il ciclo di vita del software, a supporto dei processi primari; questi includono:\begin{itemize}
\item Gestione configurazione;
\item Controllo della qualit\`a;
\item Verifica e validazione.
\end{itemize}
[def. SWEBok-v3 8-2.1.3]}
}

\newglossaryentry{organizationalprocess}
{
name=processo organizzativo,
description={I processi organizzativi provvedono al supporto all'Ingegneria del Software. Includono: \begin{itemize}
\item Formazione;
\item Analisi di misura del processo;
\item Gestione dell'infrastruttura;
\item Portfolio e riuso;
\item Organizzazione del miglioramento dei processi;
\item Gestione del modello del ciclo di vita del software.
\end{itemize}
[def. SWEBok-v3 8-2.1.3]}
}

\newglossaryentry{SDLC}
{
name=ciclo di vita dello sviluppo software (SDLC),
description={(Acronimo per \textbf{S}oftware \textbf{D}evelopment \textbf{L}ife \textbf{C}ycle) Un ciclo di vita dello sviluppo software include i processi usati per specificare e trasformare requisiti software in un prodotto software finito [def. SWEBok-v3 8-2]}
}

\newglossaryentry{SPLC}
{
name=ciclo di vita del prodotto software (SPLC),
description={(Acronimo per \textbf{S}oftware \textbf{P}roduct \textbf{L}ife \textbf{C}ycle) Un ciclo di vita del prodotto software include un SDLC oltre ad altri processi software addizionali che provvedono a: \begin{itemize}
\item Distribuzione;
\item Mantenimento;
\item Supporto;
\item Evoluzione;
\item Ritiro.
\end{itemize}
e tutti gli altri processi compresi tra l'inizio ed il ritiro, includendo processi di gestione per il controllo della configurazione e della qualit\`a applicati durante il ciclo di vita del prodotto software [def. SWEBok-v3 8-2]}
}

\newglossaryentry{camminocritico}
{
name=cammino critico,
description={Sequenza di attivit\`a-progetto che ha lo slack pi\`u piccolo}
}

\newglossaryentry{configuration}
{
name=configuration,
description={Definizioni multiple:
\begin{enumerate}
\item La configuration si basa sul concetto di sistema e la si ha dall'inizio di uno sviluppo (\textit{conception}) fino alla fine (uso operativo). Ogni componente del sistema ha il suo perch\`e, inoltre dalla conception fino all'uso operativo ha diverse \textit{configuration}. Si hanno tante configuration potenzialmente in base alle configurazioni che si avranno, inoltre va stabilito quando averranno i cambiamenti di importanza scarsa o rilevante. Questa decisione viene presa tramite l'ausilio le \textit{milestone}.
\item Una "Software Configuration" \`e l'insieme delle funzionalit\`a e delle caratteristiche di hardware o software cos\`i come indicate nella documentazione o raggiunte in un prodotto. [SWEBok 6-6]
\end{enumerate}
}
}

\newglossaryentry{configurationitem}{
name=configuration item,
description={Definizioni multiple:
\begin{enumerate}
\item Un configuration item \`e un elemento o un'aggregazione di hardware e/o software che pu\`o essere gestito come una singola entit\`a.[SWEBok 6-6]
\item Un configuration item \`e qualsiasi cosa associata ad un progetto software (progettazione, codice, dati di test, documentazione) che sia stato messo sotto un controllo di configurazione. Spesso un configuration item ha diverse versioni, e ha un nome univoco. [Sommerville, pag 684]
\end{enumerate}
}
}

\newglossaryentry{softwareconfigurationitem}{
name=software configuration item,
description={Un software configuration item \`e un'entit\`a software che \`e stata stabilita come configuration item.[SWEBok 6-6]}
}

\newglossaryentry{SCM}{
name=Software Configuration Management (SCM),
description={L'SCM è un processo di supporto al ciclo di vita del software che migliora la gestione di progetto, le attività di sviluppo e manutenzione, l'attività di garanzia di qualità, così come gli utenti e i clienti del prodotto finale.[SWEBok 6-1]}
}

\newglossaryentry{milestone}
{
name=milestone,
description={Le milestone servono per fissare dei punti di avanzamento significativi rispetto agli obiettivi stabiliti e al tempo a disposizione.
Un progettista assegna milestone che hanno una distanza tale per cui arrivarci significa raggiungere un punto importante: infatti, ogni milestone corrisponde ad una specifica configurazione del sistema.
Ogni milestone ha un proprio nome se associata e una configurazione detta \textit{baseline}}
}

\newglossaryentry{baseline}
{
name=baseline,
description={Definizioni multiple:
\begin{enumerate}
\item La baseline \`e un punto di avanzamento certo, dal quale non è possibile tornare indietro. Viene visto come un punto di una situazione certa dalla quale si potr\`a soltanto avanzare senza mai retrocedere;
\item La baseline \`e una versione approvata di un configuration item che \`e stata formalmente progettata e definita (/sistemata, "fixed") in un momento specifico del ciclo di vita del configuration item; [SWEBok 6-7]
\item Una baseline \`e una collezione delle versioni dei componenti che compongono un sistema. Le baseline sono controllate, il che significa che le versioni dei componenti che compongono il sistema non possono essere cambiate e che \`e sempre possibile ricreare una baseline a partire dai componenti che la costituiscono. [Sommerville, pag 684]
\end{enumerate}
}
}

\newglossaryentry{versione}{
name=versione,
description={Definizioni multiple:
\begin{enumerate}
\item La versione di un elemento software \`e un'istanza identificata dell'elemento stesso. Pu\`o essere pensata come uno stato di un elemento in evoluzione.[SWEBok 6-7]
\item Una versione \`e un'istanza di un configuration item che differisce, in qualche modo, dalle altre istanze di quell'item. Le versioni hanno sempre un'identificatore unico, che spesso \`e composto dal nome del configuration item pi\`u un numero di versione. [Sommerville, pag 684]
\end{enumerate}
}
}

\newglossaryentry{controllodiconfigurazione}{
name=controllo di configurazione,
description={\`E il processo che garantisce che le versioni di un sistema e i componenti siano registrati e mantenuti in modo da poter gestire i cambiamenti e poter identificare e memorizzare tutte le versioni dei componenti durante il tempo di vita del sistema. [Sommerville, pag 684]}
}

\newglossaryentry{codeline}{
name=codeline,
description={\`E un insieme di versioni di un componente software e dei configuration item dai quali dipende. In altre parole \`e una sequenza di versioni di codice sorgente nella quale le versioni successive derivano dalle precedenti. [Sommerville, pag 684/690]}
}

\newglossaryentry{mainline}{
name=mainline,
description={\`E una sequenza di baseline che rappresenta le differenti versioni di un sistema. [Sommerville, pag 684]}
}

\newglossaryentry{release}{
name=release,
description={\`E una versione di un sistema rilasciata ai consumatori. [Sommerville, pag 684]}
}

\newglossaryentry{workspace}{
name=workspace,
description={\`E un'area di lavoro privata dove il software pu\`o essere modificato senza influenze da parte degli altri sviluppatori che stanno modificando lo stesso software. [Sommerville, pag 684]}
}

\newglossaryentry{branching}{
name=branching,
description={\`E la creazione di una nuova codeline a partire da una esistente. Le due codeline possono essere sviluppate in modo indipendente. [Sommerville, pag 684]}
}

\newglossaryentry{merging}{
name=merging,
description={\`E la creazione di una nuova versione di un componente ottenuta unendo versioni separate in codeline differenti. Queste codeline possono essere state create da una precedente ramificazione (branch).  [Sommerville, pag 684]}
}

\newglossaryentry{systembuilding}{
name=system building,
description={\`E la creazione di una nuova versione eseguibile del sistema attraverso la compilazione e il \textit{linking} di versioni appropriate dei componenti e delle librerie che compongono il sistema. [Sommerville, pag 684]}
}

\newglossaryentry{stimaqualitativa}{
name = stima qualitativa,
description = {La stima si basa sul giudizio di esperti [SWEBok - 8.3.2]}
}

\newglossaryentry{stimaquantitativa}{
name = stima quantitativa,
description = {La valutazione della stima viene assegnata attraverso un punteggio sulla base delle analisi di risultati che indicano il raggiungimento dell'obiettivo e l'esito di un processo definito [SWEBok - 8.3.2]}
}

\newglossaryentry{classificazionefasi}{
name = classificazione a fasi, 
description = {La classifcazione di un processo software \`e stabilita assegnando la stessa valutazione di maturit\`a a tutti i processi all'interno di un specifico livello [SWEBok - 8.3.4]}
}

\newglossaryentry{classificazionecontinua}{
name = classificazione continua,
description = {La classificazione avviene assegnando una valutazione ad ogni processo d'interesse [SWEBok - 8.3.4]}
}

\newglossaryentry{produttività}{
name = produttivit\`a,
description = {Il rapporto tra l'output prodotto e le risorse consumate [SWEBok-v3 8.4.1], ovvero \[\frac{efficacia}{efficienza}\]}
}

\newglossaryentry{tracciamento}{
name = tracciamento,
description = {Procedimento tramite il quale per ogni baseline si sa ci\`o che si \`e fatto e perch\`e, inoltre  si conosce la qualit\`a del lavoro svolto [def. Prof. Vardanega]}
}

\newglossaryentry{framework}{
name = framework,
description = {(in italiano: quadro di lavoro) Insieme di regole che costruiscono una soluzione coerente [def. Prof. Vardanega]}
}

\newglossaryentry{javascript}{
name = javascript,
description = {linguaggio di scripting orientato agli oggetti, usato prevalentemente nella programmazione web lato client.}
}

\newglossaryentry{mongodb}{
name = mongodb,
description = {\textit{DBMS} di tipo \textit{NoSQL} tra i pi\`u diffusi.}
}

\newglossaryentry{NoSQL}{
name = NoSQL,
description = {(acronimo per Not-Only-SQL) è il nome di un tipo di \textit{DBMS} che non prevede soltanto l'utilizzo del \textit{modello relazionale}
              utilizzato dai sistemi classici di tipo \textit{SQL}.}
}

\newglossaryentry{node.js}{
name = node.js,
description = {Framework 
            per la creazione di applicazioni distribuite. Utilizza JavaScript 
            come linguaggio di scripting e gestisce le attese I/O in modo asincrono.
}
}


\newglossaryentry{angular.js}{
name = angular.js,
description = { Framework open-source per lo sviluppo di applicazioni web. 
              Fornisce una piattaforma per l'implementazione
               del \textit{pattern} MVC (Model-View-Controller)
               lato \textit{client}.
}
}

\newglossaryentry{react.js}{
name = react.js,
description = { Libreria Javascript per la creazione di interfacce utente realizzata da Facebook. 
      Tale libreria ha lo scopo di implementare la parte visiva dei dati nel pattern MVC.
      I suoi punti di forza sono la semplicità nel suo utilizzo e la modularità dei componenti  
}
}


\newglossaryentry{back end}{
name = back end,
description = { Termine inglese che indica la parte di un sistema \textit{software} che si occupa della memorizzazione
            e del recupero dei dati.
            
}
}


%todo: aggiungere 'front end'

\newglossaryentry{dsl}{
name = domain specific language,
description = { (acronimo per \textbf{D}omain \textbf{S}pecific \textbf{L}anguage) Linguaggio per computer
            specializzato per uno specifico dominio di applicazione.
}
}

\newglossaryentry{react.js}{
name = react.js,
description = { React è una libreria Javascript per la creazione di interfacce utente realizzata da Facebook. 
Tale libreria ha lo scopo di implementare la parte visiva dei dati nel pattern MVC.
I suoi punti di forza sono la semplicità del suo utilizzo e la modularità dei suoi componenti
}
}

\newglossaryentry{qualita}{
name = qualità,
description= {Insieme di caratteristiche di un'entità che ne determinano la capacità di soddisfare esigenze espresse o implicite.}
}

\newglossaryentry{team}{
name = team,
description= {Insieme di persone.}
}

\newglossaryentry{verifica}{
name = verifica,
description = {Attività volta alla ricerca di consistenza, correttezza e completezza.}
}

\newglossaryentry{validazione}{
name = validazione,
description = {Controllo effettuato sul software, per controllare se tutti i requisiti previsti sono stati coperti.}
}

\newglossaryentry{codicesorgente}{
name = codice sorgente,
description = {Testo di un programma scritto in un linguaggio di programmazione da parte di un programmatore in fase di programmazione.}
}

\newglossaryentry{programmatore}{
name = programmatore,
description = {Persona che codifica un algoritmo in uno specifico linguaggio di programmazione.}
}

\newglossaryentry{programmazione}{
name = programmazione,
description = {Insieme delle attività e tecniche che una o più persone specializzate, i programmatori, svolgono per creare un software scrivendo il relativo codice sorgente in un certo linguaggio di programmazione.}
}

\newglossaryentry{linguaggiodiprogrammazione}{
name = linguaggio di programmazione,
description = {Insieme di regole usato per scrivere il codice sorgente di un software.}
}

\newglossaryentry{bycorrection}{
name = by correction,
description = {Ottenere la correttezza di un software procedendo per correzioni, ovvero applicando un metodo iterativo. Si tratta di un metodo errato in quanto fa perdere molto tempo e non garantisce a priori la correttezza finale.}
}

\newglossaryentry{funzioneaziendale}{
name = funzione aziendale,
description = {Insieme di attività svolte all'interno dell'azienda.}
}

\newglossaryentry{committente}{
name = committente,
description = {Figura che commissiona un lavoro.}
}

\newglossaryentry{maas}{
name = mongodb as an admin service,
description = {Acronimo di \textbf{M}ongoDB} \textbf{a}s an \textbf{a}dmin \textbf{S}ervice.}
}

\newglossaryentry{maap}{
name = mongodb as an admin platform,
description = {Acronimo di \textbf{M}ongoDB} \textbf{a}s an \textbf{a}dmin \textbf{P}latform.}
}

\newglossaryentry{anomalia}{
name = anomalia,
description = {Presenza di elementi non riconducibili al modello prototipo di una classificazione o al normale svolgimento di determinate funzioni.}
}

\newglossaryentry{overhead}{
name = overhead,
description = {Risorse accessorie, richieste in sovrappiù rispetto a quelle strettamente necessarie per ottenere un determinato scopo in seguito all'introduzione di un metodo o di un processo più evoluto o più generale.}
}

\newglossaryentry{responsivita}{
name = responsività,
description = {Velocità di risposta.}
}

\newglossaryentry{proattivo}{
name = proattivo,
description = {Chi opera con il supporto di metodologie e strumenti utili a percepire anticipatamente i problemi, le tendenze o i cambiamenti futuri, al fine di pianificare le azioni opportune in tempo.}
}


\newglossaryentry{tool}{
name = tool,
description = {Applicazione che svolge un determinato compito di utilità.}
}

\newglossaryentry{spice}{
name = software process improvement and capability determination,
description = {Acronimo di \textbf{S}oftware \textbf{P}rocess \textbf{I}mprovement and \textbf{C}apability d\textbf{E}termination, è un insieme di documenti di standard tecnici che fornisce informazioni generali sui concetti di valutazione dei processi e dei suoi usi nei due contesti di miglioramento dei processi e valutazione della maturità dei processi.}
}

\newglossaryentry{pdca}{
name = plan do check act,
description = {Acronimo di \textbf{P} \textbf{D}o \textbf{C}heck \textbf{A}ct, è un metodo iterativo a quattro stadi usato per il controllo e il continuo miglioramento dei processi e dei prodotti.}
}

\newglossaryentry{gulpease}{
name = gulpease,
description = {L'indice Gulpease è un indice di leggibilità di un testo tarato sulla lingua italiana. Rispetto ad altri ha il vantaggio di utilizzare la lunghezza delle parole in lettere anziché in sillabe, semplificandone il calcolo automatico.}
}

\newglossaryentry{consuntivo}{
name = consuntivo,
description = {Rendiconto  dei risultati di un dato periodo di attività.}
}

\newglossaryentry{kloc}{
name = kilo lines of code,
description = {Acronimo di \textbf{K}ilo \textbf{L}ines \textbf{O}f \textbf{C}ode, è un'unità di misura utilizzata per esprimere la dimensione di un prodotto software in base alle migliaia di linee di codice che contiene.}
}

\newglossaryentry{fp}{
name = function point,
description = {Acronimo di \textbf{F}unction \textbf{P}oint, un'unità di misura utilizzata per esprimere la dimensione delle funzionalità fornite da un prodotto software.}
}

\newglossaryentry{package}{
name = package,
description = {In alcuni linguaggi orientati agli oggetti, tra cui Java, è un meccanismo che permette di organizzare un insieme di classi tra loro correlate che concorrono allo stesso fine.}
}

\newglossaryentry{bug}{
name = bug,
description = {Errore nel codice di un software.}
}

\newglossaryentry{pollution}{
name = pollution,
description = {Tutti i file che non devono entrare nel repository.}
}

\newglossaryentry{repository}{
name = repository,
description = {Ambiente di un sistema informativo, in cui vengono gestiti i metadati, attraverso tabelle relazionali.}
}

\newglossaryentry{ast}{
name = abstract syntax tree,
description = {Acronimo di \textbf{A}bstract \textbf{S}yntax \textbf{T}ree, è una rappresentazione ad albero della struttura sintattica astratta del codice sorgente. Ogni nodo dell'albero denota un costrutto nel codice.}
}

\newglossaryentry{linguaggiodimarkup}{
name = linguaggio di markup,
description = {Insieme di regole che descrivono meccanismi di rappresentazione del testo. Fornisce la possibilità di formattare un testo per mezzo di marcatori, cioè espressioni codificate fornite dalle convenzioni del linguaggio.}
}

\newglossaryentry{png}{
name = portable network graphics,
description = { Acronimo per \textbf{P}ortable \textbf{N}etwork \textbf{G}raphics, è un formato di file per la memorizzazione di immagini.}

\newglossaryentry{dashboard}{
name = dashboard,
description = { (in italiano `cruscotto') pagina web rappresentante lo stato corrente di un'applicazione
con interfaccia semplice e immediata.}

\newglossaryentry{owner}{
name = owner,
description = { (in italiano `proprietario') \`e il proprietario di una \textit{Company}.}

\newglossaryentry{company}{
name = company,
description = { termine derivante dal capitolato per indicare un'azienda.}

\newglossaryentry{cell}{
name = cell,
  description = { elemento dell'applicazione citato nel capitolato. Esso raffigura
  un singolo valore estratto da una \textit{query} nel \textit{database} della \textit{Company}.}


\newglossaryentry{document}{
name = document,
  description = { elemento dell'applicazione citato nel capitolato. Corrisponde ad una \textit{collection show} di una \textit{Company}.}
  
  \newglossaryentry{csv}{
name = csv,
  description = { \textbf{C}omma \textbf{S}eparated `textbf{V}alues \`e un formato di file di testo utilizzato nell-esportazione di dati tabellari.}


\newglossaryentry{software as a service}{
name = csv,
  description = { è un modello di distribuzione del software applicativo dove un produttore di software sviluppa, opera (direttamente o tramite terze parti) e gestisce un'applicazione web che mette a disposizione dei propri clienti via internet.}
