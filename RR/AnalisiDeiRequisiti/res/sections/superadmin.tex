\section{Super-Admin}

Di seguito vengono elencati i casi d'uso per il super-admin.

\newpage

\subsection{Casi d'Uso}

\subsection{UC-S 0}

    \begin{figure}[h]
      \begin{center}
        \includegraphics[width=12cm]{%todo include image}
      \caption{UC-S 0 - Gestione aziende}
      \end{center} 
    \end{figure}    
    
    %Tabella 
    \begin{center}
      \bgroup
      \def\arraystretch{1.8}     
      \begin{longtable}{  p{3.5cm} | p{8cm} } 
        
        \hline
        \multicolumn{2}{ | c | }{ \cellcolor[gray]{0.9} \textbf{UC-S 0 - Gestione aziende}} \\ 
        \hline
        
        \textbf{Attori Primari} & Super-Admin\\  
        \textbf{Precondizioni}  & l'applicazione mostra al superadmin la pagina di gestione delle aziende  \\ 
        
        \textbf{Postcondizioni} & l'applicazione mostra al superadmin la pagina di gestione delle aziende \\ 
        \textbf{Scenario principale} & 1. il superadmin pu\`o creare una nuova azienda (UC-S 0.0) \newline 2. il superadmin può visualizzare il dettaglio di un'azienda 
        \newline 3. il superadmin pu\`o modificare i dati di un'azienda \newline \\ 
        \textbf{Scenari alternativi} & 1. fallisce l'inserimento di una nuova azienda \newline 2. fallisce la modifica dei dati di un'azienda \\
      \end{longtable}
      \egroup
    \end{center}

\subsection{UC-S 0.0}
    \begin{figure}[h]
      \begin{center}
        \includegraphics[width=12cm]{%todo include image}
      \caption{UC-S 0.0 - Creazione di una nuova azienda}
      \end{center} 
    \end{figure}    
    
    %Tabella 
    \begin{center}
      \bgroup
      \def\arraystretch{1.8}     
      \begin{longtable}{  p{3.5cm} | p{8cm} } 
        
        \hline
        \multicolumn{2}{ | c | }{ \cellcolor[gray]{0.9} \textbf{UC-S 0.0 - Creazione di una nuova azienda}} \\ 
        \hline
        
        \textbf{Attori Primari} & Super-Admin\\  
        \textbf{Precondizioni}  & il sistema fornisce al super-admin un form di registrazione  \\ 
        
        \textbf{Postcondizioni} & il sistema ha aggiunto una nuova azienda e il suo owner \\ 
        \textbf{Scenario principale} & 1. l'attore inserisce il nome dell'azienda \newline 2. l'attore inserisce la propria email
        \newline 3. l'attore inserisce una password \newline \\ 
        \textbf{Scenari alternativi} & 1. l'azienda esiste gi\`a  \newline 2. la email inserita \`e gi\`a stata usata \\
      \end{longtable}
      \egroup
    \end{center}

\subsection{UC-S 0.1}
    \begin{figure}[h]
      \begin{center}
        \includegraphics[width=12cm]{%todo include image}
      \caption{UC-S 0.1 - Visualizzazione dettaglio di azienda}
      \end{center} 
    \end{figure}    
    
    %Tabella 
    \begin{center}
      \bgroup
      \def\arraystretch{1.8}     
      \begin{longtable}{  p{3.5cm} | p{8cm} } 
        
        \hline
        \multicolumn{2}{ | c | }{ \cellcolor[gray]{0.9} \textbf{UC-S 0.1 - Visualizzazione dettaglio di azienda}} \\ 
        \hline
        
        \textbf{Attori Primari} & Super-Admin\\  
        \textbf{Precondizioni}  & l'applicazione mette diposizione la pagina di visualizzazione dei dettagli di un'azienda  \\ 
        
        \textbf{Postcondizioni} & l'applicazione ha reindirizzato il super-admin alla pagina di visualizzazione dell'azienda
		     		     selezionata \\ 
        \textbf{Scenario principale} & 1. il superadmin può modificare i dati dell'azienda (UC-S 0.1.0) \newline 2. il superadmin può aggiungere un nuovo utente dell'azienda
        \newline 3. il superadmin può visualizzare uno specifico utente dell'azienda \newline 4. il super-admin pu\`o modificare i dati di uno specifico utente \\ 
        \textbf{Scenari alternativi} & 1. fallimento della modifica dei dati dell'azienda   \newline 2. fallimento dell'aggiunta di un nuovo utente \newline
        3. fallimento modifica utente \\
      \end{longtable}
      \egroup
    \end{center}


\subsection{UC-S 0.1.0}
    \begin{figure}[h]
      \begin{center}
        \includegraphics[width=12cm]{%todo include image}
      \caption{UC-S 0.1.0 - Modifica dei dati di un'azienda}
      \end{center} 
    \end{figure}    
    
    %Tabella 
    \begin{center}
      \bgroup
      \def\arraystretch{1.8}     
      \begin{longtable}{  p{3.5cm} | p{8cm} } 
        
        \hline
        \multicolumn{2}{ | c | }{ \cellcolor[gray]{0.9} \textbf{UC-S 0.1.0 - Modifica dei dati di un'azienda}} \\ 
        \hline
        
        \textbf{Attori Primari} & Super-Admin\\  
        \textbf{Precondizioni}  & l'applicazione mostra il form per la modifica dei dati dell'azienda selezionata  \\ 
        
        \textbf{Postcondizioni} & il sistema ha modificato il profilo dell'azienda sulla base dei dati inseriti dal super-admin  \\ 
        \textbf{Scenari alternativi} & 1. fallimento della modifica
      \end{longtable}
      \egroup
    \end{center}

\subsection{UC-S 0.1.1}
    \begin{figure}[h]
      \begin{center}
        \includegraphics[width=12cm]{%todo include image}
      \caption{UC-S 0.1.1 - Aggiunta di un nuovo utente nell'azienda selezionata}
      \end{center} 
    \end{figure}    
    
    %Tabella 
    \begin{center}
      \bgroup
      \def\arraystretch{1.8}     
      \begin{longtable}{  p{3.5cm} | p{8cm} } 
        
        \hline
        \multicolumn{2}{ | c | }{ \cellcolor[gray]{0.9} \textbf{UC-S 0.1.1 - Aggiunta di un nuovo utente nell'azienda selezionata}} \\ 
        \hline
        
        \textbf{Attori Primari} & Super-Admin\\  
        \textbf{Precondizioni}  & l'applicazione mostra il form per l'aggiunta di un nuovo utente  \\ 
        
        \textbf{Postcondizioni} & il sistema ha aggiunto un nuovo utente associandolo all'azienda precedentemente selezionata  \\ 
      \end{longtable}
      \egroup
    \end{center}

\subsection{UC-S 0.1.2}
    \begin{figure}[h]
      \begin{center}
        \includegraphics[width=12cm]{%todo include image}
      \caption{UC-S 0.1.2 - Visualizzazione in dettaglio di un utente dell'azienda}
      \end{center} 
    \end{figure}    
    
    %Tabella 
    \begin{center}
      \bgroup
      \def\arraystretch{1.8}     
      \begin{longtable}{  p{3.5cm} | p{8cm} } 
        
        \hline
        \multicolumn{2}{ | c | }{ \cellcolor[gray]{0.9} \textbf{UC-S 0.1.2 - Visualizzazione in dettaglio di un utente dell'azienda}} \\ 
        \hline
        
        \textbf{Attori Primari} & Super-Admin\\  
        \textbf{Scopo e descrizione} & il super-admin entra nella pagina di visualizzazione di un utente, nella quale pu\`o ispezionarne
        il profilo, inoltre pu\`o essere reindirizzato alle pagine di modifica ed eliminazione del utente in questione
        \textbf{Precondizioni}  & il sistema presenta la pagina di visualizzazione in dettaglio di un utente  \\ 
        
        \textbf{Postcondizioni} & il sistema ha ricevuto l'input dall'attore  \\ 
         \textbf{Scenario principale} & 1. il super-admin pu\`o modificare il profilo dell'utente  \newline 2. il super-admin pu\`o eliminare l'utente  \\
        
         \textbf{Scenari alternativi} & 1.   \newline 2.  \\
     
     \end{longtable}
      \egroup
    \end{center}

\subsection{UC-S 0.1.3}
    \begin{figure}[h]
      \begin{center}
        \includegraphics[width=12cm]{%todo include image}
      \caption{UC-S 0.1.3 - Modifica del profilo di un utente}
      \end{center} 
    \end{figure}    
    
    %Tabella 
    \begin{center}
      \bgroup
      \def\arraystretch{1.8}     
      \begin{longtable}{  p{3.5cm} | p{8cm} } 
        
        \hline
        \multicolumn{2}{ | c | }{ \cellcolor[gray]{0.9} \textbf{UC-S 0.1.3 - Modifica del profilo di un utente }} \\ 
        \hline
        
        \textbf{Attori Primari} & Super-Admin\\  
        \textbf{Scopo e descrizione} & l'attore entra nella pagina di modifica dell'utente, nella quale ha la possibilit\`a
        di modificarne ruolo e password
      
        \textbf{Precondizioni}  & l'applicazione predispone un form di modifica del profilo \\ 
        
        \textbf{Postcondizioni} & il sistema ha modificato il profilo dell'utente sulla base di quanto inserito dall'attore \\ 
         \textbf{Scenario principale} & 1. l'utente ha la possibilit\`a di modificare la tipologia dell'utente  \newline 2.   \\
        
         \textbf{Scenari alternativi} & 1. uscita dalla pagina senza salvataggio delle modifiche  \\
     
     \end{longtable}
      \egroup
    \end{center}



\subsection{UC-S 0.1.4}
    \begin{figure}[h]
      \begin{center}
        \includegraphics[width=12cm]{%todo include image}
      \caption{UC-S 0.1.4 - Eliminazione di un utente}
      \end{center} 
    \end{figure}    
    
    %Tabella 
    \begin{center}
      \bgroup
      \def\arraystretch{1.8}     
      \begin{longtable}{  p{3.5cm} | p{8cm} } 
        
        \hline
        \multicolumn{2}{ | c | }{ \cellcolor[gray]{0.9} \textbf{UC-S 0.1.4 - Eliminazione di un utente }} \\ 
        \hline
        
        \textbf{Attori Primari} & Super-Admin\\  
        \textbf{Scopo e descrizione} & l'attore entra nella pagina di eliminazione dell'utente, nella quale ha la possibilit\`a
        di eliminarlo
      
        \textbf{Precondizioni}  & l'applicazione richiede la conferma dell'eliminazione dell'utente \\ 
        
        \textbf{Postcondizioni} & il sistema ha seguito le indicazioni dell'attore \\ 
         \textbf{Scenario principale} & 1. l'attore ha la possibilit\`a di confermare l'eliminazione  \newline 2. l'attore
         ha la possibilit\`a ritirare la richiesta di eliminazione \\
        
         \textbf{Scenari alternativi} & 1. uscita dalla pagina senza salvataggio delle modifiche  \\
     
     \end{longtable}
      \egroup
    \end{center}


\subsection{UC-S 1}
    \begin{figure}[h]
      \begin{center}
        \includegraphics[width=12cm]{%todo include image}
      \caption{UC-S 1 - Gestione di altri super-admin}
      \end{center} 
    \end{figure}    
    
    %Tabella 
    \begin{center}
      \bgroup
      \def\arraystretch{1.8}     
      \begin{longtable}{  p{3.5cm} | p{8cm} } 
        
        \hline
        \multicolumn{2}{ | c | }{ \cellcolor[gray]{0.9} \textbf{UC-S 1 - Gestione di altri super-admin }} \\ 
        \hline
        
        \textbf{Attori Primari} & Super-Admin\\  
        \textbf{Scopo e descrizione} & L'utente è entrato nella pagina di gestione dei super-admin. In questa pagina può aggiungere un nuovo super-admin,
vedere in un elenco quelli già presenti e poterne vedere le informazioni in dettaglio cliccando nella voce dell'elenco.
      
        \textbf{Precondizioni}  & il super-admin entra nella pagina di gestione dei super-admin\\ 
        
        \textbf{Postcondizioni} & il sistema ha preso in carico le indicazioni dell'attore \\ 
         \textbf{Scenario principale} & 1. l'attore ha la possibilit\`a di aggiungere un nuovo super-admin  \newline 2. l'attore
         ha la possibilit\`a di visualizzare in dettaglio il profilo di un super-admin esistente \\
        
         \textbf{Scenari alternativi} & 1. fallimento dell'inserimento di un super-admin  \\
     
     \end{longtable}
      \egroup
    \end{center}


\subsection{UC-S 1.0}
    \begin{figure}[h]
      \begin{center}
        \includegraphics[width=12cm]{%todo include image}
      \caption{UC-S 1.0 - Creazione di un super-admin}
      \end{center} 
    \end{figure}    
    
    %Tabella 
    \begin{center}
      \bgroup
      \def\arraystretch{1.8}     
      \begin{longtable}{  p{3.5cm} | p{8cm} } 
        
        \hline
        \multicolumn{2}{ | c | }{ \cellcolor[gray]{0.9} \textbf{UC-S 1.0 - Creazione di un super-admin }} \\ 
        \hline
        
        \textbf{Attori Primari} & Super-Admin\\  
        \textbf{Scopo e descrizione} & l'attore entra nella pagina di creazione di un altro super-admin. 
        Qui deve aggiungere le informazioni (email, password) necessarie per la registrazione di un nuovo super-admin.
      
        \textbf{Precondizioni}  &  l'attore entra nella pagina di creazione di un nuovo super-admin 
        \textbf{Postcondizioni} & il sistema ha creato un nuovo super-admin nel database \\ 
         \textbf{Scenario principale} & 1. l'attore inserisce l'email dell'utente da registrare  \newline 2. l'attore
         aggiunge una password nell'apposito campo\\
        
         \textbf{Scenari alternativi} & 1. l'attore esce dalla pagina  \newline 2. l'email inserita \`e gi\`a presente
         nel database e il nuovo super-admin non pu\`o essere creato\\ 
     
     \end{longtable}
      \egroup
    \end{center}


    





