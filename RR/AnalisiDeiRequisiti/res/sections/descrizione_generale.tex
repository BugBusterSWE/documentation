\section{Descrizione generale}
\subsection{Contesto d'utilizzo}
Il \glossaryItem{progetto} \glossaryItem{MaaS} mira a fornire un servizio online per le aziende per la visualizzazione dei propri dati aziendali attraverso un'interfaccia grafica. In particolare il \glossaryItem{progetto} mira a creare un ambiente di sviluppo online per la modellazione di queste interfacce, in modo semplice e veloce per gli sviluppatori.
Tale scopo è parzialmente raggiunto dall'applicazione \glossaryItem{MaaP}: è un framework che offre gli strumenti necessari agli sviluppatori per creare tali interfacce. Quest'applicativo per\`o richiede l'installazione del framework necessario in un server per l'azienda che ne vuole usufruire, operazione onerosa e che richiede conoscenze informatiche molto spesso assenti nelle aziende. \glossaryItem{MaaS} cerca di colmare questo punto critico mirando a creare un servizio: una piattaforma raggiungibile online che non necessita dei passaggi di installazione per l'utente, ma solo di un passaggio di registrazione per creare un proprio spazio.
Oltre a questo \glossaryItem{MaaS} punta ad estendere le funzionalità di \glossaryItem{MaaP}, aggiungendo un editor per la creazione delle viste che agevola l'utente nel processo di creazione e modifica delle proprie interfacce.


\subsection{Funzione di Prodotto}
\glossaryItem{MaaS} deve fornire all'utente una piattaforma su cui effettuare la registrazione per la propria \glossaryItem{Company}. Effettuato questo passaggio un \glossaryItem{Owner} pu\`o invitare i propri collaboratori al servizio ed aggiungere gli accessi al proprio database \glossaryItem{MongoDB} per poter creare le viste.
Ciascun utente ha la possibilit\`a di creare le interfacce attraverso un editor che genera una richiesta in linguaggio \glossaryItem{DSL} che il sistema può interpretare per generare le viste.

\subsection{Entit\`a}
Le entit\`a (e le relazioni esistenti tra di loro) definite nel modello \glossaryItem{SaaS} di \glossaryItem{MaaS} vengono esposte di seguito. La loro descrizione \`e stata utile nell'identificazione delle varie tipologie di attori per la trattazione dei casi d'uso.
\begin{description}
	\item[\glossaryItem{Company}] \hfill \\\\
	Una \glossaryItem{Company} \`e un'entit\`a in relazione con gli utenti: un insieme di utenti appartiene a una \glossaryItem{Company} e ciascun utente deve appartenere a una \glossaryItem{Company};
	\item[Utente] \hfill \\\\
	Ogni utente appartenente ad una \glossaryItem{Company} ha un ruolo; a ciascun ruolo sono associate diverse funzioni.	
	
	I ruoli sono i seguenti: \glossaryItem{Owner}, \glossaryItem{Admin}, \glossaryItem{Member}, \glossaryItem{Guest}.	
	Ciascuna funzione associata ad un ruolo pu\`o essere eseguita solo nell'ambito della \glossaryItem{Company} di appartenenza. \\\\
	\textbf{Ruoli e funzioni} \hfill \\\\
		Il seguente elenco dei ruoli rispecchia la gerarchia individuata per gli attori nei casi d'uso.
		\begin{description}
			\item[\glossaryItem{Guest}] Pu\`o eseguire il \glossaryItem{DSL} a cui ha accesso.
			\item[\glossaryItem{Member}] Pu\`o eseguire, leggere e scrivere una specifica \glossaryItem{DSL} a cui ha accesso e crearne una nuova.
			\item[\glossaryItem{Admin}] Pu\`o eseguire qualsiasi operazione sul \glossaryItem{DSL} appartenente alla relativa \glossaryItem{Company}. Pu\`o invitare altri utenti, aggiungere/togliere/modificare ruoli e aggiungere/togliere/modificare permessi in lettura/scrittura al \glossaryItem{DSL}.
			\item[\glossaryItem{Owner}] Pu\`o eseguire le stesse funzioni dell'\glossaryItem{Admin}, non pu\`o essere eliminato e pu\`o impersonare altri utenti all-interno della \glossaryItem{Company}
		\end{description}
	\item[\glossaryItem{Super-Admin}] \hfill \\\\
	Il \glossaryItem{Super-Admin} non appartiene a nessuna \glossaryItem{Company}. Questo ruolo \`e utile per gli amministratori di sistema, che hanno il compito di fornire supporto agli utenti.
\end{description}
