\section{Requisiti }
I requisiti funzionali, prestazionali, di qualità e di vincolo individuati sono riportati nelle seguenti tabelle. Ogni requisito è identificato da un codice univoco.
Viene inoltre indicato se si tratta di un requisito fondamentale, desiderabile o facoltativo, una sua descrizione e il caso d'uso da cui è stato individuato. 

Ogni requisito è identificato da un codice, che segue il seguente formalismo:
\begin{center}
		\code{R\{X\}\{Y\} \{Gerarchia\}}
\end{center}

Dove:
\begin{itemize}
 \item \textbf{X} corrisponde alla tipologia del requisito e può assumere i seguenti valori:
		\begin{itemize}
		 \item[] \textbf{1} = Funzionale;
		 \item[] \textbf{2} = Prestazionale;
		 \item[] \textbf{3} = Di Qualità;
		 \item[] \textbf{4} = Vincolo.
		\end{itemize}

 \item \textbf{Y} corrisponde alla priorità del requisito e può assumere i seguenti valori:
		\begin{itemize}
		 \item[] \textbf{O} = Obbligatorio
		 \item[] \textbf{D} = Desiderabile
		 \item[] \textbf{F} = Facoltativo o Opzionale
		\end{itemize}

 \item \textbf{Gerarchia} identifica la relazione gerarchica che c'è tra i requisiti di uno stesso tipo. C'è quindi una struttura gerarchica per ogni tipologia di requisito.
\end{itemize}

\subsection{Requisiti funzionali}

		%Tabella 
			\begin{center}
			\bgroup
			\def\arraystretch{1.8}
			\begin{longtable}{ | l | p{2cm} | p{5cm} | p{1.7cm} |}
		        \hline
			\cellcolor[gray]{0.9} \textbf{Requisito} & \cellcolor[gray]{0.9} \textbf{Tipologia} 
			& \cellcolor[gray]{0.9} \textbf{Descrizione} & \cellcolor[gray]{0.9} \textbf{Fonti} \\ \hline
      
				R1O X & Funzionale \newline  Obbligatorio  & Il sistema permette all'utente non autenticato di autenticarsi tramite la visualizzazione di una pagina web, la quale conterrà al suo interno i campi di testo necessari. &  Capitolato \newline  UCU 1 \newline  \\ \hline      

			\caption{Requisiti funzionali}
			\end{longtable}
			\egroup
			\end{center}  
