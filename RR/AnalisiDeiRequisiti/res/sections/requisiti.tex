\section{Requisiti }
I requisiti funzionali, prestazionali, di qualità e di vincolo individuati sono riportati nelle seguenti tabelle. Ogni requisito è identificato da un codice univoco.
Viene inoltre indicato se si tratta di un requisito fondamentale, desiderabile o facoltativo, una sua descrizione e il caso d'uso da cui è stato individuato. 

Ogni requisito è identificato da un codice, che segue il seguente formalismo:
\begin{center}
		\textbf{RXY Gerarchia}
\end{center}

Dove:
\begin{itemize}
 \item \textbf{X} corrisponde alla tipologia del requisito e può assumere i seguenti valori:
		\begin{itemize}
		 \item[] \textbf{1} = Funzionale;
		 \item[] \textbf{2} = Prestazionale;
		 \item[] \textbf{3} = Di Qualità;
		 \item[] \textbf{4} = Vincolo.
		\end{itemize}

 \item \textbf{Y} corrisponde alla priorità del requisito e può assumere i seguenti valori:
		\begin{itemize}
		 \item[] \textbf{O} = Obbligatorio;
		 \item[] \textbf{D} = Desiderabile;
		 \item[] \textbf{F} = Facoltativo o Opzionale.
		\end{itemize}

 \item \textbf{Gerarchia} identifica la relazione gerarchica che c'è tra i requisiti di uno stesso tipo. Vi è dunque una struttura gerarchica per ogni tipologia di requisito.
\end{itemize}

\subsection{Requisiti funzionali}

%Tabella 
\begin{center}
  \bgroup
  \def\arraystretch{1.8}
  \begin{longtable}{ | l | p{2cm} | p{4.7cm} | p{2cm} |}
    \hline
    \cellcolor[gray]{0.9} \textbf{Requisito} & \cellcolor[gray]{0.9} \textbf{Tipologia} 
    & \cellcolor[gray]{0.9} \textbf{Descrizione} & \cellcolor[gray]{0.9} \textbf{Fonti} \\ \hline
    
    R1O X & Funzionale \newline Obbligatorio & L’utente non autenticato deve poter creare un account per la propria \glossaryItem{Company}, tramite una pagina web che conterrà al suo interno una form per l’inserimento dei dati (nome \glossaryItem{Company}, email \glossaryItem{Owner}, password \glossaryItem{Owner}). &  Capitolato \newline  UC-U1, UC-U1.1, UC-U1.2, UC-U1.3 \newline  \\ \hline
    
    R1O X & Funzionale \newline Obbligatorio & L’utente non autenticato che abbia ricevuto un invito dall’\glossaryItem{Owner} o da un suo delegato (uno degli amministratori dell’account della \glossaryItem{Company}), deve poter registrarsi a \glossaryItem{MaaS}.
La registrazione avviene tramite l’inserimento di una password che verrà associata all’account dell’utente (il sistema memorizza l’indirizzo email dell’utente nell’istante in cui l’invito viene spedito). &  Capitolato \newline UC-U3, Riunione con il proponente.\newline  \\ \hline
	 
	 R1O X & Funzionale \newline Obbligatorio & L’utente non autenticato registrato presso \glossaryItem{MaaS} deve poter accedere all'applicazione, tramite una pagina di login in cui vengono richiesti l’indirizzo email dell’utente e la password. Il sistema, tramite l’uso di un database indipendente verifica che l’email e la password siano associate a un utente registrato.
&  Capitolato \newline UC-U4, UC-U4.1, UC-U4.2 \newline  \\ \hline
	
	R1O X & Funzionale \newline Obbligatorio & L’utente non autenticato in possesso di un account presso \glossaryItem{MaaS} deve poter recuperare la propria password se questa viene dimenticata.
L'applicazione offre una procedura di recupero che richiede l’inserimento della mail e di un codice segreto (ottenuto dall’utente tramite email). &  Capitolato \newline UC-U6, UC-U6.1   \newline  \\ \hline
	
	R1O X & Funzionale \newline Obbligatorio & L’utente autenticato deve poter effettuare delle operazioni di modifica del profilo. &  Capitolato \newline  UC-U9  \newline  \\ \hline
	
	R1O X & Funzionale \newline Obbligatorio & L’utente autenticato deve poter modificare il proprio indirizzo email inserendone uno di sostitutivo. Il sistema, tramite l’uso di un database indipendente verifica che l’email inserita non sia già presente. &  Capitolato \newline  UC-U9, UC-U9.1  \newline  \\ \hline
	
	R1O X & Funzionale \newline Obbligatorio & L’utente autenticato deve poter modificare la propria password. &  Capitolato \newline  UC-U9, UC-U9.2  \newline  \\ \hline
	
	R1O X & Funzionale \newline Obbligatorio & L’utente autenticato deve poter rimuovere il proprio account da \glossaryItem{MaaS}. Nel caso in cui un \glossaryItem{Owner} decida di rimuoversi deve delegare ad un altro utente nella \glossaryItem{Company} tale ruolo. &  Capitolato \newline  UC-U9, UC-U9.3  \newline  \\ \hline
	
	R1O X & Funzionale \newline Obbligatorio & L’utente autenticato può accedere all’editor ed eseguire un’operazione su una \glossaryItem{DSL}. &  Capitolato \newline  UC-U11  \newline  \\ \hline
	
	R1O X & Funzionale \newline Obbligatorio & L’utente autenticato può accedere all’editor ed eseguire una specifica \glossaryItem{DSL} &  Capitolato \newline  UC-U11, UC-U11.1  \newline  \\ \hline
	
	R1O X & Funzionale \newline Obbligatorio & L’utente autenticato (con ruolo >= \glossaryItem{Member}) può accedere all’editor e modificare una specifica \glossaryItem{DSL}. &  Capitolato \newline  UC-U11, UC-U11.2  \newline  \\ \hline
	
	R1O X & Funzionale \newline Obbligatorio & L’utente autenticato (con ruolo >= \glossaryItem{Member})  può accedere all’editor e creare una specifica \glossaryItem{DSL}. &  Capitolato \newline  UC-U11, UC-U11.3  \newline  \\ \hline
	
	R1O X & Funzionale \newline Obbligatorio & L’utente autenticato (con ruolo >= \glossaryItem{Member})  può accedere all’editor e leggere una specifica \glossaryItem{DSL}. &  Capitolato \newline  UC-U11, UC-U11.4  \newline  \\ \hline
	
	R1O X & Funzionale \newline Obbligatorio & L’Admin deve poter aggiungere/togliere a un utente appartenente alla \glossaryItem{Company} di competenza un permesso in lettura/scrittura a una specifica \glossaryItem{DSL}. &  Capitolato \newline   UC-U13, UC-U13.2, UC-U13.3  \newline  \\ \hline
	
	R1O X & Funzionale \newline Obbligatorio & L’\glossaryItem{Owner} di una \glossaryItem{Company} (o un suo delegato) deve poter invitare un utente a registrarsi presso \glossaryItem{MaaS}. &  Capitolato \newline   UC-U13, UC-U13.4  \newline  \\ \hline
	
	R1O X & Funzionale \newline Obbligatorio & L’\glossaryItem{Owner} di una \glossaryItem{Company} deve poter rimuovere un utente. &  Capitolato \newline UC-U13, UC-U13.5  \newline  \\ \hline
	
	R1O X & Funzionale \newline Obbligatorio & L’\glossaryItem{Owner} di una \glossaryItem{Company} deve poter inserire manualmente un nuovo utente presso \glossaryItem{MaaS}. &  Capitolato \newline UC-U13, UC-U13.6  \newline  \\ \hline
	
	R1O X & Funzionale \newline Obbligatorio & L’utente autenticato deve poter accedere alla pagina \glossaryItem{Dashboard}, visualizzarne il contenuto ed effettuare delle operazioni sugli elementi presenti. &  Capitolato \newline UC-U15, UC-U15.1  \newline  \\ \hline
	
	R1O X & Funzionale \newline Obbligatorio & L’utente autenticato deve poter visualizzare un elemento della \glossaryItem{Dashboard}, che può essere una \glossaryItem{Cell}, un \glossaryItem{Document} oppure una \glossaryItem{Collection}. &  Capitolato \newline UC-U15, UC-U16, UC-U16.1, UC-U16.2, UC-U16.3, UC-U16.4  \newline  \\ \hline
	
	R1O X & Funzionale \newline Obbligatorio & L’utente autenticato deve poter effettuare un’operazione su un elemento della \glossaryItem{Dashboard}, che può essere una \glossaryItem{Cell}, un \glossaryItem{Document} o una \glossaryItem{Collection}. &  Capitolato \newline UC-U15, UC-U16  \newline  \\ \hline
	
	R1O X & Funzionale \newline Obbligatorio & L’utente autenticato deve poter aggiungere, modificare e ordinare un valore in una \glossaryItem{Cell}. &  Capitolato \newline UC-U15, UC-U16, UC-U16.5, UC-U16.5.1,  UC-U16.5.2, UC-U16.5.3  \newline  \\ \hline
	
	R1O X & Funzionale \newline Obbligatorio & L’utente autenticato deve poter modificare e rimuovere un \glossaryItem{Document}. &  Capitolato \newline UC-U15, UC-U16, UC-U16.6.1, UC-U16.6.2, UC-U16.6.3  \newline  \\ \hline
	
	R1O X & Funzionale \newline Obbligatorio & L’utente autenticato deve poter eseguire un’azione di default (Send mail/Export) dalla pagina \glossaryItem{Document}.&  Capitolato \newline UC-U15, UC-U16, UC-U16.6.4, UC-U16.6.5  \newline  \\ \hline
	
	R1O X & Funzionale \newline Obbligatorio & L’utente autenticato deve poter ordinare i Documents all’interno di una \glossaryItem{Collection} in base a uno dei loro campi. &  Capitolato \newline UC-U15, UC-U16, UC-U16.7.1  \newline  \\ \hline
	
	R1O X & Funzionale \newline Obbligatorio & L’utente autenticato deve poter rimuovere una \glossaryItem{Collection}. &  Capitolato \newline UC-U15, UC-U15.2, UC-U16  \newline  \\ \hline
	
	R1O X & Funzionale \newline Obbligatorio & L’utente autenticato deve poter eseguire un’azione di default(Send mail/Export) dalla pagina \glossaryItem{Collection}. &  Capitolato \newline UC-U15, UC-U16, UC-U16.7.3, UC-U16.7.4  \newline  \\ \hline
	
	R1O X & Funzionale \newline Obbligatorio & L'applicazione deve mostrare al Super-Admin la pagina di gestione delle \glossaryItem{Company}. &  Capitolato \newline UC-S0  \newline  \\ \hline
	
	
	
    R1O X & Funzionale \newline Obbligatorio & Il sistema permette all'utente di mantenere salvato un \glossaryItem{DSL} precedentemente creato. &  Capitolato \newline  UC-E1 \newline  \\ \hline
    
    R1D X & Funzionale \newline Desiderabile & Il sistema permette di scaricare nella \glossaryItem{Terminale} dell'utente un \glossaryItem{DSL}, in un qualche formato, leggibile dall'editor. &  Capitolato \newline  UC-E1 \newline \\ \hline
    
    R1D X & Funzionale \newline Desiderabile & Offrire un'interfaccia grafica per la manipolazione del \glossaryItem{DSL}. & Capitolato \newline UC-E1 \newline UC-E2 \newline UC-E3 \newline UC-E3 \\ \hline
    
    R1D X & Funzionale \newline Desiderabile & L'utente attraverso l'interfaccia grafica pu\`o eseguire l'invio del \glossaryItem{DSL} definito. & UC-E3.1 \\ \hline
    
    R1D X & Funzionale \newline Desiderabile & L'utente attraverso l'interfaccia grafica visualizza l'insieme delle \glossaryItem{DSL} a cui pu\`o accedere. & UC-E1 \\ \hline
    
    R1D X & Funzionale \newline Desiderabile & L'utente deve poter manipolare la struttura del \glossaryItem{DSL} attraverso una rappresentazione grafica. & Capitolato \newline UC-E2\\ \hline
    
    R1D X & Funzionale \newline Desiderabile & L'utente, in caso di errori, deve essere avvisato con un messaggio d'errore. & UC-E3.4 \newline UC-E3.5 \\ \hline
    
    R3D X & Funzionale \newline Desiderabile & Il messaggio d'errore in caso di fallimento della validazione del \glossaryItem{DSL} deve indicare il \glossaryItem{DSL} Element che genera l'errore. & UC-E3.4 \\ \hline
    
    R1D X & Funzionale \newline Desiderabile & Il \glossaryItem{Terminale} dell'utente deve potersi connettere attraverso Internet al sistema. & UC-E3.3 \\ \hline
    
    R1D X & Funzionale \newline Desiderabile & Ad ogni azione dell'utente compiuta nell'editor corrisponde la manipolazione della struttura del \glossaryItem{DSL}. & Capitolato \newline UC-E2\\ \hline
    
    R1D X & Funzionale \newline Desiderabile & L'applicazione deve poter determinare la correttezza di una struttura \glossaryItem{DSL} & UC-E3.2 \newline\\ \hline
    
    R1D X & Funzionale \newline Desiderabile & L'utente pu\`o usare il tipo di collegamento \glossaryItem{Riferimento} tra due elementi del \glossaryItem{DSL}. & UC-E2 \newline UC-E2.6.1 \\ \hline
    
    R1D X & Funzionale \newline Desiderabile & L'utente pu\`o usare il tipo di collegamento \glossaryItem{Associazione} tra due elementi del \glossaryItem{DSL}.
    & Capitolato \newline UC-E2 \newline UC-E2.0.3 \newline UC-E2.0.5 \newline UC-E2.1.1 \newline UC-E2.1.2 \newline UC-E2.3.2 \newline UC-E2.8.2 \newline UC-E2.6.3 \newline UC-E2.6.4 \newline UC-E2.7.3 \newline UC-E2.8.2 \newline UC-E2.9.1 \newline UC-E2.10.1 \newline UC-E2.10.2 \newline UC-E2.10.3 \newline UC-E2.11.1\\ \hline
    
    R1D X & Funzionale \newline Desiderabile & L'utente definisce i valori degli attributi del \glossaryItem{DSL} tramite editor. & UC-E2.0.2 \newline UC-E2.7.2\\ \hline
    
    R1D X & Funzionale \newline Desiderabile & L'utente deve aggiungere o rimuovere gli attributi del \glossaryItem{DSL} tramite l'editor. & UC-E2.3.1 \newline UC-E2.6.2 \newline UC-E2.7.1 \newline UC-E2.9.2\\ \hline
    
    R1D X & Funzionale \newline Desiderabile & L'utente deve aggiungere o rimuovere una struttura del \glossaryItem{DSL} tramite editor. & UC-E2.D.1 \newline UC-E2.0.4\\ \hline
    
    R1D X & Funzionale \newline Desiderabile & La \glossaryItem{Collection} deve avere una rappresentazione grafica visibile sul browser. & Capitolato \newline UC-E2 \newline UC-E2.1\\ \hline
    
    R1D X & Funzionale \newline Desiderabile & Una funzione in \glossaryItem{JavaScript} deve avere una rappresentazione grafica visibile sul browser. & UC-E2\\ \hline
    
    R1D X & Funzionale \newline Desiderabile & L'utente pu\`o scrivere la funzione \glossaryItem{JavaScript} direttamente nell'editor. & UC-E2.2 \newline UC-E2.2.1\\ \hline

    R1D X & Funzionale \newline Opzionale & L'utente attraverso l'interfaccia grafica deve essere in grado di salvare i metodi creati. & UC-E2.2.2\\ \hline
    
    R1D X & Funzionale \newline Opzionale & L'utente pu\`o importare nel \glossaryItem{DSL} corrente una funzione \glossaryItem{JavaScript} precedentemente creata. & UC-E2.2.2 \newline UC-E2.2.3\\ \hline
    
    R1D X & Funzionale \newline Desiderabile & L'utente attravero l'interfaccia grafica visualizza l'insieme dei metodi a cui pu\`o accedere. & UC-E2.0.6\\ \hline
    
    R1D X & Funzionale \newline Desiderabile & \glossaryItem{Index} deve avere una rappresentazione grafica visibile sul browser. & Capitolato \newline UC-E2 \newline UC-E2.1 \newline UC-E2.3\\ \hline
    
    R1D X & Funzionale \newline Desiderabile & \glossaryItem{Column} deve avere una rappresentazione grafica visibile sul browser. & Capitolato \newline UC-E2 \newline UC-E2.3 \newline UC-E2.4\\ \hline
    
    R1D X & Funzionale \newline Desiderabile & \glossaryItem{Row} deve avere una rappresentazione grafica visibile sul browser. & Capitolato \newline UC-E2 \newline UC-E2.5 \newline UC-E2.6\\ \hline
    
    R1D X & Funzionale \newline Desiderabile & \glossaryItem{Document} deve avere una rappresentazione grafica visibile sul browser. & Capitolato \newline UC-E2 \newline UC-E2.6\\ \hline
    
    R1D X & Funzionale \newline Desiderabile & Un dato in formato \glossaryItem{JSON} deve avere una rappresentazione grafica visibile sul browser. & UC-E2 \newline UC-E2.7 \newline UC-E2.8\\ \hline
    
    R1D X & Funzionale \newline Desiderabile & \glossaryItem{Cell} deve avere una rappresentazione grafica visibile sul browser. & Capitolato \newline UC-E2 \newline UC-E2.8\\ \hline
    
    R1D X & Funzionale \newline Desiderabile & L'utente pu\`o scegliere qual'\`e il tipo di dato rappresentato dal \glossaryItem{Cell Element} & UC-E2.8.1\\ \hline
    
    R1D X & Funzionale \newline Desiderabile & \glossaryItem{Dashboard} deve avere una rappresentazione grafica visibile sul browser. & Capitolato \newline UC-E2 \newline UC-E2.9\\ \hline
    
    R1D X & Funzionale \newline Desiderabile & \glossaryItem{DashRow} deve avere una rappresentazione grafica visibile sul browser. & UC-E2 \newline UC-E2.9 \newline UC-E2.1D\\ \hline
    
    R1D X & Funzionale \newline Desiderabile & Una \glossaryItem{Action} definita dall'amministratore della piattaforma deve avere una rappresentazione grafica visibile sul browser. & UC-E2 \newline UC-E2.11 \\ \hline
    
    R1D X & Funzionale \newline Desiderabile & Importare una \glossaryItem{Action} definita dall'amministratore nel \glossaryItem{DSL} corrente. & Capitolato \newline UC-E2 \newline UC-E2.1 \newline UC-E2.6 \newline UC-E2.9\\ \hline
    
    R1D X & Funzionale \newline Desiderabile & Impostare associazioni tra \glossaryItem{DSL} Element obbligatorie non modificabili. & Capitolato \newline UC-E2.11.2\\ \hline
    
    R1D X & Funzionale \newline Desiderabile & Selezionare la funzionalit\`a della \glossaryItem{Action}. & Capitolato \newline UC-E2.11 \newline UC-E.11.2\\ \hline
    
    R1D X & Funzionale \newline Desiderabile & La \glossaryItem{DSL} creata dall'editor deve essere inviata al server \glossaryItem{MaaS}. & Capitolato \newline UC-E3\\ \hline
    
    
    \caption{Requisiti funzionali}
  \end{longtable}
  \egroup
\end{center} 

\subsection{Requisiti di qualità}

\begin{center}
  \bgroup
  \def\arraystretch{1.8}
  \begin{longtable}{ | l | p{2cm} | p{4.7cm} | p{2cm} |}
    \hline
    \cellcolor[gray]{0.9} \textbf{Requisito} & \cellcolor[gray]{0.9} \textbf{Tipologia} 
    & \cellcolor[gray]{0.9} \textbf{Descrizione} & \cellcolor[gray]{0.9} \textbf{Fonti} \\ \hline
    R3O X & Qualità \newline Obbligatorio & Devono essere scritti e rilasciati manuali d’uso ed ogni altra documentazione tecnica (in lingua inglese) necessaria per l’utilizzo del prodotto. & Capitolato, Incontro con il proponente \\ \hline
    R3O X & Qualità \newline Obbligatorio & Per lo sviluppo del prodotto richiesto verranno rispettate tutte le norme descritte nel documento Norme di Progetto v. & Capitolato \\ \hline
    \end{longtable}
  \egroup
\end{center}  

\subsection{Requisiti di vincolo}
\begin{center}
  \bgroup
  \def\arraystretch{1.8}
  \begin{longtable}{ | l | p{2cm} | p{4.7cm} | p{2cm} |}
    \hline
    \cellcolor[gray]{0.9} \textbf{Requisito} & \cellcolor[gray]{0.9} \textbf{Tipologia} 
    & \cellcolor[gray]{0.9} \textbf{Descrizione} & \cellcolor[gray]{0.9} \textbf{Fonti} \\ \hline
    R4O X & Vincolo \newline Obbligatorio & Il codice sorgente deve essere reso pubblico e posto sotto il controllo di versione usando github o bitbucket & Capitolato \\ \hline
    R4O X & Vincolo \newline Obbligatorio & Lo \textit{stack tecnologico} da usare deve includere: \newline
- \glossaryItem{Node.js} per il \glossaryItem{back end}. \newline 
- \glossaryItem{MongoDB} (versione >= 3.x) per il database dell'applicazione. 
& Capitolato \\ \hline
    R4O X & Vincolo \newline Obbligatorio &  Obbligo di effettuare il \glossaryItem{deployment} su Heroku.
      & Capitolato \\ \hline
    \end{longtable}
  \egroup
\end{center}   

\subsection{Tracciamento requisiti-fonti}

\begin{center}
  \bgroup
  \def\arraystretch{1.8}
  \begin{longtable}{ | l | p{2cm} | p{4.7cm} | p{2cm} |}
    \hline
    \cellcolor[gray]{0.9} \textbf{Requisito} &   
    \cellcolor[gray]{0.9} \textbf{Fonte}\\ \hline
    \end{longtable}
  \egroup
\end{center} 

\subsection{Tracciamento fonti-requisiti}

\begin{center}
  \bgroup
  \def\arraystretch{1.8}
  \begin{longtable}{ | l | p{2cm} | p{4.7cm} | p{2cm} |}
    \hline
    \cellcolor[gray]{0.9} \textbf{Fonte} &   
    \cellcolor[gray]{0.9} \textbf{Requisito}\\ \hline
    \end{longtable}
  \egroup
\end{center} 
