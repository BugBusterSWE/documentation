\section{Editor}

Di seguito viengono elencati i casi d'uso per l'editor.

\newpage

\subsection{Casi d'Uso}

\subsubsection{UC-E}

    %\begin{figure}[h]
    %  \begin{center}
    %    \includegraphics[width=12cm]{res/img/UCEditor/UC-E0-test}
    %  \caption{UC-E0 - Visualizzazione editor}
    %  \end{center} 
    %\end{figure}    
    
    %Tabella 
    \begin{center}
      \bgroup
      \def\arraystretch{1.8}     
      \begin{longtable}{  p{3.5cm} | p{8cm} } 
        
        \hline
        \multicolumn{2}{ | c | }{ \cellcolor[gray]{0.9} \textbf{UC-E - Operazioni ad alto livello - Uso dell'editor}} \\ 
        \hline
        
        \textbf{Attori Primari} & Super-Admin, Direttore d'azienda, Admin, User \\ 
        \textbf{Scopo e Descrizione} & Con l'uso dell'editor \`e possibile realizzare nuove DSL oppure modificare quelle presenti. La rappresentazione verr\`a con elementi grafici astraendo il linguaggio DSL permettendo all'utente di non conoscerne la struttura \\ 
        
        \textbf{Precondizioni}  & Il browser carica la pagina dell'editor ed il sistema riconosce l'identificatore dell'utente\\ 
        
        \textbf{Postcondizioni} & La DSL formata dall'insieme delle azioni compiute \`e caricata e visibile nel browser \\ 
        \textbf{Flusso Principale} & 1. UC-E0 2. UC-E7 3. UC-E9 4. UC-E8 \\ %da aggiungere?
        \textbf{Estensioni} & a. L'utente si disconnette dalla rete (UC-E3) \\ %vedere lista puntata inline https://en.wikibooks.org/wiki/LaTeX/List_Structures#Inline_lists
      \end{longtable}
      \egroup
    \end{center}

\subsubsection{UC-E0}

    \begin{figure}[h]
      \begin{center}
        \includegraphics[width=12cm]{res/img/UCEditor/UC-E0-test}
      \caption{UC-E0 - Visualizzazione editor}
      \end{center} 
    \end{figure}    
    
    %Tabella 
    \begin{center}
      \bgroup
      \def\arraystretch{1.8}     
      \begin{longtable}{  p{3.5cm} | p{8cm} } 
        
        \hline
        \multicolumn{2}{ | c | }{ \cellcolor[gray]{0.9} \textbf{UC-E0 - Visualizzazione editor}} \\ 
        \hline
        
        \textbf{Attori Primari} & Super-Admin, Direttore d'azienda, Admin, User \\ 
        \textbf{Scopo e Descrizione} & L'utente richiede la visualizzzione dell'editor. \\ 
        
        \textbf{Precondizioni}  & L'attore ha i diritti per accesso all'editor.\\ 
        
        \textbf{Postcondizioni} & L'attore visualizza l'editor. \\ 
        \textbf{Flusso Principale} & Nessuno \\ %da aggiungere?
        \textbf{Estensioni} & a. Il supporto alla tecnologia JavaScript non \`e presente (UC-E1) \\ %vedere lista puntata inline https://en.wikibooks.org/wiki/LaTeX/List_Structures#Inline_lists
      \end{longtable}
      \egroup
    \end{center}

    
    \subsubsection{UC-E1}
    
    %Tabella 
    \begin{center}
      \bgroup
      \def\arraystretch{1.8}     
      \begin{longtable}{  p{3.5cm} | p{8cm} } 
        
        \hline
        \multicolumn{2}{ | c | }{ \cellcolor[gray]{0.9} \textbf{UC-E1 - JavaScript non presente}} \\ 
        \hline
        
        \textbf{Attori Primari} & Super-Admin, Direttore d'azienda, Admin, User \\ 
        \textbf{Scopo e Descrizione} & Viene visualizzato un messaggio a schermo che informa l'attore primario dell'impossibilit\`a di eseguire codice JavaScript \\ 
        
        \textbf{Precondizioni}  & Il sistema non ha potuto eseguire il codice JavaScript.\\ 
        
        \textbf{Postcondizioni} & L'applicazione ha mostrato un messaggio d'errore. \\ 
        \textbf{Flusso Principale} & Nessuno \\ %da aggiungere?
      \end{longtable}
      \egroup
    \end{center}

    \newpage

    \subsubsection{UC-E2}

    \begin{figure}[h]
      \begin{center}
        \includegraphics[width=12cm]{res/img/UCEditor/UC-E2-test}
      \caption{UC-E2 - Uso dell'editor}
      \end{center} 
    \end{figure}    
    
    %Tabella 
    \begin{center}
      \bgroup
      \def\arraystretch{1.8}     
      \begin{longtable}{  p{3.5cm} | p{8cm} } 
        
        \hline
        \multicolumn{2}{ | c | }{ \cellcolor[gray]{0.9} \textbf{UC-E2 - Uso dell'editor}} \\ 
        \hline
        
        \textbf{Attori Primari} & Super-Admin, Direttore d'azienda, Admin, User \\ 
        \textbf{Scopo e Descrizione} & L'utente pu\`o usare i servizi dell'editor: a. Salvataggio, b. Scrittura, c. Caricamento salvataggi, d. Utilizzo dell'editor offline. \\ 
        
        \textbf{Precondizioni}  & L'utente sta visualizzando l'editor.\\ 
        
        \textbf{Postcondizioni} & L'utente ha usato i servizi offerti dall'editor per svolgere le sue attivit\`a. \\ 
        \textbf{Flusso Principale} & Nessuno \\ %da aggiungere?
        \textbf{Estensioni} & a. Il supporto alla tecnologia JavaScript non \`e presente (UC-E1) \\ %vedere lista puntata inline https://en.wikibooks.org/wiki/LaTeX/List_Structures#Inline_lists
      \end{longtable}
      \egroup
    \end{center}


    \subsubsection{UC-E3}
    
    %Tabella 
    \begin{center}
      \bgroup
      \def\arraystretch{1.8}     
      \begin{longtable}{  p{3.5cm} | p{8cm} } 
        
        \hline
        \multicolumn{2}{ | c | }{ \cellcolor[gray]{0.9} \textbf{UC-E3 - Disconnessione dalla rete}} \\ 
        \hline
        
        \textbf{Attori Primari} & Super-Admin, Direttore d'azienda, Admin, User \\ 
        \textbf{Scopo e Descrizione} & Viene visualizzato un messaggio a schermo che informa l'attore primario dell'impossibilit\`a di connettersi alla rete internet. \\ 
        
        \textbf{Precondizioni}  & L'utente si \`e disconnesso dalla rete. \\ 
        
        \textbf{Postcondizioni} & L'attore non \`e pi\`u in grado di lavorare con il sistema. \\ 
      \end{longtable}
      \egroup
    \end{center}

    
\subsubsection{UC-E4}

    \begin{figure}[h]
      \begin{center}
        \includegraphics[width=12cm]{res/img/UCEditor/UC-E4-test}
      \caption{UC-E4 - Caricamento del DSL}
      \end{center} 
    \end{figure}    
    
    %Tabella 
    \begin{center}
      \bgroup
      \def\arraystretch{1.8}     
      \begin{longtable}{  p{3.5cm} | p{8cm} } 
        
        \hline
        \multicolumn{2}{ | c | }{ \cellcolor[gray]{0.9} \textbf{UC-E4 - Caricamento del DSL}} \\ 
        \hline
        
        \textbf{Attori Primari} & Super-Admin, Direttore d'azienda, Admin, User \\ 
        \textbf{Scopo e Descrizione} & Lo scopo principale \`e inviare il DSL corretto al server in modo tale da creare una Collection \\ 
        
        \textbf{Precondizioni}  & Lo scopo principale \`e inviare il DSL corretto al server in modo tale da creare una Collection \\ 
        
        \textbf{Postcondizioni} & \`E stato avvisato l'utente dell'effettivo caricamento del DSL nel server \\ 
        \textbf{Flusso Principale} & Nessuno \\ %da aggiungere?
        \textbf{Estensioni} & a. L'utente si disconnette dalla rete (UC-E3)
      \end{longtable}
      \egroup
    \end{center}


\subsubsection{UC-E5}

    \begin{figure}[h]
      \begin{center}
        \includegraphics[width=12cm]{res/img/UCEditor/UC-E5-test}
      \caption{UC-E5 - Validazione DSL}
      \end{center} 
    \end{figure}    
    
    %Tabella 
    \begin{center}
      \bgroup
      \def\arraystretch{1.8}     
      \begin{longtable}{  p{3.5cm} | p{8cm} } 
        
        \hline
        \multicolumn{2}{ | c | }{ \cellcolor[gray]{0.9} \textbf{UC-E5 - Validazione DSL}} \\ 
        \hline
        
        \textbf{Attori Primari} & Super-Admin, Direttore d'azienda, Admin, User \\ 
        \textbf{Scopo e Descrizione} & Verificare e validare il DSL prodotto dall'editor \\ 
        
        \textbf{Precondizioni}  & Verificare e validare il DSL prodotto dall'editor \\ 
        
        \textbf{Postcondizioni} & Il DSL \`e stato validato con successo \\ 
        \textbf{Flusso Principale} & Nessuno \\ %da aggiungere?
        \textbf{Inclusione} & a. Il DSL valido viene caricato nel Server MaaS (UC-E4) \\
        \textbf{Estensioni} & a. Il DSL non \`e valido
      \end{longtable}
      \egroup
    \end{center} 


    \subsubsection{UC-E6}
    
    %Tabella 
    \begin{center}
      \bgroup
      \def\arraystretch{1.8}     
      \begin{longtable}{  p{3.5cm} | p{8cm} } 
        
        \hline
        \multicolumn{2}{ | c | }{ \cellcolor[gray]{0.9} \textbf{UC-E6 - Il DSL non \`e valido}} \\ 
        \hline
        
        \textbf{Attori Primari} & Super-Admin, Direttore d'azienda, Admin, User \\ 
        \textbf{Scopo e Descrizione} & Viene visualizzato un messaggio a schermo che informa l'attore primario della non validit\`a del DSL \\ 
        
        \textbf{Precondizioni}  & Il DSL non ha passato il test del validatore \\ 
        
        \textbf{Postcondizioni} & L'applicazione ha mostrato un messaggio d'errore. \\ 
      \end{longtable}
      \egroup
    \end{center}


\subsubsection{UC-E7}

    %\begin{figure}[h]
    %  \begin{center}
    %    \includegraphics[width=12cm]{res/img/UCEditor/UC-E7-test}
    %  \caption{UC-E7 - Gestione DSL Element}
    %  \end{center} 
    %\end{figure}    
    
    %Tabella 
    \begin{center}
      \bgroup
      \def\arraystretch{1.8}     
      \begin{longtable}{  p{3.5cm} | p{8cm} } 
        
        \hline
        \multicolumn{2}{ | c | }{ \cellcolor[gray]{0.9} \textbf{UC-E7 - Gestione DSL Element}} \\ 
        \hline
        
        \textbf{Attori Primari} & Super-Admin, Direttore d'azienda, Admin, User \\ 
        \textbf{Scopo e Descrizione} & Il DSL Element \`e l'unit\`a base per la realizzazione del DSL. L'insieme di tutti i DSL Element collegati formeranno la DSL completa. \\ 
        
        \textbf{Precondizioni}  & Il DSL Element \`e l'unit\`a base per la realizzazione del DSL. L'insieme di tutti i DSL Element collegati formeranno la DSL completa. \\ 
        
        \textbf{Postcondizioni} & Viene generato un DSL Element valido da collegare ad altri DSL Element validi. \\ 
        \textbf{Flusso Principale} & 1. L'attore principale aggiunge un nuovo DSL Element (UC-E7.1)
2. L'attore principale compila i campi obbligatori dell'DSL Element (UC-E7.2)
3. L'attore principale compila i campi opzionali dell'DSL Element (UC-E7.3)
4. L'attore principale crea un DSL Dependent Element all'DSL Element creato (UC-E7.4)
5. L'attore principale rimuove un DSL Element (UC-E7.5) \\ %da aggiungere?
        \textbf{Estensioni} & Nessuna
      \end{longtable}
      \egroup
    \end{center} 


\subsubsection{UC-E8}

    %\begin{figure}[h]
    %  \begin{center}
    %    \includegraphics[width=12cm]{res/img/UCEditor/UC-E8-test}
    %  \caption{UC-E8 - Invio DSL}
    %  \end{center} 
    %\end{figure}    
    
    %Tabella 
    \begin{center}
      \bgroup
      \def\arraystretch{1.8}     
      \begin{longtable}{  p{3.5cm} | p{8cm} } 
        
        \hline
        \multicolumn{2}{ | c | }{ \cellcolor[gray]{0.9} \textbf{UC-E8 - Invio DSL}} \\ 
        \hline
        
        \textbf{Attori Primari} & Super-Admin, Direttore d'azienda, Admin, User \\ 
        \textbf{Scopo e Descrizione} & Una volta conclusa la fase di elaborazione della DSL viene prima controllata ed in mancanza di errori viene caricata sull'archivio sul server delle DSL dell'utente registrato \\ 
        
        \textbf{Precondizioni}  & Una volta conclusa la fase di elaborazione della DSL viene prima controllata ed in mancanza di errori viene caricata sull'archivio sul server delle DSL dell'utente registrato \\ 
        
        \textbf{Postcondizioni} & Il DSL viene validato e caricato sull'archivio delle DSL dell'utente \\ 
        \textbf{Flusso Principale} & Nessuno \\ %da aggiungere?
        \textbf{Inclusione} & a. Viene validato il DSL (UC-E5) \\
        \textbf{Estensioni} & Nessuna
      \end{longtable}
      \egroup
    \end{center} 


\subsubsection{UC-E9}

    %\begin{figure}[h]
    %  \begin{center}
    %    \includegraphics[width=12cm]{res/img/UCEditor/UC-E9}
    %  \caption{UC-E9 - Importare una DSL esistente}
    %  \end{center} 
    %\end{figure}    
    
    %Tabella 
    \begin{center}
      \bgroup
      \def\arraystretch{1.8}     
      \begin{longtable}{  p{3.5cm} | p{8cm} } 
        
        \hline
        \multicolumn{2}{ | c | }{ \cellcolor[gray]{0.9} \textbf{UC-E9 - Importare una DSL esistente}} \\ 
        \hline
        
        \textbf{Attori Primari} & Super-Admin, Direttore d'azienda, Admin, User \\ 
        \textbf{Scopo e Descrizione} & L'editor deve permettere anche la modifica di DSL perci\`o \`e necessario poter caricare uno gi\`a presente nel Server MaaS \\ 
        
        \textbf{Precondizioni}  & L'editor deve permettere anche la modifica di DSL perci\`o \`e necessario poter caricare uno gi\`a presente nel Server MaaS \\ 
        
        \textbf{Postcondizioni} & Appare nell'area editabile la rappresentazione grafica del DSL e sono settate variabili d'ambienti necessarie per eseguire correttamente tutte le operazioni. \\ 
        \textbf{Flusso Principale} & Nessuno. \\ %da aggiungere?
        \textbf{Estensioni} & Nessuno.
      \end{longtable}
      \egroup
    \end{center} 


\subsubsection{UC-E10}

    %\begin{figure}[h]
    %  \begin{center}
    %    \includegraphics[width=12cm]{res/img/UCEditor/UC-E10}
    %  \caption{UC-E10 - Gestione della Function Element}
    %  \end{center} 
    %\end{figure}    
    
    %Tabella 
    \begin{center}
      \bgroup
      \def\arraystretch{1.8}     
      \begin{longtable}{  p{3.5cm} | p{8cm} } 
        
        \hline
        \multicolumn{2}{ | c | }{ \cellcolor[gray]{0.9} \textbf{UC-E10 - Gestione della Function Element}} \\ 
        \hline
        
        \textbf{Attori Primari} & Super-Admin, Direttore d'azienda, Admin, User \\ 
        \textbf{Scopo e Descrizione} & Rappresentazione grafica di una funzione in JavaScript, comprendendo parametri d'entrata e parametri d'uscita. \\ 
        
        \textbf{Precondizioni}  & Rappresentazione grafica di una funzione in JavaScript, comprendendo parametri d'entrata e parametri d'uscita. \\ 
        
        \textbf{Postcondizioni} & Estende le funzionalit\`a di un DSL Element aggiungendo le funzionalit\`a di Function Element. \\ 
        \textbf{Flusso Principale} & 1. L'attore principale definisce gli statement da eseguire dalla Function Element. (UC-E10.1)
2. L'attore principale salva la Function Element. (UC-E10.2)
3. L'attore principale carica una Function Element \\ %da aggiungere?
        \textbf{Estensioni} & Nessuna.
      \end{longtable}
      \egroup
    \end{center} 

    \newpage

\subsubsection{UC-E11}

    %\begin{figure}[h]
    %  \begin{center}
    %    \includegraphics[width=12cm]{res/img/UCEditor/UC-E11}
    %  \caption{UC-E11 - Gestione di Collection Element}
    %  \end{center} 
    %\end{figure}    
    
    %Tabella 
    \begin{center}
      \bgroup
      \def\arraystretch{1.8}     
      \begin{longtable}{  p{3.5cm} | p{8cm} } 
        
        \hline
        \multicolumn{2}{ | c | }{ \cellcolor[gray]{0.9} \textbf{UC-E11 - Gestione di Collection Element}} \\ 
        \hline
        
        \textbf{Attori Primari} & Super-Admin, Direttore d'azienda, Admin, User \\ 
        \textbf{Scopo e Descrizione} & Rappresentazione grafica di una Collection in DSL. \\ 
        
        \textbf{Precondizioni}  & Rappresentazione grafica di una sezione index. \\ 
        
        \textbf{Postcondizioni} & Estende le funzionalit\`a di un DSL Element aggiungendo le funzionalit\`a di Collection Element \\ 
        \textbf{Flusso Principale} & 1. Permette di creare o agganciare una Function Element da Label (UC-E11.1)
2. Permette di creare una Index Element da Index (UC-E11.2)
3. Permette di creare una Show Element da Show (UC-11.3) \\ %da aggiungere?
        \textbf{Estensioni} & Nessuna.
      \end{longtable}
      \egroup
    \end{center} 


\subsubsection{UC-E12}

    %\begin{figure}[h]
    %  \begin{center}
    %    \includegraphics[width=12cm]{res/img/UCEditor/UC-E12}
    %  \caption{UC-E12 - Gestione di Index Element}
    %  \end{center} 
    %\end{figure}    
    
    %Tabella 
    \begin{center}
      \bgroup
      \def\arraystretch{1.8}     
      \begin{longtable}{  p{3.5cm} | p{8cm} } 
        
        \hline
        \multicolumn{2}{ | c | }{ \cellcolor[gray]{0.9} \textbf{UC-E12 - Gestione di Index Element}} \\ 
        \hline
        
        \textbf{Attori Primari} & Super-Admin, Direttore d'azienda, Admin, User \\ 
        \textbf{Scopo e Descrizione} & Rappresentazione grafica di una section index. \\ 
        
        \textbf{Precondizioni}  & Rappresentazione grafica di una section index. \\ 
        
        \textbf{Postcondizioni} & Estende le funzionalit\`a di un DSL Element aggiundo le funzionalit\`a di Index Element. \\ 
        \textbf{Flusso Principale} & 1. Permette di creare una Column Element da Sortby (UC-E12.1)
2. Permette di creare o agganciare una Function Element da Query (UC-E12.2)
3. Permette di creare o aggiungere una Column Element da Colum (UC-E12.3) \\ %da aggiungere?
        \textbf{Estensioni} & Nessuna.
      \end{longtable}
      \egroup
    \end{center} 


\subsubsection{UC-E13}

    %\begin{figure}[h]
    %  \begin{center}
    %    \includegraphics[width=12cm]{res/img/UCEditor/UC-E13}
    %  \caption{UC-E13 - Gestione di Column Element}
    %  \end{center} 
    %\end{figure}    
    
    %Tabella 
    \begin{center}
      \bgroup
      \def\arraystretch{1.8}     
      \begin{longtable}{  p{3.5cm} | p{8cm} } 
        
        \hline
        \multicolumn{2}{ | c | }{ \cellcolor[gray]{0.9} \textbf{UC-E13 - Gestione di Column Element}} \\ 
        \hline
        
        \textbf{Attori Primari} & Super-Admin, Direttore d'azienda, Admin, User \\ 
        \textbf{Scopo e Descrizione} & Rappresentazione grafica di una Column del DSL \\ 
        
        \textbf{Precondizioni}  & Rappresentazione grafica di una Column del DSL \\ 
        
        \textbf{Postcondizioni} & Estende le funzionalit\`a di un DSL Element aggiungendo le funzionalit\`a di Column Element. \\ 
        \textbf{Flusso Principale} & 1. Permette di creare o agganciare una Function Element da Name (UC-E13.1)
2. Permette di creare o agganciare una Function Element da Label (UC-E13.2)
3. Permette di creare o agganciare una Function Element da Transformation (UC-E13.3) \\ %da aggiungere?
        \textbf{Estensioni} & Nessuna.
      \end{longtable}
      \egroup
    \end{center} 


\subsubsection{UC-E14}

    %\begin{figure}[h]
    %  \begin{center}
    %    \includegraphics[width=12cm]{res/img/UCEditor/UC-E14}
    %  \caption{UC-E14 - Gestione di Show Element}
    %  \end{center} 
    %\end{figure}    
    
    %Tabella 
    \begin{center}
      \bgroup
      \def\arraystretch{1.8}     
      \begin{longtable}{  p{3.5cm} | p{8cm} } 
        
        \hline
        \multicolumn{2}{ | c | }{ \cellcolor[gray]{0.9} \textbf{UC-E14 - Gestione di Show Element}} \\ 
        \hline
        
        \textbf{Attori Primari} & Super-Admin, Direttore d'azienda, Admin, User \\ 
        \textbf{Scopo e Descrizione} & Rappresentazione grafica di una sezione Show del DSL. \\ 
        
        \textbf{Precondizioni}  & Rappresentazione grafica di una sezione Show del DSL. \\ 
        
        \textbf{Postcondizioni} & Estende le funzionalit\`a di un DSL Element aggiungendo le funzionalit\`a di Show Element \\ 
        \textbf{Flusso Principale} & 1. Permette di creare un Column Element da Populate (UC-E14.1)
2. Permette di creare o aggiungere un Row Element da Row (UC-E14.2) \\ %da aggiungere?
        \textbf{Estensioni} & Nessuna.
      \end{longtable}
      \egroup
    \end{center} 


\subsubsection{UC-E15}

    %\begin{figure}[h]
    %  \begin{center}
    %    \includegraphics[width=12cm]{res/img/UCEditor/UC-E15}
    %  \caption{UC-E15 - Gestione di Row Element}
    %  \end{center} 
    %\end{figure}    
    
    %Tabella 
    \begin{center}
      \bgroup
      \def\arraystretch{1.8}     
      \begin{longtable}{  p{3.5cm} | p{8cm} } 
        
        \hline
        \multicolumn{2}{ | c | }{ \cellcolor[gray]{0.9} \textbf{UC-E15 - Gestione di Row Element}} \\ 
        \hline
        
        \textbf{Attori Primari} & Super-Admin, Direttore d'azienda, Admin, User \\ 
        \textbf{Scopo e Descrizione} & Rappresentazione di una Row del DSL. \\ 
        
        \textbf{Precondizioni}  & Rappresentazione di una Row del DSL. \\ 
        
        \textbf{Postcondizioni} & Estende le funzionalit\`a di un DSL Element aggiungendo le funzionalit\`a di Row Element. \\ 
        \textbf{Flusso Principale} & 1. Permette di creare o agganciare una Function Element da Name (UC-E15.1)
2. Permette di creare o aggiungere una Function Element da Transformation (UC-E15.2) \\ %da aggiungere?
        \textbf{Estensioni} & Nessuna.
      \end{longtable}
      \egroup
    \end{center} 


\subsubsection{UC-E16}

    %\begin{figure}[h]
    %  \begin{center}
    %    \includegraphics[width=12cm]{res/img/UCEditor/UC-E16}
    %  \caption{UC-E16 - Gestione di Document Element}
    %  \end{center} 
    %\end{figure}    
    
    %Tabella 
    \begin{center}
      \bgroup
      \def\arraystretch{1.8}     
      \begin{longtable}{  p{3.5cm} | p{8cm} } 
        
        \hline
        \multicolumn{2}{ | c | }{ \cellcolor[gray]{0.9} \textbf{UC-E16 - Gestione di Document Element}} \\ 
        \hline
        
        \textbf{Attori Primari} & Super-Admin, Direttore d'azienda, Admin, User \\ 
        \textbf{Scopo e Descrizione} & Rappresentazione grafica di un JSON \\ 
        
        \textbf{Precondizioni}  & Rappresentazione grafica di un JSON \\ 
        
        \textbf{Postcondizioni} & Estende le funzionalit\`a di un DSL Element aggiungendo le funzionalit\`a di Document Element. \\ 
        \textbf{Flusso Principale} & 1. Permette di creare una voce attributo (UC-E16.1)
2. Permette di modificare una voce attributo (UC-E16.2)
3. Permette di eliminare una voce attributo (UC-E16.3)
4. Ogni attributo pu\`o essere composto da sottoattributi (UC-E16.4) \\ %da aggiungere?
        \textbf{Estensioni} & Nessuna.
      \end{longtable}
      \egroup
    \end{center} 
