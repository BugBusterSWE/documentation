\chapter{Editor}

Di seguito viengono elencati i casi d'uso per l'editor.

\newpage

\section{Casi d'Uso}

\subsection{UC-E0}

    \begin{figure}[h]
      \begin{center}
        \includegraphics[width=12cm]{res/img/UCEditor/UC-E0-test}
      \caption{UC-E0 - Visualizzazione editor}
      \end{center} 
    \end{figure}    
    
    %Tabella 
    \begin{center}
      \bgroup
      \def\arraystretch{1.8}     
      \begin{longtable}{  p{3.5cm} | p{8cm} } 
        
        \hline
        \multicolumn{2}{ | c | }{ \cellcolor[gray]{0.9} \textbf{UC-E0 - Visualizzazione editor}} \\ 
        \hline
        
        \textbf{Attori Primari} & Super-Admin, Direttore d'azienda, Admin, User \\ 
        \textbf{Scopo e Descrizione} & L'utente richiede la visualizzzione dell'editor. \\ 
        
        \textbf{Precondizioni}  & L'attore ha i diritti per accesso all'editor.\\ 
        
        \textbf{Postcondizioni} & L'attore visualizza l'editor. \\ 
        \textbf{Flusso Principale} & Nessuno \\ %da aggiungere?
        \textbf{Estensioni} & a. Il supporto alla tecnologia JavaScript non \`e presente (UC-E1) \\ %vedere lista puntata inline https://en.wikibooks.org/wiki/LaTeX/List_Structures#Inline_lists
      \end{longtable}
      \egroup
    \end{center}

    
    \subsection{UC-E1}
    
    %Tabella 
    \begin{center}
      \bgroup
      \def\arraystretch{1.8}     
      \begin{longtable}{  p{3.5cm} | p{8cm} } 
        
        \hline
        \multicolumn{2}{ | c | }{ \cellcolor[gray]{0.9} \textbf{UC-E1 - JavaScript non presente}} \\ 
        \hline
        
        \textbf{Attori Primari} & Super-Admin, Direttore d'azienda, Admin, User \\ 
        \textbf{Scopo e Descrizione} & Viene visualizzato un messaggio a schermo che informa l'attore primario dell'impossibilit\`a di eseguire codice JavaScript \\ 
        
        \textbf{Precondizioni}  & Il sistema non ha potuto eseguire il codice JavaScript.\\ 
        
        \textbf{Postcondizioni} & L'applicazione ha mostrato un messaggio d'errore. \\ 
        \textbf{Flusso Principale} & Nessuno \\ %da aggiungere?
      \end{longtable}
      \egroup
    \end{center}


    \subsection{UC-E2}

    \begin{figure}[h]
      \begin{center}
        \includegraphics[width=12cm]{res/img/UCEditor/UC-E2-test}
      \caption{UC-E2 - Uso dell'editor}
      \end{center} 
    \end{figure}    
    
    %Tabella 
    \begin{center}
      \bgroup
      \def\arraystretch{1.8}     
      \begin{longtable}{  p{3.5cm} | p{8cm} } 
        
        \hline
        \multicolumn{2}{ | c | }{ \cellcolor[gray]{0.9} \textbf{UC-E2 - Uso dell'editor}} \\ 
        \hline
        
        \textbf{Attori Primari} & Super-Admin, Direttore d'azienda, Admin, User \\ 
        \textbf{Scopo e Descrizione} & L'utente pu\`o usare i servizi dell'editor: a. Salvataggio, b. Scrittura, c. Caricamento salvataggi, d. Utilizzo dell'editor offline. \\ 
        
        \textbf{Precondizioni}  & L'utente sta visualizzando l'editor.\\ 
        
        \textbf{Postcondizioni} & L'utente ha usato i servizi offerti dall'editor per svolgere le sue attivit\`a. \\ 
        \textbf{Flusso Principale} & Nessuno \\ %da aggiungere?
        \textbf{Estensioni} & a. Il supporto alla tecnologia JavaScript non \`e presente (UC-E1) \\ %vedere lista puntata inline https://en.wikibooks.org/wiki/LaTeX/List_Structures#Inline_lists
      \end{longtable}
      \egroup
    \end{center}
