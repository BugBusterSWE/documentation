\section{Capitolato C4 - MaaS}

\subsection{Descrizione}
%todo: give a high-level description of the product
L'obiettivo di questo progetto \`e la creazione di un servizio web che incorpora \textit{Maap} e lo rende diponibile
tramite web per pi\`u aziende.

\subsection{Valutazioni sul dominio}
\subsubsection{Dominio applicativo}
Il prodotto che il gruppo si impegna a sviluppare si inserisce nel dominio della persistenza dei dati
mediante l'utilizzo di \textit{database} %parola da glossario
di tipo \textit{NoSQL}. %parola da glossario
%todo: there's still a lot to write

\subsubsection{Dominio tecnologico}
\begin{itemize}
 %todo: node.js da glossario
  \item \textbf{Node.js} per la realizzazione della componente \textit{back end} %todo: back end da glossario
  \item \textbf{MongoDB} per la memorizzazione persistente ed il recupero dei dati %todo: mongoDB da glossario
  \item \textbf{Angular.js} oppure \textbf{React.js} per la realizzazione della componente \textit{front end} %todo: angular, react and front end from glossary
  \item competenze nella definizione di linguaggi astratti (\tetxit{DSL})
\end{itemize}


\subsection{Valutazione complessiva}
\begin{itemize}
\item Aspetti positivi
  \begin{enumerate}
    \item Nello sviluppo del prodotto i membri del gruppo apprenderanno tecnologie innovative 
      attualmente richieste dal mercato
    \item Chiarezza dei requisiti richiesti dal proponente
    \item Disponibilit\`a da parte del proponente nella negoziazione dei requisiti
  \end{enumerate}
\item{Aspetti negativi}
  \begin{enumerate}
    \item Solo alcuni membri del gruppo conoscono profondamente le tecnologie adottate nel progetto
    \item Almeno nella fase iniziale, lo sviluppo del progetto sembra richiedere una notevole mole di lavoro
  \end{enumerate}
\end{itemize}
