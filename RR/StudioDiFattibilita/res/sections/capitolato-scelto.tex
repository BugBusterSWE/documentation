\section{Capitolato scelto - C4: MaaS: MongoDB as an admin Service}
\subsection{Descrizione}
Il progetto MaaS intende proporsi come un servizio per le aziende nell'ambito della gestione e della visualizzazione di dati aziendali. 
Tale progetto si basa su \textit{MaaP}, un'applicazione web gi\`a esistente che fornisce agli utilizzatori un sistema elegante di visualizzazione
dei propri dati aziendali letti nel database mongoDB in possesso dell'azienda.
MaaP costituisce il cuore di MaaS: al momento questa applicazione richiede un'installazione in un server web e 
una configurazione ad hoc da parte di un tecnico che ha accesso diretto al server in questione. MaaS punta a rimuovere 
questi limiti tecnici fornendo MaaP "as a service", cioè punta a creare una piattaforma comune in cui gli utenti possono 
registrarsi e accedere al servizio offerto da MaaP. Il progetto consiste inoltre nell'ampliamento del funzionamento di MaaP tramite la creazione di 
un editor per semplificare agli utenti la creazione di viste per i propri dati.


\subsection{Valutazioni sul dominio del progetto}
L'ambito operativo in cui si colloca il progetto \`e collegato alla persistenza dei dati tramite l'utilizzo
di \textit{basi di dati} di tipo \textit{NoSQL}, in particolare \textit{MongoDB}.

\subsubsection{Dominio applicativo}

\subsubsection{Dominio tecnologico}
\begin{itemize}
%required technologies list
\end{itemize}


\subsection{Motivazioni della scelta}
\subsubsection{Aspetti positivi}

\subsubsection{Aspetti negativi}

