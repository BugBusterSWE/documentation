\section{Capitolato scelto - C4: \glossaryItem{MaaS}: \glossaryItem{MongoDB} as an admin Service}
\subsection{Descrizione}
Il \glossaryItem{progetto} \glossaryItem{MaaS} (\textbf{M}ongoDB \textbf{a}s an \textbf{A}dmin \textbf{S}ervice) intende proporsi come un servizio per le aziende nell'ambito della gestione e della visualizzazione di dati aziendali. 
Tale \glossaryItem{progetto} si basa su \glossaryItem{MaaP} (\textbf{M}ongoDB \textbf{a}s an \textbf{A}dmin \textbf{P}latform), un'applicazione web gi\`a esistente  che fornisce all'utente un sistema di visualizzazione
dei propri dati aziendali ricavati dal database \glossaryItem{MongoDB} posseduto dall'azienda.
\glossaryItem{MaaP} costituisce il cuore di \glossaryItem{MaaS}: al momento questa applicazione richiede un'installazione in un server web e 
una configurazione ad hoc da parte di un tecnico che ha accesso diretto al server in questione. \glossaryItem{MaaS} punta a rimuovere 
questi limiti tecnici fornendo \glossaryItem{MaaP} \textit{as a service}, cioè intende creare una piattaforma comune in cui gli utenti possano 
registrarsi e accedere al servizio offerto da \glossaryItem{Maap}. Il \glossaryItem{progetto} consiste inoltre nell'ampliamento del funzionamento di \glossaryItem{MaaP} tramite la creazione di 
un editor per semplificare agli utenti la creazione di viste per i propri dati.


\subsection{Valutazioni sul dominio del \glossaryItem{progetto}}
\subsubsection{Dominio applicativo}
L'ambito operativo in cui si colloca il \glossaryItem{progetto} \`e collegato alla persistenza dei dati tramite l'utilizzo
di \textit{basi di dati} di tipo \glossaryItem{NoSQL} (\textbf{N}ot \textbf{o}nly \textbf{SQL}), in particolare \glossaryItem{MongoDB}.

\subsubsection{Dominio tecnologico}
\begin{itemize}
%required technologies list
\item \glossaryItem{Node.js} per la realizzazione del componente \glossaryItem{back end};
\item \glossaryItem{Angular.js} per la realizzazione del componente \glossaryItem{front end};
\item \glossaryItem{React.js} in alternativa a \glossaryItem{Angular.js}, con lo stesso scopo
\item \glossaryItem{MongoDB} per la memorizzazione ed il recupero dei dati;
\item competenze nella definizione di linguaggi astratti \glossaryItem{DSL}.
\end{itemize}


\subsection{Motivazioni della scelta}
\subsubsection{Aspetti positivi}
\begin{itemize}
\item Il gruppo ha valutato positivamente l'opportunit\`a di apprendere tecnologie innovative
  direttamente applicabili nel mondo del lavoro;
\item I requisiti utente sono ben delineati;
\item Il proponente si \`e mostrato disponibile nella negoziazione dei requisiti.
\end{itemize}

\subsubsection{Aspetti negativi}
\begin{itemize}
\item Alcuni membri del gruppo non hanno competenze avanzate nell'ambito di sviluppo del presente \glossaryItem{progetto};
\item Almeno nella prima fase, la mole di lavoro per la realizzazione del \glossaryItem{progetto} appare notevole.
\end{itemize}



