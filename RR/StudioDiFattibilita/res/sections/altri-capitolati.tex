\section{Valutazioni sugli altri capitolati}

\subsection{C1 - Actorbase: a NoSQL DB based on the Actor model}
\subsubsection{Aspetti positivi}
Il gruppo ha considerato questo progetto molto interessante per la materia trattata; inoltre
i componenti del gruppo si sono mostrati disponibili all'apprendimento di un linguaggio 
moderno e innovativo come \textit{Scala}, il quale \`e richiesto in questo progetto.
%todo: 'Scala in glossario'
\subsubsection{Aspetti negativi}
Trattandosi tuttavia di un progetto interno all'universit\`a, il gruppo lo ha giudicato scarsamente
fruibile sul mercato. Inoltre sono stati previsti tempi di sviluppo troppo lunghi.


\subsection{C2 - CLIPS: Communication & Localisation with Indoor Positioning Systems}
\subsubsection{Aspetti positivi}
L'area di studio del progetto \`è interessante ed innovativa. Inoltre, tutti i membri
del gruppo si sono mostrati interessati ad un ambito di studio nuovo come quello dei \textit{beacon}.
\subsubsection{Aspetti negativi}
Il presente capitolato non descrive dettagliatamente cosa il proponente richiede di sviluppare,
in quanto il progetto in questione ha come obiettivo la ricerca di nuovi scenari per la navigazione \textit{indoor}
applicata a diversi ambiti.\\
Questo aspetto \`e stato giudicato portatore di numerosi rischi, principalmente
in quanto l'elaborazione di un'idea innovativa e accettabile da parte del proponente avrebbe richiesto troppo tempo,
perci\`o l'attivi\`a di \textit{Analisi dei requisiti} sarebbe stata iniziata con un ritardo inevitabile. 


\subsection{C3 - UMAP: un motore per l'analisi predittiva in ambiente Internet of Things}
\subsubsection{Aspetti positivi}
L'utilit\`a del prodotto previsto dal presente capitolato \`e innegabile per qualsiasi tipo di azienda,
un prodotto di questo genere permetterebbe infatti una maggiore automatizzazione di alcuni
processi di controllo spesso eseguiti da umani.
Inoltre le tecnologie richieste per la sua implementazione sono innovative e direttamente fruibili sul mercato.

\subsubsection{Aspetti negativi}
L'ideazione di un algoritmo predittivo di analisi di dati generici è stata considerata troppo impegnativa,
visti anche i curricula di studi informatici dei componenti del gruppo, che non comprendono esami
o ricerche negli ambiti dell'\textit{Intelligenza artificiale} o dell'\textit{Apprendimento automatico}.


\subsection{C5 - Quizzipedia: software per la gestione di questionari}
\subsubsection{Aspetti positivi}
Il gruppo lo ritiene il pi\`u semplice e comprensibile nelle richieste, almeno in quelle
riguardanti i requisiti minimi obbligatori; per questo i membri hanno pensato che la scelta 
di questo capitolato permetterebbe di sviluppare un prodotto accettabile in un tempo minore
rispetto agli altri. 
Inoltre \`e stata valutata positivamente la possibilit\`a di lavorare `liberamente'
su numerosi requisiti opzionali. 
Oltre a questo, le tecnologie da adottare sono innovative e richieste dal mercato.
\subsubsection{Aspetti negativi}
Il presente progetto \`e stato considerato scarsamente `sfidante' nelle richieste e nello scopo.

\subsection{C6 - SiVoDiM: Sintesi Vocale per Dispositivi Mobili}
%todo: ............................
\subsubsection{Aspetti positivi}
Il gruppo ha valutato positivamente questo capitolato per la prevista facilità nella sua realizzazione e 
la semplicità e chiarezza delle richieste.
\subsubsection{Aspetti negativi}
Il gruppo non lo considera un banco di prova per l'apprendimento di nuove tecnologie.

