\section{Gestione amministrativa della revisione}
\subsection{Comunicazione delle anomalie}
Il processo di Software Quality Management è finalizzato alla ricerca dei difetti. L'identificazione delle anomalie mantiene aggiornato il Responsabile sullo stato del prodotto, e permette di prendere dei provvedimenti per correggerle. A tal fine, è utile distinguere e catalogare le anomalie, per discuterne durante revisioni e riunioni. Il team ha scelto di adottare le definizioni di anomalie presenti nello standard IEEE 610.12.90:
\begin{itemize}
\item \textbf{Error}: differenza riscontrata tra il risultato di una computazione e il valore teorico atteso (e.g. uscita dal range di accettazione degli indici di misurazione);
\item \textbf{Fault}: passo, processo o dato definito in modo erroneo (e.g. violazioni delle norme tipografiche di un documento). \`E anche noto con il nome di bug.
\item \textbf{Failure}: la conseguenza di un fault (e.g. incongruenza del prodotto rispetto alle funzionalità indicate nell'analisi dei requisiti);
\item \textbf{Mistake}: azione umana che produce un risultato errato.
\end{itemize}
Questa catalogazione permette di definire delle metriche in grado di valutare l'andamento del prodotto e, in alcuni casi, di predirlo. Un esempio è la metrica che conteggia il numero di \textit{bug per lines of code}.

\section{Controlli per la qualità di processo}
Le procedure di controllo per la qualità di processo sono finalizzate a migliorare la qualità del prodotto e/o diminuire i costi e tempi di sviluppo. Esistono due approcci principali:
\begin{itemize}
\item \textbf{A maturità di processo}: riflette le buone pratiche di management e tecniche di sviluppo. L'obiettivo primario è la qualità del prodotto e la prevedibilità dei processi;
\item \textbf{Agile}: sviluppo iterativo senza l'overhead della documentazione e di tutti gli aspetti predeterminabili. Ha come caratteristica la responsività ai cambiamenti dei requisti cliente e uno sviluppo rapido.
\end{itemize}
Il primo approccio è maggiormente indicato ad un team con poca esperienza: verrà pertanto scelto quello. Con una visione proattiva si cerca di avere maggior controllo e previsione sulle attività da svolgere. Questa viene anche indicata come best practice} per gruppi poco esperti.\\
Il processo con maggiore influenza sulla qualità del sistema non è quello di sviluppo ma quello di progettazione. È qui che le capacità e le esperienze dei singoli danno un contributo decisivo.\\
%todo immagine
Il miglioramento dei processi è un processo ciclico, composto da tre sotto-processi:
\begin{itemize}
\item \textbf{Misurazione del processo}: misura gli attributi del progetto, punta ad allineare gli obiettivi con le misurazioni effettuate. Questo forma una baseline che aiuta a capire se i miglioramenti hanno avuto effetto;
\item \textbf{Analisi del processo}: vengono identificate le problematiche ed i colli di bottiglia dei processi;
\item \textbf{Modifiche del processo}: i cambiamenti vengono proposti in risposta alle problematiche riscontrate.
\end{itemize}
Il team procederà nel seguente modo: 
\begin{itemize}
% todo riferimenti alle sezioni
\item Nella sezione \textit{Dettaglio delle verifiche tramite analisi} (\ref{})  verranno inserite le misurazioni rilevate sulle le metriche descritte in \textit{Misure e Metriche} (\ref{});
\item L'analisi viene effettuata i giorni precedenti alle consegne previste dal committente; il \textit{Riassunto delle attività di verifica} (\ref{}) contiene l'analisi del processo, le relative considerazioni  comprendenti le problematiche riscontrate;
\item Le modifiche al processo vengono attuate all'inizio del processo incrementale successivo. Queste attività sono programmate nel \PianoDiProgetto.
\end{itemize}
