\section{Introduzione}
Lo scopo primario è la \textit{qualit\`a} del prodotto e dei suoi processi, ottenibile solo attraverso una serie continua di controlli. Il \textit{team} si impone una rigida e costante attivit\'a di \textit{verifica}, per poter individuare e correggere eventuali errori presenti nei documenti e nel \textit{codice sorgente}. L'assenza di tali controlli porterebbe al deterioramento del prodotto, soprattutto se in presenza di un team giovane e con poca esperienza alle spalle. \\
\`E volont\`a del team non operare \textit{by correction} che comporterebbe il rischio di ritardi nella maturazione del prodotto.

\subsection{Scopo del documento}
Il presente documento contiene le strategie di verifica e validazione che il team BugBusters ha deciso di adottare. La ricerca della qualit\`a, nei prodotti e nei processi, non rientra nei ruoli canonici di un progetto, ma \`e una \textit{funzione aziendale}; il \textit{committente}, potrà, attraverso la strategie qui documentata, valutare su basi oggettive il prodotto e disporre di una solida batteria di test.

\subsection{Scopo del prodotto}
L'obiettivo che si pone \textit{MaaS} (\textit{\textbf{M}ongoDB} \textbf{a}s an \textbf{a}dmin \textbf{S}ervice) \`e quello di rendere il gi\`a funzionante \textit{MaaP} (\textbf{M}ongoDB \textbf{a}s an \textbf{a}dmin \textbf{P}latform) un \textit{servizio}, ovvero quello di renderlo disponibile a tutti, senza richiederne l'installazione. MaaS si propone di essere un'estensione di MaaP anche dal punto di vista delle funzionalit\`a offerte all'utente finale. Sar\`a in grado di supportare i ruoli basilari di una \textit{compagnia}, permettendo ad utenti diversi di eseguire operazioni diverse. \\
MaaS verr\`a realizzato utilizzando principalmente \textit{Node.js} e MongoDB.

\subsection{Glossario}
% TODO, come verranno marcati i termini da glossario?
Ogni occorrenza di acronimi, dei termini tecnici o di dominio \`e evidenziata con il corsivo e marcata con la lettera G in pedice. Nel documento \textit{Glossario v1.0.0} sono riportati i significati corrispondenti.

\subsection{Riferimenti}
Di seguito  sono elencati i riferimenti sui quali si basano il presente documento e le attività di verifica e validazione.

\subsubsection{Normativi}
\begin{itemize}
	\item \textbf{Norme di Progetto}: \textit{Norme di Progetto v1.0.0};
	\item \textbf{Capitolato d'appalto C4}: \textit{RedBabel}, MaaS \url{http://www.math.unipd.it/~tullio/IS-1/2015/Progetto/C4.pdf};
	\item \textbf{Standard ISO/IEC 15504}: \url{http://en.wikipedia.org/wiki/ISO/IEC_15504};
	\item \textbf{Standard ISO/IEC 9126}: \url{http://en.wikipedia.org/wiki/ISO/IEC_9126}.
\end{itemize}
	
\subsubsection{Informativi}
% TODO link??
\begin{itemize}
	\item \textbf{Piano di Progetto}: \textit{Piano di Progetto v1.0.0};
	\item \textbf{SWEBOK v3}: capitoli ;% TODO
	\item \textbf{Slides del corso di Ingegneria del Software mod. A}: Qualit\`a del software \url{http://www.math.unipd.it/~tullio/IS-1/2015/Dispense/L08.pdf};
	\item \textbf{Slides del corso di Ingegneria del Software mod. A}: Qualit\`a del processo \url{http://www.math.unipd.it/~tullio/IS-1/2015/Dispense/L09.pdf};
	\item \textbf{Software Engineering 9th - I. Sommerville (Pearson, 2011)}: capitoli ; % TODO
	\item \textbf{Metriche del software G - Ercole F. Colonese}: \url{http://www.colonese.it/00-Manuali_Pubblicatii/08-Metriche%20del%20software_v1.0.pdf};
	\item \textbf{Ciclo di Deming}: \url{https://it.wikipedia.org/wiki/Ciclo_di_Deming};
	\item \textbf{Indice Gulpease}: \url{https://it.wikipedia.org/wiki/Indice_Gulpease};
\end{itemize}