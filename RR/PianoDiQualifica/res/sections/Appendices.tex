\newpage
\appendix
\section{Resoconto delle attività di verifica} \label{App:AppendixA}
	\subsection{Riassunto delle attività di verifica} \label{App:AppendixA}
		\subsubsection{Revisione dei requisiti} \label{App:AppendixA}
			
			L'attività di verifica svolta dai verificatori è avvenuta come determinato dal Piano di Progetto v1.x.y al termine della stesura di ogni documento in ingresso alla revisione dei requisiti.
			
			La verifica svolta sui documenti e sui processi è avvenuta seguendo le indicazioni delle Norme di Progetto v1.x.y e misurando le metriche indicate in 3.6.1 e 3.6.2.
			
\section{Dettaglio delle verifiche tramite analisi} \label{App:AppendixB}
	\subsection{Processi} \label{App:AppendixB}
	
		Vengono riportati i valori degli indici \textit{Schedule Variance} e \textit{Budget Variance}, descritti nella sezione 3.6.2.
		
		Abbiamo suddiviso l'analisi dei requisiti nelle macro attività che hanno prodotto come output i documenti riportati in tabella. 
		
		Tali macro-attività includono anche tutte le attività di verifica svolte per ogni documento.
		
		\begin{table}[!ht]
			\centering
				\begin{tabular}{|l|l|l|ll}
					\cline{1-3}
					 \textbf{Macro-attività}  & \textbf{SV}  & \textbf{BV}  &  \\ \cline{1-3}
					 Norme di Progetto  &  &  &  \\ \cline{1-3}
					 Studio Fattibilità &  &  &  \\ \cline{1-3}
					 Analisi Requisiti &  &  &  \\ \cline{1-3}
					 Piano di Progetto &  &  &  \\ \cline{1-3}
					 Piano di Qualifica &  &  &  \\ \cline{1-3}
					 Glossario &  &  &  \\ \cline{1-3}
				\end{tabular}
				\caption{Esiti verifica processi}
		\end{table}
		
		Complessivamente, il processo di analisi ha:
		\begin{itemize}
			\item SV = 
			\item BV = 
		\end{itemize}
	\newpage
	
	\subsection{Documenti} \label{App:AppendixB}
	
		Vengono riportati i valori dell’\textit{indice Gulpease} per ogni documento durante la fase di Analisi. 
		
		\begin{table}[!ht]
			\centering
				\begin{tabular}{|l|l|l|ll}
					\cline{1-3}
					 \textbf{Macro-attività}  & \textbf{Indice Gulpease}  & \textbf{Esito}  &  \\ \cline{1-3}
					 Norme di Progetto  &  &  &  \\ \cline{1-3}
					 Studio Fattibilità &  &  &  \\ \cline{1-3}
					 Analisi Requisiti &  &  &  \\ \cline{1-3}
					 Piano di Progetto &  &  &  \\ \cline{1-3}
					 Piano di Qualifica &  &  &  \\ \cline{1-3}
					 Glossario &  &  &  \\ \cline{1-3}
				\end{tabular}
				\caption{Esiti verifica documenti - Analisi}
		\end{table}
		\newpage
		
\section{Standard di qualità} \label{App:AppendixC}

	\subsection{Standard ISO/IEC 15504}

		ISO/IEC 15504 denominato anche Software Process Improvement and Capability dEtermination (\textbf{SPICE}), consiste di un insieme di normative e linee guida per lo sviluppo di processi software.
	
		Definisce un modello di riferimento per dare una valutazione complessiva della maturità dei processi in un'organizzazione del settore IT.\\
	
		\textbf{Livelli di maturità dei processi}\\

		Per ogni processo, il modello ISO/IEC 15504 definisce un livello di maturità che può variare nella seguente scala:
		
		\begin{enumerate}

			\item Optimizing process
			\item Predictable process
			\item Established process
			\item Managed process
			\item Performed process
			\item Incomplete process

		\end{enumerate}
		
		Ciascun livello di maturità viene misurato sulla base dei seguenti attributi, che possono essere non posseduti (0 \% - 15 \%), parzialmente soddisfatti (16 \% - 50 \%), largamente soddisfatti (51 \% - 85 \%), pienamente soddisfatti (86 \% - 100 \%):
		
		\begin{enumerate}
		
			\item Process performance
			\item Performance management
			\item Work product management
			\item Process definition
			\item Process deployment
			\item Process measurement
			\item Process control
			\item Process innovation
			\item Process optimization
			
		\end{enumerate}
		
		\subsubsection{Ciclo di Deming} \label{App:AppendixC}
		
			Il ciclo PDCA si suddivide nelle seguenti 4 fasi:
			
			\begin{enumerate}
			
				\item \textbf{Plan}: fase di pianificazione. Si definisce cosa andrà realizzato e come andrà controllato (se si sta ripetendo il ciclo questa fase ha lo scopo di apportare dei miglioramenti).
				\item \textbf{Do}: fase di esecuzione. Si agisce in base a quanto pianificato nella fase Plan.
				\item \textbf{Check}: fase di verifica. Vengono controllati i risultati ottenuti dalla fase Do e confrontati con i risultati attesi dalla fase Plan. Le informazioni raccolte saranno utili alla prossima fase.
				\item \textbf{Act}: Fase di “miglioramento continuo”. Se i risultati prodotti in Do hanno apportato un miglioramento o un peggioramento rispetto a quelli attesi dalla fase Plan, allora probabilmente è necessario ripetere di nuovo il ciclo PDCA.
				
			\end{enumerate}

	\subsection{Standard ISO/IEC 9126}
	
		Lo standard ISO/IEC 9126 consiste di una serie di normative e linee guida che hanno lo scopo di descrivere un modello di qualità del software. 
		
		Tale modello di qualità è classificato da un insieme di 6 caratteristiche generali, a loro volta suddivise in sotto caratteristiche.
		
		Si fa notare che lo standard descritto in questa sezione è stato sostituito dallo standard \href{http://www.iso.org/iso/iso_catalogue/catalogue_tc/catalogue_detail.htm?csnumber=35733}{ISO/IEC 25010:2011}, nel quale sono state apportate delle modifiche al modello di qualità (sono state aggiunte/rimosse/modificate alcune caratteristiche e sotto caratteristiche).\\
		La norma tecnica relativa alla qualità del software si compone di quattro parti:
		\begin{itemize}
		
			\item Modello della qualità del software
			\item Metriche per la qualità esterna
			\item Metriche per la qualità interna
			\item Metriche per la qualità in uso
			
		\end{itemize}
		
		\textbf{Modello di qualità - caratteristiche}
		
		\begin{itemize}
		
			\item \textbf{Funzionalità}: capacità di un prodotto software di fornire funzioni che soddisfino esigenze stabilite.
			
			Sotto caratteristiche:
			
			\begin{itemize}
			
				\item Appropriatezza
				\item Accuratezza
				\item Interoperabilità
				\item Conformità
				\item Sicurezza

			\end{itemize}

		\item \textbf{Affidabilità}: capacità del prodotto software di mantenere un certo livello di prestazioni quando usato in date condizioni per un dato periodo.
		
			Sotto caratteristiche:
			
			\begin{itemize}
			
				\item Maturità
				\item Tolleranza agli errori
				\item Resistenza agli errori
				\item Aderenza

			\end{itemize}

		\item \textbf{Efficienza}: capacità di fornire appropriate prestazioni relativamente alla quantità di risorse usate.
		
			Sotto caratteristiche:
			
			\begin{itemize}
			
				\item Comportamento rispetto al tempo
				\item Utilizzo delle risorse
				\item Conformità
				
			\end{itemize}

		\item \textbf{Usabilità}: capacità del prodotto software di essere capito, appreso e usato ottimamente dall'utente, sotto condizioni specificate.
		
			Sotto caratteristiche:
			
			\begin{itemize}
			
				\item Comprensibilità
				\item Apprendibilità
				\item Operabilità
				\item Attrattivà
				\item Conformità

			\end{itemize}

		\item \textbf{Manutenibilità}: capacità del software di essere modificato, includendo correzioni, miglioramenti o adattamenti.
		
			Sotto caratteristiche:
		
			\begin{itemize}
			
				\item Analizzabilità
				\item Modificabilità
				\item Stabilità
				\item Testabilità
				
			\end{itemize}

		\item \textbf{Portabilità}: capacità del software di essere trasportato da un ambiente di lavoro ad un altro. (Ambiente che può variare dall'hardware al sistema operativo).
		
			Sotto caratteristiche:
			
			\begin{itemize}
			
				\item Adattabilità
				\item Installabilità
				\item Sostituibilità
				\item Conformità
				
			\end{itemize}
			
		\end{itemize}
		
		\textbf{Metriche per la qualità esterna}\\
		
			Le metriche esterne, specificate nella norma ISO/IEC 9126-2, misurano i comportamenti del software sulla base dei test, dell'operatività e dell'osservazione durante la sua esecuzione, in funzione degli obiettivi stabiliti in un contesto tecnico rilevante o di business.\\
			
		\textbf{Metriche per la qualità interna}\\
		
			La qualità interna è specificata nella norma ISO/IEC 9126-3 e si applica al software non eseguibile (ad esempio il codice sorgente) durante le fasi di progettazione e codifica. 

			Le metriche interne permettono di individuare eventuali problemi che potrebbero influire sulla qualità finale del prodotto prima che sia realizzato il software eseguibile.\\

		\textbf{Metriche per la qualità in uso}\\
		
			La qualità in uso rappresenta il punto di vista dell'utente sul software. La qualità in uso permette di abilitare specificati utenti a raggiungere specificati obiettivi con efficacia, produttività, sicurezza e soddisfazione.


		
			
			
		