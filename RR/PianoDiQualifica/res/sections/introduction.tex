\section{Introduzione}
Lo scopo primario è la \glossaryItem{\glossaryItem{qualità}} del prodotto e dei suoi processi, ottenibile solo attraverso una serie continua di controlli. Il \glossaryItem{team} si impone una rigida e costante attività di \glossaryItem{verifica}, per poter individuare e correggere eventuali errori presenti nei documenti e nel \glossaryItem{codice sorgente}. L'assenza di tali controlli porterebbe al deterioramento del prodotto, soprattutto se in presenza di un \glossaryItem{team} giovane e con poca esperienza alle spalle. \\
È volontà del \glossaryItem{team} non operare \glossaryItem{by correction} che comporterebbe il rischio di ritardi nella maturazione del prodotto.

\subsection{Scopo del documento}
Il presente documento contiene le strategie di \glossaryItem{verifica} e \glossaryItem{validazione} che il \glossaryItem{team} BugBusters ha deciso di adottare. La ricerca della \glossaryItem{qualità}, nei prodotti e nei processi, non rientra nei ruoli canonici di un \glossaryItem{progetto}, ma è una \glossaryItem{funzione aziendale}; il \glossaryItem{committente}, potrà, attraverso le strategie qui documentate, valutare su basi oggettive il prodotto e disporre di una solida batteria di test.

\subsection{Scopo del prodotto}
L'obiettivo che si pone \glossaryItem{MaaS} (\textbf{M}ongoDB \textbf{a}s an \textbf{a}dmin \textbf{S}ervice) è quello di rendere il già funzionante \glossaryItem{MaaP} (\textbf{M}ongoDB \textbf{a}s an \textbf{a}dmin \textbf{P}latform) un servizio, ovvero quello di renderlo disponibile a tutti, senza richiederne l'installazione. \glossaryItem{MaaS} si propone di essere un'estensione di \glossaryItem{MaaP} anche dal punto di vista delle funzionalità offerte all'utente finale. Sarà in grado di supportare i ruoli basilari di una company, permettendo ad utenti diversi di eseguire operazioni diverse. \\
\glossaryItem{MaaS} verrà realizzato utilizzando principalmente Node.js e MongoDB.

\subsection{Glossario}
Ogni occorrenza di acronimi, dei termini tecnici o di dominio è evidenziata con il corsivo e marcata con la lettera G in pedice. Nel documento \Glossario sono riportati i significati corrispondenti.

\subsection{Riferimenti}
Di seguito  sono elencati i riferimenti sui quali si basano il presente documento e le attività di \glossaryItem{verifica} e \glossaryItem{validazione}.

\subsubsection{Normativi}
\begin{itemize}
\item \textbf{Norme di progetto}: \NormeDiProgetto;
\item \textbf{Capitolato d'appalto C4}: RedBabel, \glossaryItem{MaaS} \url{http://www.math.unipd.it/~tullio/IS-1/2015/Progetto/C4.pdf};
\item \textbf{Standard ISO/IEC 15504}: \url{http://en.wikipedia.org/wiki/ISO/IEC_15504};
\item \textbf{Standard ISO/IEC 9126}: \url{http://en.wikipedia.org/wiki/ISO/IEC_9126};
\item \textbf{Standard IEEE 610.12-90}: \url{https://cow.ceng.metu.edu.tr/Courses/download_courseFile.php?id=2677}.
\end{itemize}
	
\subsubsection{Informativi}
\begin{itemize}
\item \textbf{Piano di Progetto}: \PianoDiProgetto;
\item \textbf{SWEBOK v3}: capitolo 10;
\item \textbf{Slides del corso di Ingegneria del Software mod. A}: \glossaryItem{Qualità} del software \url{http://www.math.unipd.it/~tullio/IS-1/2015/Dispense/L08.pdf};
\item \textbf{Slides del corso di Ingegneria del Software mod. A}: \glossaryItem{Qualità} del \glossaryItem{processo} \url{http://www.math.unipd.it/~tullio/IS-1/2015/Dispense/L09.pdf};
\item \textbf{Software Engineering 9th - I. Sommerville (Pearson, 2011)}: capitoli 24 e 26;
\item \textbf{Metriche del software G - Ercole F. Colonese}: \url{http://www.colonese.it/00-Manuali_Pubblicatii/08-Metriche%20del%20software_v1.0.pdf};
\item \textbf{Ciclo di Deming}: \url{https://it.wikipedia.org/wiki/Ciclo_di_Deming};
\item \textbf{Indice Gulpease}: \url{https://it.wikipedia.org/wiki/Indice_Gulpease};
\end{itemize}
