%ISO/IEC 12207:1995 is a standard for process with a GENERIC wiew
%'the customer/client uses process standard to ease control activity' 
%we always refer to ISO/IEC:12207 but there exist other standards
%We'll deal with the following Primary Processes: 1)Supply process, 2)Development process

%###################
%Why did we choose to structure the documentation in this way?: 
%"la buona organizzazione di un’azienda si basa sul riconoscimento dei propri processi, 
%la loro adozione consapevole ed efficace e il loro supporto efficiente 
%###################
\chapter{Processi primari}
	\section{Processo di sviluppo}
		\subsection{Attività di Analisi dei requisiti}
			\subsubsection{Scopo}
				Gli analisti del gruppo dovranno ricavare i requisiti utili per il progetto
				dal capitolato e dagli incontri con il proponente, avendo come obiettivo la
				produzione del documento 'Analisi dei Requisiti'.
			
			%parlare di: IEEE 830-1998: Recommended Practice
			%for Software Requirements Specifications (slide 32 in ING_REQUISITI)
			La presente attività ha come scopo la produzione una specifica dei
			requisiti conforme ai principi dello standard IEEE 830-1998, espressi in 8 qualità
			essenziali:
				\begin{itemize}
				\item Priva di ambiguità (UNAMBIGOUS)
				\item Corretta (CORRECT)
				\item Completa (COMPLETE)
				\item Verificabile (VERIFIABLE)
				\item Consistente (CONSISTENT)
				\item Modificabile (MODIFIABLE)
				\item Tracciabile (TRACEABLE)
				\item Ordinata per rilevanza (RANKED)
				\end{itemize}
			
			
			
			
			
			
			%Obiettivi: slide 36 di ING_REQUISITI
			
			
			%slide 40 di ING_REQUISITI: Cause di abbandono
			La stesura di un' Analisi dei Requisiti di qualità è cruciale, infatti
			da un rapporto dello Standish Group del 1995 si nota come due delle cause primarie
			di abbandono dei progetti siano le seguenti:
				\begin{itemize}
				\item Requisiti incompleti
				\item Volatilità di specifiche e requisiti
				\end{itemize}
			
			
			%parlare degli STRUMENTI utilizzati
			\subsubsection{Strumenti}
				Per i diagrammi dei casi d'uso verrà utilizzato il linguaggio UML 2.0,
				in quanto noto a tutti i membri del gruppo dato che è stato trattato a lezione
				di Ingegneria del Software.
				Si è scelto di utilizzare l'editor grafico online draw.io.
			
			
		\subsection{Attività di Progettazione}
		 %Costruzione a priori perseguendo la correttezza per costruzione
		 %invece che inseguendo la correttezza per correzione     <<  slide PROGETTAZIONE
		
		 %Procede dall'analisi dei requisiti   <<  slide PROGETTAZIONE
		 
		 %si useranno Design patterns
		
		 
		 %parlare degli STRUMENTI utilizzati
		
		\subsection{Attività di Codifica}
		
	\section{Processo di Fornitura}
