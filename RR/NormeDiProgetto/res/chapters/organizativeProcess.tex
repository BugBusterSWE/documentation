
\section{Processi Organizzativi}

\subsection{Processo di pianificazione}

\subsubsection{Scopo}

Lo scopo di questo processo è la produzione e la comunicazione del piano di progetto ai membri del gruppo. Tutto ciò aiuta nella determinazione nell'ambito della gestione del progetto e le attività tecniche, identifica gli output del processo e ne stabilisce un calendario.


\subsubsection{Risultati osservabili}

Questo processo produce:
\begin{itemize}

\item l'identificazione dell'ambito del progetto;
\item l'analisi di fattibilità degli obiettivi prefissati
\item una stima esatta dei task necessari per completare il lavoro
\item un piano per l'esecuzione del lavoro programmato

\end{itemize}

\subsubsection{Descrizione}

\paragraph*{Soglia oraria giornaliera}È stato deciso di porre la soglia oraria ai seguenti orari: la mattina dalle ore (?) alle ore (?), il pomeriggio  ore (?) alle ore (?) %da decidere

\paragraph*{Assegnazione dei task}Il responsabile si occuperà di assegnare i task in base alla disponibilità dei membri del gruppo e alla rotazione dei ruoli

\subsubsection{Rotazione dei ruoli}

Durante lo sviluppo del progetto vi sono diversi ruoli che devono essere ricoperti dai membri del gruppo BugBuster. Il \textit{Piano di progetto} si occupa di assegnare le attività principali e specifiche per ogni ruolo. Il \textit{Reponsabile} di progetto ha l'onere di far rispettare i ruoli assegnati durante l'attività e il \textit{Verificatore} deve essere in grado di individuare eventuali incongruenze e di segnalarle opportunamente.
\paragraph*{Incongruenze tra ruoli e risoluzioni}
Esistono due principali tipi di incongruenze incontrabili:
\begin{itemize}

\item Il compito svolto non fa parte dei compiti propri di un dato ruolo
\item La stessa persona verifica ciò che ha precedentemente prodotto
\end{itemize}

È importante tenere conto inoltre di altri tre fattori per un buono svolgimento del progetto:
\begin{itemize}

\item Una persona non deve impiegare più del 50\% delle ore di lavoro in un unico ruolo: questo impedirebbe una corretta rotazione dei ruoli.
\item Una persona non dovrebbe ricere più di un task al giorno, onde evitare un eccessivo carico di lavoro, che rischierebbe di non essere effettuato corettamente.
\item I ruoli verranno ruotati %definire ogni quanto

\end{itemize}
