\chapter{Processi Organizzativi}

\section{Processo di pianificazione}

\subsection{Scopo}

Lo scopo di questo processo è la produzione e la comunicazione del piano di progetto ai membri del gruppo. Tutto ciò aiuta nella determinazione nell'ambito della gestione del progetto e le attività tecniche, identifica gli output del processo e ne stabilisce un calendario.


\subsection{Risultati osservabili}

Questo processo produce:
\begin{itemize}

\item l'identificazione dell'ambito del progetto;
\item l'analisi di fattibilità degli obiettivi prefissati
\item una stima esatta dei task necessari per completare il lavoro
\item un piano per l'esecuzione del lavoro programmato

\end{itemize}

\subsection{Descrizione}

\paragraph*{Soglia oraria giornaliera}È stato deciso di porre la soglia oraria ai seguenti orari: la mattina dalle ore (?) alle ore (?), il pomeriggio  ore (?) alle ore (?) %da decidere

\paragraph*{Assegnazione dei task}Il responsabile si occuperà di assegnare i task in base alla disponibilità dei membri del gruppo e alla rotazione dei ruoli

\subsection{Rotazione dei ruoli}

Durante lo sviluppo del progetto vi sono diversi ruoli che devono essere ricoperti dai membri del gruppo BugBuster. 
