%ISO/IEC 12207:1995 is a standard for process with a GENERIC wiew
%'the customer/client uses process standard to ease control activity' 
%we always refer to ISO/IEC:12207 but there exist other standards
%We'll deal with the following Primary Processes: 1)Supply process, 2)Development process

%###################
%Why did we choose to structure the documentation in this way?: 
%"la buona organizzazione di un’azienda si basa sul riconoscimento dei propri processi, 
%la loro adozione consapevole ed efficace e il loro supporto efficiente 
%###################

%\href{nome}{url}
\section{Processi primari}
	\section{Processo di sviluppo}
		\subsubsection{Attività di Analisi dei requisiti}
			\paragraph*{Scopo}
				Gli analisti del gruppo dovranno ricavare i requisiti utili per il progetto
				dal capitolato e dagli incontri con il proponente, avendo come obiettivo la
				produzione del documento `Analisi dei Requisiti'.
			
			%parlare di: IEEE 830-1998: Recommended Practice
			%for Software Requirements Specifications (slide 32 in ING_REQUISITI)
			La presente attività ha come scopo la produzione una specifica dei
			requisiti conforme ai principi dello standard IEEE 830-1998, espressi in 8 qualità
			essenziali:
				\begin{itemize}
				\item Priva di ambiguità (UNAMBIGOUS)
				\item Corretta (CORRECT)
				\item Completa (COMPLETE)
				\item Verificabile (VERIFIABLE)
				\item Consistente (CONSISTENT)
				\item Modificabile (MODIFIABLE)
				\item Tracciabile (TRACEABLE)
				\item Ordinata per rilevanza (RANKED)
				\end{itemize}
			
			
			
			%casi d'uso: si parte da 0 nella numerazione
			\paragraph*{Casi d'uso}
			Ogni caso d'uso dovrà presentare i seguenti campi:
				\begin{itemize}
				\item Codice identificativo
					\paragraph Precisazione sul punto corrente
					\textbf{UC-{X} x.y.z}
					Descrizione:
						\begin{itemize}
						\item \textbf{X}: rappresenta uno degli ambiti di riferimento individuati
						nell'Analisi dei Requisiti, in particolare:
							\begin{itemize}
							\item [] \textbf{B} = Ambito Backend  %backend va precisato
							\item [] \textbf{G} = Ambito utente Guest %guest va precisato
							\item [] \textbf{S} = Ambito SuperAdmin %superadmin va precisato
							\item [] \textbf{O} = Ambito Owner %owner va precisato
							\item [] \textbf{E} = Ambito Editor %editor va precisato
							\end{itemize}
						\end{itemize}
				\item Titolo
				\item Diagramma UML
				\item Attori
				\item Scopo e descrizione
				\item Precondizioni
				\item Postcondizione
				\item Flusso principale
				\end{itemize}
			
			
			
			%Obiettivi: slide 36 di ING_REQUISITI
			
			
			%slide 40 di ING_REQUISITI: Cause di abbandono
			La stesura di un' Analisi dei Requisiti di qualità è cruciale, infatti
			da un rapporto dello Standish Group del 1995 si nota come due delle cause primarie
			di abbandono dei progetti siano le seguenti:
				\begin{itemize}
				\item Requisiti incompleti
				\item Volatilità di specifiche e requisiti
				\end{itemize}
			
			
			%parlare degli STRUMENTI utilizzati
			\paragraph*{Strumenti}
				Per i diagrammi dei casi d'uso verrà utilizzato il linguaggio UML 2.0,
				in quanto noto a tutti i membri del gruppo dato che è stato trattato a lezione
				di Ingegneria del Software.
				Si scelto inoltre di adottare l'editor online \emph{draw.io}.
			
			
		\subsubsection{Attività di Progettazione}
			\paragraph*{Scopo}
		 %Costruzione a priori perseguendo la correttezza per costruzione
		 %invece che inseguendo la correttezza per correzione     <<  slide PROGETTAZIONE
			 %Procede dall'analisi dei requisiti   <<  slide PROGETTAZIONE	
		     La presente attività, che procede dall'Analisi dei Requisiti, persegue l'obiettivo
		     della correttezza per costruzione, permettendo di ridurre al minimo le attività di correzione
		     in fase di codifica.
		     
                     I documenti elaborati in questa fase sono:
                     \begin{itemize}
                       \item Specifica Tecnica, nel quale vengono individuate le componenti macroscopiche del sistema,
                         procedura utile per la paralellizazione del lavoro
                       \item Definizione di Prodotto, documento che fornisce al \textit{programmatore} tutte le direttive necessarie per la codifica
                     \end{itemize}
		 
		 %si useranno Design patterns
		 Per la progettazione al livello di sistema si useranno \textit{Design pattern}
                 
                 %diagrammi UML prodotti
                      \paragraph*{Diagrammi prodotti}
                      Nel corso di questa fase verranno prodotti quattro tipi di diagrammi \textit{UML}:
                      \begin{itemize}
                        \item Diagrammi delle classi
                        \item Diagrammi dei \textit{package}
                        \item Diagrammi di sequenza
                        \item Diagrammi di attività
                      \end{itemize}
                      I tipi di diagrammi sopra elencati andranno a far parte dei due documenti menzionati sopra.
		
		 
		 %parlare degli STRUMENTI utilizzati
		 \paragraph*{Strumenti}
		 Anche per questa attività, i membri del gruppo hanno stabilito l'utilizzo del linguaggio UML 2.0 per i seguenti
		 diagrammi:
		 \begin{itemize}
			\item Diagrammi delle package  
			\item Diagrammi delle classi	
			\item Diagrammi di attività
		\end{itemize}
			 
		
		\subsubsection{Attività di Codifica}
                %citare https://github.com/RisingStack/node-style-guide/blob/master/README.md
                \subsubsection{Scopo}
                La presente attività ha come scopo la traduzione in codice sorgente dei risultati ottenuti in sede
                di \textit{Specifica Tecnica} e \textit{Definizione di Prodotto}.

                


                
                
                
		
	\subsection{Processo di Fornitura}
