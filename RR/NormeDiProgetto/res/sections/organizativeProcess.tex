\section{Processi Organizzativi}

\subsection{\glossaryItem{Progetto} di pianificazione}

\subsubsection{Scopo}

Lo scopo di questo \glossaryItem{progetto} è la realizzazione e la comunicazione del piano di \glossaryItem{progetto} ai membri del gruppo. Tutto ciò aiuta nella determinazione dell'ambito della gestione del \glossaryItem{progetto} e delle attività tecniche, identifica gli output del \glossaryItem{progetto} e ne stabilisce un calendario.


\subsubsection{Risultati osservabili}

Questo \glossaryItem{progetto} produce:
\begin{itemize}

\item L'identificazione dell'ambito del \glossaryItem{progetto};
\item L'analisi di fattibilità degli obiettivi prefissati;
\item Una stima esatta dei task necessari per completare il lavoro;
\item Un piano per l'esecuzione del lavoro programmato.

\end{itemize}

\subsubsection{Descrizione}

\paragraph*{Soglia oraria giornaliera}È stato deciso di porre la soglia oraria ai seguenti orari: la mattina dalle ore 8:30 alle ore 11:30, il pomeriggio dalle ore 14 alle 18. Le ore di lavoro al di fuori di questa fascia oraria saranno considerate come lavoro straordinario.

\paragraph*{Assegnazione dei task}Il responsabile si occuperà di assegnare i task in base alla disponibilità dei membri del gruppo e alla rotazione dei ruoli

\subsubsection{Rotazione dei ruoli}

Durante lo sviluppo del \glossaryItem{progetto} vi sono diversi ruoli che devono essere ricoperti dai membri del gruppo BugBuster. Il \textit{Piano di progetto} si occupa di assegnare le attività principali e specifiche per ogni ruolo. Il \textit{Reponsabile} di progetto ha l'onere di far rispettare i ruoli assegnati durante l'attività e il \textit{Verificatore} deve essere in grado di individuare eventuali incongruenze e di segnalarle opportunamente.
\paragraph*{Incongruenze tra ruoli e risoluzioni}
Esistono due tipi principali di incongruenze riscontrabili:
\begin{itemize}

\item Il compito svolto non fa parte dei compiti propri di un dato ruolo;
\item La stessa persona verifica ciò che ha precedentemente prodotto.
\end{itemize}

È importante tenere conto inoltre di altri tre fattori per un buono svolgimento del \glossaryItem{progetto}:
\begin{itemize}

\item Una persona non deve impiegare più del 50\% delle ore di lavoro in un unico ruolo: questo impedirebbe una corretta rotazione dei ruoli;
\item Una persona non dovrebbe ricevere più di un task al giorno, onde evitare un eccessivo carico di lavoro, che rischierebbe cos\`i di non essere svolto correttamente;
\item I ruoli verranno ruotati ogni due settimane.

\end{itemize}

Il resposabile nell'azione di assegnazione dei ticket dovr\`a indicare il ruolo che il destinatario ricopre nell'adempimento di tale task. Questo \`e necessario per il rendiconto economico. I ruoli sono assegnati basandosi sul \PianoDiProgetto.

\paragraph*{Identificazione dei ruoli}

\begin{description}

\item[Amministratore] \hfill \\ L'\textit{Amministratore} equipaggia, organizza e gestisce l'ambiente di lavoro e di produzione. Collabora con il \textit{Responsabile} alla stesura delle \textit{Norme di Progetto} e del \textit{Piano di Progetto}, e si assicura di:
  \begin{itemize}

  \item Attuare scelte tecnologice concordate con il \textit{Responsabile};
  \item Gestire le liste di distribuzione e assicurarne il rispetto;
  \item Controllare versioni e configurazioni del prodotto;
  \item Risolvere i problemi legati alla gestione dei processi.
    
  \end{itemize}

\item[Analista] \hfill \\ L'\textit{Analista} \`e responsabile dell'attivit\`a di analisi, e ha il compito di comprendere appieno il dominio applicativo. Si occupa di redigere lo \textit{Studio di Fattibilit\`a} e l'\textit{Analisi dei Requisiti}.

\item[Progettista] \hfill \\ Il \textit{Progettista} \`e il responsabile dell'attivit\`a di progettazione, ha una profonda conoscenza dello \textit{stack tecnologico} utilizzato e presenta adeguate competenze tecniche. Ha una forte influenza sugli aspetti tecnici e tecnologici del \glossaryItem{progetto}.

\item[Programmatore] \hfill \\ Il \textit{Programmatore} ha responsabilit\`a circoscritte, e si occupa dell' attivit\`a di codifica, rispettando le \textit{Norme di Progetto}. Si occupa di realizzare il prodotto e le componenti di ausiliarie funzionali all'esecuzione delle prove di \glossaryItem{verifica} e \glossaryItem{validazione}.

\item[Responsabile] \hfill \\ Il \textit{Responsabile} ha l'ultima voce in capitolo per quanto concerne le decisioni sul \glossaryItem{progetto}, \`e il responsabile ultimo dei risultati, infatti approva l'emissione dei documenti, oltre a redigere insieme all'\textit{Amministratore} il \textit{Piano di Progetto}. Il \textit{Responsabile} ha diverse responsabilit\`a, di seguito elencate:
  \begin{itemize}

  \item Pianificare e organizzare lo sviluppo del \glossaryItem{progetto}, stimando tempi, costi e assegnazione delle attivit\`a ai componenti del gruppo;
  \item Riportare lo stato del \glossaryItem{progetto} ai committenti;
  \item Analizzare i rischi in cui è possibile incorrere, monitorarli e prendere provvedimenti a riguardo;
  \item Stabilire una \textit{way of working} per ogni componente del gruppo, per esercitare un'influenza positiva sulle performance del gruppo.
    
  \end{itemize}

\item[Verificatore] \hfill \\ Il \textit{Verificatore} organizza ed attua le attivit\`a di \glossaryItem{verifica} e controlla che le attivit\`a siano conformi alle norme definite. Redige nel \textit{Piano di Qualifica} la parte che documenta le attivit\`a svolte e i risultati ottenuti, confrontandoli con le metriche del \PianoDiQualifica. Le ore assegnate al verificatore devono corrispondere almeno al 30\% delle ore totali suddivise tra i ruoli.
  
\end{description}

\subsubsection{Revisioni di \glossaryItem{progetto}}

Le revisioni, specificate nel \PianoDiProgetto, consistono in \textit{review}\footnote{Revisioni informali.} e \textit{audit}\footnote{Revisioni formali.}, ma essendo le modalit\`a con cui vengono affrontate dal gruppo le medesime verranno denominate presentazioni o revisioni. Secondo quanto indicato nell'IEEE 1028, vengono stabilite le seguenti procedure:
\begin{enumerate}
	\setcounter{enumi}{-1}
	\item Entry evaluation: il Responsabile di Progetto prepara una checklist e si assicura che esistano le condizioni che permettano il successo della presentazione;
	\item Management preparation: il Responsabile si assicura che tutti i membri del gruppo possano partecipare e l'ambiente di lavoro rispetti i vincoli imposti o suggeriti dal committente. A tal fine è stata elaborata una lista di indicazioni (\ref{presentazione});
	\item Planning the review: il Responsabile deve identificare e confermare gli obiettivi della revisione, inoltre si assicura che tutto il team sia equipaggiato con le risorse necessarie per affrontare la revisione;
	\item Overview of review procedures: ci si assicura che tutti i membri del gruppo conoscano gli obiettivi, le procedure e il materiale portato in revisione;
	\item Individual preparation: ogni membro del gruppo si prepara individualmente per la presentazione;
	\item Group examination: il gruppo si incontra al completo e prova la presentazione confermando il rispetto dei tempi e degli obiettivi;
	\item Rework/follow-up: vengono corretti, laddove possibile,  gli errori e/o i difetti;
	\item Exit evaluation: il Responsabile si assicura che tutti gli output prodotti siano pronti per la revisione.
\end{enumerate}

\label{itm:presentazione}
Le indicazioni su come costruire una presentazione sono le seguenti:
\begin{itemize}

\item Le diapositive devono seguire la regola del \textit{5 x 5}: ovvero per una buona leggibilit\`a dovrebbero essere presenti al pi\`u 5 punti ognuno di lunghezza massima di 5 parole.

\item Evitare l'utilizzo di immagini troppo piccole o dettagliate;

\item Argomentare come \`e stata applicata la teoria nel \glossaryItem{progetto}, senza limitarsi a spiegarla;

\item La spiegazione delle slide dev'essere effettuata rivolgendosi al pubblico, evitando sempre di dargli le spalle;

\item Le mani non devono stare nelle tasche ma aiutare a sottolineare le parti importanti nella spiegazione;

\item Ogni slide deve essere numerata;

\end{itemize}
