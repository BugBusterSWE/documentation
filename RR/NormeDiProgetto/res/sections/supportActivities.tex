\section{Processi di supporto}
\section{Processo di documentazione}
\subsection{Scopo del processo}
Qeusto processo ha come scopo di ordurre e mantenere la documentazione prodotta durante il ciclo di vita del software.
\subsection{Risultati attesi dal processo}
Tale processo deve garanire:
\begin{itemize}
\item l'individuazione dei documenti da produrre per le varie fasi del ciclo di sviluppo
\item la corretta implementazione dei documenti segnalati
\item l'individuazione degli standard a cui tali documenti devono aderire
\end{itemize}
  \subsection{Procedura}
  Per produrre la documentazione è stato deciso di utilizzare il linguaggio di markup \Hologo{LaTeX}. Tale scelta è stata effettuata per permettere un miglior versionamento dei file e per permettere una formattazione dei testi uniforme.
  Per la realizzazione di ciascun documento \`e stato realizzato un modello .tex reperibile presso il repository dei documenti (.......)

  \subsection{Struttura comune dei documenti}
  \subsubsection{Prima pagina}
  La prima pagina del documento deve sempre seguire questa struttura:
  \begin{itemize}
    \item logo e nome del gruppo devono essere riportati nella metà alta del documento, centrati orizzontalmente
    \item titolo del documento al di sotto del logo e del nome del gruppo
    \item tabella riassuntiva del documento riportante:
      \begin{itemize}
      \item versione del documento
      \item nome e cognome dei redatori del doucemtno
      \item nome e cognome dei verificatori del doucemnto
      \item nome e cognome del responsabile del progetto che approva il documento
      \item tipologia d'uso del documento
      \item lista di distribuzione
      \end{itemize}
    \item breve descrizione del contenuto del documento
  \end{itemize}

  \subsubsection{Diario delle modifiche}
  All'interno del documento deve essere presente, subito dopo la proma pagina, una sezione riguardante le modifiche apportate al documento in questione. Tale sezione deve consistere in una tabella cos\`i formattata:
  \begin{itemize}
  \item Sommario delle modifiche svolte
  \item nome e cognome dell'autore della modifica
  \item ruolo dell'autore della modifica
  \item data della modifica
  \item versione del documento aggiornata a dopo la modifica
  \end{itemize}
  Questa tabella deve essere ordinata per data in modo decrescente, affinch\`e appaiano per prime le utlime modifiche subite dal documento e che la versione riportata dalla prima riga della tabella sia la stessa riportata nella prima pagina del documento.

  \subsubsection{Indici}
  In ogni documento deve essere presente un indice per le sezioni, uno per le figure ed uno per le tabelle. Mentre il primo deve essere sempre presente, nel caso di assenza di figure e/o tabelle i relativi indici devono essere omessi.

  \subsubsection{Formattazione generale delle pagine}
  Ciascuna pagina del documento deve contenere un'intestazione riportante:
  \begin{itemize}
  \item logo del gruppo
  \item nome del gruppo
  \item sezione corrente del documento
    \end{itemize}
  Ciascuna pagina inoltre deve riportare anche un pi\'e di pagina con le seguenti informazioni:
  \beign{itemize}
  \item Nome e versione del documento
  \item pagina corrente del documento nel formato ``X di N'' dove X è il numero di pagina corrente ed N è il numero di pagine totali
    \end{itemize}

\subsection{Norme tipografiche}
In questa sezione vengono trattate le convenzioni riguardanti tipografia, ortografia e stile per il testo dei documenti.
\subsubsection{Punteggiatura}
\begin{itemize}
\item parentesi:il testo racchiuso tra parentesi non deve presentare spazi addiacenti al carattere di parentesi e non deve chiudersi con un simbolo di punteggiatura
\item punteggiatura: un carattere di punteggiatura non deve essere mai preceduto da uno spazio
\item le lettere maiuscole vanno poste dopo i punti, i punti di domanda, i punti esclamativi e all'inizio di ogni voce per gli elenchi puntati, oltre che dove previsto dalla lingua italiana (o dalla lingua inglese per la documentazione da consegnare in tale lingua). Il nome del team, del progetto, dei documenti, dei ruoli, delle fasi di lavoro e le parole Proponente e Commitente devono sempre cominciare con lettera maiuscola.
  \item elenchi puntati: devono terminare con punto e virgola, ad eccezione dell'ultimo che deve terminare con un punto
\end{itemize}

\subsubsection{Stile di testo}
\begin{itemize}
\item Corsivo per le seguenti occasioni:
  \begin{itemize}
  \item Citazioni
  \item Abbreviazioni
  \item Riferimenti ad altri documenti
  \item Nomi di documenti
  \item Nomi di programmi o framework
  \item Ruoli di progetto
  \item Nomi di societ\`a o aziende
  \item Nome del gruppo
  \end{itemize}
\item Grassetto per i seguenti casi:
  \begin{itemize}
  \item Negli elenchi puntati per evidenziare il concetto sviluppato dal punto
  \item Per evidenziare parole chiave
  \end{itemize}
\item Monospace per formattare parti di codice o riportare indirizzi web e percorsi
\item Maiuscolo per intere parole solo nei casi di acronimi
  
