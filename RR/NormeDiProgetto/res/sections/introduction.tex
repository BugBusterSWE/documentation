\section{Introduzione}
\subsection{Scopo del documento}
Questo documento definisce le norme, gli strumenti e le procedure che tutti i membri del Gruppo devono adottare per lo svoglimento del progetto MaaS. Tutti i membri sono tenuti a leggere il documento e ad applicare le regole definite al fine di garantire conformità del materiale prodotto e di ridurre il numero di errori. Qualora siano necessarie modifiche a questo documento è necessario il tempestivo avviso agli altri membri del gruppo.

\subsection{Scopo del progetto}
Lo scopo del progetto MaaS è la realizzazione di un servizio per le aziende, raggiungibile in un server web, per la visualizzazione di dati aziendali. Tale progetto si basa su Maap, un'applicazione già esistente che ha lo scopo di fornire la visualizzazione dei dati letti dal database mongoDB in possesso dell'azienda. Il progetto verte sulla conversione di Maap da applicazione web a servizio, estendendone le potenzialità con un editor per facilitare i nuovi utenti la creazione di viste per i propri dati.

\subsection{Ambiguit\`a}
Al fine di evitare ambiguità dovute al linguaggio impiegato nei documenti, viene fornito il \Glossario contente la definizione dei termini marcati.

\subsection{Riferimenti a documenti esterni}
\begin{itemize}
\item Normativa ISO 12207 per ciclo di vita del software (\href{https://en.wikipedia.org/wiki/ISO/IEC_12207})
\item Swebook (\href{http://www.computer.org/web/swebok/v3})
\item Capitolato (\href{http://www.math.unipd.it/~tullio/IS-1/2015/Progetto/C4.pdf}
  \end{itemize}
