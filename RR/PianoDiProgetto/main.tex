\documentclass[11pt,a4paper]{article}

\usepackage[utf8]{inputenc}

%it is a document for us, we can use italian here
\usepackage[italian]{babel}

\usepackage{hyperref}
\hypersetup{
  colorlinks,
  citecolor=gray,
  filecolor=red,
  linkcolor=blue,
  urlcolor=blue
}

\usepackage{amsmath}
\usepackage{graphicx}
\usepackage{amsfonts}
\usepackage{listings}
\lstset{%
	commentstyle=\color{green},
	frame=single,
	keepspaces=true,
	keywordstyle=\color{blue},
	numbers=left,
	numberstyle=\tiny\color{black},
	rulecolor=\color{black},
        basicstyle=\ttfamily
}


\title{\textbf{Piano di Progetto}}
\author{BugBuster}

\date{22 Gennaio 2016}

\begin{document}

\maketitle

\tableofcontents

\section*{Versioni del documento}

\begin{center}

  \begin{table}[H]
    \centering
    \label{versioniDocumento}
    \begin{tabular}{ >{\centering}p{1.8cm} | >{\centering}p{2.2cm} | >{\centering}p{3cm} | >{\centering}p{6cm} }
      \textbf{Versione} & \textbf{Data} & \textbf{Persone coinvolte} & \textbf{Descrizione} \tabularnewline \hline
      %periodo di Progettazione Architetturale
      4.0.0 & 2016/04/10 & Giovanni Mazzocchin \linebreak (Responsabile) & Approvazione documento \tabularnewline \hline
      3.1.0 & 2016/04/09 & Davide Rigoni \linebreak (Verificatore) & Verifica \tabularnewline \hline
      3.0.1 & 2016/04/01 & Giovanni Mazzocchin \linebreak (Responsabile) & Aggiornato il preventivo a finire \tabularnewline \hline
      3.0.1 & 2016/04/01 & Giovanni Mazzocchin \linebreak (Responsabile) & Aggiunto consuntivo del periodo di Progettazione Architetturale \tabularnewline \hline
      %periodo di Analisi Miglioramenti
      3.0.0 & 2016/02/23 & Davide Polonio \linebreak (Responsabile) & Approvazione documento \tabularnewline \hline
      2.1.0 & 2016/02/23 & Giovanni Mazzocchin \linebreak (Verificatore) & Verifica \tabularnewline \hline
      2.0.4 & 2016/02/20 & Davide Polonio \linebreak (Responsabile) & Suddivisione della sezione "Consuntivo e Preventivo a finire" \tabularnewline \hline
      2.0.3 & 2016/02/20 & Davide Polonio \linebreak (Responsabile) & Modifica del periodo "Verifica e Validazione" a "Validazione" \tabularnewline \hline
      2.0.2 & 2016/02/20 & Davide Polonio \linebreak (Responsabile) & Aggiustamento sezione dei rischi \tabularnewline \hline
      2.0.1 & 2016/02/20 & Davide Polonio \linebreak (Responsabile) & Aggiunto dettaglio su consegna RPmin \tabularnewline \hline
      %periodo di Analisi
      2.0.0 & 2016/01/18 & Matteo Di Pirro \linebreak (Responsabile) & Approvazione documento \tabularnewline \hline
      1.4.0 & 2016/01/15 & Giovanni Mazzocchin \linebreak (Verificatore) & Verifica PDC e VV \tabularnewline \hline
      1.3.1 & 2016/01/13 & Matteo Di Pirro \linebreak (Responsabile) & Stesura PDC e VV \tabularnewline \hline
      1.3.0 & 2016/01/12 & Giovanni Mazzocchin \linebreak (Verificatore) & Verifica AM e PA \tabularnewline \hline
      1.2.1 & 2016/01/07 & Matteo Di Pirro \linebreak (Responsabile) & Stesura AM e PA \tabularnewline \hline
      1.2.0 & 2016/01/02 & Giovanni Mazzocchin \linebreak (Verificatore) & Verifica AM \tabularnewline \hline
      1.1.1 & 2016/12/28 & Davide Rigoni \linebreak (Responsabile) & Stesura AM \tabularnewline \hline
      1.1.0 & 2016/12/28 & Luca Bianco \linebreak (Verificatore) & Verifica rischi tecnologici e sul personale \tabularnewline \hline
      1.0.1 & 2016/12/28 & Davide Polonio \linebreak (Amministratore) & Stesura rischi tecnologici e sul personale \tabularnewline \hline
      1.0.0 & 2015/12/28 & Davide Rigoni \linebreak (Responsabile) & Stesura introduzione e organigramma \tabularnewline \hline
    \end{tabular}
  \end{table}
  
\end{center}

%\input{res/sections/mySectionFile.tex}



\section{Introduzione}
\subsection{Scopo del documento}
Questo documento ha l’obiettivo di identificare e dettagliare la pianificazione del gruppo BugBuster
relativa allo sviluppo del progetto MaaS. La ripartizione del carico di lavoro e di responsabilità tra i
componenti del gruppo, e il conto economico preventivo sono oggetto di primo piano in tale documento.

\subsection{Scopo del prodotto}
Lo scopo del progetto è la realizzazione di un Software as a Service che si basa sul progetto MaaP . Il prodotto atteso si chiama MaaS ossia MongoDB as an Admin Service.

\subsection{Glossario}
Al fine di evitare ogni ambiguità relativa al linguaggio impiegato nei documenti viene fornito il Glossario
v1.1 , contenente la definizione dei termini marcati con una G pedice.

\subsection{Riferimenti}
\subsubsection{Interni}
\begin{itemize}
\item Norme di Progetto v1.1
\item Capitolato d'appalto C4: MaaS: MongoDB as an admin Service: \\ \hyperlink{http://www.math.unipd.it/~tullio/IS-1/2015/Progetto/C4p.pdf}{http://www.math.unipd.it/~tullio/IS-1/2015/Progetto/C4p.pdf} 
\item Vincoli sull’organigramma del gruppo e sull’offerta tecnico-economica: \\
\hyperlink{http://www.math.unipd.it/~tullio/IS-1/2015/Progetto/PD01b.html}{http://www.math.unipd.it/~tullio/IS-1/2015/Progetto/PD01b.html}
\end{itemize}
\subsubsection{Esterni}
\begin{itemize}
\item IAN SOMMERVILLE, Software Engineering, Part 4: Software Management, 9th edition, Boston, Pearson Education, 2011
\end{itemize}

\subsection{Ciclo di vita}
L’interesse del committente è limitato al segmento di ciclo di vita che va dall’analisi dei requisiti al
rilascio del prodotto, escludendo dunque la successiva manutenzione ed il ritiro. Il modello di ciclo di
vita scelto è il modello incrementale, ritenuto preferibile in quanto permette di scomporre in sottosistemi
il problema principale, riducendo i rischi derivati dalla scarsa conoscenza da parte del gruppo delle
tecnologie necessarie, come illustrato nello \textit{Studio di fattibilità}. Questo modello permette inoltre di:
\begin{itemize}
\item Soddisfare primariamente i requisiti principali, e dedicarsi successivamente a quelli opzionali,
potendo però offrire al proponente un sistema funzionante;
\item Minimizzare i rischi di ritardo rispetto ai tempi stabiliti in quanto i cicli hanno durata breve e
sono precedentemente pianificati;
\item Rendere più semplice la verifica.
\end{itemize}

\section{Scadenze}
Di seguito sono presentate le scadenze che il gruppo ha deciso di rispettare e sulle quali si baserà la pianificazione del progetto:
\begin{itemize}
\item Revisione dei Requisiti (RR): 2016-01-22;
\item Revisione di Progetto (RP): 2016-04-11;
\item Revisione di Qualifica (RQ): 2016-05-16;
\item Revisione di Accettazione (RA): 2016-06-10;
\end{itemize}

\section{Ruoli e costi}
Durante lo sviluppo del progetto vi sono diversi ruoli, che ogni membro del gruppo BugBuster è tenuto a ricoprire almeno una volta, evitando conflitti d’interesse al momento della verifica. Nelle Norme di \textit{Progetto v1.1} sono descritte le responsabilità che competono ogni ruolo. I ruoli che ogni componente del gruppo ricoprirà in tempi diversi sono: \textit{Amministratore}, \textit{Analista}, \textit{Progettista}, \textit{Programmatore}, \textit{Responsabile} e \textit{Verificatore}.
Ciascun ruolo ha il proprio costo orario come segnalato nei 
\href{http://www.math.unipd.it/~tullio/IS-1/2015/Progetto/PD01b.html}{Vincoli sull’organigramma del gruppo e sull’offerta tecnico-economica}

\section{Pianificazione}






\end{document}
