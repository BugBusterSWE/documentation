\documentclass[11pt,a4paper]{article}

\usepackage[utf8]{inputenc}

%it is a document for us, we can use italian here
\usepackage[italian]{babel}

\usepackage{hyperref}
\hypersetup{
  colorlinks,
  citecolor=gray,
  filecolor=red,
  linkcolor=blue,
  urlcolor=blue
}

\usepackage{amsmath}
\usepackage{graphicx}
\usepackage{amsfonts}
\usepackage{listings}
\lstset{%
	commentstyle=\color{green},
	frame=single,
	keepspaces=true,
	keywordstyle=\color{blue},
	numbers=left,
	numberstyle=\tiny\color{black},
	rulecolor=\color{black},
        basicstyle=\ttfamily
}


\title{\textbf{Piano di Progetto}}
\author{BugBuster}

\date{22 Gennaio 2016}

\begin{document}

\maketitle

\tableofcontents


\section*{Versioni del documento}

\begin{center}

  \begin{table}[H]
    \centering
    \label{versioniDocumento}
    \begin{tabular}{ >{\centering}p{1.8cm} | >{\centering}p{2.2cm} | >{\centering}p{3cm} | >{\centering}p{6cm} }
      \textbf{Versione} & \textbf{Data} & \textbf{Persone coinvolte} & \textbf{Descrizione} \tabularnewline \hline
		2.0.0 & 2015/12/31 & Andrea Mantovani \linebreak (Responsabile) & Approvazione documento \tabularnewline \hline \hline
		1.0.3 & 2015/12/31 & Luca Bianco \linebreak (Verificatore) & Verifica Valutazione sugli altri capitolati \tabularnewline \hline
		1.0.2 & 2015/12/31 & Luca Bianco \linebreak (Verificatore) & Verifica Capitolato scelto - C4: MaaS G : MongoDB G as an admin Service  \tabularnewline \hline
		1.0.1 & 2015/12/28 & Giovanni Mazzocchin \linebreak (Analista) & Stesura Valutazione sugli altri capitolati \tabularnewline \hline
		1.0.0 & 2015/12/27 & Giovanni Mazzocchin \linebreak (Analista) & Stesura Capitolato scelto - C4: MaaS G : MongoDB G as an admin Service \tabularnewline \hline
    \end{tabular}
  \end{table}
  
\end{center}
%how to: \input{res/chapters/argumentOfChapter}

\section{Processi Organizzativi}

\subsection{Processo di pianificazione}

\subsubsection{Scopo}

Lo scopo di questo processo è la produzione e la comunicazione del piano di progetto ai membri del gruppo. Tutto ciò aiuta nella determinazione nell'ambito della gestione del progetto e le attività tecniche, identifica gli output del processo e ne stabilisce un calendario.


\subsubsection{Risultati osservabili}

Questo processo produce:
\begin{itemize}

\item l'identificazione dell'ambito del progetto;
\item l'analisi di fattibilità degli obiettivi prefissati
\item una stima esatta dei task necessari per completare il lavoro
\item un piano per l'esecuzione del lavoro programmato

\end{itemize}

\subsubsection{Descrizione}

\paragraph*{Soglia oraria giornaliera}È stato deciso di porre la soglia oraria ai seguenti orari: la mattina dalle ore (?) alle ore (?), il pomeriggio  ore (?) alle ore (?) %da decidere

\paragraph*{Assegnazione dei task}Il responsabile si occuperà di assegnare i task in base alla disponibilità dei membri del gruppo e alla rotazione dei ruoli

\subsubsection{Rotazione dei ruoli}

Durante lo sviluppo del progetto vi sono diversi ruoli che devono essere ricoperti dai membri del gruppo BugBuster. Il \textit{Piano di progetto} si occupa di assegnare le attività principali e specifiche per ogni ruolo. Il \textit{Reponsabile} di progetto ha l'onere di far rispettare i ruoli assegnati durante l'attività e il \textit{Verificatore} deve essere in grado di individuare eventuali incongruenze e di segnalarle opportunamente.
\paragraph*{Incongruenze tra ruoli e risoluzioni}
Esistono due principali tipi di incongruenze incontrabili:
\begin{itemize}

\item Il compito svolto non fa parte dei compiti propri di un dato ruolo
\item La stessa persona verifica ciò che ha precedentemente prodotto
\end{itemize}

È importante tenere conto inoltre di altri tre fattori per un buono svolgimento del progetto:
\begin{itemize}

\item Una persona non deve impiegare più del 50\% delle ore di lavoro in un unico ruolo: questo impedirebbe una corretta rotazione dei ruoli.
\item Una persona non dovrebbe ricere più di un task al giorno, onde evitare un eccessivo carico di lavoro, che rischierebbe di non essere effettuato corettamente.
\item I ruoli verranno ruotati ogni due settimane.

\end{itemize}

Il resposabile nell'azione di assegnazione dei ticket dovr\`a indicare il ruolo che il destinatario ricopre nell'adempimento di tale task. Questo \`e necessario per il rendiconto economico. I ruoli sono assegnati basandosi sul \PianoDiProgetto.

\paragraph*{Identificazione dei ruoli}

\begin{itemize}

\item[Amministratore] L'\textit{Amministratore} equipaggia, organizza e gestisce l'ambiente di lavoro e di produzione. Collabora con il \textit{Responsabile} alla stesura delle \textit{Norme di Progetto} e del \textit{Piano di Progetto}, e si assicura di:
  \begin{itemize}

  \item Attuare scelte tecnologice concordate con il \textit{Responsabile}
  \item Gestire le liste di distribuzione e assicurarne il rispetto
  \item Controllare versioni e configurazioni del prodotto
  \item Risolvere i problemi legati alla gestione dei processi
    
  \end{itemize}

\item[Analista] L'\textit{Analista} \`e responsabile dell'attivit\`a di analisi, e ha il compito di comprendere appieno il dominio applicativo. Si occupa di redigere lo \textit{Studio di Fattibilit\`a} e l'\textit{Analisi dei Requisiti}.

\item[Progettista] Il \textit{Progettista} \`e il responsabile dell'attivit\`a di progettazione, ha una profonda conoscenza dello stack tecnologico utilizzato e presenta adeguate competenze tecniche. Ha una forte influnza su aspetti tecnici e tecnologici del progetto.

\item[Programmatore] Il \textit{Programmatore} ha responsabilit\`a circoscritte, e si occupa della attivit\`a di codifica rispettando le \textit{Norme di Progetto}. Si occupa di realizzare il prodotto e le componenti di ausilio necessarie per l'esecuzione delle prove di verifica e validazione.

\item[Responsabile] Il \textit{Responsabile} ha l'ultima voce in capitolo per quanto concerne le decisioni sul progetto, \`e il responsabile ultimo dei risultati, infatti approva l'emissione dei documenti, oltre a redigere insieme all'\textit{Amministratore} il \textit{Piano di Progetto}. Il \textit{Responsabile} ha molte responsabilit\`a, che sono:
  \begin{itemize}

  \item Pianificazione e organizzazione dello sviluppo del progetto, stimanto tempi, costi e assgnazione delle attivit\`a ai componenti del gruppo;
  \item Riportare lo stato del progetto ai committenti;
  \item Analizzare i rischi che possono incorrere, monitorarli e prendere provvedimenti a riguardo;
  \item Stabilire una \textit{ways of working} per ogni componente del gruppo, ai fini di un'influenza positiva delle performance del gruppo.
    
  \end{itemize}

\item[Verificatore] Il \textit{Verificatore} organizza ed attua le attivit\`a di verifica e controlla che le attivit\`a siano conforme alle norme definite. Redige nel \textit{Piano di Qualifica} la parte che documenta le attivit\`a svolte e i risultati ottenuti, confrontandoli con le metriche del \PianoDiQualifica. Le ore assegnate al verificatore devono corrispondere almeno al 30\% delle ore totali suddivise tra i ruoli.
  
\end{itemize}

\subsubsection{Revisioni di progetto}

Le revisioni, specificate in \PianoDiProgetto, consistono in \textit{review}\footnote{Revisioni informali.} e \textit{audit}\footnote{Revisioni formali.}, ma essendo le modalit\`a con cui vengono affrontate dal gruppo le medesime verranno denominate presentazioni o revisioni. Secondo quanto stabilito nell'IEE 1028, vengono stabilite le seguenti procedure:
\begin{enumerate}
	\setcounter{enumi}{-1}
	\item Entry evaluation: il Responsabile di Progetto prepara una checklist e si assicura che esistano le condizioni che permettano il successo della presentazione;
	\item Management preparation: il Responsabile si assicura che tutti i membri del gruppo possano partecipare e l'ambiente di lavoro rispetti i vincoli imposti o suggeriti dal committente. A tal fine è stato elaborata una lista di indicazioni (\ref{presentazione});
	\item Planning the review: il Responsabile deve identificare e confermare gli obiettivi della revisione, inoltre si assicura che tutto il team sia equipaggiato con le risorse necessarie per affrontare la revisione;
	\item Overview of review procedures: ci si assicura che tutti i membri del gruppo conoscano gli obiettivi, le procedure e il materiale portato in revisione;
	\item Individual preparation: ogni membro del gruppo si prepara individualmente per la presentazione;
	\item Group examination: il gruppo si incontra al completo e prova la presentazione confermando il rispetto dei tempi e degli obiettivi;
	\item Rework/follow-up: vengono corretti, laddove possibile,  gli errori e/o i difetti;
	\item Exit evaluation: il Responsabile si assicura che tutti gli output prodotti siano pronti per la revisione.
\end{enumerate}

\label{presentazione}
Le indicazioni su come costruire una presentazione sono le seguenti:
\begin{itemize}

\item Le diapositive devono seguire la regola del \textit{5 x 5}: ovvero per una buona leggibilit\`a dovrebbero essere presenti alpi\`u 5 punti ognuno di lunghezza massima di 5 parole.

\item Evitare l'utilizzo di immagini troppo piccole o dettagliate;

\item Argomentare come \`e stata applicata la teoria al progetto, senza limitarsi a spiegarla;

\item La spiegazione delle slide deve avvenure rivolgendosi al pubblico evitando sempre di dargli le spalle;

\item Le mani non devono stare nelle tasche ma aiutare a sottolineare le parti importanti nella spiegazione;

\item Ogni slide deve essere numerata;

\end{itemize}

%ISO/IEC 12207:1995 is a standard for process with a GENERIC wiew
%'the customer/client uses process standard to ease control activity' 
%we always refer to ISO/IEC:12207 but there exist other standards
%We'll deal with the following Primary Processes: 1)Supply process, 2)Development process

%###################
%Why did we choose to structure the documentation in this way?: 
%"la buona organizzazione di un’azienda si basa sul riconoscimento dei propri processi, 
%la loro adozione consapevole ed efficace e il loro supporto efficiente 
%###################

%\href{nome}{url}
\section{Processi primari}
	\section{Processo di sviluppo}
		\subsubsection{Attività di Analisi dei requisiti}
			\paragraph*{Scopo}
				Gli analisti del gruppo dovranno ricavare i requisiti utili per il progetto
				dal capitolato e dagli incontri con il proponente, avendo come obiettivo la
				produzione del documento `Analisi dei Requisiti'.
			
			%parlare di: IEEE 830-1998: Recommended Practice
			%for Software Requirements Specifications (slide 32 in ING_REQUISITI)
			La presente attività ha come scopo la produzione una specifica dei
			requisiti conforme ai principi dello standard IEEE 830-1998, espressi in 8 qualit\`a
			essenziali:
				\begin{itemize}
				\item Priva di ambiguit\`a (UNAMBIGOUS)
				\item Corretta (CORRECT)
				\item Completa (COMPLETE)
				\item Verificabile (VERIFIABLE)
				\item Consistente (CONSISTENT)
				\item Modificabile (MODIFIABLE)
				\item Tracciabile (TRACEABLE)
				\item Ordinata per rilevanza (RANKED)
				\end{itemize}
			
			
			
			%casi d'uso: si parte da 0 nella numerazione
			\paragraph*{Casi d'uso}
			Ogni caso d'uso dovrà presentare i seguenti campi:
				\begin{itemize}
				\item Codice identificativo
					\paragraph Precisazione sul punto corrente
					\textbf{UC-{X} x.y.z}
					Descrizione:
						\begin{itemize}
						\item \textbf{X}: rappresenta uno degli ambiti di riferimento individuati
						nell'Analisi dei Requisiti, in particolare:
							\begin{itemize}
							\item [] \textbf{B} = Ambito Backend  %backend va precisato
							\item [] \textbf{G} = Ambito utente Guest %guest va precisato
							\item [] \textbf{S} = Ambito SuperAdmin %superadmin va precisato
							\item [] \textbf{O} = Ambito Owner %owner va precisato
							\item [] \textbf{E} = Ambito Editor %editor va precisato
							\end{itemize}
						\end{itemize}
				\item Titolo
				\item Diagramma UML
				\item Attori
				\item Scopo e descrizione
				\item Precondizioni
				\item Postcondizione
				\item Flusso principale
				\end{itemize}
			
			
			
			%Obiettivi: slide 36 di ING_REQUISITI
			
			
			%slide 40 di ING_REQUISITI: Cause di abbandono
			La stesura di un' Analisi dei Requisiti di qualità è cruciale, infatti
			da un rapporto dello Standish Group del 1995 si nota come due delle cause primarie
			di abbandono dei progetti siano le seguenti:
				\begin{itemize}
				\item Requisiti incompleti
				\item Volatilità di specifiche e requisiti
				\end{itemize}
			
			
			%parlare degli STRUMENTI utilizzati
			\paragraph*{Strumenti}
				Per i diagrammi dei casi d'uso verrà utilizzato il linguaggio UML 2.0,
				in quanto noto a tutti i membri del gruppo dato che è stato trattato a lezione
				di Ingegneria del Software.
				Si scelto inoltre di adottare l'editor online \emph{draw.io}.
			
			
		\subsubsection{Attività di Progettazione}
			\paragraph*{Scopo}
		 %Costruzione a priori perseguendo la correttezza per costruzione
		 %invece che inseguendo la correttezza per correzione     <<  slide PROGETTAZIONE
			 %Procede dall'analisi dei requisiti   <<  slide PROGETTAZIONE	
		     La presente attivit\`a, che procede dall'Analisi dei Requisiti, persegue l'obiettivo
		     della correttezza per costruzione, permettendo di ridurre al minimo le attività di correzione
		     in fase di codifica.
		     
                     I documenti elaborati in questa fase sono:
                     \begin{itemize}
                       \item Specifica Tecnica, nel quale vengono individuate le componenti macroscopiche del sistema,
                         procedura utile per la paralellizazione del lavoro
                       \item Definizione di Prodotto, documento che fornisce al \textit{programmatore} tutte le direttive necessarie per la codifica
                     \end{itemize}
		 
		 %si useranno Design patterns
		 Per la progettazione al livello di sistema si useranno \textit{Design pattern}
                 
                 %diagrammi UML prodotti
                      \paragraph*{Diagrammi prodotti}
                      Nel corso di questa fase verranno prodotti quattro tipi di diagrammi \textit{UML}:
                      \begin{itemize}
                        \item Diagrammi delle classi
                        \item Diagrammi dei \textit{package}
                        \item Diagrammi di sequenza
                        \item Diagrammi di attivit\`a
                      \end{itemize}
                      I tipi di diagrammi sopra elencati andranno a far parte dei due documenti menzionati sopra.
		
		 
		 %parlare degli STRUMENTI utilizzati
		 \paragraph*{Strumenti}
		 Anche per questa attivit\`a, i membri del gruppo hanno stabilito l'utilizzo del linguaggio UML 2.0 per i seguenti
		 diagrammi:
		 \begin{itemize}
			\item Diagrammi delle package  
			\item Diagrammi delle classi	
			\item Diagrammi di attivit\`a
		\end{itemize}
			 
		
		\subsubsection{Attività di Codifica}
                %citare https://github.com/RisingStack/node-style-guide/blob/master/README.md
                \subsubsection{Scopo}
                La presente attività ha come scopo la traduzione in codice sorgente dei risultati ottenuti in sede
                di \textit{Specifica Tecnica} e \textit{Definizione di Prodotto}.
                In sede di \textit{Revisione dei Requisiti} non è ancora possibile fornire una visione precisa sulla presente attivit\`a,
                tuttavia per il codice \textit{Javascript} %termine da glossario
                è stata stabilita l'adozione delle direttive presenti in \href{https://github.com/RisingStack/node-style-guide/blob/master/README.md}{RisingStack Node.js Style Guide()}
     		%<<< slide 22/30 di ANALISI_STATICA
		%L’adozione di standard di codifica e di sottoinsiemi del linguaggio appropriati deve essere coerente con la scelta dei metodi di verifica adottati   
		%La verifica solo retrospettiva (a valle dello sviluppo) è spesso inadeguata
		%Il costo di rilevazione e correzione di un errore è tanto maggiore quanto più avanzato è lo stadio di sviluppo        

		%<<< slide 30/30 di ANALISI_STATICA
		%L’efficacia dei metodi di analisi è funzione della qualità di strutturazione del codice
                
                
                
		
	\subsection{Processo di Fornitura}




\section{Introduzione}
\subsection{Scopo del documento}
Questo documento ha l’obiettivo di identificare e dettagliare la pianificazione del gruppo BugBuster
relativa allo sviluppo del progetto MaaS. La ripartizione del carico di lavoro e di responsabilità tra i
componenti del gruppo, e il conto economico preventivo sono oggetto di primo piano in tale documento.

\subsection{Scopo del prodotto}
Lo scopo del progetto è la realizzazione di un Software as a Service che si basa sul progetto MaaP . Il prodotto atteso si chiama MaaS ossia MongoDB as an Admin Service.

\subsection{Glossario}
Al fine di evitare ogni ambiguità relativa al linguaggio impiegato nei documenti viene fornito il Glossario
v1.1 , contenente la definizione dei termini marcati con una G pedice.

\subsection{Riferimenti}
\subsubsection{Interni}
\begin{itemize}
\item Norme di Progetto v1.1
\item Capitolato d'appalto C4: MaaS: MongoDB as an admin Service: \\ \hyperlink{http://www.math.unipd.it/~tullio/IS-1/2015/Progetto/C4p.pdf}{http://www.math.unipd.it/~tullio/IS-1/2015/Progetto/C4p.pdf} 
\item Vincoli sull’organigramma del gruppo e sull’offerta tecnico-economica: \\
\hyperlink{http://www.math.unipd.it/~tullio/IS-1/2015/Progetto/PD01b.html}{http://www.math.unipd.it/~tullio/IS-1/2015/Progetto/PD01b.html}
\end{itemize}
\subsubsection{Esterni}
\begin{itemize}
\item IAN SOMMERVILLE, Software Engineering, Part 4: Software Management, 9th edition, Boston, Pearson Education, 2011
\end{itemize}

\subsection{Ciclo di vita}
L’interesse del committente è limitato al segmento di ciclo di vita che va dall’analisi dei requisiti al
rilascio del prodotto, escludendo dunque la successiva manutenzione ed il ritiro. Il modello di ciclo di
vita scelto è il modello incrementale, ritenuto preferibile in quanto permette di scomporre in sottosistemi
il problema principale, riducendo i rischi derivati dalla scarsa conoscenza da parte del gruppo delle
tecnologie necessarie, come illustrato nello \textit{Studio di fattibilità}. Questo modello permette inoltre di:
\begin{itemize}
\item Soddisfare primariamente i requisiti principali, e dedicarsi successivamente a quelli opzionali,
potendo però offrire al proponente un sistema funzionante;
\item Minimizzare i rischi di ritardo rispetto ai tempi stabiliti in quanto i cicli hanno durata breve e
sono precedentemente pianificati;
\item Rendere più semplice la verifica.
\end{itemize}

\section{Scadenze}
Di seguito sono presentate le scadenze che il gruppo ha deciso di rispettare e sulle quali si baserà la pianificazione del progetto:
\begin{itemize}
\item Revisione dei Requisiti (RR): 2016-01-22;
\item Revisione di Progetto (RP): 2016-04-11;
\item Revisione di Qualifica (RQ): 2016-05-16;
\item Revisione di Accettazione (RA): 2016-06-10;
\end{itemize}

\section{Ruoli e costi}
Durante lo sviluppo del progetto vi sono diversi ruoli, che ogni membro del gruppo BugBuster è tenuto a ricoprire almeno una volta, evitando conflitti d’interesse al momento della verifica. Nelle Norme di \textit{Progetto v1.1} sono descritte le responsabilità che competono ogni ruolo. I ruoli che ogni componente del gruppo ricoprirà in tempi diversi sono: \textit{Amministratore}, \textit{Analista}, \textit{Progettista}, \textit{Programmatore}, \textit{Responsabile} e \textit{Verificatore}.
Ciascun ruolo ha il proprio costo orario come segnalato nei 
\href{http://www.math.unipd.it/~tullio/IS-1/2015/Progetto/PD01b.html}{Vincoli sull’organigramma del gruppo e sull’offerta tecnico-economica}

\section{Pianificazione}






\end{document}
