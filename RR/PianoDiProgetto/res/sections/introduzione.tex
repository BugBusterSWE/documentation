\section{Introduzione}
\subsection{Scopo del documento}
Questo documento ha l’obiettivo di identificare e dettagliare la pianificazione del gruppo BugBusters
relativa allo sviluppo del progetto MaaS. La ripartizione del carico di lavoro e di responsabilit\`a tra i
componenti del gruppo, e il conto economico preventivo sono oggetto di primo piano in tale documento.

\subsection{Scopo del prodotto}
Lo scopo del progetto \`e la realizzazione di un \textit{Software as a Service} basato sul prodotto MaaP. Il prodotto atteso si chiama MaaS, ossia \textit{MongoDB as an admin Service}.

\subsection{Glossario}
Al fine di evitare ogni ambiguit\`a relativa al linguaggio impiegato nei documenti viene fornito il Glossario
v1.0.0 , contenente la definizione dei termini marcati con una G pedice.

\subsection{Riferimenti}
\subsubsection{Interni}
\begin{itemize}
\item Norme di Progetto v1.1
\item Capitolato d'appalto C4: MaaS: MongoDB as an admin Service: \\ \href{http://www.math.unipd.it/~tullio/IS-1/2015/Progetto/C4p.pdf}{http://www.math.unipd.it/\~tullio/IS-1/2015/Progetto/C4p.pdf} 
\item Vincoli sull’organigramma del gruppo e sull’offerta tecnico-economica: \\
\href{http://www.math.unipd.it/~tullio/IS-1/2015/Progetto/PD01b.html}{http://www.math.unipd.it/\~tullio/IS-1/2015/Progetto/PD01b.html}
\end{itemize}
\subsubsection{Esterni}
\begin{itemize}
\item IAN SOMMERVILLE, Software Engineering, Part 4: Software Management, 9th edition, Boston, Pearson Education, 2011
\item The Guide to the Software Engineering Body of Knowledge V3 (SWEBOK)
\end{itemize}

\subsection{Ciclo di vita}
L’interesse del committente è limitato al segmento di ciclo di vita che va dall’analisi dei requisiti al
rilascio del prodotto, escludendo dunque la successiva manutenzione ed il ritiro. Il modello di ciclo di
vita scelto \`e quello incrementale, ritenuto preferibile in quanto:
\begin{itemize}
\item Permette di monitorare l'evoluzione del progetto senza mai eseguire iterazioni, ma solo incrementi
\item Alla conclusione di ogni processo si ha una base verificata che ne permette un veloce successivo incremento
\item Permette l'alternanza tra le attivit\`a
\end{itemize}
