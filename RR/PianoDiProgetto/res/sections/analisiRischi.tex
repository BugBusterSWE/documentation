\section{Analisi dei rischi}
Sezione dove il gruppo BugBusters elenca i rischi incontrati durente lo svolgimento del progetto e gli strumenti e tecniche adoperare per ridurlo o debellarlo. Inoltre \`e presente le descrizione di ogni elemento informativo che il gruppo BugBusters ha trovato sufficiente per categorizzare un rischio.
Ad ogni rischio \`e assegnato un identificativo nel formato: 

\begin{center}
\textit{[codice gruppo][codice probabilit\`a rischio][indice numerico crescente]}  
\end{center}

dando al lettore una veloce comprensione del rischio.

Il \textit{codice gruppo} indica l'appartentenza ad una delle seguenti categorie:
\begin{itemize}
\item \textbf{T} - Rischi tecnologici: guasti o malfunzionamenti hardware;
\item \textbf{P} - Rischi sulle persone: riguardante i membri del gruppo;
\item \textbf{M} - Rischi organizzativi: derivati dall'ambiente e dalle necessit\`a organizzative;
\item \textbf{S} - Rischi sugli strumenti software: problemi sul software impiegato a supporto del progetto;
\item \textbf{R} - Rischi sui requisiti: mal comprensione, mancanza o necessit\`a di aggiunta di requisiti;
\item \textbf{V} - Rischi sulle stime: sottostima dei tempi e dei costi necessari;
\end{itemize}

Il \textit{codice propabilit\`a rischio} \`e assiociato alle seguenti probabilit\`a di rischio: 
\begin{itemize}
\item \textbf{L} - Molto Bassa
\item \textbf{S} - Bassa
\item \textbf{M} - Media
\item \textbf{H} - Alta
\end{itemize}

L' \textit{indice numerico crescente} indica la posizione all'interno della sua categoria. 

\`E anche stabilit un \textit{Effetto} che quel rischio ha sul progetto. Non sono indicati nell'identificativo del rischio in quanto criterio di ricerca merno rilevante. Il livello degli effetti sono:
\begin{itemize}
\item Tollerabili;
\item Seri;
\item Catastrofici;
\end{itemize}

L'analisi dei rischi di suddivide in 4 parti:
\begin{itemize}
\item \textit{Identificazione}: breve descrizione del rischio analizzato;
\item \textit{Probabilit\`a}': probabilit\`a di rischio;
\item \textit{Effetti}: tipologia effetto;
\item \textit{Pianificazione}: la procedura che il gruppo BugBusters adotta in caso si verificasse il rischio;
\item \textit{Controllo}: metodo per la verifica del successo della pianificazione;
\end{itemize}

\subsection{Rischi tecnologici}
\subsubsection{Guasto Hardware - TS1}
Ogni membro del gruppo \`e provvisto di un computer portatile o fisso in cui svolgere il proprio ruolo. Il rischio sta in un guasto ai computer di un elemente del gruppo che gli rende impossibile il proseguimento dei compiti assegnatogli.
\begin{enumerate}
\item \textit{Probabilit\`a}: Bassa;
\item \textit{Effetti}: Tollerabile; 
\item \textit{Pianificazione}: l'Universit\`a di Padova metta a disposizione laboratori informatici per gli studenti che ne richiedono l'uso;
\item \textit{Controllo}: il numero dei computer rimane invariato durante tutto lo sviluppo del software, quindi il rischio resta invariato.
\end{enumerate}

\subsection{Rischi sulle persone}
\subsubsection{Problemi dei componenti del gruppo - PM1}
Ogni membro del gruppo ha impegni personali perci\`o non potr\`a svolgere un orario regolare di lavoro al progetto. Inoltre \`e da prendere in considerazioni, per motivi lavorativi o di salute, periodi di assenza prolungati.
\begin{enumerate}
\item \textit{Probabilit\`a}: Media;
\item \textit{Effetti}: Tollerabile;
\item \textit{Pianificazione}: nel caso un elemento del gruppo sia impossibilitat\`o a svolgere per un periodo limitato i propri task, il responsabile provveder\`a a riassegnare i task ad altri;
\item \textit{Controllo}: utilizzo del calendario di gruppo per individuare le fasi critiche.
\end{enumerate}

\subsubsection{Problemi tra i componenti del gruppo - PM2}
Per ogni membro del gruppo \`e la prima esperienza su un progetto con un gruppo numeroso di persone ed un lungo periodo di lavoro. La probabilit\`a di insorgere in conflitti aumenta con l'incrementare del lavoro e l'avvicinamento alle date di consegna.
\begin{enumerate}
\item \textit{Probabilit\`a}: Media;
\item \textit{Effetti}: Seri;
\item \textit{Pianificazione}: nel caso di forte contrasto, sar\`a compito del \textit{Responsabile di progetto} mediare al fine di risolvere il conflitto;
\item \textit{Controllo}: l'\textit{Amministratore} ha il compito di mantenere un clima cooperativo all'interno del gruppo.
\end{enumerate} 

\subsection{Scarsa conoscenza delle tecnologie - PA3}
Essendo un progetto ad ampio spettro \`e certo che si dovranno usare tecnologie non trattate durante il percorso di studi o offrontate personalemente.
\begin{enumerate}
\item \textit{Probabilit\`a}: Alta;
\item \textit{Effetti}: Seri;
\item \textit{Pianificazione}: ogni membro del gruppo \`e tenuto a studiare le tecnologie coinvolte durante lo sviluppo. \`E inoltre data la disponibilit\`a del proponente per discutere degli argomenti pi\`u complessi;
\item \textit{Controllo}: il \textit{Responsabile} ha il compito di verificare il grado delle conoscenze sulle tecnologie da parte dei membri del gruppo.
\end{enumerate}

\subsection{Rischi organizzativi}
\subsubsection{Rotazione dei ruoli - MS1}
Il cambio di ruolo pu\`o creare difficolt\`a ai componenti del gruppo a causa del cambio di responsabilit\`a e competenze assiociati al ruolo da coprire. Ne potrebbe conseguire un periodo d'abbassamento d'efficacia di tutto il gruppo.
\begin{enumerate}
\item \textit{Probabilit\`a}: Bassa;
\item \textit{Effetti}: Tollerabile;
\item \textit{Pianificazione}: la rotazione dei ruoli viene prestabilit\`a prima di avviare i lavori, perci\`o ogni membro del gruppo ha la possibilit\`a di studiare preventivamente i compiti e le metodologie del nuovo ruolo;
\item \textit{Controllo}: il \textit{Responsabile} verifica che ogni membro del gruppo ricopra tutti i ruoli previsti dalle \textit{Norme di progetto}.
\end{enumerate}

\subsection{Rischi sugli strumenti software}
\subsubsection{Piattaforme fuori servizio - SL1}
Le piattaforme coinvolte sono Teamwork e GitHub.
\begin{enumerate}
\item \textit{Probabilit\`a}: Molto bassa;
\item \textit{Effetti}: Catastrofici;
\item \textit{Pianificazione}: suddivio nelle piattaforme coinvolte.
  \begin{itemize}
    \item Teamwork dichiara di appoggiarsi ai servizi di backup offerti da Amazon;
    \item GitHub dichiara di effettuare backup su tre differenti server, di cui uno fuori sede. All'indirizzo \href{https://help.github.com/articles/github-security/}{https://help.github.com/articles/github-security/} \`e possibile trovare tutte le informazioni rilasciate da GitHub sulle misure di sicurezza adottate. Inoltre, ogni elemento del gruppo ha un copia in locale del progetto, consentendo un recupero parziale o totale.
  \end{itemize}
\item \textit{Controllo}: ci si affida alle misure di sicurezza adottare dai servizi adottati.
\end{enumerate}

\subsubsection{Cambio di un programma - SS2}
Data l'inesperienza sulla gestione di un gruppo cosi numeroso e i softaware pi\'u capaci a supportarlo non si pu\'o escludere il passaggio ad un programma valutato migliore di quello adottato fin'ora.
\begin{enumerate}
\item \textit{Probabilit\`a}: Bassa;
\item \textit{Effetti}: Seri;
\item \textit{Pianificazione}: prima della reale sostituzione l'\textit{Amministratore} procede alla verifica che il software sostitutivo soddisfi i seguenti requisiti:
  \begin{itemize}
    \item retrocompatibilit\`a - tutto il materiale precedentemente prodotto deve essere riprodotto con il nuovo programma;
    \item migliorante - il nuovo programma adottato dovr\`a consentire di svolgere lo stesso lavoro in tempo minori e/o con una qualit\`a superiore.
  \end{itemize}
\item \textit{Controllo}: il \textit{Verificatore} controller\`a che il materiale prodotto con il nuovo softaware rispetti i criteri attesi.
\end{enumerate}

\subsection{Rischi sui requisiti}
\subsubsection{Comprensione errata dei requisiti - RM1}
Data l'inesperienza dei componenti del gruppo nell'analisi dei requisiti, \`e possibile un'errata compensione dei requisiti comportando un'offerta non conforme alla richieste.
\begin{enumerate}
\item \textit{Probabilit\`a}: Media;
\item \textit{Effetti}: Seri;
\item \textit{Pianificazione}: i membri del gruppo sono tenuti a colmare le lacune sui fondamenti dell'analisi dei requisiti;
\item \textit{Controllo}: in caso di dubbi rivolgersi agli altri membri del gruppo per chiarimenti o contattare il Prof. Riccardo Cardin.
\end{enumerate}

\subsubsection{Modifica dei requisiti - RS2}
Durante gli incontri con il proponente \`e emersa la possibilit\`a di poter modificare i requisiti segnati nel capitolato.
\begin{enumerate}
\item \textit{Probabilit\`a}: Bassa;
\item \textit{Effetti}: Seri;
\item \textit{Pianificazione}: il cambio di requisiti deve svolgersi prima della prima data di revisione del progetto in modo da non rallentare o bloccare i \textit{Progettisti} durante la stesura della Specifica tecnica. Un ulteriore controllo verr\`a effettuato prima dell'entrata in codifica;
\item \textit{Controllo}: lavorando a stretto contatto con il proponente si avr\`a una visione d'insieme pi\`u chiara mitigando questo problema.
\end{enumerate}

\subsection{Rischi sulle stime}
\subsubsection{Sottostima dei tempi necessari}
Data l'inesperienza dei membri del gruppo nella pianificazione di progetto e l'attuazione della stessa su un arco di tempo medio-lungo, la sottostima dei tempi necessari alla realizzazione del progetto risulta un rischio concreto.
\item \textit{Probabilit\`a}: Alta;
\item \textit{Effetti}: Tollerabili;
\item \textit{Pianificazione}: i gruppi di attivit\`a pianificate relative alle scadenze fissate dal committente non ricoprono tutto l'arco di tempo a dispozione lasciando uno slack prima di ogni consegna;
\item \textit{Controllo}: il \textit{Responsabile}, grazie alla dashboard pu\`o verificare lo stato di avanzamento delle attivit\`a;


