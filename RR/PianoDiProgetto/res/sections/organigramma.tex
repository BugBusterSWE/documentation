\section*{Organigramma} %non deve comparire nell'indice
\subsection*{Accettazione da parte dei componenti}
\begin{center}
  \begin{tabular}{ l | c | c }
    \hline
    \textbf{Nome e Cognome} & \textbf{Data} & \textbf{Email} \\
    \hline
    Andrea Mantovani & 2016/01/19 &  \\ \hline
    Davide Polonio & 2016/01/19 &  \\ \hline
    Davide Rigoni & 2016/01/19 &  \\ \hline
    Emanuele Carraro & 2016/01/19 &  \\ \hline
    Giovanni Mazzocchin & 2016/01/19 &  \\ \hline
    Luca Bianco & 2016/01/19 &  \\ \hline
    Matteo Di Pirro & 2016/01/19 &  \\
    \hline
  \end{tabular}
\end{center}


\subsection*{Componenti}
\begin{center}
  \begin{tabular}{ l | c | c }
    \hline
    \textbf{Nome e Cognome} & \textbf{Matricola} & \textbf{Email} \\
    \hline
    Andrea Mantovani & 1070258 & andreamnt94@gmail.com \\ \hline
    Davide Polonio & 1070162 & poloniodavide@gmail.com \\ \hline
    Davide Rigoni & 1073176 & davider1994@gmail.com \\ \hline
    Emanuele Carraro & 1070742 & emanuele.carraro.94@gmail.com \\ \hline
    Giovanni Mazzocchin & 1071619 & giovanni.mazzocchin@gmail.com \\ \hline
    Luca Bianco & 1073397 & luca.biancow@gmail.com \\ \hline
    Matteo Di Pirro & 1074041 & matteodipirro@gmail.com \\
    \hline
  \end{tabular}
\end{center}



\subsection*{Definizione dei ruoli}
I componenti del team ripartiranno tra loro i ruoli e le responsabilità. Essi rappresentano le relative figure aziendali, e ad ognuno corrisponde un costo orario espresso in euro.
Ogni componente del team deve ricoprire almeno una volta ogni ruolo, con la sicurezza di rispettare i vincoli dati (una persona non può, ad esempio, essere verificatrice di sé stessa).
