\section{Suddivisione del lavoro e prospetto orario}
Ogni componente del gruppo dovrà ricoprire ogni ruolo almeno una volta nel corso del progetto.
Durante lo stesso periodo un componente può ricoprire più ruoli, a condizione che le mansioni previste non vadano in conflitto tra loro; ad esempio non è ammessa la verifica del proprio lavoro.
Segue il prospetto orario suddiviso per periodi e totale. \\

Nell'intestazione utilizzata per le tabelle di questo capitolo sono state impiegate \textbf{abbreviazioni} per i nomi dei ruoli.
Di seguito viene riportato il loro significato, \textbf{nell'ordine in cui sono utilizzate} nell'intestazione:
\begin{itemize}
\item Amm.: \textit{Amministratore};
\item Ana.: \textit{Analista};
\item Pgt.: \textit{Progettista};
\item Pgr.: \textit{Programmatore};
\item Res.: \textit{Responsabile};
\item Ver.: \textit{Verificatore}.
\end{itemize}

%-----------------------------------------------------------------------------------------------------
%-------------------------------------- ANALISI ------------------------------------------------------
%-----------------------------------------------------------------------------------------------------
\pagebreak
\subsection{Analisi}
Nel periodo di Analisi dei requisiti ciascun componente dovrà rivestire i seguenti ruoli:

\begin{table}[H]
\begin{tabular}{lccccccc}
\toprule
    \textbf{Nome}  & \multicolumn{6}{c}{\textbf{Ore per ruolo}} & \textbf{Ore totali} \\
     & Amm. & Ana. & Pgt. & Pgr. & Res. & Ver. & \\
    \midrule
    
	   Andrea Mantovani & 15 & 0 & 0 & 0 & 10 & 0 & 25 \\
	     Davide Polonio & 18 & 0 & 4 & 0 & 0 & 0 & 22 \\
	      Davide Rigoni & 0 & 14 & 0 & 0 & 0 & 9 & 23 \\
	   Emanuele Carraro & 0 & 12 & 0 & 0 & 0 & 9 & 21 \\
	Giovanni Mazzocchin & 0 & 12 & 0 & 0 & 0 & 9 & 21 \\
	        Luca Bianco & 0 & 12 & 0 & 0 & 0 & 9 & 21 \\
	    Matteo di Pirro & 0 & 0 & 4 & 0 & 23 & 0 & 27 \\
    %totale ore: 160 
    
    \bottomrule
\end{tabular}
\caption{Ore per componente, fase di Analisi}
\end{table}


I valori in tabella sono riassunti nel seguente grafico: \\ 

    \begin{figure}[H]
      \begin{center}
        \includegraphics[width=12cm]{res/img/orePerComponenteAnalisi.png}
      \caption{Ore per componente, fase di Analisi}
      \end{center} 
    \end{figure}    
    
Si fa notare che le ore sopra indicate non sono incluse nelle 105 ore rappresentanti il tetto massimo di ore somministrabile da ciascun componente.


%-----------------------------------------------------------------------------------------------------
%-------------------------------------- ANALISI  MIGLIORAMENTI ---------------------------------------
%-----------------------------------------------------------------------------------------------------
\pagebreak
\subsection{Analisi Miglioramenti}
Nel periodo di Analisi dei requisiti ciascun componente dovrà rivestire i seguenti ruoli:

\begin{table}[H]
\begin{tabular}{lccccccc}
\toprule
    \textbf{Nome}  & \multicolumn{6}{c}{\textbf{Ore per ruolo}} & \textbf{Ore totali} \\
     & Amm. & Ana. & Pgt. & Pgr. & Res. & Ver. & \\
    \midrule
    
	   Andrea Mantovani & 0 & 0 & 3 & 0 & 0 & 0 & 3 \\
	     Davide Polonio & 0 & 0 & 0 & 0 & 1 & 0 & 1 \\
	      Davide Rigoni & 0 & 0 & 0 & 0 & 0 & 0 & 0 \\
	   Emanuele Carraro & 11 & 0 & 0 & 0 & 0 & 0 & 11 \\
	Giovanni Mazzocchin & 0 & 0 & 0 & 0 & 0 & 5 & 5 \\
	        Luca Bianco & 0 & 0 & 0 & 0 & 0 & 6 & 6 \\
	    Matteo di Pirro & 0 & 4 & 0 & 0 & 0 & 0 & 4 \\
    %totale ore: 30 
    
    \bottomrule
\end{tabular}
\caption{Ore per componente, fase di Analisi}
\end{table}


I valori in tabella sono riassunti nel seguente grafico: \\ 

    \begin{figure}[H]
      \begin{center}
        \includegraphics[width=12cm]{res/img/orePerComponenteAnalisi.png}
      \caption{Ore per componente, fase di Analisi}
      \end{center} 
    \end{figure}    
    
Si fa notare che le ore sopra indicate non sono incluse nelle 105 ore rappresentanti il tetto massimo di ore somministrabile da ciascun componente.


%-----------------------------------------------------------------------------------------------------
%-------------------------------------- PROGETTAZIONE ARCHITETTURALE ---------------------------------
%-----------------------------------------------------------------------------------------------------
\pagebreak
\subsection{Progettazione Architetturale}
Nella fase di Progettazione architetturale ciascun componente dovrà rivestire i seguenti ruoli:

\begin{table}[H]
\begin{tabular}{lccccccc}
\toprule
    \textbf{Nome}  & \multicolumn{6}{c}{\textbf{Ore per ruolo}} & \textbf{Ore totali} \\
     & Amm. & Ana. & Pgt. & Pgr. & Res. & Ver. & \\
    \midrule
    
	   Andrea Mantovani & 0 & 3 & 16 & 0 & 0 & 14 & 33 \\
	     Davide Polonio & 0 & 0 & 20 & 0 & 0 & 10 & 30 \\
	      Davide Rigoni & 8 & 0 & 15 & 0 & 0 & 10 & 33 \\
	   Emanuele Carraro & 0 & 0 & 25 & 0 & 0 & 10 & 35 \\
	Giovanni Mazzocchin & 0 & 0 & 20 & 0 & 2 & 10 & 32 \\
	        Luca Bianco & 0 & 0 & 22 & 0 & 0 & 10 & 32 \\
	    Matteo di Pirro & 0 & 0 & 23 & 0 & 0 & 10 & 33 \\
           %totale ore: 228
    
    \bottomrule
\end{tabular}
\caption{Ore per componente, fase di Progettazione architetturale}
\end{table}

I valori in tabella sono riassunti nel seguente grafico: \\ 

    \begin{figure}[H]
      \begin{center}
        \includegraphics[width=12cm]{res/img/orePerComponenteProgettazioneArchitetturale.png}
      \caption{Ore per componente, fase di Progettazione architetturale}
      \end{center} 
    \end{figure}    
    
    
    
    
%-----------------------------------------------------------------------------------------------------
%-------------------------------------- PROGETTAZIONE DI DETTAGLIO E CODIFICA ------------------------
%-----------------------------------------------------------------------------------------------------
\pagebreak
\subsection{Progettazione di Dettaglio e Codifica}
Nella fase di Progettazione di dettaglio e codifica ciascun componente dovrà rivestire i seguenti ruoli:

\begin{table}[H]
\begin{tabular}{lccccccc}
\toprule
    \textbf{Nome}  & \multicolumn{6}{c}{\textbf{Ore per ruolo}} & \textbf{Ore totali} \\
     & Amm. & Ana. & Pgt. & Pgr. & Res. & Ver. & \\
    \midrule
    
	   Andrea Mantovani & 0 & 0 & 20 & 21 & 0 & 12 & 53 \\
	     Davide Polonio & 0 & 3 & 30 & 20 & 0 & 0 & 53 \\
	      Davide Rigoni & 0 & 0 & 20 & 20 & 0 & 14 & 54 \\
	   Emanuele Carraro & 0 & 0 & 0 & 30 & 3 & 16 & 49 \\
	Giovanni Mazzocchin & 25 & 0 & 15 & 15 & 0 & 0 & 55 \\
	        Luca Bianco & 25 & 0 & 15 & 15 & 0 & 0 & 55 \\
	    Matteo di Pirro & 0 & 0 & 20 & 20 & 0 & 12 & 52 \\
           %totale ore: 371
    
    
    \bottomrule
\end{tabular}
\caption{Ore per componente, fase di Progettazione di dettaglio e codifica}
\end{table}

I valori in tabella sono riassunti nel seguente grafico: \\ 

    \begin{figure}[H]
      \begin{center}
        \includegraphics[width=12cm]{res/img/orePerComponenteProgettazioneDettaglioCodifica.png}
      \caption{Ore per componente, fase di Progettazione di dettaglio e codifica}
      \end{center} 
    \end{figure}    
    
    
    
%-----------------------------------------------------------------------------------------------------
%-------------------------------------- VALIDAZIONE --------------------------------------------------
%-----------------------------------------------------------------------------------------------------
\pagebreak
\subsection{Validazione}
Nella fase di Validazione ciascun componente dovrà rivestire i seguenti ruoli:

\begin{table}[H]
\begin{tabular}{lccccccc}
\toprule
    \textbf{Nome}  & \multicolumn{6}{c}{\textbf{Ore per ruolo}} & \textbf{Ore totali} \\
     & Amm. & Ana. & Pgt. & Pgr. & Res. & Ver. & \\
    \midrule
    
	   Andrea Mantovani & 0 & 0 & 6 & 5 & 0 & 5 & 16 \\
	     Davide Polonio & 0 & 0 & 6 & 0 & 0 & 10 & 16 \\
	      Davide Rigoni & 0 & 0 & 0 & 0 & 0 & 15 & 15 \\
	   Emanuele Carraro & 0 & 0 & 0 & 0 & 0 & 15 & 15 \\
	Giovanni Mazzocchin & 0 & 0 & 0 & 0 & 0 & 15 & 15 \\
	        Luca Bianco & 0 & 0 & 0 & 0 & 1 & 10 & 11 \\
	    Matteo di Pirro & 13 & 0 & 0 & 0 & 0 & 5 & 18 \\
           %totale ore: 106 --> totale 895 -160 = 735
    
    \bottomrule
\end{tabular}
\caption{Ore per componente, fase di Verifica}
\end{table}

I valori in tabella sono riassunti nel seguente grafico: \\ 

    \begin{figure}[H]
      \begin{center}
        \includegraphics[width=12cm]{res/img/orePerComponenteVerifica.png}
      \caption{Ore per componente, fase di Verifica}
      \end{center} 
    \end{figure}    
   
    
    
%-----------------------------------------------------------------------------------------------------
%-------------------------------------- TOTALE -------------------------------------------------------
%-----------------------------------------------------------------------------------------------------
\pagebreak
\subsection{Totale}
Il totale delle ore, comprensive delle ore di Analisi dei requisiti che saranno fornite da ciascun membro
del gruppo nel corso dell’intero progetto sono le seguenti:

\begin{table}[H]
\begin{tabular}{lccccccc}
\toprule
    \textbf{Nome}  & \multicolumn{6}{c}{\textbf{Ore per ruolo}} & \textbf{Ore totali} \\
     & Amm. & Ana. & Pgt. & Pgr. & Res. & Ver. & \\
    \midrule
   
	   Andrea Mantovani & 15 & 3 & 45 & 26 & 10 & 31 & 130 \\
	     Davide Polonio & 18 & 3 & 60 & 20 & 1 & 20 & 122 \\
	      Davide Rigoni & 8 & 14 & 35 & 20 & 0 & 48 & 125 \\
	   Emanuele Carraro & 11 & 12 & 25 & 30 & 3 & 50 & 131 \\
	Giovanni Mazzocchin & 25 & 12 & 35 & 15 & 2 & 39 & 128 \\ 
	        Luca Bianco & 25 & 12 & 37 & 15 & 1 & 35 & 125 \\ 
	    Matteo di Pirro & 13 & 4 & 47 & 20 & 23 & 27 & 134 \\ 
   			%totale ore: 895
   
    \bottomrule
\end{tabular}
\caption{Ore per componente totali, inclusa la fase di Analisi}
\end{table}

I valori in tabella sono riassunti nel seguente grafico: \\ 

    \begin{figure}[H]
      \begin{center}
        \includegraphics[width=12cm]{res/img/orePerComponenteTotaliAnalisi.png}
      \caption{Ore per componente totali, inclusa la fase di Analisi}
      \end{center} 
    \end{figure}    
    

Quelle esenti dall'Analisi dei requisiti :

\begin{table}[H]
\begin{tabular}{lccccccc}
\toprule
    \textbf{Nome}  & \multicolumn{6}{c}{\textbf{Ore per ruolo}} & \textbf{Ore totali} \\
     & Amm. & Ana. & Pgt. & Pgr. & Res. & Ver. & \\
    \midrule
   
	   Andrea Mantovani & 0 & 3 & 45 & 26 & 0 & 31 & 105 \\
	     Davide Polonio & 0 & 3 & 56 & 20 & 1 & 20 & 100 \\
	      Davide Rigoni & 8 & 0 & 35 & 20 & 0 & 39 & 102 \\
	   Emanuele Carraro & 11 & 0 & 25 & 30 & 3 & 41 & 110 \\
	Giovanni Mazzocchin & 25 & 0 & 35 & 15 & 2 & 30 & 107 \\
	        Luca Bianco & 25 & 0 & 37 & 15 & 1 & 26 & 104 \\
	    Matteo di Pirro & 13 & 4 & 43 & 20 & 0 & 27 & 107 \\
   			%totale ore: 735
   
    \bottomrule
\end{tabular}
\caption{Ore per componente totali, inclusa la fase di Analisi}
\end{table}

I valori in tabella sono riassunti nel seguente grafico: \\ 

    \begin{figure}[H]
      \begin{center}
        \includegraphics[width=12cm]{res/img/orePerComponenteTotaliAnalisi.png}
      \caption{Ore per componente totali, inclusa la fase di Analisi}
      \end{center} 
    \end{figure}    
    
    
