\section{Suddivisione del lavoro e prospetto orario}
Ogni componente del gruppo dovrà ricoprire ogni ruolo almeno una volta nel corso del progetto.
Durante lo stesso periodo un componente può ricoprire più ruoli, a condizione che le mansioni previste non vadano in conflitto tra loro, ad esempio non si può verificare il proprio lavoro.
Segue il prospetto orario suddiviso per periodi e totale. \\

Nell'intestazione utilizzata per le tabelle di questo capitolo sono state impiegate \textbf{abbreviazioni} per i nomi dei ruoli.
Di seguito viene riportato il loro significato, \textbf{nell'ordine in cui sono utilizzate} nell'intestazione:
\begin{itemize}
\item Amm.: \textit{Amministratore};
\item Ana.: \textit{Analista};
\item Pgt.: \textit{Progettista};
\item Pgr.: \textit{Programmatore};
\item Res.: \textit{Responsabile};
\item Ver.: \textit{Verificatore}.
\end{itemize}

\pagebreak
\subsubsection{Analisi}
Nel periodo di Analisi dei requisiti ciascun componente dovrà rivestire i seguenti ruoli:
\noindent
\begin{table}[H]
\begin{tabular}{lccccccc}
\toprule
    \textbf{Nome}  & \multicolumn{6}{c}{\textbf{Ore per ruolo}} & \textbf{Ore totali} \\
     & Amm. & Ana. & Pgt. & Pgr. & Res. & Ver. & \\
    \midrule
    
	Andrea Mantovani & 0 & 12 & 9 & 0 & 0 & 2 & 23 \\
	Davide Polonio & 0 & 9 & 0 & 0 & 4 & 9 & 22 \\
	Davide Rigoni & 9 & 2 & 0 & 0 & 5 & 4 & 20 \\
	Emanuele Carraro & 0 & 10 & 0 & 0 & 0 & 11 & 21 \\
	Giovanni Zecchin & 6 & 11 & 0 & 0 & 0 & 0 & 17 \\
	Luca Bianco & 15 & 0 & 0 & 0 & 5 & 0 & 20 \\
	Matteo di Pirro & 0 & 14 & 0 & 0 & 0 & 5 & 19 \\
    
    \bottomrule
\end{tabular}
\caption{Ore per componente, periodo di Analisi}
\end{table}

I valori in tabella sono riassunti nel seguente grafico: \\ 

Si fa notare che le ore sopra indicate non sono incluse nelle 105 ore rappresentanti il tetto massimo di ore somministrabile da ciascun componente.