\documentclass[11pt]{meetingmins}
\usepackage[utf8]{inputenc}

\setcommittee{BugBusters}

\setmembers{

  D.~Polonio,
  D.~Rigoni,
  M.~Di Pirro,
  G.~Mazzocchin,
  L.~Bianco,
  A.~Mantovani,
  E.~Carraro
}

\setdate{February 23, 2016}

\setpresent{
  
  D.~Polonio,
  D.~Rigoni,
  A.~Mantovani,
  E.~Carraro,
  M.~Di Pirro
}

\absent{
  G.~Mazzocchin
  L.~Bianco
}

\alsopresent{None}

\begin{document}

\maketitle

\section{Ordine del giorno}

\begin{enumerate}

  %da vedere slack
  \item Nuova gestione dei repository
  \item Plugin per Chronos
  \item Compiti da assegnare

\end{enumerate}

\section{Orari}

\begin{itemize}
\item[Inizio]: ore 14:15
\item[Fine]: ore 16:00
\item[Tempo]: Previsto di 1:30 ora, effettivo di 1:45 ore

\end{itemize}

\section{Verbale}

\subsection{Nuova gestione dei repository}
Il repository documentation ha cambiato struttura. Inoltre non è più permesso fare push in master; per aggiungere qualcosa nel repository va prima creato un branch associato a una issue, fare una pull request e solo se le nuove aggiunte compilano è permesso mergiare in master.

\subsection{Plugin per Chronos}

Da fare:

\begin{enumerate}
\item Ambiente di lavoro, Pre Commit, Ricezione task
\item Glossario, Gulpease, Aspell
\item Dashboard
\end{enumerate}

Descrizione\\

\textbf{1) Ambiente di lavoro}

\begin{itemize}
\item Aggiungere autore, issue
\item Visualizzazione finestra per commit
\item Registro modifiche
\item Ricezione task
\item Creare automaticamente il branch
\end{itemize}

\textbf{2) Documentazione}

\begin{itemize}
\item Parole del glossario
\item Gulpease
\item Aspell
\end{itemize}

\textbf{3) Dashboard}\\

Proposta di Andrea:

\begin{itemize}
\item Invio task
\item Associazione persone - task
\item Tempistiche
\item Calendario
\end{itemize}

(queste funzionalità sono già disponibili su Teamwork)

\subsection{Compiti}

\begin{itemize}
\item Sistemare i documenti seguendo le indicazioni di Tullio e le issue su Github. (ASSEGNATO A: G.~Mazzocchin, E.~Carraro, L.~Bianco)
\item Scrivere i vari plugin per Chronos. (ASSEGNATO A: M.~Di Pirro, A.~Mantovani)
\item Inserire i requisiti su Tracy. (ASSEGNATO A D.~Polonio, D.~Rigoni.)
\item Va presa una decisione sulla revisione di progettazione a cui partecipare (RPmin o RPmax).
\end{itemize}

La fase di miglioramento (stabilita nel PdP) termina il 26/01. Durante (ed entro) questo periodo va completata prima di tutto la revisione dei documenti, correggendo gli errori indicati da Tullio e seguendo le issue su Github.
Si è deciso di posticipare la scadenza (cioè la data entro la quale devono essere svolti i compiti) a mercoledì 2 marzo.

L'incontro è terminato alle 16:00 (durata: 1h45min)\\

\nextmeeting{March 02, 2016}

\end{document}
