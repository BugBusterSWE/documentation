\documentclass[11pt]{meetingmins}
\usepackage[utf8]{inputenc}

\setcommittee{BugBusters}

\setmembers{

  D.~Polonio,
  D.~Rigoni,
  M.~Di Pirro,
  G.~Mazzocchin,
  L.~Bianco,
  A.~Mantovani,
  E.~Carraro
}

\setdate{March 30, 2016}

\setpresent{
  
  D.~Polonio,
  L.~Bianco,
  G.~Mazzocchin,
  M.~Di Pirro,
  A.~Mantovani,
  E.~Carraro,
  D.~Rigoni
}

\absent{None}

\alsopresent{None}

\begin{document}

\maketitle

\section{Ordine del giorno}

\begin{enumerate}

  \item Punto della situazione sull'andamento;
  \item Proposta stesura presentazione;
  \item Spiegare il funzionamento e l'uso di Chronos;
  \item Visione di CodeMirror;
  \item Pianificazione delle attività per la prossima fase;
  \item Task di stesura della presentazione per RP.

\end{enumerate}

\section{Orari}

\begin{itemize}
\item[Inizio]: ore 09:50
\item[Fine]: ore 11:10
\item[Tempo]: Previsto di 90 min, effettivo di 80 min

\end{itemize}

\section{Verbale}

\subsection{Punto della situazione sull'andamento}

Il gruppo analizza la situazione e fa notare i difetti dell'attuale organizzazione e propone alternative in vista dei prossimi periodi. Vengone discusse le metodologie di svolgimento dei test.

\subsection{Proposta stesura presentazione}

Viene deciso di seguire la regola 5x5, come scritto nelle norme di progetto. Verranno aggiunte più immagini per spiegare quanto verrà detto. Ogni slide dovrà avere durata di circa 60", non ci saranno ';' alla fine degli elenchi e ':' prima. Lo spazio bianco verrà riempito il più possibile. I diagrammi saranno il più possibile leggibili, anche da lontano.

\subsection{Spiegare il funzionamento e l'uso di Chronos}

Viene deciso di utilizzare pienamente Chronos per il prossimo periodo, in particolare per controllo grammaticale (Aspell), controllo di leggibilità (Gulpease) e gestione del glossario. 

\subsection{Visione di CodeMirror}

Viene deciso di integrare CodeMirror per l'editor testuale di MaaS.

\subsection{Pianificazione delle attività per la prossima fase}

Viene deciso di pianificare a priori le attività da effettuare per il prossimo periodo. L'assegnazione dei singoli task avverrà per ogni settimana. Viene deciso che i test saranno scritti prima del codice da testare. 

\subsection{Task di stesura della presentazione per RP}

Viene deciso che D.~Polonio stenderà la struttura della presentazione.

\nextmeeting{da definire}

\end{document}
