\documentclass[11pt]{meetingmins}
\usepackage[utf8]{inputenc}

\setcommittee{BugBusters}

\setmembers{

  D.~Polonio,
  D.~Rigoni,
  M.~Di Pirro,
  G.~Mazzocchin,
  L.~Bianco,
  A.~Mantovani,
  E.~Carraro
}

\setdate{March 7, 2016}

\setpresent{
  
  D.~Polonio,
  L.~Bianco,
  G.~Mazzocchin,
  M.~Di Pirro,
  A.~Mantovani,
  E.~Carraro
}

\absent{
   D.~Rigoni
}

\alsopresent{None}

\begin{document}

\maketitle

\section{Ordine del giorno}

\begin{enumerate}

  %da vedere slack
  \item Decisione sulla struttura della Specifica Tecnica da adottare
  \item Decisione sulla partecipazione ad RPmin o RPmax 
  \item Decisione della ripartizione dei compiti da svolgere per il seguente periodo

\end{enumerate}

\section{Orari}

\begin{itemize}
\item[Inizio]: ore 11:15
\item[Fine]: ore 11:45
\item[Tempo]: Previsto di 30 min, effettivo di 30 min

\end{itemize}

\section{Verbale}

\subsection{Struttura della specifica tecnica da adottare}

A seguito delle considerazioni fatte sulla strategia da utilizzare per la stesura della specifica tecnica, il gruppo ha deciso di decidere lo schema finale da adottare per la stesura della specifica tecnica in seguito ad un periodo esplorativo, sia per i contenuti che per la struttura ottimale da utilizzare per il progetto in questione.

Le parti importate da MaaP non andranno documentate da 0 ma riportate come fossero librerie esterne.

\subsection{Partecipazione ad RPmin o RPmax}

Il gruppo decide che la revisione da sostenere sarà RPmin

\subsection{Ripartizione del lavoro}

Si è decisa la ripartizione del lavoro per il gruppo per il prossimo periodo.

\begin{itemize}
\item A.~Mantovani si occuperà della fine dello sviluppo di Chronos
\item D.~Rigoni finirà la configurazione e l'importazione dei dati in Tracy
\item D.~Polonio finirà la configurazione di Jenkins
\item Il resto del gruppo si dedicherà alla progettazione e alla stesura della specifica tecnica
\end{itemize}

\nextmeeting{March 14, 2016}

\end{document}
