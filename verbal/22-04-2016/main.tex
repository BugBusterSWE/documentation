\documentclass[11pt]{meetingmins}
\usepackage[utf8]{inputenc}

\setcommittee{BugBusters}

\setmembers{

  D.~Polonio,
  D.~Rigoni,
  M.~Di Pirro,
  G.~Mazzocchin,
  L.~Bianco,
  A.~Mantovani,
  E.~Carraro
}

\setdate{April 22, 2016}

\setpresent{
  
  D.~Polonio,
  L.~Bianco,
  M.~Di Pirro,
  E.~Carraro,
  D.~Rigoni
}

\absent{

  G.~Mazzocchin,
  A.~Mantovani
}

\alsopresent{None}

\begin{document}

\maketitle

\section{Ordine del giorno}

\begin{enumerate}

\item Discussione su come pianificare la codifica per l'organizzazione del gruppo
\item Discussione su assegnazione dei seguenti task da svolgere:
  
  \begin{itemize}

  \item creazione del repository per MaaS
  \item configurazione di Jenkins per NodeJs
  \item configurazione repository MaaS per Jenkins
  \item configurazione repository MaaS per Travis-CI
  \item messa a punto di Tracy per l'utilizzo della funzionalità di tracciamento delle classi
  \item aggiunta funzionalità al bot per l'assegnazione automatica dei verificatori alle pull requests.
  \end{itemize}

\end{enumerate}

\section{Orari}

\begin{itemize}
\item[Inizio]: ore 17:30
\item[Fine]: ore 18:30
\item[Tempo]: Previsto di 60 min, effettivo di 60 min

\end{itemize}

\section{Verbale}

\subsection{Discussione su come pianificare la codifica per l'organizzazione del gruppo}

Si discute su come iniziare a scrivere MaaS, elencando tutti i vantaggi e gli svantaggi di diverse strategie di implementazioni. Inoltre, sono state definite le attività da svolgere e sono stati identificati i principali task.

\nextmeeting{da definire}

\end{document}
