\documentclass[11pt]{meetingmins}
\usepackage[utf8]{inputenc}

\setcommittee{BugBusters}

\setmembers{

  D.~Polonio,
  D.~Rigoni,
  M.~Di Pirro,
  G.~Mazzocchin,
  L.~Bianco,
  A.~Mantovani,
  E.~Carraro
}

\setdate{May 24, 2016}

\setpresent{
  
  D.~Polonio,
  E.~Carraro,
  D.~Rigoni,
}

\absent{

  A.~Mantovani,
  L.~Bianco,
  M.~Di Pirro,
  G.~Mazzocchin

}

\alsopresent{None}

\begin{document}

\maketitle

\section{Ordine del giorno}

\begin{enumerate}

\item Decidere il formato migliore per il manuale sviluppatore
\item Spiegazione del Workflow per Heroku
\end{enumerate}

\section{Orari}

\begin{itemize}
\item[Inizio]: ore xx:xx
\item[Fine]: ore xx:xx
\item[Tempo]: Previsto di xx min, effettivo di xx min

\end{itemize}

\section{Verbale}

\subsubsection{Decidere il formato migliore per il manuale sviluppatore}

\subsubsection{Spiegazione del Workflow per Heroku}
L'amministratore spiega il workflow per Heroku. Vengono descritte le funzionalità di Heroku, e viene spiegato il sistema di automatizzazione implementato: è presente una pipeline, e a ogni pull request viene creata una istanza temporanea di Heroku (di durata massima di cinque giorni) che permette a un verificatore, oltre ai test, di vedere i risultati di un lavoro svolto da un membro del gruppo.
Una particolare istanza di Heroku viene aggiornata a ogni modifica nel branch master del repository se e solo se i controlli di Travis CI passano, mentre a ogni nuova release di MaaS viene aggiornata una istanza diversa di Heroku posta in ``produzione''.
Con questo workflow viene evidenziato come non sia necessario l'intervento dei membri del gruppo, a meno di interventi di manutenzione o di modifica del workflow stesso.

\nextmeeting{June 06, 2016}

\end{document}
