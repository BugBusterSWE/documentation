\input{../../common/style/global.tex}
\NeedsTeXFormat{LaTeX2e}[1994/06/01]
\ProvidesPackage{bugbusters}[2016/01/10 Custom Package]


%-------------------Package used -----------------------------------------
\RequirePackage[utf8]{inputenc} % we use uft8
\RequirePackage[italian]{babel} % language: note that this isn't for all documents
\RequirePackage{graphicx}
\RequirePackage{url} % per scrivere gli indirizzi In
\RequirePackage{hyperref}
\RequirePackage{amsmath}
\RequirePackage{amsfonts}
\RequirePackage{listings}
%-------------------------------------------------------------------------

%%%%%%%%%%%%%%
%  COSTANTI  %
%%%%%%%%%%%%%%

% In questa prima parte vanno definite le 'costanti' utilizzate soltanto da questo documento.
% Devono iniziare con una lettera maiuscola per distinguersi dalle funzioni.

\newcommand{\DocTitle}{Glossario}
\newcommand{\DocVersion}{\VersioneG{}}

\newcommand{\DocRedazione}{Matteo Di Pirro}
\newcommand{\DocVerifica}{Luca Bianco, Davide Rigoni}
\newcommand{\DocApprovazione}{\Responsabile}

\newcommand{\DocUso}{Esterno}
\newcommand{\DocDistribuzione}{
	\Committente{} \\
	Gruppo \GroupName{}
}

% La descrizione del documento
\newcommand{\DocDescription}{
  Questo documento contiene il Glossario dei termini della documentazione del progetto.
}

%%%%%%%%%%%%%%
%  FUNZIONI  %
%%%%%%%%%%%%%%

% In questa seconda parte vanno definite le 'funzioni' utilizzate soltanto da questo documento.


\title{\textbf{Norme di Progetto}}
\author{BugBuster}

\date{17 giugno 2016}

\begin{document}

\makeFrontPage

\paragraph*{Data}: 2016/06/06

\paragraph*{Membri del gruppo}
\begin{itemize}

\item D. Polonio
\item D. Rigoni
\item M. Di Pirro
\item G. Mazzocchin
\item L. Bianco
\item A. Mantovani
\item E. Carraro
\end{itemize}

\paragraph*{Membri presenti}
\begin{itemize}
\item D. Polonio
\item D. Rigoni
\item A. Mantovani
\end{itemize}

\paragraph*{Membri assenti}
\begin{itemize}
\item M. Di Pirro
\item G. Mazzocchin
\item L. Bianco
\item E. Carraro
\end{itemize}

\section{Ordine del giorno}

\begin{enumerate}
\item Punto della situazione
\end{enumerate}

\section{Orari}

\begin{itemize}
\item[Inizio]: ore 10:00
\item[Fine]: ore 10:30
\item[Tempo]: Previsto di 30 min, effettivo di 30 min

\end{itemize}

\section{Verbale}

\subsubsection{Punto della situazione}
Il gruppo fa il punto della situazione su eventuali problemi e stilando i task completati e nota che l'andamento consuma due giorni di slack previsti dal grafico di Gantt, pianificato dal responsabile.
\paragraph*{Prossimo incontro}: 2016/06/13

\end{document}
