\documentclass[11pt]{meetingmins}
\usepackage[utf8]{inputenc}

\setcommittee{BugBusters}

\setmembers{

  D.~Polonio,
  D.~Rigoni,
  M.~Di Pirro,
  G.~Mazzocchin,
  L.~Bianco,
  A.~Mantovani,
  E.~Carraro
}

\setdate{May 5, 2016}

\setpresent{
  
  D.~Polonio,
  E.~Carraro,
  D.~Rigoni,
  A.~Mantovani
  G.~Mazzocchin
}

\absent{

  L.~Bianco,
  M.~Di Pirro
}

\alsopresent{None}

\begin{document}

\maketitle

\section{Ordine del giorno}

\begin{enumerate}

\item Note dal taccuino sulle Norme
\item Pianificazione e assegnazione dei \textit{task} per la stesura di MaaS
\item Controllare codice \textit{backend} presentato da Luca Bianco
\item Definire le regole da inserire nelle norme in base ai punti segnati nel taccuino


\end{enumerate}

\section{Orari}

\begin{itemize}
\item[Inizio]: ore 11:45
\item[Fine]: ore 12:30
\item[Tempo]: Previsto di 45 min, effettivo di 45 min

\end{itemize}

\section{Verbale}

\subsection{Note dal taccuino di Teamwork sulle Norme}
	\paragraph*{Creazione Pull Request}
		Inserire le stesse \textit{label} del \textit{task} corrispondente, inoltre va inserita la 
		La procedura: un membro termina il proprio \textit{task}, apre la \textit{Pull Request} con le stesse 
		\textit{label} del \textit{task}, inoltre anche la \textit{milestone} deve essere 
		uguale a quella del task. Il verificatore chiude il task.
		Quando si termina un task va chiusa la issue corrispondente.

	\paragraph*{Descrizione commit, come specificare le modifiche a seconda del tipo di file}
		Al momento dell'apertura di una Pull Request, il suo creatore deve segnalare le sezioni modificate e
		le motivazioni (va inserita una descrizione abbastanza approfondita).

	\paragraph*{Definire una regola corretta per i nomi del glossario}
		Il responsabile dovr\`a occuparsi di questo problema.

	\paragraph*{Nome fasi in maiuscolo o minuscolo}
		\`E stato stabilito che i nomi delle fasi vanno tutti in maiuscolo.		
		Se una voce \`e composta da pi\`u termini, tutti i singoli termini devono avere l'iniziale maiuscola.

	\paragraph*{Procedura apertura issue. Il branch dell'attivit\`a da chi \`e aperto? Quando viene concluso un task cosa bisogna fare?}
		L'assegnatario dell'attivitit\`a apre il \textit{branch} corrispondente all'activity.				

	\paragraph*{Se l'attivit\`a riguarda una modifica quale versione va considerata?}
		La versione nel branch in cui verr\`a applicata la modifica? O sull'ultima \textit{release}?)
		\`E stato stabilito che il responsabile applicher\`a la procedura adeguata per il problema in esame.	

	\paragraph*{Conflitti nelle Pull Request}
		\`E stato stabilito che in caso di conflitto il verificatore operer\`a congiuntamente all'assegnatario della
		\textit{issue} per risolvere problema.		
	        
	

	\paragraph*{Per ogni tecnologia bisogna mettere vantaggi e svantaggi?}
		Vantaggi e svantaggi andranno sempre inseriti.

	\paragraph*{\textbf{Specifica tecnica}: l'\textit{editor} e l'interprete hanno i titoli in cui ogni elemento è preceduto 
		dal package (es: DSLCreator::DSLElement) mentre il \textit{frontend} e \textit{backend} non lo fanno. La notazione andrebbe uniformata}
		Da definire.
		
		
	\paragraph*{Decidere il momento della chiusura delle \textit{issue}: una volta completato il task o dopo l'approvazione della PR}
		La issue va chiusa immediatamente dopo la chiusura di una Pull Request.
		
	\paragraph*{Convenzione nomi file}
		I nomi dei file andranno in inglese, in minuscolo e seguendo la convenzione \textit{camelCase}.

\subsection{Pianificazione e assegnazione dei task per la stesura di MaaS}


\subsection{Controllare codice \textit{backend} presentato da Luca Bianco}


\subsection{Definire le regole da inserire nelle norme in base ai punti segnati nel taccuino}




\nextmeeting{09/05/2016}

\end{document}
