\documentclass[11pt]{meetingmins}
\usepackage[utf8]{inputenc}

\setcommittee{BugBusters}

\setmembers{

  D.~Polonio,
  D.~Rigoni,
  M.~Di Pirro,
  G.~Mazzocchin,
  L.~Bianco,
  A.~Mantovani,
  E.~Carraro
}

\setdate{February 23, 2016}

\setpresent{
  
  D.~Polonio,
  D.~Rigoni,
  A.~Mantovani,
  E.~Carraro
}

\absent{
   M.~Di Pirro
}

\alsopresent{None}

\begin{document}

\maketitle

\section{Ordine del giorno}

\begin{enumerate}

  %da vedere slack
  \item Stile codifica e nome cartelle
  \item Prossimi incontri
  \item Opzioni di Tracy
  \item Chronos

\end{enumerate}

\section{Orari}

\begin{itemize}
\item[Inizio]: ore 11:00
\item[Fine]: ore 11:30
\item[Tempo]: Previsto di 30 min, effettivo di 30 min

\end{itemize}

\section{Verbale}

\subsection{Stile codifica e nome cartelle}

Si \`{e} scelto di usare Standard (JavaScript Standard Style - https://github.com/feross/standard) per validare lo stile di codice JavaScript.\\
Si \`{e} scelto di usare RisingStack Node.js Style Guide come convenzioni per lo stile di codifica. https://github.com/RisingStack/node-style-guide\\

Si \`{e} scelto di usare la segunte convenzione (python-like) per i nomi delle cartelle: lettere minuscole separate da underscore.\\
Si \`{e} scelto di usare la convenzione CamelCase per i nomi delle classi e delle variabili.

\subsection{Organizzazione incontri}

Si \`{e} proposto di incontrarsi: luned\`{i}, marted\`{i}, mercoled\`{i} della prossima settimana, dopo le lezioni di ingegneria del software e mercoled\`{i} alle prime ore.\\

\subsection{Opzioni di Tracy}

Sono state valutate le opzioni da mantenere e rimuovere da Tracy (compito assegnato a: D.~Rigoni) ed è stato deciso che:
\begin{itemize}
\item verrà utilizzato solamente per la creazione dei casi d'uso e il tracciamento dei requisiti;
\item le altre funzioni devono essere rimosse;
\item se possibile si vorrebbe integrare la gestione del glossario.
\end{itemize}

\subsection{Chronos}

Spiegazione di Andrea su come utilizzare Chronos.\\

L'incontro è terminato alle 11:30 (durata: 30min)\\

\nextmeeting{March 07, 2016}

\end{document}
