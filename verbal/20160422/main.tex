\documentclass[11pt]{meetingmins}
\usepackage[utf8]{inputenc}

\setcommittee{BugBusters}

\setmembers{

  D.~Polonio,
  D.~Rigoni,
  M.~Di Pirro,
  G.~Mazzocchin,
  L.~Bianco,
  A.~Mantovani,
  E.~Carraro
}

\setdate{April 22, 2016}

\setpresent{
  
  D.~Polonio,
  L.~Bianco,
  M.~Di Pirro,
  E.~Carraro,
  D.~Rigoni
}

\absent{

  G.~Mazzocchin,
  A.~Mantovani
}

\alsopresent{None}

\begin{document}

\maketitle

\section{Ordine del giorno}

\begin{enumerate}

\item Discussione su come pianificare l'organizzazione del gruppo

\item Discussione su nuova ridenominazione dei verbali

\item Discussione su l'ordine di codifica di MaaS

\end{enumerate}

\section{Orari}

\begin{itemize}
\item[Inizio]: ore 17:30
\item[Fine]: ore 18:30
\item[Tempo]: Previsto di 60 min, effettivo di 60 min

\end{itemize}

\section{Verbale}

\subsection{Discussione su come pianificare l'organizzazione del gruppo}

Si discute su come iniziare a scrivere MaaS, elencando tutti i vantaggi e gli svantaggi di diverse strategie di implementazioni. Inoltre, sono state definite le attività da svolgere e sono stati identificati i principali task.

\subsection{Discussione su nuova ridenominazione dei verbali}

Viene approvato da tutti di rinominare i verbali secondo l'identificativo suggerito in fase di correzione della Revisione di Progettazione. La rinominazione viene stabilito avere effetto retroattivo, per poter uniformare l'archivio di tutti i verbali fin'ora scritti.

\subsection{Discussione su l'ordine di codifica di MaaS}

Si discute su una possibile strategia per la codifica di MaaS, valutando attentamente svantaggi e vantaggi di eventuali scelte strategiche. Alla fine, viene deciso per sviluppare MaaS nel seguente ordine:
\begin{enumerate}

\item Backend
\item Frontend
\item Interprete
\item Editor grafico
\end{enumerate}

Con questo ordine di sviluppo sar\`a possibile ridurre al minimo il numero di \textit{stub} e \textit{driver} necessari e consentir\`a al gruppo una maggiore velocit\`a nello sviluppo. L'editor grafico viene tenuto alla fine in quanto essendo facoltativo verr\`a codificato nei tempi di \textit{slack} accomulato per essere terminato, se possibile, nella prossima fase (revisione d'accettazione).


\nextmeeting{da definire}

\end{document}
