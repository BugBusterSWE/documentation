\documentclass[11pt]{meetingmins}
\usepackage[utf8]{inputenc}

\setcommittee{BugBusters}

\setmembers{

  D.~Polonio,
  D.~Rigoni,
  M.~Di Pirro,
  G.~Mazzocchin,
  L.~Bianco,
  A.~Mantovani,
  E.~Carraro
}

\setdate{April 22, 2016}

\setpresent{
  
  D.~Polonio,
  L.~Bianco,
  M.~Di Pirro,
  E.~Carraro,
  D.~Rigoni
}

\absent{

  G.~Mazzocchin,
  A.~Mantovani
}

\alsopresent{None}

\begin{document}

\maketitle

\section{Ordine del giorno}

\begin{enumerate}

\item Discussione su come pianificare l'organizzazione del gruppo

\item Discussione su nuova ridenominazione dei verbali

\end{enumerate}

\section{Orari}

\begin{itemize}
\item[Inizio]: ore 17:30
\item[Fine]: ore 18:30
\item[Tempo]: Previsto di 60 min, effettivo di 60 min

\end{itemize}

\section{Verbale}

\subsection{Discussione su come pianificare la codifica per l'organizzazione del gruppo}

Si discute su come iniziare a scrivere MaaS, elencando tutti i vantaggi e gli svantaggi di diverse strategie di implementazioni. Inoltre, sono state definite le attività da svolgere e sono stati identificati i principali task.

\substction{Discussione su nuova ridenominazione dei verbali}

Viene approvato da tutti di rinominare i verbali secondo l'identificativo suggerito in fase di correzione della Revisione di Progettazione. La rinominazione viene stabilito avere effetto retroattivo, per poter uniformare l'archivio di tutti i verbali fin'ora scritti.

\nextmeeting{da definire}

\end{document}
