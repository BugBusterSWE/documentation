\documentclass[11pt]{meetingmins}
\usepackage[utf8]{inputenc}

\setcommittee{BugBusters}

\setmembers{

  D.~Polonio,
  D.~Rigoni,
  M.~Di Pirro,
  G.~Mazzocchin,
  L.~Bianco,
  A.~Mantovani,
  E.~Carraro
}

\setdate{March 14, 2016}

\setpresent{
  
  D.~Polonio,
  D.~Rigoni,
  A.~Mantovani,
  E.~Carraro,
  M.~Di Pirro,
  G.~Mazzocchin,
  L.~Bianco
}

\absent{}

\alsopresent{None}

\begin{document}

\maketitle

\section{Ordine del giorno}

\begin{enumerate}
 \item Scelta tra progettazione ad oggetti o funzionale
 \item Scelta tra HTML5 o XHTML
 \item Vale la pena ancora utilizzare teamwork per la gestione dei task?
 \item Codice univoco per decisioni delle riunioni 
\end{enumerate}

\section{Orari}

\begin{itemize}
\item[Inizio]: ore 11:15
\item[Fine]: ore 11:50
\item[Tempo]: Previsto di 45 minuti, effettivo di 35 minuti

\end{itemize}

\section{Verbale}

\subsection{Progettazione ad oggetti o funzionale}
La progettazione funzionale in Javascript risulterebbe interessante ed esplorativa ma la poca documentazione trovata in Internet su tale argomento evidenzia il fatto che una scelta del genere sia un potenziale rischio per il progetto.
Pertanto si è deciso di continuare con la progettazione ad oggetti.

\subsection{Scelta tra HTML 5 o XHTML}

All'unanimità il gruppo decide HTML5

\subsection{Vale ancora la pena di utilizzare teamwork per la gestione dei task?}

Teamwork è ancora utile per la gestione dei task ma risulta difficoltoso al momento nell'utilizzo per la troppa eterogeneità  dei template utilizzati per i task aperti. SI sceglie quindi di lasciare la gestione della creazione dei task a solo il responsabile di progetto. Gli altri membri avranno l'accesso solo in lettura

\subsection{Codice univoco per le decisioni prese durante le riunioni}
È sorta la necessità di tracciare le decisioni prese nelle riunioni interne. Pertanto viene introdotta la notazione nella documentazione al riferimento dei verbali. La notazione sarà del tipo:

RI-ggmm-s

dove: 
\begin{itemize}
\item RI: identifica "Riunione interna" 
\item gg: identifica il giorno
\item mm: identifica il mese
\item s: identifica il punto nel verbale
\end{itemize}

\nextmeeting{March 21, 2016}

\end{document}
